\documentclass[11pt]{jarticle}

\usepackage{amsmath}
\usepackage{graphics}
\usepackage{hyperref}

\setlength{\oddsidemargin}{-0.7cm}
\setlength{\topmargin}{-1.5cm}
\setlength{\textwidth}{16.5cm}
\setlength{\textheight}{26cm}
\pagestyle{empty}

\begin{document}

\noindent
{\bf\large 「計算機実験I」実習課題(EX4)}
\\[-0.5em]

\noindent
\begin{itemize}

\item サンプルプログラム: example-1-L4.zip
  
\item 準備練習
  \begin{enumerate}
  \item ハウスホルダー法による対角化のサンプルプログラム({\tt diag.c})をコンパイル・実行せよ
    \begin{quote} \tt
      \$ \underline{cc diag.c -o diag -llapack -lm} \\
      \$ \underline{./diag matrix1.dat}
    \end{quote}
  \item {\tt diag.c}で得られた固有ベクトルが互いに正規直交していることを確認するコードを作成し実行せよ。(それぞれの固有ベクトルを行とする行列とその転置行列をかけて、単位行列になることを確認すればよい。可能であればBLASライブラリの{\tt dgemm} [EX3応用課題3]を使ってみよ)
  \end{enumerate}

\item 基本課題
  \begin{enumerate}
  \item 対角成分は$(n,n)$成分のみが1でそれ以外は全て2、副対角成分$(i,\pm i)$は全て-1の$n \times n$の三重対角行列を考える。その固有値は、
    \[ \lambda_k = 2 (1 - \cos (\pi (2 k - 1) / (2 n + 1))) \ \ (k=1,\cdots,n)\]
    で与えられる。べき乗法を用いて、最大固有値を計算するプログラムを作成し、その結果を理論式と比較せよ。(上の式では添字は1から始まっているが、C言語では0から始まることに注意)

    また、$n$を変えた時、最大固有値の収束の速さがどのように変わるか調べよ。計算時間をハウスホルダー法と比較せよ
  \item ファイル{\tt measurement1.dat}に、ある実験で得られたデータが収められている。1カラム目は$x$、2カラム目は$y$、3カラム目は$y$の誤差の値である。最小二乗法によりデータを多項式でフィッティングするプログラムを作成せよ。(実習EX3で用いたLU分解、あるいは応用課題2の特異値分解を使って良い)

    多項式の最大次数を大きくしていくとフィッティング結果はどのように変化するか?何次の多項式を使うのが最も良いと考えられるか考察せよ

    また、同様の解析をファイル{\tt measurement2.dat}に対して行ってみよ
  \end{enumerate}  
\item 応用課題
  \begin{enumerate}
  \item Lanczos法により固有値を計算するプログラムを作成せよ。Ritz値が、繰り返しに従ってどのように振る舞うか図示してみよ。収束の速さをべき乗法と比較せよ
  \item 特異値分解のサンプルプログラム({\tt svd.c})をコンパイル・実行せよ。入力{\tt matrix2.dat}を用いて、講義L4の例が再現されるかどうか確認せよ
    \begin{quote} \tt
      \$ \underline{cc svd.c -o svd -llapack -lm} \\
      \$ \underline{./svd matrix2.dat}
    \end{quote}
    さらに、完全SVDを行うプログラム例({\tt full\_svd.c})、$m<n$の行列({\tt matrix3.dat})、ランク落ちしている正方行列({\tt matrix4.dat})も用意されている。出力結果を確認してみよ
  \item 画像ファイルを行列形式に変換\footnote{\url{https://github.com/todo-group/computer-experiments/tree/master/misc}に、JPEGやPNGなどの形式の画像ファイルをグレイスケールに変換し、行列の形で書き出すPythonスクリプトがある}し、SVDで圧縮してみよ。どの程度まで圧縮可能か?
  \end{enumerate}  

\item レポート課題No.2

  レポート課題の内容、提出方法、締切については、ITC-LMSに掲示するので確認すること
\end{itemize}

\end{document}
