\documentclass[11pt]{jarticle}

\usepackage{amsmath}
\usepackage{graphics}
\usepackage{hyperref}

\setlength{\oddsidemargin}{-0.7cm}
\setlength{\topmargin}{-1.5cm}
\setlength{\textwidth}{16.5cm}
\setlength{\textheight}{26cm}
\pagestyle{empty}

\begin{document}

\noindent
{\bf\large 「計算機実験」実習課題(EX5)}
\\[-0.5em]

\noindent
\begin{itemize}
\item 講義のページ: \verb+http://exa.phys.s.u-tokyo.ac.jp/ja/lectures/2016S-computer+

\item サンプルプログラム: \\ {\small \verb+https://github.com/todo-group/computer-experiments/tree/master/exercise/svd+}
   \\ {\small \verb+https://github.com/todo-group/computer-experiments/tree/master/exercise/linear_regression+}
  
\item 準備練習
  \begin{enumerate}
  \item 特異値分解のサンプルプログラム({\tt svd.c})をコンパイル・実行せよ。入力{\tt matrix1.dat}を用いて、講義L4の例が再現されるかどうか確認せよ
    \begin{quote} \tt
      \$ \underline{cc svd.c -o svd -llapack -lm} \\
      \$ \underline{./svd matrix1.dat}
    \end{quote}
  \item {\tt regression.c}は数値データを読み込み、一次式で最小二乗フィッテイングを行うプログラムである。実行すると最終行に一次式の係数が出力される
    \begin{quote} \tt
      \$ \underline{./regression measurement1.dat}
    \end{quote}
    ファイル{\tt measurement1.dat}の1カラム目は$x$、2カラム目は$y$、3カラム目は$y$の誤差の値(計算の中では使っていない)である。元データとフィッティング結果をグラフにせよ
  \end{enumerate}

\item 基本課題
  \begin{enumerate}
  \item {\tt svd.c}の最後では行列のランク$[{\rm min}(m,n)-1]$近似を計算している。コマンドライン引数で近似のランク数を指定できるように、また近似の誤差(フロベニウスノルム)を計算・出力するようプログラムを修正せよ。{\tt matrix2.dat}、{\tt matrix3.dat}について、近似度合いを変えながら、その出力を観察せよ。
  \item {\tt regression.c}中で、15行目の{\tt nbase}は基底関数の数を表し、18行目からの関数{\tt f}は{\tt i}番目の基底関数の$x$における値を返す関数である。二次式でフィッティングが行えるよう、15行目から26行目を修正し、
実行結果をグラフにせよ。一方、{\tt measurement2.dat}では、なめらかなバックグラウンドの上に中心$\mu=3.3$、分散$\sigma^2=1.3$のGaussianが乗っていることが分かっている。{\tt regression.c}を修正し、Gaussianの係数の大きさを見積もれ
  \item {\tt regression.c}ではLU分解を用いて連立一次方程式を解いている。そのため、基底関数の中に互いに線形従属なもの(例: 1, $x$, $1+x$)がある場合には、行列がランク落ちするため実行時にエラーとなる。特異値分解による一般化逆行列(講義L4 p.35)を用いる方法に修正し、どのような解が得られるか確認せよ
  \end{enumerate}

\item 応用課題
  \begin{enumerate}
  \item {\tt svd.c}中のLAPACKの特異値分解{\tt dgesvd}の呼び出し(54行目)では、行列の次元({\tt m}と{\tt n})、左特異ベクトル({\tt u})と右特異ベクトル({\tt vt})の順番が、もともとの{\tt dgesvd}のドキュメント\footnote{\url{http://www.netlib.org/lapack/explore-html/d8/d2d/dgesvd_8f.html}}とは逆になっている。このプログラムが正しく動作するのはなぜか? (ヒント: 講義L3 p.29「Cで作成した行列をFortranに渡すと転置されてしまう」)
  \item {\tt regression.c}では、測定量の誤差の値は使っていない。残差を誤差で重み付けするようにプログラムを改良せよ。また、フィッティング結果(係数)の誤差はどのようにすれば見積もることができるか?
  \item 画像ファイルを行列形式に変換し、SVDで圧縮してみよ。どの程度まで圧縮可能か?
  \item 物理学実験で得られたデータをフィッティングしてみよう。フィッティング関数が係数に関して非線形である場合は、どのように解を求めればよいか?
  \end{enumerate}  
\end{itemize}

\end{document}
