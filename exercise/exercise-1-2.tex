\documentclass[11pt]{jarticle}

\usepackage{amsmath}
\usepackage{graphics}
\usepackage{hyperref}

\setlength{\oddsidemargin}{-0.7cm}
\setlength{\topmargin}{-1.5cm}
\setlength{\textwidth}{16.5cm}
\setlength{\textheight}{26cm}
\pagestyle{empty}

\begin{document}

\noindent
{\bf\large 「計算機実験I」実習課題(EX2)}
\\[-0.5em]

\noindent
\begin{itemize}

\item サンプルプログラム: example-1-2.zip

\item 準備練習
  \begin{enumerate}
  \item 計算結果をグラフにする際には、以下の点に特に注意する必要がある
    \begin{itemize}
    \item グラフの横軸や縦軸が整数値を取る変数の場合、小数(0.5, 1.5など)の目盛や数字は付けない
    \item グラフの縦軸や横軸の値が非常に小さい(大きい)時には、10の冪表示とする。(例: 0.00000001ではなく$10^{-8}$)
    \item グラフの縦軸と横軸には変数名を(必要であれば単位も)付ける
    \item プロットしているデータが一種類の時は、レジェンド(凡例)は不要
    \item レジェンドは意味のあるものに。[例: ファイル名``prog-1.dat''ではなく``Runge-Kutta (h=0.01)'']
    \item 収束の様子(冪)を見る(見せる)には、収束先の値を引いた上で両logプロットする。指数関数的な収束の場合には片logプロットを使う
    \end{itemize}
    gnuplot (あるいは他を使っている場合はそのソフト)で、これらをどのように設定するか調べよ
  \item EX1の基本課題1、2で作成したグラフを改善せよ
  \item サンプルプログラムpointer.cをコンパイル・実行せよ。なぜそのような出力結果が得られるか考えよ(講義L2スライドpp.21,26)
  \end{enumerate}

\item 基本課題
  \begin{enumerate}
  \item 空気による摩擦のあるバネの問題を考える。壁にバネが繋がれ、バネの先には質量$m$の物体が繋がっている。床との摩擦は考えないものとする。バネの伸びる方向に$x$座標を取り、自然長の位置を原点とすると、物体の運動方程式は以下のように与えられる
    \[
    m\frac{\mathrm{d} ^2x}{\mathrm{d} t^2} = -kx - \kappa \frac{\mathrm{d} x}{\mathrm{d} t} 
    \]
    ここで、$k$はバネ定数、$\kappa$は摩擦の比例定数とする。Euler法(L2スライドp.4)を使い$x(t)$を$t=30$まで計算せよ。その際、刻み幅$h$の大きさを変化させ、解の変わる様子を確認せよ。ただし、$k=2$、$\kappa=0.2$、$m = 1$、初期条件は$x(0)=10$、$x'(0)=0$とする
  \item 中点法(L2スライドp.7)、4次のRunge-Kutta法(L2スライドp.8)を用いて同様の計算を行い、精度の向上の様子を調べよ
  \item 摩擦が無い場合($\kappa=0$)についてEuler法、4次のRunge-Kutta法を用いてシミュレーションを行い、全エネルギー(運動エネルギーとポテンシャルエネルギーの和)の時間変化を観察せよ。さらに、シンプレクティック法(L2スライドp.17)を用いたプログラムを作成し、全エネルギーがどのように振る舞うか調べよ
  \end{enumerate}
  
\item 応用課題
  \begin{enumerate}
  \item 方程式によっては、刻み幅を小さくしても、なかなか精度が上がらないものがある.一つの例として、``硬い方程式''知られている。``硬い方程式''とは何か、これを精度良く解くためにはどうすれば良いか調べよ。また、具体的な問題について計算を行ってみよ
  \item Numerov法とシューティング法を用いて、一次元井戸型ポテンシャル中の粒子のシュレディンガー方程式の固有エネルギーと固有関数の組をいくつか求めよ
  \end{enumerate}

\item レポート課題No.1

  レポート課題の内容、提出方法、締切については、ITC-LMSに掲示するので確認すること

\end{itemize}

\end{document}
