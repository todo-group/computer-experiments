\documentclass[11pt]{jarticle}

\usepackage{amsmath}
\usepackage{graphics}
\usepackage{hyperref}
\usepackage{color}

\setlength{\oddsidemargin}{-1.2cm}
\setlength{\topmargin}{-2.3cm}
\setlength{\textwidth}{17.5cm}
\setlength{\textheight}{27cm}
\pagestyle{empty}

\begin{document}

\noindent
{\bf\large 「計算機実験I」実習課題(EX0)}
\\[-0.5em]

\noindent
\begin{itemize}
\item 講義のページ: \verb+http://exa.phys.s.u-tokyo.ac.jp/ja/lectures/2019s-computer1+

\item 準備
  \begin{enumerate}
  \item Mac, Windows上で動作する仮想UNIX環境 MateriApps LIVE! のUSBメモリを配布するので、USBメモリ貸出簿の「USB-ID」の番号のついたUSBメモリを受け取り、貸出簿に名前を記入すること(初回の出席の代わりとする)。
    インストール方法はUSBメモリ内の\href{https://github.com/cmsi/MateriAppsLive/wiki/MateriAppsLive-ltx}{README.html}、\href{https://github.com/cmsi/MateriAppsLive-setup/blob/master/ova/setup.pdf}{setup.pdf}を参照のこと。
    USBメモリは5月中旬を目処に返却すること

  \item ECCクラウドメールが読み書きできることを確認する(SSH公開鍵登録フォームとMATLABのインストールに必要)。ログインID/パスワードが不明な場合には、講義のページを参照のこと
  \item ITC-LMS (\verb+https://itc-lms.ecc.u-tokyo.ac.jp+)にログインし、「計算機実験I」に自分が登録されていることを確認する(UTASで履修登録をしてもITC-LMSに反映されるまで1〜2日はかかる模様)。
    「setting」で転送先メールアドレスか通知用SNS(LINE)アカウントを登録し、お知らせが受け取れるようにしておくことが望ましい
    
  \item ECCS端末(iMac)へログイン。ターミナルを開き、SSHの鍵ペアを作成する
    \begin{quote} \tt
      \$ \underline{ssh-keygen -t rsa} \\
      Generating public/private rsa key pair.\\
      Enter file in which to save the key (/xxx/.ssh/id\_rsa): \underline{(何も入力せずreturn)}\\
      Enter passphrase (empty for no passphrase): \underline{(パスフレーズを入力)}\\
      Enter same passphrase again: \underline{(パスフレーズを再度入力)}
    \end{quote}
    秘密鍵が {\tt \$HOME/.ssh/id\_rsa} に、公開鍵が {\tt \$HOME/.ssh/id\_rsa.pub} に作成される
    
    {\bf ここで入力する「パスフレーズ」は、ECCSの「パスワード」とは別のものである。「パスフレーズ」は自分で決める文字列であり、「パスワード」とは異なるものでなければならない。また、{\color{red} 設定した「パスフレーズ」は忘れずに覚えておくこと}}

  \item 作成した公開鍵を「SSH公開鍵登録フォーム」(\verb+https://forms.gle/2PURA8TJrAaQA7k57+)から登録する。フォームにはECCSクラウドメールのアカウント・パスワードを使ってログインする。締切は「{\bf 4/12(金)15:00}」とする

    TIPS: macOSでファイル({\tt \$HOME/.ssh/id\_rsa.pub})の中身をクリップボードにコピーするには{\tt pbcopy}コマンドを使えばよい。例:
    \begin{quote} \tt
      \$ \underline{pbcopy $<$ \$HOME/.ssh/id\_rsa.pub}
    \end{quote}
  \end{enumerate}
\item 基本課題
  \begin{enumerate}
  \item エディタ(Emacs他)を使ってハンドブック例3.1.1のファイルを作成する。Cコンパイラでコンパイル・実行(ハンドブック3.1.1節)
  \item ハンドブック3.1.1〜3.1.3節, 3.2.1〜3.2.2節の例題を試せ
  \item フィボナッチ数列($a_{n+2}=a_{n+1}+a_n$ ($n \ge 0$), $a_0=0$, $a_1=1$)を計算するプログラムを作成し、$a_{20}$, $a_{30}$, $a_{40}$, $a_{50}$, $a_{60}$を求めよ。桁あふれに注意すること。結果は、\LaTeX の{\tt tabular}環境を使って表にまとめよ
  \end{enumerate}
\item 追加課題 (自宅で)
  \begin{enumerate}
  \item C言語・\LaTeX・gnuplotのインストールに関する情報

    (\verb+https://github.com/todo-group/ComputerExperiments/wiki/InstallTeX+)を参考に、自分のPCに環境を整備せよ。基本課題1〜3を行ってみよ
  \item MATLAB Quick-Start Part I (\verb+https://elf-c.he.u-tokyo.ac.jp/courses/375+)にしたがって、PC、タブレット、スマートフォン等にMATLABをインストールせよ。MATLABを使ってフィボナッチ数を計算し、基本課題3の結果を検証せよ
  \end{enumerate}
\end{itemize}

\end{document}
