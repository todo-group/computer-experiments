\documentclass[11pt]{jarticle}

\usepackage{amsmath}
\usepackage{graphics}
\usepackage{hyperref}

\setlength{\oddsidemargin}{-0.7cm}
\setlength{\topmargin}{-1.5cm}
\setlength{\textwidth}{16.5cm}
\setlength{\textheight}{26cm}
\pagestyle{empty}

\begin{document}

\noindent
{\bf\large 「計算機実験」実習課題(EX0)}
\\[-0.5em]

\noindent
\begin{itemize}
\item 講義のページ: \verb+http://exa.phys.s.u-tokyo.ac.jp/ja/lectures/2018s-computer1+
\item ITC-LMS: \verb+https://itc-lms.ecc.u-tokyo.ac.jp+ にログインし、「計算機実験I」に自分が登録されていることを確認すること。

  「個人設定」でメールアドレスの登録と更新通知の設定(「1日1回まとめて転送」)を行うこと。
\item Mac, Windows上で動作する仮想UNIX環境 MateriApps LIVE! のUSBメモリを配布する。
  \begin{itemize}
  \item USBメモリ貸出簿に名前を記入すること。(初回の出席の代わりとする)
  \item インストール方法はUSBメモリ内の\href{https://github.com/cmsi/MateriAppsLive/wiki/MateriAppsLive-ltx}{README.html}、\href{https://github.com/cmsi/MateriAppsLive-setup/blob/master/ova/setup.pdf}{setup.pdf}を参照のこと
  \item USBメモリは5月中旬を目処に返却すること
  \end{itemize}

\item 準備練習
  \begin{enumerate}
  \item ECCS端末(iMac)へログイン。ターミナルを開き、SSHの鍵ペアを作成する
    \begin{quote} \tt
      \$ \underline{ssh-keygen -t rsa} \\
      Generating public/private rsa key pair.\\
      Enter file in which to save the key (/xxx/.ssh/id\_rsa): \underline{(何も入力せずreturn)}\\
      Enter passphrase (empty for no passphrase): \underline{(パスフレーズを入力)}\\
      Enter same passphrase again: \underline{(パスフレーズを再度入力)}
    \end{quote}
    秘密鍵が {\tt \$HOME/.ssh/id\_rsa} に、公開鍵が {\tt \$HOME/.ssh/id\_rsa.pub} に作成される。
    
    {\bf ここで入力する「パスフレーズ」は、ECCSの「パスワード」とは別のものである。「パスフレーズ」は自分で決める文字列であり、「パスワード」とは異なるものでなければならない。また、設定した「パスフレーズ」は忘れずに覚えておくこと}
  \item 1.で作成した「公開鍵」を computer@exa.phys.s.u-tokyo.ac.jp あてに電子メールで送付せよ。({\bf 間違って「秘密鍵」を送らないこと!}) メールはECCSのアカウントから送ること。その際、タイトル(サブジェクト)は「計算機実験 SSH公開鍵」、また本文中に学籍番号と氏名を明記すること。締切は「{\bf 4/20(金)15:00}」とする。

    ヒント: macOSでファイル({\tt \$HOME/.ssh/id\_rsa.pub})の中身をクリップボードにコピーするには{\tt pbcopy}コマンドを使えば良い:
    \begin{quote} \tt
      \$ \underline{pbcopy $<$ \$HOME/.ssh/id\_rsa.pub}
    \end{quote}
  \end{enumerate}
\item 基本課題
  \begin{enumerate}
  \item エディタ(emacs他)を使って、ハンドブック例3.1.1のファイルを作成する。Cコンパイラでコンパイルし、実行(ハンドブック3.1.1節)
  \item ハンドブック3.1.1〜3.1.3節, 3.2.1〜3.2.2節の例題を試せ
  \item フィボナッチ数列($a_{n+2}=a_{n+1}+a_n$ ($n \ge 0$), $a_0=0$, $a_1=1$)を計算するプログラムを作成し、$a_{20}$, $a_{30}$, $a_{40}$, $a_{50}$, $a_{60}$を求めよ。桁あふれに注意すること。結果は、\LaTeX の{\tt tabular}環境を使って表にまとめよ
  \end{enumerate}
\item 追加課題 (自宅で)
  \begin{enumerate}
  \item 配布した MateriApps LIVE! USB メモリの中の {\tt setup.pdf} にしたがい、自分の PC に VirtualBox と MateriApps LIVE! をインストールせよ。MateriApps LIVE! の中で基本課題1〜3を行え
  \end{enumerate}
\end{itemize}

\end{document}
