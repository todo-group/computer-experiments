% -*- coding: utf-8 -*-

\documentclass[10pt,dvipdfmx]{beamer}
\usepackage{tutorial}

\title{計算機実験I (第6回)}
\date{2020/06/03}

\begin{document}

\begin{frame}
  \titlepage
  \tableofcontents
\end{frame}

\begin{frame}[t]{講義予定}
  \begin{itemize}
    %\setlength{\itemsep}{1em}
  {\color{gray} \item 2020-04-22: 第1回 環境整備}
  {\color{gray} \item 2020-05-07: 第2回 計算機実験の基礎}
  {\color{gray} \item 2020-05-13: 第3回 常微分方程式の解法 \\
    \hspace*{5em} (レポートNo.1出題 5/29締切)}
  {\color{gray} \item 2020-05-20: 第4回 行列演算とライブラリ}
  {\color{gray} \item 2020-05-27: 第5回 連立一次方程式の解法 \\
  \hspace*{5em} (レポートNo.2出題 6/12締切)}
\item 2020-06-03: 第6回 行列の対角化
\item 2020-06-10: 第7回 疎行列に対する反復解法
\item 2020-06-17: 第8回 特異値分解と最小二乗法 \\
  \hspace*{5em} (レポートNo.3出題 7/3締切)
  \end{itemize}
\end{frame}

\begin{frame}[t]{本日の課題}
  \begin{itemize}
    %\setlength{\itemsep}{1em}
  \item 「\href{https://utphys-comp.github.io}{計算機実験のための環境整備}」({\small \href{https://utphys-comp.github.io}{https://utphys-comp.github.io}})が完了していない人は至急連絡を!
  \item 実習
    \begin{itemize}
    \item 講義資料の中の{\tt diag.c}をコンパイル・実行。ソースコードの中身を確認 \\
      サンプルコード一式: \href{https://github.com/todo-group/ComputerExperiments/releases/tag/2020s-computer1}{example-1-6.zip} (前回までのものも含む)
    \item 実習課題一覧\href{https://github.com/todo-group/ComputerExperiments/releases/tag/2020s-computer1}{exercise-1.pdf}の課題17〜20 (あるいはそれ以外)から適宜選び実習
    \end{itemize}
  \item 質問はSlackの「\# 5\_対角化」あるいは他の適当と思われるチャンネルで \\[2em]
  \end{itemize}
\end{frame}

\section{行列の対角化}
% -*- coding: utf-8 -*-

\documentclass[10pt,dvipdfmx]{beamer}
\usepackage{tutorial}

\begin{document}
\section{行列とLAPACK}
\begin{frame}[t,fragile]{二次元配列}
  \begin{itemize}
    \setlength{\itemsep}{1em}
  \item C言語では、二次元配列は一次元配列の先頭をさす(ポインタ)の配列として表される(と理解しておけば良い)
  \item \verb+a[i]+は、要素\verb+a[i][0]+を指すポインタ
    \begin{itemize}
    \item \verb+a+ と \verb+&a[0]+ は等価 (\verb+&a[0][0]+ ではない)
    \item \verb+a[0]+ と \verb+&a[0][0]+ は等価
    \item \verb+a[2]+ と \verb+&a[2][0]+ は等価
    \item \verb^(a+2)^ と \verb^&a[2]^ は等価
    \item \verb^(*(a+2))[3]^ と \verb^*(*(a+2)+3)^ と \verb^a[2][3]^ は等価
    \item \verb^*(a+2)[3]^ と \verb^*((a+2)[3])^ と \verb^*(a[5])^ と\verb^a[5][0]^ は等価
    \item \verb^[]^は\verb^*^よりも強い
    \end{itemize}
  \item ポインタ確認プログラム: \href{https://github.com/todo-group/computer-experiments/blob/master/exercise/matrix/pointer-matrix.c}{pointer-matrix.c}
  \end{itemize}
\end{frame}

\begin{frame}[t,fragile]{動的二次元配列の確保}
  \begin{itemize}
    \setlength{\itemsep}{1em}
  \item 各行を表す配列とそれぞれの先頭アドレスを保持する配列の二種類が必要
\begin{lstlisting}
double **a;
m = 10;  
n = 10;  
a = (double**)malloc((size_t)(m * sizeof(double*));
for (int i = 0; i < m; ++i)
  a[i] = (double*)malloc((size_t)(n * sizeof(double));
\end{lstlisting}
\item 各行を保持する配列が、メモリ上で連続に確保される保証はない
\item 行列用のライブラリ(LAPACK等)を使うときに問題となる
  \end{itemize}
\end{frame}

\begin{frame}[t,fragile]{BLASライブラリ}
  \begin{itemize}
    \setlength{\itemsep}{1em}
  \item 行列・行列積、行列・ベクトル積などを高速に行う最適化された関数群
  \item 行列・行列積を計算するサブルーチン {\tt dgemm} \\
    \url{http://www.netlib.org/lapack/explore-html/d7/d2b/dgemm_8f.html}
    \begin{itemize}
    \item $C = \alpha A \times B + \beta C$ を計算
    \item BLASもFortranで書かれている
    \end{itemize}
  \item 例: \href{https://github.com/todo-group/computer-experiments/blob/master/exercise/matrix/multiply.c}{multiply.c}, \href{https://github.com/todo-group/computer-experiments/blob/master/exercise/matrix/multiply_dgemm.c}{multiply\_dgemm.c}
  \end{itemize}
\end{frame}

\begin{frame}[t,fragile]{LAPACK (Linear Algebra PACKage)}
  \begin{itemize}
    %\setlength{\itemsep}{1em}
  \item 線形計算のための高品質な数値計算ライブラリ
    \begin{itemize}
    \item \url{http://www.netlib.org/lapack}
    \item 線形方程式、固有値問題、特異値問題、線形最小二乗問題など
    \item (FFT 高速フーリエ変換は入っていない)
    \item LAPACK自体はFortranで書かれている
    \end{itemize}
  \item ほぼ全てのPC、ワークステーション、スーパーコンピュータで利用可 (インストール済)
  \item Netlibでソースが公開されているリファレンス実装は遅いが、それぞれのベンダー(Intel、Fujitsu、etc)による最適化されたLAPACKが用意されている場合が多い(MKL、SSL2、etc)
  \item LAPACKを使うことにより、高速で信頼性が高く、ポータブルなコードを書くことが可能になる
  \end{itemize}
\end{frame}

\begin{frame}[t,fragile]{LAPACKによる連立一次方程式の求解}
  \begin{itemize}
    \setlength{\itemsep}{1em}
  \item LU分解を行うサブルーチン {\tt dgetrf} \\
    \url{http://www.netlib.org/lapack/explore-html/d3/d6a/dgetrf_8f.html}
  \item Fortranによる関数宣言
\begin{lstlisting}
subroutine dgetrf(integer M, integer N,
         double precision, dimension(lda, *) A,
         integer LDA, integer, dimension(*) IPIV,
         integer INFO)
\end{lstlisting}
\item {\tt A}: 左辺の行列、{\tt M,N}: 次元、{\tt IPIV}: 選択されたピボット行のリスト、{\tt lda}: 通常{\tt M} (行数)と同じで良い
  \end{itemize}
\end{frame}

\begin{frame}[t,fragile]{CからBLAS/LAPACKを呼び出す際の注意事項}
  \begin{itemize}
    %\setlength{\itemsep}{1em}
  \item (もともとFortran言語で書かれていたことによる制限)
  \item 関数名はすべて小文字、最後に \verb+_+ (下線)を付ける
  \item スカラー、ベクトル、行列は全て「ポインタ渡し」とする
  \item ベクトルや行列は最初の要素へのポインタを渡す (サイズは別に渡す)
  \item 行列の要素は(0,0) $\rightarrow$ (1,0) $\rightarrow$ (2,0) $\rightarrow\cdots\rightarrow$ $(m-1,0)$ $\rightarrow$ (0,1) $\rightarrow$ (1,1) $\rightarrow\cdots\rightarrow$ $(m-1,n-1)$の順で連続して並んでいなければならない(column-major)
    \begin{itemize}
    \item C言語の二次元配列では \verb+a[i][j]+ の次には \verb%a[i][j+1]%が入っている(row-major)
    \item 行列が転置されて解釈されてしまう!
    \end{itemize}
  \item コンパイル時には{\tt -llapack -lblas}オプションを指定し、LAPACKライブラリとBLASライブラリをリンクする(ハンドブック2.1.6節)
  \end{itemize}
\end{frame}

\begin{frame}[t,fragile]{cmatrix.hライブラリ}
  \begin{itemize}
    %\setlength{\itemsep}{1em}
  \item Column-major形式の二次元配列の確保({\tt alloc\_dmatrix})、開放({\tt free\_dmatrix})、出力({\tt print\_dmatrix})、読み込み({\tt read\_dmatrix})を行うためのユーティリティ関数、(i,j)成分にアクセスするためのマクロ({\tt mat\_elem})他を準備
  \item ソースコード: \href{https://github.com/todo-group/computer-experiments/blob/master/exercise/include/cmatrix.h}{cmatrix.h}
  \item 使用例
\begin{lstlisting}
#include "cmatrix.h"
...
double **mat;
mat = alloc_dmatrix(m, n);
mat_elem(mat, 1, 3) = 5.0;
...
free_dmatrix(mat);
\end{lstlisting}
  \item サンプルコード: \href{https://github.com/todo-group/computer-experiments/blob/master/exercise/matrix/matrix_example.c}{matrix\_example.c}
  \end{itemize}
\end{frame}

\begin{frame}[t,fragile]{alloc\_dmatrixでの動的二次元配列の確保}
  \begin{itemize}
    %\setlength{\itemsep}{1em}
  \item 長さ$m \times n$の一次元配列を用意し、各列(それぞれ$m$要素)の先頭アドレスを長さ$n$のポインター配列に格納する
\begin{lstlisting}
double **a;
m = 10;  
n = 10;  
a = (double**)malloc((size_t)(n * sizeof(double*));
a[0] = (double*)malloc((size_t)(m*n * sizeof(double));
for (int i = 1; i < n; ++i)
  a[i] = a[i-1] + m;
\end{lstlisting}
\item 行列の(i,j)成分を\verb+a[j][i]+に格納することにする (column-major)
  \end{itemize}
\end{frame}

\begin{frame}[t,fragile]{要素アクセス・先頭アドレス}
  \begin{itemize}
    % \setlength{\itemsep}{1em}
  \item 行列の(i,j)成分を\verb+a[j][i]+に格納することにする(column-major)
    \begin{itemize}
      \item \href{https://github.com/todo-group/computer-experiments/blob/master/exercise/matrix/cmatrix.h}{cmatrix.h}ではマクロ(\verb+mat_elem+)を準備
\begin{lstlisting}
#define mat_elem(mat, i, j) (mat)[j][i]
\end{lstlisting}
\item このマクロを使うと、例えば(i,j)成分への代入は以下のように書ける
\begin{lstlisting}
mat_elem(a, i, j) = 1;
\end{lstlisting}
\end{itemize}
  \item LAPACKにベクトルや行列の最初の要素へのポインタを渡す
    \begin{itemize}
      \item ベクトルの最初の要素(0)へのポインタ: \verb+&v[0]+
      \item 行列の最初の要素(0,0)へのポインタ: \verb+&a[0][0]+
      \item \href{https://github.com/todo-group/computer-experiments/blob/master/exercise/matrix/cmatrix.h}{cmatrix.h}にマクロ({\tt vec\_ptr}、{\tt mat\_ptr})が準備されているのでそれぞれ、{\tt vec\_ptr(v)}、{\tt mat\_ptr(a)}と書ける
    \end{itemize}
  \end{itemize}
\end{frame}

\begin{frame}[t,fragile]{LAPACKによる連立一次方程式の求解}
  \begin{itemize}
    \setlength{\itemsep}{1em}
  \item C言語から呼び出すための関数宣言を作成 (ハンドブック3.6.4節)
\begin{lstlisting}
void dgetrf_(int *M, int *N, double *A,
             int *LDA, int*IPIV, int *INFO);
\end{lstlisting}
関数名は全て小文字。関数名の最後に {\tt \_} (下線)を付ける
\item LU分解の例
\begin{lstlisting}
m = 10;
n = 10;
a = alloc_dmatrix(m, n);
...
dgetrf_(&m, &n, mat_ptr(a), &m, vec_ptr(ipiv), &info);
\end{lstlisting}
完全なソースコード: \href{https://github.com/todo-group/computer-experiments/blob/master/exercise/linear_system/lu_decomp.c}{lu\_decomp.c}
  \end{itemize}
\end{frame}

\end{document}

\begin{frame}[t,fragile]{連立一次方程式のあらわれる例}
  \begin{itemize}
    %\setlength{\itemsep}{1em}
  \item 偏微分方程式の境界値問題の差分法による求解
  \item 非線形連立方程式に対するニュートン法
    \[ {\bf x}' = {\bf x} - \Big( \frac{\partial {\bf f}({\bf x})}{\partial {\bf x}} \Big)^{-1} {\bf f}({\bf x}) \]
    \begin{itemize}
    \item 微分方程式の初期値問題の陰解法など
    \item 逆行列を求めベクトルに掛ける代わりに連立一次方程式を解く
      \[ {\bf x} = A^{-1} {\bf b} \ \ \Rightarrow \ \ A {\bf x} = {\bf b} \]
    \end{itemize}
  \end{itemize}
\end{frame}

%\begin{frame}[t,fragile]{ポアソン方程式の境界値問題}
  \begin{itemize}
    %\setlength{\itemsep}{1em}
  \item 二次元ポアソン方程式
    \[ \frac{\partial^2 u(x,y)}{\partial x^2} + \frac{\partial^2 u(x,y)}{\partial y^2} = f(x,y) \qquad 0 \le x \le 1, \ 0 \le y \le 1\]
  \item ディリクレ型境界条件: $u(x,y) = g(x,y)$ on $\partial \Omega$
  \item 有限差分法により離散化
    \begin{itemize}
    \item $x$方向、$y$方向をそれぞれ$n$等分: $(x_i,y_j) = (i/n, j/n)$
    \item $(n+1)^2$個の格子点の上で$u(x_i,y_j)=u_{ij}$が定義される
    \item そのうち$4n$個の値は境界条件で定まる
    \item ポアソン方程式を中心差分で近似 ($h=1/n$)
      \[
      \frac{u_{i+1,j}-2u_{ij}+u_{i-1,j}}{h^2} + \frac{u_{i,j+1}-2u_{ij}+u_{i,j-1}}{h^2} = f_{ij}
      \]
      残り$(n-1)^2$個の未知数に対する連立一次方程式
    \end{itemize}
  \end{itemize}
\end{frame}

\begin{frame}[t,fragile]{シュレディンガー方程式の行列表示}
  \begin{itemize}
    %\setlength{\itemsep}{1em}
  \item シュレディンガー方程式
    \[
    [-\frac{d^2}{dx^2}+V(x)]\psi(x) = E \psi(x)
    \]
  \item 連立差分方程式を行列の形で表す($\psi(x_0)=\psi(x_n)=0$)
    \[\begin{small}\hspace*{-4em}
    \begin{pmatrix}
      \frac{2}{h^2}+V(x_1) & -\frac{1}{h^2} \\
      -\frac{1}{h^2} & \frac{2}{h^2}+V(x_2) & -\frac{1}{h^2} \\
      & -\frac{1}{h^2} & \frac{2}{h^2}+V(x_3) & -\frac{1}{h^2} \\
      & & \ddots & \ddots \\
      & & & -\frac{1}{h^2} & \frac{2}{h^2}+V(x_{n-1}) \\
    \end{pmatrix}
    \begin{pmatrix}
      \psi(x_1) \\
      \psi(x_2) \\
      \psi(x_3) \\
      \vdots \\
      \psi(x_{n-1}) \\
    \end{pmatrix}
    = \cdots % E
    %% \begin{pmatrix}
    %%   \psi(x_1) \\
    %%   \psi(x_2) \\
    %%   \psi(x_3) \\
    %%   \vdots \\
    %%   \psi(x_{n-1}) \\
    %% \end{pmatrix}
    \end{small}
    \]
  \item $(n-1) \times (n-1)$の疎行列の固有値問題
    \begin{itemize}
    \item 固有値: 固有エネルギー
    \item 固有ベクトル: 波動関数
    \end{itemize}
  \end{itemize}
\end{frame}

\begin{frame}[t,fragile]{固体物理・量子統計物理に現れる行列}
  \begin{itemize}
    %\setlength{\itemsep}{1em}
  \item 強束縛近似(tight-binding approx.)のもとでの第二量子化表示
    \[
    H = -t \sum_{\langle i,j \rangle \sigma} (c_{i,\sigma}^\dagger c_{j,\sigma} + h.c.) + \text{(相互作用)}
    \]
  \item 局所スピン模型(ハイゼンベルグ模型)
    \[
    H = -J\sum_{\langle i,j \rangle} S_i \cdot S_j
    = -J\sum_{\langle i,j \rangle} [S_i^z S_j^z +\frac{1}{2} (S_i^+ S_j^- + S_i^- S_j^+) ]
    \]
  \item 格子点の数を$n$とすると、ハミルトニアンはそれぞれ$4^n \times 4^n$、$2^n \times 2^n$の(疎)行列で表される。
  \item $n$が大きくなると、行列の次元は指数関数的に増加
  \item 量子多体系に共通する困難
  \end{itemize}
\end{frame}

\begin{frame}[t,fragile]{実対称行列(エルミート行列)の性質}
  \begin{itemize}
    %\setlength{\itemsep}{1em}
  \item $n \times n$実対称行列$A$ ($=A^T$)の固有値問題
    \[
    A x = \lambda x
    \]
  \item $n$個の固有値($\lambda_1,\lambda_2,\cdots,\lambda_n$)は全て実。固有ベクトル($\xi_1,\xi_2,\cdots,\xi_n$)は互いに正規直交するようにとることができる。行列$U$を
    \[
    U = \begin{bmatrix} \xi_1 \, \xi_2 \, \cdots \, \xi_n \end{bmatrix}
    \]
    と定義すると、$U$は直交(ユニタリ)行列($U^T U = U^{-1} U = E$)
  \item $A$の固有分解(固有値分解)
    \[
    A = U \Lambda U^T \qquad \Lambda = \text{diag}(\lambda_1,\cdots,\lambda_n)
    \]
  \end{itemize}
\end{frame}

\begin{frame}[t,fragile]{行列のべき乗・指数関数}
  \begin{itemize}
    %\setlength{\itemsep}{1em}
  \item 行列のべき乗
    \begin{align*}
      A^p &= (U \Lambda U^T)(U \Lambda U^T) \cdots (U \Lambda U^T) \\
      &= U \Lambda^p U^T \qquad \Lambda^p = \text{diag}(\lambda_1^p,\cdots,\lambda_n^p)
    \end{align*}
  \item 行列の指数関数
    \begin{align*}
      e^{xA} &= \sum_{k=0}^{\infty} \frac{1}{k!}(xA)^k = U \Big[ \sum_{k=0}^{\infty} \frac{1}{k!}(x\Lambda)^k \Big] U^T \\
      &= U e^{x \Lambda} U^T \qquad e^{x \Lambda} = \text{diag}(e^{x\lambda_1},\cdots,e^{x\lambda_n})
    \end{align*}
  \item 逆行列 $A^{-1} = U \Lambda^{-1} U^T$
  \item 行列式 $|A| = \prod_i \lambda_i$、対角和(トレース) ${\rm tr} A = \sum_i \lambda_i$
  \end{itemize}
\end{frame}

\begin{frame}[t,fragile]{行列の数値対角化}
  \begin{itemize}
    %\setlength{\itemsep}{1em}
  \item 一般的に次元が5以上の行列の固有値は、あらかじめ定まる有限回の手続きでは求まらない
    \begin{itemize}
    \item 必ず何らかの反復法(+収束判定)が必要となる
    \end{itemize}
  \item 密行列向きの方法
    \begin{itemize}
    \item Jacobi法
    \item Givens変換・Householder法(三重対角化) + QR法など
    \end{itemize}
  \item 疎行列向きの方法
    \begin{itemize}
    \item べき乗法
    \item Lanczos法(三重対角化) + QR法など
    \end{itemize}
  \item 固有ベクトル
    \begin{itemize}
    \item QR法で求めたものを逆変換
    \item 逆反復法で精度改善
    \end{itemize}
  \end{itemize}
\end{frame}


\section{密行列の対角化}

\begin{frame}[t,fragile]{基本方針}
  \begin{itemize}
    %\setlength{\itemsep}{1em}
  \item やってはいけない方法: 特性方程式
    \[
    |\lambda E - A| = 0
    \]
    の係数を求めて、代数方程式として解く
    \begin{itemize}
    \item 数値的に不安定 (代数方程式の解は係数の誤差に対して敏感)
    \item 計算コスト大[$\sim O(N!)$]
    \end{itemize}
  \item スタンダードな方法: 行列を次々に直交変換して、対角行列(あるいは三重対角行列)に近づけていく
    \[
    A \rightarrow U_1^T A U_1 \rightarrow U_2^T (U_1^T A U_1) U_2 \rightarrow U_3^T (U_2^T (U_1^T A U_1) U_2) U_3 \rightarrow \cdots
    \]
  \item 固有値は変換された行列の固有値、固有ベクトルは変換後の行列の固有ベクトルに左から$U_1 U_2 U_3 \cdots$を掛けたもの
  \end{itemize}
\end{frame}

\begin{frame}[t,fragile]{Jacobi法}
  \begin{itemize}
    \setlength{\itemsep}{1em}
  \item 直交行列$U_{pq}$を以下のように選ぶ ($(p,p),(p,q),(q,p),(q,q)$成分を除くと単位行列)
    \[
    U_{pq} =
    \begin{bmatrix}
      1 \\
      & \ddots \\
      & & 1 \\
      & & & \cos \theta & & & \sin \theta \\
      & & & & 1 \\
      & & & & & 1 \\
      & & & -\sin \theta & & & \cos \theta \\
      & & & & & & & 1 \\
      & & & & & & & & \ddots \\
      & & & & & & & & & 1 \\
    \end{bmatrix}
    \]
  \end{itemize}
\end{frame}

\begin{frame}[t,fragile]{Jacobi法による相似変換}
  \begin{itemize}
    %\setlength{\itemsep}{1em}
  \item $B=U_{pq}^{-1} A U_{pq}$により、$A$の$p$行、$q$行、$p$列、$q$列のみが変更を受ける
    \begin{align*}
      b_{pk} &= b_{kp} = a_{pk} \cos \theta - a_{qk} \sin \theta \qquad k \ne p,q \\
      b_{qk} &= b_{kq} = a_{pk} \sin \theta + a_{qk} \cos \theta \qquad k \ne p,q \\
      b_{pp} &= \frac{a_{pp}+a_{qq}}{2} + \frac{a_{pp}-a_{qq}}{2} \cos 2 \theta - a_{pq} \sin 2 \theta \\
      b_{qq} &= \frac{a_{pp}+a_{qq}}{2} - \frac{a_{pp}-a_{qq}}{2} \cos 2 \theta + a_{pq} \sin 2 \theta \\
      b_{pq} &= b_{qp} = \frac{a_{pp}-a_{qq}}{2} \sin 2 \theta + a_{pq} \cos 2 \theta
    \end{align*}
  \item $b_{pq} = b_{qp} = 0$とするには、$\theta$を次のように選べば良い
    \[
    \tan 2 \theta = - \frac{2 a_{pq}}{a_{pp}-a_{qq}}
    \]
  \end{itemize}
\end{frame}

\begin{frame}[t,fragile]{Jacobi法の収束}
  \begin{itemize}
    %\setlength{\itemsep}{1em}
  \item 相似変換により対角和は不変に保たれるので
    \[
      {\rm tr} \, A^T A = {\rm tr} \, B^T B \ \ \Rightarrow \ \
      \sum_{i,j} a_{ij}^2 = \sum_{i,j} b_{ij}^2
    \]
  \item 一方、この変換で
    \[
    b_{pp}^2 + b_{qq}^2 = b_{pp}^2 + 2 b_{pq}^2 + b_{qq}^2 = a_{pp}^2 + 2 a_{pq}^2 + a_{qq}^2
    \]
    すなわち、変換により、対角成分の二乗和は増加する $\Rightarrow$ 非対角成分の二乗和は単調減少
  \item 全ての非対角成分が十分小さくなるまで繰り返す
  \item 固有値=対角成分、固有ベクトル$=U_1 U_2 U_3 \cdots$
  \end{itemize}
\end{frame}

\begin{frame}[t,fragile]{3重対角化}
  \begin{itemize}
    %\setlength{\itemsep}{1em}
  \item 対角化は有限回の手続きでは行えない
  \item 3重対角化であれば、$O(N^3)$の有限回の計算で決定論的に行える
  \item Givens変換: Jacobi変換と同じ相似変換を利用
    \begin{itemize}
    \item $U_{32}$で(3,1)と(1,3)を消去 $\Rightarrow$ $U_{42}$で(4,1)と(1,4)を消去 $\Rightarrow$ $U_{52}$で(5,1)と(1,5)を消去 $\Rightarrow$ $U_{62},\cdots,U_{N,2}$ $\Rightarrow$ $U_{43},U_{53},\cdots,U_{N,3}$ $\Rightarrow$ $\cdots$ $\Rightarrow$ $U_{n,n-1}$で($n,n-2$)と($n-2,n$)を消去
    \item $(4/3)N^3$回の乗算と$(2/3)N^3$回の加減算で3重対角化される
    \end{itemize}
  \item Householder変換: $U = I - 2 w w^T / |w|^2$
    \begin{itemize}
    \item $(2/3)N^3$回の乗算と加減算で3重対角化される
    \item Givens変換に比べ少し効率的なので、こちらが広く使われている
    \end{itemize}
  \end{itemize}
\end{frame}

\begin{frame}[t,fragile]{3重対角行列の対角化}
  \begin{itemize}
    %\setlength{\itemsep}{1em}
  \item 二分法、QR分解、分割統治法、MRRRなど様々な方法が知られている
  \item 固有ベクトル
    \begin{itemize}
    \item QR分解では3重対角行列の固有ベクトルも同時に求まる
    \item あるいは、固有値を求めた後、逆反復法を用いて固有ベクトルを求める
    \end{itemize}
  \item 逆反復法
    \begin{itemize}
    \item 近似固有値を$\mu$とするとき、行列$(A - \mu E)^{-1}$を考えると、固有ベクトルは$A$と同じ、固有値は$(\lambda-\mu)^{-1}$。
    \item $\mu$が十分に正確であれば、$(\lambda-\mu)^{-1}$は絶対値最大の固有値。行列$(A - \mu E)^{-1}$を適当な初期ベクトルにかけ続けると$\lambda$に対応する固有ベクトルに収束(c.f. べき乗法)
    \item 実際には$(A-\mu E) x' = x$という連立方程式を繰り返し解く
    \end{itemize}
  \end{itemize}
\end{frame}

\begin{frame}[t,fragile]{QR法}
  \begin{itemize}
    %\setlength{\itemsep}{1em}
  \item QR分解
    \begin{itemize}
    \item 行列$A$を直交(エルミート)行列$Q$と上三角行列$R$の積に分解: $A=QR$
    \item Gram-Schmidtの直交化と等価
    \end{itemize}
  \item QR法による固有値と固有ベクトルの計算
    \begin{itemize}
    \item 行列$A_1$をQR分解($A_1=Q_1R_1$) → $A_2 = R_1Q_1$ 
    \item 行列$A_2$をQR分解($A_2=Q_2R_2$) → $A_3 = R_2Q_2$ 
    \item 行列$A_k$をQR分解($A_k=Q_kR_k$) → $A_{k+1} = R_kQ_k$
    \item 繰り返していくと対角より下の全ての成分は零に収束し、対角成分は固有値に収束する(証明略)
    \end{itemize}
  \item 連続した直交変換: $A_{k+1} = R_kQ_k = Q_k^{-1}Q_kR_kQ_k = Q_k^{-1}A_kQ_k$
    \begin{itemize}
    \item $A_1$が対称(エルミート)三重対角行列の場合、$A_k$も対称(エルミート)三重対角
    \item 密行列に対して最初からQR法を適用するより、Householder法で三重対角化した後で使う方が効率がよい
    \end{itemize}
  \end{itemize}
\end{frame}

\begin{frame}[t,fragile]{LAPACKの対角化ルーチン}
  \begin{itemize}
    %\setlength{\itemsep}{1em}
  \item 様々な対角化ルーチンが準備されている
    \begin{itemize}
    \item 倍精度実対称行列の対角化 {\tt dsyev}
      \url{http://www.netlib.org/lapack/explore-html/dd/d4c/dsyev_8f.html}
    \item Fortranによる関数宣言
\begin{lstlisting}
subroutine dsyev(character JOBZ, character UPLO,
  integer N, double precision, dimension(lda, *) A,
  integer LDA, double precision, dimension(*) W,
  double precision, dimension(*) WORK,
  integer LWORK, integer INFO)		
\end{lstlisting}
    \end{itemize}
  \item 他にも{\tt dsyevd}、{\tt dsyevr}、{\tt dsyevx}などがある \\
    3重対角化までは同じ。3重対角行列の対角化が異なる
  \item 単精度版の{\tt ssyev}、複素(エルミート行列)版の{\tt zheev}など
  \item {\tt dsyev}の使用例: \href{https://github.com/todo-group/computer-experiments/blob/master/exercise/eigenvalue_problem/diag.c}{diag.c}
  \end{itemize}
\end{frame}

\begin{frame}[t,fragile]{複素エルミート行列の固有分解}
  \begin{itemize}
    %\setlength{\itemsep}{1em}
  \item 固有値は実数
  \item これまでの方法がそのまま使える (ただし、転置 $\rightarrow$ 複素転置)
  \item 実対称行列用のサブルーチンを使っての対角化も可能
    \begin{itemize}
    \item エルミート行列を実部と虚部に分ける: $A = R + iW$
    \item エルミート行列の固有値問題 $(R + iW)(u+iv) = \lambda(u+iv)$を$2n \times 2n$の実対称行列の問題に書き換える
      \[
      \begin{pmatrix} R & -W \\ W & R \end{pmatrix}
      \begin{pmatrix} u \\ v \end{pmatrix}
      = 
      \lambda \begin{pmatrix} u \\ v \end{pmatrix}
      \]
    \item 固有値は同じ固有値が2度づつ現れる
    \item 対応する複素行列の固有ベクトルは、$u+iv$と$-v+iu$
    \end{itemize}
  \end{itemize}
\end{frame}


\begin{frame}[t]{本日の課題}
  \begin{itemize}
    %\setlength{\itemsep}{1em}
  \item 「\href{https://utphys-comp.github.io}{計算機実験のための環境整備}」({\small \href{https://utphys-comp.github.io}{https://utphys-comp.github.io}})が完了していない人は至急連絡を!
  \item 実習
    \begin{itemize}
    \item 講義資料の中の{\tt diag.c}をコンパイル・実行。ソースコードの中身を確認 \\
      サンプルコード一式: \href{https://github.com/todo-group/ComputerExperiments/releases/tag/2020s-computer1}{example-1-6.zip} (前回までのものも含む)
    \item 実習課題一覧\href{https://github.com/todo-group/ComputerExperiments/releases/tag/2020s-computer1}{exercise-1.pdf}の課題17〜20 (あるいはそれ以外)から適宜選び実習
    \end{itemize}
  \item 質問はSlackの「\# 5\_対角化」あるいは他の適当と思われるチャンネルで \\[2em]
  \end{itemize}
\end{frame}

\end{document}
