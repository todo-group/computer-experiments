%-*- coding:utf-8 -*-

\documentclass[10pt,dvipdfmx]{beamer}
\usepackage{tutorial}

\title{計算機実験II (L6) --- 分子動力学}
\date{2020/11/13}

\begin{document}

\begin{frame}
  \titlepage
  \tableofcontents
\end{frame}

\begin{frame}[t]{講義日程 (予定)}
  \begin{itemize}
    % \setlength{\itemsep}{1em}
  \item 全8回 (金曜5限 16:50-18:35)
    \begin{itemize}
    \item 10月4日(金) 講義1: 多体系の統計力学とモンテカルロ法
    \item {\color{gray} 10月11日(金) 休講 (物理学教室コロキウム)}
    \item 10月18日(金) 実習1
    \item {\color{gray} 10月25日(金) 休講}
    \item 11月1日(金) 講義2: 偏微分方程式と多体系の量子力学
    \item 11月8日(金) 実習2
    \item {\color{gray} 11月15日(金) 休講}
    \item 11月29日(金) 講義3: 少数多体系・分子動力学
    \item 12月6日(金) 実習3
    \item {\color{gray} 12月13日(金) 休講 (物理学教室コロキウム)}
    \item {\color{gray} 12月20日(金) 休講 (ニュートン祭)}
    \item 12月27日(金) 講義4: 最適化問題
    \item 1月10日(金) 実習4
    \item {\color{gray} 1月24日(金) 休講 (物理学教室コロキウム)}
    \end{itemize}
  \end{itemize}
\end{frame}


\input{6_monte_carlo/rng-07.tex}

\section{多体系の統計力学(復習)}

%-*- coding:utf-8 -*-

\begin{frame}[t,fragile]{古典多粒子系}
  \begin{itemize}
    %\setlength{\itemsep}{1em}
  \item ハミルトニアン
    \[
    H = \sum \frac{p_i^2}{2m} + \sum_{ij} V(x_i, x_j)
    \]
  \item 分配関数
    \[
    Z(T) = \int \exp [- \beta H(p,x) ] \, dp \, dx
    \]
    \begin{itemize}
    \item 運動量$p$に関する積分は簡単に実行できる(ガウス積分)
    \item 位置$x$に関する積分: 数値積分、マルコフ連鎖モンテカルロ法、分子動力学法
    \end{itemize}
  \end{itemize}
\end{frame}

%-*- coding:utf-8 -*-

\begin{frame}[t,fragile]{様々な数値計算手法}
  \begin{itemize}
    % \setlength{\itemsep}{1em}
  \item 数え上げ (離散変数)
    \begin{itemize}
      \item 計算コスト × (指数関数的)
      \item メモリコスト ○ (${\cal O}(1)$)
    \end{itemize}
  \item 転送行列法 (離散変数)
    \begin{itemize}
      \item 計算コスト △ (指数関数的)
      \item メモリコスト △ (指数関数的)
    \end{itemize}
  \item 分子動力学法 (連続変数)
    \begin{itemize}
      \item 計算コスト ○ (${\cal O}(N)$)
      \item メモリコスト ○ (${\cal O}(N)$)
      \item 統計誤差あり
    \end{itemize}
  \item マルコフ連鎖モンテカルロ法 (離散変数・連続変数)
    \begin{itemize}
      \item 計算コスト ○ (${\cal O}(N)$)
      \item メモリコスト ○ (${\cal O}(N)$)
      \item 統計誤差あり
    \end{itemize}
  \end{itemize}
\end{frame}

%-*- coding:utf-8 -*-

\begin{frame}[t,fragile]{数値積分}
  \begin{itemize}
    %\setlength{\itemsep}{1em}
  \item 一次元の場合
    \begin{itemize}
    \item 台形公式: 積分区間$(a,b)$を$M$個の幅$\Delta=(b-a)/M$の区間に分けて線形関数で近似
      \begin{align*}
        \int_a^b f(x) \, dx &= \sum_{i=1}^M \int_{a+\Delta (i-1)}^{a+\Delta i} f(x) \, dx \\
        &\simeq \sum_{i=1}^M \Delta \frac{f({a+\Delta (i-1)}) + f({a+\Delta i})}{2}
      \end{align*}
    \item より高次の公式: シンプソンの公式(区間を二次式で近似)など
    \end{itemize}
    \item 高次元になると区間の数が指数関数的に増える (次元の呪い)
  \end{itemize}
\end{frame}

%-*- coding:utf-8 -*-

\begin{frame}[t,fragile]{マルコフ連鎖モンテカルロ法}
  \begin{itemize}
    \setlength{\itemsep}{1em}
  \item メトロポリス法
    \begin{itemize}
    \item 一つの粒子を選ぶ(位置$x$)
    \item 新しい位置の候補をある確率分布に従って選ぶ($x'$)
    \item ポテンシャルエネルギーの変化量($\Delta E$)を計算し、確率$P=\min(1,exp[-\beta\Delta E])$で新しい位置を採択
    \end{itemize}
  \item 新しい位置の選び方
    \begin{itemize}
    \item 大きく変えすぎると棄却率が増加
    \item もとの位置を中心とする局所的な分布
    \item $\sigma={\cal O}(1)$の標準偏差をもつ正規分布など: $N(x, \sigma^2)$
    \end{itemize}
  \end{itemize}
\end{frame}


\section{分子動力学法}

%-*- coding:utf-8 -*-

\begin{frame}[t,fragile]{分子動力学法}
  \begin{itemize}
    \setlength{\itemsep}{1em}
  \item 適当な初期条件から、運動方程式に従って位置と運動量を時間発展させる
    \begin{itemize}
    \item Euler法、Runge-Kutta法、リープ・フロッグ法、速度ベルレ法など
    \item $6N$次元の連立微分方程式
    \end{itemize}
  \item 時間発展に関する物理量の時間平均から平均を評価
    \begin{align*}
      \langle A(p,x) \rangle &= \frac{1}{Z(E)} \int A(p,x) \, \delta(H(p,x)-E) \, dp \, dx \\
      &\simeq \frac{1}{t_{\rm max}} \int_0^{t_{\rm max}} A(p(t),x(t)) \, dt
    \end{align*}
    \begin{itemize}
    \item ハミルトニアンが時間依存しない場合は全エネルギーが保存する→ミクロカノニカル分布
    \end{itemize}
  \item 平衡状態における平均値だけでなく、熱や電荷の輸送などの動的現象、非平衡状態からの緩和現象などもシミュレーションできる
  \end{itemize}
\end{frame}

%-*- coding:utf-8 -*-

\begin{frame}[t,fragile]{フローチャート}
  \begin{center}
    \resizebox{0.45\textwidth}{!}{\includegraphics{image/md-flowchart.pdf}}
  \end{center}
\end{frame}

%-*- coding:utf-8 -*-

\begin{frame}[t,fragile]{境界条件と力の計算}
  \begin{itemize}
    \setlength{\itemsep}{1em}
  \item 気体・液体・固体などの熱力学的極限を調べたい場合
    \begin{itemize}
    \item 端の効果を取り除くために周期境界条件を採用
    \item 周期的に同じパターンが続く
    \end{itemize}
  \item ポテンシャルの計算
    \begin{align*}
      U = \frac{1}{2} \sum_i \sum_{j \ne i} \sum_{n_x=-\infty}^{\infty} \sum_{n_y=-\infty}^{\infty} \sum_{n_z=-\infty}^{\infty} U[\mathbf{r}_i - \mathbf{r}_j + L(n_x,n_y,n_z)]
    \end{align*}
  \item 短距離力
    \begin{itemize}
    \item カットオフを入れる
    \item 最も近いイメージだけ考慮(minimum image convention)
    \end{itemize}
  \item 長距離力
    \begin{itemize}
    \item カットオフを入れると物理が変わる
    \item エバルト法、ツリー法、高速多重極展開
    \end{itemize}
  \end{itemize}
\end{frame}


\section{ビリアル定理}

%-*- coding:utf-8 -*-

\begin{frame}[t,fragile]{ビリアル定理}
  \begin{itemize}
    %\setlength{\itemsep}{1em}
  \item ビリアル定理
    \[
    \langle T \rangle = -\frac{1}{2} \sum_i^N \langle \mathbf{f}_i \cdot \mathbf{r}_i \rangle
    \]
    $T$: 運動エネルギー、$\mathbf{f}_i$: 粒子$i$に働く力、$\mathbf{r}_i$: 粒子$i$の位置
  \item ビリアル
    \[
    G = \sum_i^N \mathbf{p}_i \cdot \mathbf{r}_i
    \]
    \begin{itemize}
    \item $G$の時間微分
      \[
      \frac{dG}{dt} = \sum_i \mathbf{p}_i \cdot \frac{d\mathbf{r}_i}{dt} + \sum_i \frac{d\mathbf{p}_i}{dt} \cdot \mathbf{r}_i = 2 T + \sum_i \mathbf{f}_i \cdot \mathbf{r}_i
      \]
    \item 長時間平均をとると左辺は0 $\Rightarrow$ ビリアル定理
    \end{itemize}
  \end{itemize}
\end{frame}

%-*- coding:utf-8 -*-

\begin{frame}[t,fragile]{ビリアル定理}
  \begin{itemize}
    %\setlength{\itemsep}{1em}
  \item ベキ的な相互作用の場合: $U = \sum_{i<j} a_{ij} | \mathbf{r}_i - \mathbf{r}_j |^{n+1}$
    \[
    \mathbf{f}_i = -\nabla_i U = - \sum_{j \ne i} a_{ij} (n+1) (\mathbf{r}_i - \mathbf{r}_j) | \mathbf{r}_i - \mathbf{r}_j |^{n-1} \equiv \sum_{j \ne i} \mathbf{f}_{ij}
    \]
    \begin{align*}
      -\frac{1}{2} \sum_i \mathbf{f}_i \cdot \mathbf{r}_i &= -\frac{1}{2} \sum_i \sum_{j \ne i} \mathbf{f}_{ij} \cdot \mathbf{r}_i = -\frac{1}{2} \sum_{i < j} (\mathbf{f}_{ij} \cdot \mathbf{r}_i + \mathbf{f}_{ji} \cdot \mathbf{r}_j) \\
      &= -\frac{1}{2} \sum_{i < j} \mathbf{f}_{ij} \cdot (\mathbf{r}_i - \mathbf{r}_j) = \frac{1}{2} (n+1) U
    \end{align*}
    $\Rightarrow$ $\displaystyle \langle T \rangle = \frac{n+1}{2} \langle U \rangle$
  \item 特に重力、静電気力の場合($n=-2$)
    \[
     \langle T \rangle = -\frac{1}{2} \langle U \rangle \qquad \text{ビリアル比: $\displaystyle r_{\rm v} \equiv \frac{\langle T \rangle}{| \langle U \rangle |} = \frac{1}{2}$}
    \]
  \end{itemize}
\end{frame}

%-*- coding:utf-8 -*-

\begin{frame}[t,fragile]{圧力・温度の計算}
  \begin{itemize}
    %\setlength{\itemsep}{1em}
  \item 壁から受ける力を考えると
    \begin{align*}
      \mathbf{f}_i &= \mathbf{f}_i^\text{int} + \mathbf{f}_i^\text{ext} = -\nabla_i U + \mathbf{f}_i^\text{ext} \\
      \sum_i \mathbf{f}_i^\text{ext} \cdot \mathbf{r}_i &= - \int_{\partial V} (P \mathbf{n}) \cdot \mathbf{r} \, dS = - P \int \nabla \cdot \mathbf{r} \, dV = -3PV
    \end{align*}
  \item ビリアル定理と組み合わせて
    \[
    PV = \frac{2}{3} \langle T \rangle -\frac{1}{3} \langle \sum_i \nabla_i U \cdot \mathbf{r}_i \rangle
    \]
  \item 温度の計算: エネルギー等分配則より
    \[
    \langle T \rangle = \frac{3}{2} k_B T N
    \]
    %\begin{itemize}
    %\item
    (全運動量を0に固定している場合は$\langle T \rangle = \frac{3}{2} k_B T (N-1)$)
    %\end{itemize}
  \end{itemize}
\end{frame}


\section{温度の制御}

%-*- coding:utf-8 -*-

\begin{frame}[t,fragile]{温度の制御}
  \begin{itemize}
    %\setlength{\itemsep}{1em}
  \item カノニカル分布を実現するには?
    \begin{itemize}
    \item 巨大な環境(熱浴)を付ける必要がある?
    \item シミュレーションで巨大な環境を用意するのは非現実的
    \end{itemize}
  \item 速度スケーリング(Velocity Scaling)法
    \begin{itemize}
    \item 平均の運動エネルギーが対応する温度に一致するように毎回速度(運動量)をスケールしなおす
      \begin{align*}
        p_i \Rightarrow &p'_i = \sqrt{ \frac{3mNk_BT}{\sum_i p_i^2} } p_i \\
        & \sum_i \frac{p_i^{'2}}{2m} = \frac{3mNk_BT}{\sum_i p_i^2} \sum_i \frac{p_i^2}{2m} = \frac{3}{2} N k_B T
      \end{align*}
    \item 位置エネルギーも運動エネルギーに引きずられてカノニカル分布に収束? $\Rightarrow$ 理論的根拠なし
    \end{itemize}
  \end{itemize}
\end{frame}

%-*- coding:utf-8 -*-

\begin{frame}[t,fragile]{温度の制御}
  \begin{itemize}
    %\setlength{\itemsep}{1em}
  \item ランジュバン(Langevin)法
    \begin{itemize}
    \item 乱数を使う方法
    \item 摩擦項と揺動項(ランダム力)を付け加える
      \begin{align*}
        \frac{dp_i}{dt} = - \frac{\partial H}{\partial q_i} - \gamma_i p_i + R_i(t)
      \end{align*}
    \item $\gamma_i$: 摩擦係数
    \item $R_i(T)$: 平均零の白色ノイズ
      \begin{align*}
        \langle R_i(t) R_i(t') \rangle = 2 m \gamma_i k_B T \delta(t-t')
      \end{align*}
    \item 摩擦項と揺動項がつりあうところで、温度$T$のカノニカル分布が実現する
    \end{itemize}
  \item Andersen法
    \begin{itemize}
    \item 各粒子と熱浴の結合係数: \(\nu\)
    \item 時間幅\(h\)の時間後、各粒子に対して確率\(\nu h\)でボルツマン分布から速度を再設定する。
    \end{itemize}
  \end{itemize}
\end{frame}

%-*- coding:utf-8 -*-

\begin{frame}[t,fragile]{Nose-Hoover熱浴}
  \begin{itemize}
    %\setlength{\itemsep}{1em}
  \item Nose (能勢)-Hoover法
    \begin{itemize}
    \item 熱浴をたった1つの自由度($s$)だけで実現する!
    \item 現実系のハミルトニアン
      \begin{align*}
        H(\mathbf{p},\mathbf{x}) &= \sum_i \frac{p_i^2}{2m} + U(\mathbf{x})
      \end{align*}
    \item 仮想系のハミルトニアン (温度$T$をパラメータとして含む)
      \begin{align*}
        H'(\mathbf{p}',\mathbf{x}',{\color{red}p_s},{\color{red}s}) &= \sum_i \frac{{p'_i}^2}{2m{\color{red}s^2}} + U(\mathbf{x}') + {\color{red}\frac{p_s^2}{2Q} + g k_B T\log s}
      \end{align*}
      $s$: 熱浴の自由度、$p_s$: $s$に共役な運動量、$Q$: 熱浴の「質量」、$g$: 系の自由度($3N+1$または$3N$)
    \end{itemize}
  \end{itemize}
\end{frame}


%-*- coding:utf-8 -*-

\begin{frame}[t,fragile]{Nose-Hoover熱浴}
  \begin{itemize}
  \item 仮想系の運動方程式
    \begin{align*}
      \frac{dx_i'}{dt'} &= \frac{\partial H'}{\partial p_i'} = \frac{p_i'}{ms^2} \\
      \frac{dp_i'}{dt'} &= -\frac{\partial H'}{\partial x_i'} = -\frac{\partial U}{\partial x_i'} \\
      \frac{ds}{dt'} &= \frac{\partial H'}{\partial p_s} = \frac{p_s}{Q} \\
      \frac{dp_s}{dt'} &= -\frac{\partial H'}{\partial s} = \sum_i \frac{{p'_i}^2}{ms^3} - \frac{g k_B T}{s}
    \end{align*}
  \item 現実系と仮想系との間に以下の関係を仮定する

    $x_i=x_i'$、$p_i=p_i'/{\color{red}s}$、$t=\int^t s^{-1} dt'$、$dt=dt'/{\color{red}s}$
  \end{itemize}
\end{frame}

%-*- coding:utf-8 -*-

\begin{frame}[t,fragile]{Nose-Hoover熱浴}
  \begin{itemize}
    %\setlength{\itemsep}{1em}
  \item 現実系の変数による書き換え
    \begin{align*}
      \frac{dx_i}{dt} &= \frac{\partial H}{\partial p_i} \\
      \frac{dp_i}{dt} &= -\frac{\partial U}{\partial x_i} -\frac{p_s}{Q} p_i \\
      \frac{dp_s}{dt} &= 2 \big[ \sum_i \frac{{p_i}^2}{2m} - \frac{g k_B T}{2} \big]
    \end{align*}
  \item あるいは、$x_s = \log s$を導入すると、最後の2つの式は
    \begin{align*}
      % \frac{dx_i}{dt} &= \frac{\partial H}{\partial p_i} \\
      \frac{dp_i}{dt} &= -\frac{\partial U}{\partial x_i} -\frac{dx_s}{dt} p_i \\
      \frac{d^2x_s}{dt^2} &= \frac{2}{Q} \big[ \sum_i \frac{{p_i}^2}{2m} - \frac{g k_B T}{2} \big]
    \end{align*}
  \end{itemize}
\end{frame}

%-*- coding:utf-8 -*-

\begin{frame}[t,fragile]{カノニカル分布の実現}
  \begin{itemize}
    %\setlength{\itemsep}{1em}
  \item $H'(\mathbf{p}',\mathbf{x}',p_s,s)$による仮想時間発展によりエネルギー$E'$のミクロカノニカルアンサンブルが実現しているとする

    $\Leftrightarrow$ 仮想時間$t'$でサンプルすると$2(3N+1)$次元の位相空間上で$H'=E'$の曲面上に均等に分布(エルゴード性)
    
    $\Leftrightarrow$ 分布関数: $\delta [ H'(\mathbf{p}',\mathbf{x}',p_s,s) - E']$
    
    $\Rightarrow$ 現実系の変数$(p,x)$に関する周辺分布を考える
  \item 仮想系のミクロカノニカル分配関数
    \begin{align*}
      Z'&=\int d\mathbf{p}' d\mathbf{x}' dp_s ds \, \delta [ H'(\mathbf{p}',\mathbf{x}',p_s,s) - E'] \\
      &=\int d\mathbf{{\color{red}p}} d\mathbf{{\color{red}x}} dp_s ds \, {\color{red}s^{3N}} \delta [ {\color{red}H(\mathbf{p},\mathbf{x})} + \frac{p_s^2}{2Q} + g k_B T \log s - E']
    \end{align*}
  \end{itemize}
\end{frame}

%-*- coding:utf-8 -*-

\begin{frame}[t,fragile]{カノニカル分布の実現}
  \begin{itemize}
    %\setlength{\itemsep}{1em}
  \item $s$について積分(*)
    \begin{align*}
      \int ds \, s^{3N} \delta[f(s)] = s_0^{3N} / | f'(s_0) | = s_0^{3N+1} / g k_B T
    \end{align*}
    $f(s) \equiv H(\mathbf{p},\mathbf{x}) + \frac{p_s^2}{2Q} + g k_B T \log s - E'$、$s_0$は$f(s)=0$の解
    \begin{align*}
      s_0 = \exp \Big[ -\frac{1}{gk_BT} \Big( H(p,x) + \frac{p_s^2}{2Q} - E' \Big) \Big]
    \end{align*}
  \item さらに$p_s$についてガウス積分を行い、$g=3N+1$とすると
    \begin{align*}
      Z' = \frac{1}{3N+1} \sqrt{\frac{2Q\pi}{k_BT}} e^{E'/k_BT} \int d\mathbf{p}d\mathbf{x} \, {\color{red}\exp [- H(p,x)/k_BT ]}
    \end{align*}
  \end{itemize}
\end{frame}

%%-*- coding:utf-8 -*-

\begin{frame}[t,fragile]{補足(*)}
  \begin{itemize}
    %\setlength{\itemsep}{1em}
  \item $f(s)=0$の解を$s_0$とすると、$\displaystyle \delta [f(s)] = \frac{\delta(s-s_0)}{|f'(s_0)|}$が成り立つ

    証明: $u=f(s)$とおくと、$du = f'(s)ds$から
    \begin{align*}
      \int h(s) {\color{red}\delta [f(s)]} \, ds= \int h(f^{-1}(u)) \delta(u) \frac{du}{|f'(f^{-1}(u))|}
    \end{align*}
    $f^{-1}(0)=s_0$なので
    \begin{align*}
      = \frac{h(s_0)}{|f'(s_0)|} = \int h(s) {\color{red}\frac{\delta(s-s_0)}{|f'(s_0)|}} ds
    \end{align*}
  \end{itemize}
\end{frame}

%-*- coding:utf-8 -*-

\begin{frame}[t,fragile]{実時間発展の場合}
  \begin{itemize}
    %\setlength{\itemsep}{1em}
  \item 実時間に直した方程式の時間発展を計算した場合、物理量$A(\mathbf{p},\mathbf{x})$の実時間平均は
    \begin{align*}
      \langle A \rangle_t &= \lim_{\tau\rightarrow\infty} \frac{1}{\tau} \int_0^\tau dt \, A(\mathbf{p}(t),\mathbf{x}(t)) \\
      &= \lim_{\tau\rightarrow\infty} \frac{\tau'}{\tau} \frac{1}{\tau'} \int_0^{\tau'} dt' \, A(\mathbf{p}'(t')/s(t'),\mathbf{x}'(t')) / s(t')
    \end{align*}
  \item $\tau = \int_0^{\tau} dt = \int_0^{\tau'} dt'/s(t')$なので
    \begin{align*}
      \langle A \rangle_t &= \frac{\lim_{\tau'\rightarrow\infty} \frac{1}{\tau'} \int_0^{\tau'} dt' \, A(\mathbf{p}'(t')/s(t'),\mathbf{x}'(t')) / s(t')}{\lim_{\tau'\rightarrow\infty} \frac{1}{\tau'} \int_0^{\tau'} dt' \, 1/ s(t')} \\
      &= \langle A(\mathbf{p},\mathbf{x}) / s \rangle_{t'} / \langle 1 / s \rangle_{t'}
    \end{align*}
  \end{itemize}
\end{frame}

%-*- coding:utf-8 -*-

\begin{frame}[t,fragile]{実時間発展の場合}
  \begin{itemize}
    %\setlength{\itemsep}{1em}
  \item 分子・分母の$t'$に関する期待値を計算すると、$s_0$が1つキャンセルするので
    \begin{align*}
      \langle A \rangle_t &= \frac{\int d\mathbf{p} d\mathbf{x} \, A(\mathbf{p},\mathbf{x}) \exp [ -\frac{3N}{gk_BT} H(\mathbf{p}, \mathbf{x})]}{\int d\mathbf{p} d\mathbf{x} \, \exp [ -\frac{3N}{gk_BT} H(\mathbf{p}, \mathbf{x})]}
    \end{align*}
    \item {\color{red}$g=3N$}とすると、実時間発展の長時間平均とカノニカル分布における位相平均が一致
  \end{itemize}
\end{frame}

%-*- coding:utf-8 -*-

\begin{frame}[t,fragile]{温度の制御}
  \begin{itemize}
    %\setlength{\itemsep}{1em}
  \item Nose-Hoover熱浴
    \begin{itemize}
    \item 運動方程式(実時間発展の場合)
      \begin{align*}
        \frac{dx_i}{dt} &= \frac{\partial H}{\partial p_i} \\
        \frac{dp_i}{dt} &= -\frac{\partial U}{\partial x_i} -\frac{p_s}{Q} p_i \\
        \frac{dp_s}{dt} &= 2 \big[ \sum_i \frac{{p_i}^2}{2m} - \frac{g k_B T}{2} \big]
      \end{align*}
    \item 「摩擦係数」$p_s$にネガティブフィードバックがかかる
    \item $g=3N$ととれば、$(p,x)$の周辺分布はカノニカル分布になる
    \end{itemize}
  \item 調和振動子系など簡単な系では保存量が生じてエルゴード性が破れる場合も $\Rightarrow$ 自由度$s$の温度を制御するもう一つの自由度$r$を追加(Nose-Hoover Chain法) 
  \item 温度圧力一定(NPTアンサンブル)も体積を表すもう1つの自由度を追加することで実現可
  \end{itemize}
\end{frame}


\section{}
\begin{frame}[t]{本日の課題}
  \begin{itemize}
    %\setlength{\itemsep}{1em}
  \item 実習
    \begin{itemize}
    \item 正規分布にしたがう乱数の生成法
    \item 実習課題一覧\href{https://github.com/todo-group/ComputerExperiments/releases/tag/2020a-computer2}{exercise-2.pdf}から常微分方程式(あるいは別の)課題を選び実習
    \end{itemize}
  \item 質問はSlackの「\# 3\_常微分方程式」あるいは他の適当と思われるチャンネルで
  \item 次回講義(12/4)の前日までにITC-LMSのアンケート「作業レポートNo.6」に回答
  \end{itemize}
\end{frame}

\end{document}
