%-*- coding:utf-8 -*-

\documentclass[10pt,dvipdfmx]{beamer}
\usepackage{tutorial}

\title{計算機実験II (L3) --- 量子力学}
\date{2020/10/16}

\begin{document}

\begin{frame}
  \titlepage
  \tableofcontents
\end{frame}

\begin{frame}[t]{講義日程 (予定)}
  \begin{itemize}
    % \setlength{\itemsep}{1em}
  \item 全8回 (金曜5限 16:50-18:35)
    \begin{itemize}
    \item 10月4日(金) 講義1: 多体系の統計力学とモンテカルロ法
    \item {\color{gray} 10月11日(金) 休講 (物理学教室コロキウム)}
    \item 10月18日(金) 実習1
    \item {\color{gray} 10月25日(金) 休講}
    \item 11月1日(金) 講義2: 偏微分方程式と多体系の量子力学
    \item 11月8日(金) 実習2
    \item {\color{gray} 11月15日(金) 休講}
    \item 11月29日(金) 講義3: 少数多体系・分子動力学
    \item 12月6日(金) 実習3
    \item {\color{gray} 12月13日(金) 休講 (物理学教室コロキウム)}
    \item {\color{gray} 12月20日(金) 休講 (ニュートン祭)}
    \item 12月27日(金) 講義4: 最適化問題
    \item 1月10日(金) 実習4
    \item {\color{gray} 1月24日(金) 休講 (物理学教室コロキウム)}
    \end{itemize}
  \end{itemize}
\end{frame}


\section{シュレディンガー方程式の固有値問題 (復習)}
\begin{frame}[t,fragile]{時間依存しないシュレディンガー方程式}
  \begin{itemize}
    %\setlength{\itemsep}{1em}
  \item 井戸型ポテンシャル中の一粒子問題
    \begin{align*}
      \big[ -\frac{\hbar^2}{2m}\frac{d^2}{dx^2} + V(x) \big] \psi(x) = E \psi(x) \\
      V(x) = \begin{cases}
        0 & \text{$a \le x \le b$} \\ \infty & \text{otherwise}
      \end{cases}
    \end{align*}
  \item $\hbar^2/2m = 1$、$a=0$、$b=1$となるように変数変換して
    \begin{align*}
      \big( \frac{d^2}{dx^2} + E \big) \psi(x) = 0 \qquad 0 \le x \le 1
    \end{align*}
    を境界条件$\psi(0) = \psi(1) = 0$のもとで解けば良い
  \end{itemize}
\end{frame}

\begin{frame}[t,fragile]{固有値問題の解法(シューティング)}
  \begin{itemize}
    %\setlength{\itemsep}{1em}
  \item $x_i=h \times i$ ($h=1/n$)、$x_0=0$、$x_n=1$とする
  \item $\psi(x_0)=0$、$\psi(x_1) = 1$を仮定 ($\psi'(x_0)=1/h$と与えたことに相当)
  \item $E = 0$とおく
  \item Runge-Kutta法、Numerov法などを用いて$x=x_n$まで積分
  \item $\psi(x_n)$の符号がかわるまで、$E$を少しずつ増やす
  \item 符号が変わったら、$E$の区間を半分ずつに狭めていき、$\psi(x_n)=0$となる$E$ (固有エネルギー)と$\psi(x)$ (波動関数)を得る
  \end{itemize}
\end{frame}

\begin{frame}[t,fragile]{シュレディンガー方程式の行列表示}
  \begin{itemize}
    %\setlength{\itemsep}{1em}
  \item シュレディンガー方程式
    \[
    [-\frac{d^2}{dx^2}+V(x)]\psi(x) = E \psi(x)
    \]
  \item 連立差分方程式を行列の形で表す($\psi(x_0)=\psi(x_n)=0$)
    \[\begin{small}\hspace*{-4em}
    \begin{pmatrix}
      \frac{2}{h^2}+V(x_1) & -\frac{1}{h^2} \\
      -\frac{1}{h^2} & \frac{2}{h^2}+V(x_2) & -\frac{1}{h^2} \\
      & -\frac{1}{h^2} & \frac{2}{h^2}+V(x_3) & -\frac{1}{h^2} \\
      & & \ddots & \ddots \\
      & & & -\frac{1}{h^2} & \frac{2}{h^2}+V(x_{n-1}) \\
    \end{pmatrix}
    \begin{pmatrix}
      \psi(x_1) \\
      \psi(x_2) \\
      \psi(x_3) \\
      \vdots \\
      \psi(x_{n-1}) \\
    \end{pmatrix}
    = \cdots % E
    %% \begin{pmatrix}
    %%   \psi(x_1) \\
    %%   \psi(x_2) \\
    %%   \psi(x_3) \\
    %%   \vdots \\
    %%   \psi(x_{n-1}) \\
    %% \end{pmatrix}
    \end{small}
    \]
  \item $(n-1) \times (n-1)$の疎行列の固有値問題
    \begin{itemize}
    \item 固有値: 固有エネルギー
    \item 固有ベクトル: 波動関数
    \end{itemize}
  \end{itemize}
\end{frame}


%-*- coding:utf-8 -*-

\begin{frame}[t,fragile]{変分法}
  \begin{itemize}
    %\setlength{\itemsep}{1em}
  \item 波動関数を互いに直交する正規化された波動関数(基底関数)の線形結合で近似する (変分波動関数、試行関数)
    \begin{align*}
      | \psi \rangle = \sum_{p=1}^m C_p | \phi_p \rangle \qquad (\langle \phi_p | \phi_q \rangle = \delta_{pq})
    \end{align*}
  \item エネルギーの期待値
    \begin{align*}
      E &= \frac{\langle \psi | H | \psi \rangle}{\langle \psi | \psi \rangle} = \frac{\sum_{p,q} C_p^* H_{pq} C_q}{\sum_{p,q} C_p^* \delta_{pq} C_q} \\
      H_{pq} &= \langle \phi_p | H | \phi_q \rangle
    \end{align*}
  \item $E$ができるだけ小さくなるよう係数$C_p$を最適化 (変分原理)
  \end{itemize}
\end{frame}

%-*- coding:utf-8 -*-

\begin{frame}[t,fragile]{変分法}
  \begin{itemize}
    %\setlength{\itemsep}{1em}
  \item $\delta E = 0$から
    \begin{align*}
      \sum_{q} (H_{pq} - E \delta_{pq} ) C_q = 0 \qquad \text{for $^\forall p$}
    \end{align*}
  \item $H_{pq}$, $\delta_{pq}$を$m \times m$行列と考えると、固有値問題とみなせる
    \begin{align*}
      H C = E C
    \end{align*}
  \item $H$はエルミート行列
  \item $\{ \phi_p \}$の張る部分空間での最適化 (= Rayleigh-Ritzの方法)
  \item 変分波動関数と真の波動関数の差が$\epsilon$程度の時、$E$と真の固有エネルギーの差は$\epsilon^2$程度
  \end{itemize}
\end{frame}

%-*- coding:utf-8 -*-

\begin{frame}[t,fragile]{非直交基底関数による変分法}
  \begin{itemize}
    %\setlength{\itemsep}{1em}
  \item 重なり積分
    \begin{align*}
      S_{pq} = \langle \phi_p | \phi_q \rangle \ne \delta_{pq}
    \end{align*}
  \item 変分波動関数の正規化条件
    \begin{align*}
      \langle \psi | \psi \rangle = \sum_{p,q} C_p^* \langle \phi_p | \phi_q \rangle C_q = \sum_{p,q} C_p^* S_{pq} C_q = 1
    \end{align*}
  \item エネルギー期待値
    \begin{align*}
      E = \frac{\sum_{p,q} C_p^* H_{pq} C_q}{\sum_{p,q} C_p^* S_{pq} C_q}
    \end{align*}
  \item $\delta E = 0$から
    \begin{align*}
      \sum_q (H_{pq} - E S_{pq}) C_q = 0 \ \Rightarrow \ HC = ESC \ \text{(一般化固有値問題)}
    \end{align*}
  \end{itemize}
\end{frame}


\section{偏微分方程式の境界値問題 (復習)}

\begin{frame}[t,fragile]{ポアソン方程式の境界値問題}
  \begin{itemize}
    %\setlength{\itemsep}{1em}
  \item 二次元ポアソン方程式
    \[ \frac{\partial^2 u(x,y)}{\partial x^2} + \frac{\partial^2 u(x,y)}{\partial y^2} = f(x,y) \qquad 0 \le x \le 1, \ 0 \le y \le 1\]
  \item ディリクレ型境界条件: $u(x,y) = g(x,y)$ on $\partial \Omega$
  \item 有限差分法により離散化
    \begin{itemize}
    \item $x$方向、$y$方向をそれぞれ$n$等分: $(x_i,y_j) = (i/n, j/n)$
    \item $(n+1)^2$個の格子点の上で$u(x_i,y_j)=u_{ij}$が定義される
    \item そのうち$4n$個の値は境界条件で定まる
    \item ポアソン方程式を中心差分で近似 ($h=1/n$)
      \[
      \frac{u_{i+1,j}-2u_{ij}+u_{i-1,j}}{h^2} + \frac{u_{i,j+1}-2u_{ij}+u_{i,j-1}}{h^2} = f_{ij}
      \]
      残り$(n-1)^2$個の未知数に対する連立一次方程式
    \end{itemize}
  \end{itemize}
\end{frame}


%\section{偏微分方程式の初期値問題}
%-*- coding:utf-8 -*-

\section{偏微分方程式の初期値問題}

\begin{frame}[t]{一次元拡散方程式(放物型)}
  \begin{itemize}
  \item 一次元拡散方程式: $u=u(x,t)$, $q=q(x,t)$
    \[
    \frac{\partial u}{\partial t} - D \frac{\partial^2 u}{\partial x^2} = q
    \]
    \begin{itemize}
    \item 初期条件: $u(x,0) = f(x)$
    \item 境界条件: $u(0,t) = u(1,t) = 0$
    \end{itemize}
  \item 時間$t$と位置$x$に関して離散化
    \begin{align*}
      & u_j^n = u(x_j, t_n) \\
      & q_j^n = q(x_j, t_n) \\
      & t_0 = 0, t_1=\Delta t, t_2=2 \Delta t, \cdots, t_n=n \Delta t, \cdots \\
      & x_0 = 0, x_1=\Delta x, x_2=2 \Delta x, \cdots, x_N=N \Delta x = 1 \qquad (\Delta x = 1/N)
    \end{align*}
  \end{itemize}
\end{frame}

\begin{frame}[t]{有限差分法}
  \begin{itemize}
  \item $t$に関して前進差分を考える
    \[
    \frac{\partial u}{\partial t} \Big|_{(j \Delta x, n \Delta t)} = \frac{u_j^{n+1} - u_j^n}{\Delta t} + {\cal O}(\Delta t)
    \]
  \item $x$に関しては中心差分を考える
    \[
    \frac{\partial^2 u}{\partial x^2} \Big|_{(j \Delta x, n \Delta t)} = \frac{u_{j+1}^{n} - 2 u_{j}^{n} + u_{j-1}^{n}}{\Delta x^2} + {\cal O}(\Delta x^2)
    \]
  \item 拡散方程式に代入して整理すると
    \[
    u_{j}^{n+1} = u_{j}^{n} + r (u_{j+1}^{n} - 2 u_{j}^{n} + u_{j-1}^{n}) + \Delta t q_{j}^{n} \qquad (r = D\frac{\Delta t}{\Delta x^2})
    \]
  \item FTCS (Forward-Time Centered Space)法
  \end{itemize}
\end{frame}

\begin{frame}[t]{FTCS法}
  \begin{itemize}
  \item $O(\Delta t) + O(\Delta x^2)$の陽解法
    \begin{center}
      \resizebox{0.4\textwidth}{!}{\includegraphics{image/ftcs-1.pdf}}
    \end{center}
  \item 初期条件
    \[
    u_j^0 = f(j\Delta x) \ \ (j=0,1,\cdots,N)
    \]
  \item 境界条件
    \[
    u_0^n = u_N^n = 0 \ \ (n=0,1,2,\cdots)
    \]
  \end{itemize}
\end{frame}

\begin{frame}[t]{有限差分法の安定性}
  \begin{itemize}
  \item (陽的)有限差分法においては、$\Delta t$、$\Delta x$は小さければ小さいほどよいというわけではない
  \item 一次元拡散方程式の場合
    \begin{align*}
      \begin{cases}
        r \le 1/2 & \text{安定} \\
        r > 1/2 & \text{\color{red}不安定}
      \end{cases}
    \end{align*}
  \item $\Delta x$を半分にしたら、$\Delta t$は1/4にしなければならない

    $\Rightarrow$ 計算量は8倍
  \end{itemize}
\end{frame}

\begin{frame}[t]{一次元波動方程式(双極型)}
  \begin{itemize}
  \item 一次元波動方程式
    \[
    \frac{\partial^2 u}{\partial t^2} = c^2 \frac{\partial^2 u}{\partial x^2} \qquad u(x,0)=f(x), \frac{\partial u}{\partial t} (x,0) = g(x)
    \]
  \item $t$に関する中心差分
    \[
    \frac{\partial^2 u}{\partial t^2} \Big|_{(j \Delta x, n \Delta t)} = \frac{u_{j}^{n+1} - 2 u_{j}^{n} + u_{j}^{n-1}}{\Delta t^2} + {\cal O}(\Delta t^2)
    \]
  \item 代入して整理すると
    \[
    u_{j}^{n+1} = 2u_{j}^{n} - u_{j}^{n-1} + \alpha^2 (u_{j+1}^{n} - 2 u_{j}^{n} + u_{j-1}^{n}) \qquad (\alpha = c\frac{\Delta t}{\Delta x})
    \]
  \end{itemize}
\end{frame}

\begin{frame}[t]{波動方程式に対するFTCS法}
  \begin{itemize}
  \item $O(\Delta t^2) + O(\Delta x^2)$の陽解法
    \begin{center}
      \resizebox{0.4\textwidth}{!}{\includegraphics{image/ftcs-2.pdf}}
    \end{center}
  \item 初期条件
    \[
    u_j^0 = f(j\Delta x) \ \ (j=0,1,\cdots,N)
    \]
    初期速度については$n=0$に関する中心差分を考えて
    \[
    \frac{u_j^1 - u_j^{-1}}{2 \Delta t} = g_j \ \ \Rightarrow \ \ u_j^1 = u_j^0 + \Delta t g_j + \frac{\alpha^2}{2} (u_{j+1}^{n} - 2 u_{j}^{n} + u_{j-1}^{n})
    \]
  \end{itemize}
\end{frame}

\begin{frame}[t]{時間に依存するシュレディンガー方程式}
  \begin{itemize}
  \item 時間に依存するシュレディンガー方程式
    \[
    i \hbar \frac{\partial \Psi}{\partial t}(x,t) = H(x,t) \Psi(x,t) = \Big[ - \frac{\hbar^2}{2m} \frac{\partial^2}{\partial x^2} + V(x,t) \Big] \Psi(x,t)
    \]
    \begin{itemize}
    \item 波動関数のノルム $\displaystyle \int | \Psi(x,t) |^2 \, dx$ は保存
    \item $V(x,t)$が時間$t$に依存しない場合、エネルギーの期待値は保存
      \[
      \langle H \rangle = \frac{\displaystyle \int \Psi^* H \Psi \, dx}{\displaystyle \int | \Psi |^2 \, dx}
      \]
    \end{itemize}
  \item 以下では無次元化して$\hbar = m = 1$とおく
  \end{itemize}
\end{frame}

\begin{frame}[t]{時間に依存するシュレディンガー方程式}
  \begin{itemize}
  \item シュレディンガー方程式の形式解 ($V$が時間に依存しない場合)
    \[
    \Psi(x,t) = e^{-i H t} \Psi(x,0)
    \]
  \item $t$に関して前進差分
    \begin{align*}
      e^{-i H t} &= [e^{-i H \Delta t}]^M \approx [1 - i H \Delta t]^M \qquad (\Delta t = t / M) \\
      \Psi^{n+1} &= (1 -  i \Delta t H) \Psi^{n}
    \end{align*}
  \item $H$は対称(エルミート)行列 $\Rightarrow$ 時間発展演算子$e^{-i H \Delta t}$はユニタリー行列
  \item 差分近似$(1 -  i \Delta t H)$はユニタリーではない
    \begin{align*}
      & (e^{-i H \Delta t})^\dagger e^{-i H \Delta t} = e^{i H \Delta t} e^{-i H \Delta t} = 1 \\
      & (1 -  i \Delta t H)^\dagger (1 -  i \Delta t H) = (1 +  i \Delta t H) (1 -  i \Delta t H) = 1 + {\color{red} \Delta t^2 H^2}
    \end{align*}
  \end{itemize}
\end{frame}

\begin{frame}[t]{クランク・ニコルソン法}
  \begin{itemize}
  \item クランク・ニコルソン法
    \[
    \Psi^{n+1} = \frac{1 -  i \frac{\Delta t}{2} H}{1 +  i \frac{\Delta t}{2} H} \Psi^{n}
    \]
  \item (数値精度の範囲で)ユニタリー行列であるので、ノルムは保存
  \item $(1 +  i \frac{\Delta t}{2} H)^{-1}$を掛ける $\Rightarrow$ 連立一次方程式を解く必要がある
    \begin{itemize}
    \item まず、$\Psi = (1 - i \frac{\Delta t}{2} H) \Psi^n$ を計算
    \item 次に、$(1 +  i \frac{\Delta t}{2} H) \Psi^{n+1} = \Psi$ を解く(連立一次方程式)
    \end{itemize}
  \item 陰解法の一種
  \end{itemize}
\end{frame}

\begin{frame}[t]{陰解法}
  \begin{itemize}
  \item 時刻$t$関して後退差分を使う
    \[
    \frac{\partial u}{\partial t} \Big|_{(j \Delta x, n \Delta t)} = \frac{u_j^{n} - u_j^{n-1}}{\Delta t} + {\cal O}(\Delta t)
    \]
  \item $x$に関する中心差分と組み合わせ、$n \rightarrow n+1$と書き直すと
    \[
    u_{j}^{n+1} = u_{j}^{n} + r (u_{j+1}^{n+1} - 2 u_{j}^{n+1} + u_{j-1}^{n+1})
    \]
    $u^{n+1}$が両辺に現れる $\Rightarrow$ 陰解法
  \item $O(\Delta t) + O(\Delta x^2)$
  \item $r$の値によらず{\color{red}常に安定}
  \end{itemize}
\end{frame}

\begin{frame}[t]{クランク・ニコルソン法}
  \begin{itemize}
  \item さらに、時間方向にきざみ幅$\Delta t/2$の中心差分を使うと
    \[
    \frac{\partial u}{\partial t} \Big|_{(j \Delta x, n \Delta t)} = \frac{u_j^{n+\frac{1}{2}} - u_j^{n-\frac{1}{2}}}{\Delta t} + {\cal O}(\Delta t^2)
    \]
  \item $x$に関する中心差分と組み合わせ、$n \rightarrow n+\frac{1}{2}$し、さらに$u_j^{n+\frac{1}{2}}$を$(u_j^{n+1}+u_j^{n})/2$で近似すると
    \[
    u_{j}^{n+1} = u_{j}^{n} + \frac{r}{2} (u_{j+1}^{n+1} - 2 u_{j}^{n+1}  +u_{j-1}^{n+1} + u_{j+1}^{n} - 2 u_{j}^{n} + u_{j-1}^{n})
    \]
    あるいは
    \[
    u_{j}^{n+1} - \frac{r}{2} (u_{j+1}^{n+1} - 2 u_{j}^{n+1} + u_{j-1}^{n+1}) = u_{j}^{n} + \frac{r}{2} (u_{j+1}^{n} - 2 u_{j}^{n} + u_{j-1}^{n})
    \]
    $\Rightarrow$ クランク・ニコルソン法 [$O(\Delta t^2) + O(\Delta x^2)$]
  \end{itemize}
\end{frame}

%% \begin{frame}[t]{}
%%   \begin{itemize}
%%   \item
%%   \end{itemize}
%% \end{frame}


\section{}
\begin{frame}[t]{本日の課題}
  \begin{itemize}
    %\setlength{\itemsep}{1em}
  \item 実習
    \begin{itemize}
    \item 連立一次方程式の解法の復習 (計算機実験I 第5回)
    \item 実習課題一覧\href{https://github.com/todo-group/ComputerExperiments/releases/tag/2020a-computer2}{exercise-2.pdf}から偏微分方程式(あるいは別の)課題を選び実習
    \end{itemize}
  \item 質問はSlackの「\# 7\_偏微分方程式」あるいは他の適当と思われるチャンネルで
  \item 次回講義(10/23)の前日までにITC-LMSのアンケート「作業レポートNo.3」に回答
  \end{itemize}
\end{frame}

\end{document}
