%-*- coding:utf-8 -*-

\documentclass[10pt,dvipdfmx]{beamer}
\usepackage{tutorial}

\title{計算機実験II (L2) --- 偏微分方程式と多体系の量子力学}
\date{2025/10/24}

\begin{document}

\begin{frame}
  \titlepage
  \tableofcontents
\end{frame}

\section{偏微分方程式の初期値問題}

\input{2_ode/ode-02.tex}
\input{2_ode/pde-01.tex}
\input{2_ode/pde-02.tex}
\input{2_ode/pde-03.tex}
\input{2_ode/pde-04.tex}
\begin{frame}[t]{Von Neumannの安定性解析}
  \begin{itemize}
  \item 空間方向にフーリエ変換し、時間方向には指数関数型の解を仮定する
    \begin{align*}
    u_j^n &= \sum_k (\mu_k)^n e^{ik\Delta x \cdot j}
    \end{align*}
  \item FTCS法の式に代入し、波数$k$の項を取り出すと
    \[
    \mu_k =  (1 + 2r(\cos k \Delta x - 1))
    \]
  \item 全ての$k$に対し発散しない(\(=\mu_k\)が1以下になる)ためには
    \[
      | 1 - 4r | \le 1 \qquad \text{すなわち} \ 0 < r \le \frac{1}{2}
    \]
  \end{itemize}
\end{frame}

\input{2_ode/pde-05.tex}
\begin{frame}[t]{波動方程式に対するFTCS法}
  \begin{itemize}
  \item $O(\Delta t^2) + O(\Delta x^2)$の陽解法
    \begin{center}
      \resizebox{0.4\textwidth}{!}{\includegraphics{image/ftcs-2.pdf}}
    \end{center}
  \item 初期条件
    \[
    u_j^0 = f(j\Delta x) \ \ (j=0,1,\cdots,N)
    \]
    初期速度については$n=0$に関する中心差分を考えて
    \[
    \frac{u_j^1 - u_j^{-1}}{2 \Delta t} = g_j \ \ \Rightarrow \ \ u_j^1 = u_j^0 + \Delta t g_j + \frac{\alpha^2}{2} (u_{j+1}^{0} - 2 u_{j}^{0} + u_{j-1}^{0})
    \]
  \end{itemize}
\end{frame}

\begin{frame}[t]{波動方程式の安定性解析}
  \begin{itemize}
  \item 空間方向にフーリエ変換し、時間方向には指数関数型の解を仮定する
    \begin{align*}
    u_j^n &= \sum_k (\mu_k)^n e^{ik\Delta x \cdot j}
    \end{align*}
  \item FTCS法の式に代入し、波数$k$の項を取り出すと
    \begin{align*}
    \mu_k^2 =  2\mu_k - 1 + 2\alpha^2(\cos k \Delta x - 1) \mu_k \\
    \mu_k^2 - 2 [1 + \alpha^2(\cos k \Delta x - 1)] \mu_k + 1 = 0
    \end{align*}
  \item 二つの解の積は1なので、互いに異なる2つの実数解が存在する場合、解の一つは必ず1より大きくなり、波動方程式の解は発散する
  \item 全ての波数\(k\)について重解あるいは複素解となるためには
    \[
      | 1 - 2 \alpha^2 | \le 1 \qquad \text{すなわち} \ 0 < \alpha \le 1
    \]
  \end{itemize}
\end{frame}

\input{2_ode/pde-07.tex}
\begin{frame}[t]{時間に依存するシュレディンガー方程式}
  \begin{itemize}
  \item シュレディンガー方程式の形式解 ($V$が時間に依存しない場合)
    \[
    \Psi(x,t) = e^{-i t H} \Psi(x,0)
    \]
  \item $t$に関して前進差分
    \begin{align*}
      e^{-i t H} &= [e^{-i \Delta t H}]^M \approx [1 - i \Delta t H]^M \qquad (\Delta t = t / M) \\
      \Psi^{n+1} &= (1 -  i \Delta t H) \Psi^{n}
    \end{align*}
  \item $H$は対称(エルミート)行列 $\Rightarrow$ 時間発展演算子$e^{-i \Delta t H}$はユニタリー行列
  \item 差分近似$(1 -  i \Delta t H)$はユニタリーではない
    \begin{align*}
      & (e^{-i \Delta t H})^\dagger e^{-i \Delta t H} = e^{i \Delta t H} e^{-i \Delta t H} = 1 \\
      & (1 -  i \Delta t H)^\dagger (1 -  i \Delta t H) = (1 +  i \Delta t H) (1 -  i \Delta t H) = 1 + {\color{red} \Delta t^2 H^2}
    \end{align*}
  \end{itemize}
\end{frame}

\input{2_ode/pde-09.tex}
\input{2_ode/pde-10.tex}
\input{2_ode/pde-11.tex}

% \section{対角化 (復習)}
% \begin{frame}[t,fragile]{ポアソン方程式の境界値問題}
  \begin{itemize}
    %\setlength{\itemsep}{1em}
  \item ノイマン型境界条件の場合
    \begin{itemize}
    \item 境界上で$u(x,y)$の微分が定義される。
    \item 例) $\partial u(0,y) / \partial x = h(0,y)$
    \end{itemize}
  \item 境界条件を差分近似で表す
    \[
    \frac{u_{1j} - u_{0j}}{h} = h_{0j} \qquad j=1 \cdots (n-1)
    \]
    $(n+1)^2-4$個の未知数に対して、ポアソン方程式の差分近似とあわせて、合計$(n-1)^2+4(n-1)=(n+1)^2-4$個の連立一次方程式
  \item (全ての$u_ij$を定数だけシフトしても方程式は変わらない → 独立な方程式は\((n+1)^2-3\)個 → 追加で1点の値を与える必要がある) \\[2em]
  \item 二次元グリッド上の点$(i,j)$と長さ$(n+1)^2$のベクトルの要素との対応関係をきちんと定義することが大事
  \end{itemize}
\end{frame}

% \input{4_eigenvalue_problem/intro-04.tex}
% \input{4_eigenvalue_problem/intro-07.tex}

\section{横磁場イジング模型}

\input{4_eigenvalue_problem/tfi-01.tex}
\input{4_eigenvalue_problem/tfi-02.tex}
\input{4_eigenvalue_problem/tfi-03.tex}

\section{多体量子系の時間発展}

\input{4_eigenvalue_problem/tfi-04.tex}
\input{4_eigenvalue_problem/tfi-05.tex}
\input{4_eigenvalue_problem/tfi-06.tex}

\section{量子回路シミュレーション}

\input{4_eigenvalue_problem/tfi-07.tex}
\input{4_eigenvalue_problem/tfi-08.tex}
\input{4_eigenvalue_problem/tfi-09.tex}
\input{4_eigenvalue_problem/tfi-10.tex}
\input{4_eigenvalue_problem/tfi-12.tex}

\section{}
\begin{frame}[t]{講義日程 (予定)}
  \begin{itemize}
    % \setlength{\itemsep}{1em}
  \item 全8回 (金曜5限 16:50-18:35)
    \begin{itemize}
    \item 10月10日(金) 講義1: 多体系の統計力学とモンテカルロ法
    \item 10月17日(金) 実習1
    \item 10月24日(金) 講義2: 偏微分方程式と多体系の量子力学
    \item 10月31日(金) 実習2
    \item 11月7日(金) 講義3: 少数多体系・分子動力学
    \item 11月14日(金) 実習3
    \item 11月28日(金) 講義4: 最適化問題
    \item 12月5日(金) 実習4
    \end{itemize}
  \end{itemize}
\end{frame}


\end{document}
