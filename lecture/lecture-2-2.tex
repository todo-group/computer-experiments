%-*- coding:utf-8 -*-

\documentclass[10pt,dvipdfmx]{beamer}
\usepackage{tutorial}

\title{計算機実験II (L2) --- 偏微分方程式と多体系の量子力学}
\date{2025/10/24}

\begin{document}

\begin{frame}
  \titlepage
  \tableofcontents
\end{frame}

\section{偏微分方程式の初期値問題}

\begin{frame}[t,fragile]{初期値問題と境界値問題}
  \begin{itemize}
    \setlength{\itemsep}{1em}
  \item 初期値問題
    \begin{itemize}
    \item 微分方程式において、ある1点に関する全ての境界条件(初期値)が与えられているもの
    \item 質点の運動など(時系列の問題)
  \end{itemize}
  \item 境界値問題
    \begin{itemize}
    \item 複数の点に関する境界条件が与えられているもの
    \item 物体のゆがみの計算や静電場の計算など(空間的に解く問題)
  \end{itemize}
  \item 初期値問題は初期値から逐次的に解くことが可能
  \item 境界値問題は初期値問題に比べて計算法が複雑
  \end{itemize}
\end{frame}

\begin{frame}[t]{一次元拡散方程式(放物型)}
  \begin{itemize}
  \item 一次元拡散方程式: $u=u(x,t)$, $q=q(x,t)$
    \[
    \frac{\partial u}{\partial t} - D \frac{\partial^2 u}{\partial x^2} = q
    \]
    \begin{itemize}
    \item 初期条件: $u(x,0) = f(x)$
    \item 境界条件: $u(0,t) = u(1,t) = 0$
    \end{itemize}
  \item 時間$t$と位置$x$に関して離散化
    \begin{align*}
      & u_j^n = u(x_j, t_n) \\
      & q_j^n = q(x_j, t_n) \\
      & t_0 = 0, t_1=\Delta t, t_2=2 \Delta t, \cdots, t_n=n \Delta t, \cdots \\
      & x_0 = 0, x_1=\Delta x, x_2=2 \Delta x, \cdots, x_N=N \Delta x = 1 \qquad (\Delta x = 1/N)
    \end{align*}
  \end{itemize}
\end{frame}

\begin{frame}[t]{有限差分法}
  \begin{itemize}
  \item $t$に関して前進差分を考える
    \[
    \frac{\partial u}{\partial t} \Big|_{(j \Delta x, n \Delta t)} = \frac{u_j^{n+1} - u_j^n}{\Delta t} + {\cal O}(\Delta t)
    \]
  \item $x$に関しては中心差分を考える
    \[
    \frac{\partial^2 u}{\partial x^2} \Big|_{(j \Delta x, n \Delta t)} = \frac{u_{j+1}^{n} - 2 u_{j}^{n} + u_{j-1}^{n}}{\Delta x^2} + {\cal O}(\Delta x^2)
    \]
  \item 拡散方程式に代入して整理すると
    \[
    u_{j}^{n+1} = u_{j}^{n} + r (u_{j+1}^{n} - 2 u_{j}^{n} + u_{j-1}^{n}) + \Delta t q_{j}^{n} \qquad (r = D\frac{\Delta t}{\Delta x^2})
    \]
  \item FTCS (Forward-Time Centered Space)法
  \end{itemize}
\end{frame}

\begin{frame}[t]{FTCS法}
  \begin{itemize}
  \item $O(\Delta t) + O(\Delta x^2)$の陽解法
    \begin{center}
      \resizebox{0.4\textwidth}{!}{\includegraphics{image/ftcs-1.pdf}}
    \end{center}
  \item 初期条件
    \[
    u_j^0 = f(j\Delta x) \ \ (j=0,1,\cdots,N)
    \]
  \item 境界条件
    \[
    u_0^n = u_N^n = 0 \ \ (n=0,1,2,\cdots)
    \]
  \end{itemize}
\end{frame}

\begin{frame}[t]{有限差分法の安定性}
  \begin{itemize}
  \item (陽的)有限差分法においては、$\Delta t$、$\Delta x$は小さければ小さいほどよいというわけではない
  \item 一次元拡散方程式の場合
    \begin{align*}
      \begin{cases}
        r \le 1/2 & \text{安定} \\
        r > 1/2 & \text{\color{red}不安定}
      \end{cases}
    \end{align*}
  \item $\Delta x$を半分にしたら、$\Delta t$は1/4にしなければならない

    $\Rightarrow$ 計算量は8倍
  \end{itemize}
\end{frame}

\begin{frame}[t]{Von Neumannの安定性解析}
  \begin{itemize}
  \item $u(x,t)$のフーリエ変換を導入する
    \begin{align*}
    u(x,t) &= \sum_k v(k,t) e^{ikx} \\
    u_j^n &= \sum_k v_k^n e^{ik\Delta x \cdot j}
    \end{align*}
  \item FTCS法の式に代入し、$k$の項を取り出すと
    \[
    v_{k}^{n+1} =  (1 + 2r(\cos k \Delta x - 1)) v_{k}^{n}
    \]
  \item 全ての$k$に対し発散しない(=係数の絶対値が1以下になる)ためには
    \[
      | 1 - 4r | \le 1 \qquad \text{すなわち} \ 0 < r \le \frac{1}{2}
    \]
  \end{itemize}
\end{frame}

\begin{frame}[t]{一次元波動方程式(双極型)}
  \begin{itemize}
  \item 一次元波動方程式
    \[
    \frac{\partial^2 u}{\partial t^2} = c^2 \frac{\partial^2 u}{\partial x^2} \qquad u(x,0)=f(x), \frac{\partial u}{\partial t} (x,0) = g(x)
    \]
  \item $t$に関する中心差分
    \[
    \frac{\partial^2 u}{\partial t^2} \Big|_{(j \Delta x, n \Delta t)} = \frac{u_{j}^{n+1} - 2 u_{j}^{n} + u_{j}^{n-1}}{\Delta t^2} + {\cal O}(\Delta t^2)
    \]
  \item 代入して整理すると
    \[
    u_{j}^{n+1} = 2u_{j}^{n} - u_{j}^{n-1} + \alpha^2 (u_{j+1}^{n} - 2 u_{j}^{n} + u_{j-1}^{n}) \qquad (\alpha = c\frac{\Delta t}{\Delta x})
    \]
  \item 解が安定であるための条件
    \[
    \alpha = c\frac{\Delta t}{\Delta x} \le 1
    \]
  \end{itemize}
\end{frame}

\begin{frame}[t]{波動方程式に対するFTCS法}
  \begin{itemize}
  \item $O(\Delta t^2) + O(\Delta x^2)$の陽解法
    \begin{center}
      \resizebox{0.4\textwidth}{!}{\includegraphics{image/ftcs-2.pdf}}
    \end{center}
  \item 初期条件
    \[
    u_j^0 = f(j\Delta x) \ \ (j=0,1,\cdots,N)
    \]
    初期速度については$n=0$に関する中心差分を考えて
    \[
    \frac{u_j^1 - u_j^{-1}}{2 \Delta t} = g_j \ \ \Rightarrow \ \ u_j^1 = u_j^0 + \Delta t g_j + \frac{\alpha^2}{2} (u_{j+1}^{n} - 2 u_{j}^{n} + u_{j-1}^{n})
    \]
  \end{itemize}
\end{frame}

\begin{frame}[t]{波動方程式の安定性解析}
  \begin{itemize}
  \item 空間方向にフーリエ変換し、時間方向には指数関数型の解を仮定する
    \begin{align*}
    u_j^n &= \sum_k (\mu_k)^n e^{ik\Delta x \cdot j}
    \end{align*}
  \item FTCS法の式に代入し、波数$k$の項を取り出すと
    \begin{align*}
    \mu_k^2 =  2\mu_k - 1 + 2\alpha^2(\cos k \Delta x - 1) \mu_k \\
    \mu_k^2 - 2 [1 + \alpha^2(\cos k \Delta x - 1)] \mu_k + 1 = 0
    \end{align*}
  \item 二つの解の積は1なので、互いに異なる2つの実数解が存在する場合、解の一つは必ず1より大きくなり、波動方程式の解は発散する
  \item 全ての波数\(k\)について重解あるいは複素解となるためには
    \[
      | 1 - 2 \alpha^2 | \le 1 \qquad \text{すなわち} \ 0 < \alpha \le 1
    \]
  \end{itemize}
\end{frame}

\begin{frame}[t]{時間に依存するシュレディンガー方程式}
  \begin{itemize}
  \item 時間に依存するシュレディンガー方程式
    \[
    i \hbar \frac{\partial \Psi}{\partial t}(x,t) = H(x,t) \Psi(x,t) = \Big[ - \frac{\hbar^2}{2m} \frac{\partial^2}{\partial x^2} + V(x,t) \Big] \Psi(x,t)
    \]
    \begin{itemize}
    \item 波動関数のノルム $\displaystyle \int | \Psi(x,t) |^2 \, dx$ は保存
    \item $V(x,t)$が時間$t$に依存しない場合、エネルギーの期待値は保存
      \[
      \langle H \rangle = \frac{\displaystyle \int \Psi^* H \Psi \, dx}{\displaystyle \int | \Psi |^2 \, dx}
      \]
    \end{itemize}
  \item 以下では無次元化して$\hbar = m = 1$とおく
  \end{itemize}
\end{frame}

\begin{frame}[t]{時間に依存するシュレディンガー方程式}
  \begin{itemize}
  \item シュレディンガー方程式の形式解 ($V$が時間に依存しない場合)
    \[
    \Psi(x,t) = e^{-i H t} \Psi(x,0)
    \]
  \item $t$に関して前進差分
    \begin{align*}
      e^{-i H t} &= [e^{-i H \Delta t}]^M \approx [1 - i H \Delta t]^M \qquad (\Delta t = t / M) \\
      \Psi^{n+1} &= (1 -  i \Delta t H) \Psi^{n}
    \end{align*}
  \item $H$は対称(エルミート)行列 $\Rightarrow$ 時間発展演算子$e^{-i H \Delta t}$はユニタリー行列
  \item 差分近似$(1 -  i \Delta t H)$はユニタリーではない
    \begin{align*}
      & (e^{-i H \Delta t})^\dagger e^{-i H \Delta t} = e^{i H \Delta t} e^{-i H \Delta t} = 1 \\
      & (1 -  i \Delta t H)^\dagger (1 -  i \Delta t H) = (1 +  i \Delta t H) (1 -  i \Delta t H) = 1 + {\color{red} \Delta t^2 H^2}
    \end{align*}
  \end{itemize}
\end{frame}

\begin{frame}[t]{クランク・ニコルソン法}
  \begin{itemize}
  \item クランク・ニコルソン法
    \[
    \Psi^{n+1} = \frac{1 -  i \frac{\Delta t}{2} H}{1 +  i \frac{\Delta t}{2} H} \Psi^{n}
    \]
  \item (数値精度の範囲で)ユニタリー行列であるので、ノルムは保存
  \item $(1 +  i \frac{\Delta t}{2} H)^{-1}$を掛ける $\Rightarrow$ 連立一次方程式を解く必要がある
    \begin{itemize}
    \item まず、$\Psi = (1 - i \frac{\Delta t}{2} H) \Psi^n$ を計算
    \item 次に、$(1 +  i \frac{\Delta t}{2} H) \Psi^{n+1} = \Psi$ を解く(連立一次方程式)
    \end{itemize}
  \item 陰解法の一種
  \end{itemize}
\end{frame}

\begin{frame}[t]{拡散方程式に対する陰解法}
  \begin{itemize}
  \item 時刻$t$関して後退差分を使う
    \[
    \frac{\partial u}{\partial t} \Big|_{(j \Delta x, n \Delta t)} = \frac{u_j^{n} - u_j^{n-1}}{\Delta t} + {\cal O}(\Delta t)
    \]
  \item $x$に関する中心差分と組み合わせ、$n \rightarrow n+1$と書き直すと
    \[
    u_{j}^{n+1} = u_{j}^{n} + r (u_{j+1}^{n+1} - 2 u_{j}^{n+1} + u_{j-1}^{n+1})
    \]
    $u^{n+1}$が両辺に現れる $\Rightarrow$ 陰解法
  \item $O(\Delta t) + O(\Delta x^2)$
  \item $r$の値によらず{\color{red}常に安定}
  \end{itemize}
\end{frame}

\begin{frame}[t]{拡散方程式に対するクランク・ニコルソン法}
  \begin{itemize}
  \item さらに、時間方向にきざみ幅$\Delta t/2$の中心差分を使うと
    \[
    \frac{\partial u}{\partial t} \Big|_{(j \Delta x, n \Delta t)} = \frac{u_j^{n+\frac{1}{2}} - u_j^{n-\frac{1}{2}}}{\Delta t} + {\cal O}(\Delta t^2)
    \]
  \item $x$に関する中心差分と組み合わせ、$n \rightarrow n+\frac{1}{2}$し、さらに$u_j^{n+\frac{1}{2}}$を$(u_j^{n+1}+u_j^{n})/2$で近似すると
    \[
    u_{j}^{n+1} = u_{j}^{n} + \frac{r}{2} (u_{j+1}^{n+1} - 2 u_{j}^{n+1}  +u_{j-1}^{n+1} + u_{j+1}^{n} - 2 u_{j}^{n} + u_{j-1}^{n})
    \]
    あるいは
    \[
    u_{j}^{n+1} - \frac{r}{2} (u_{j+1}^{n+1} - 2 u_{j}^{n+1} + u_{j-1}^{n+1}) = u_{j}^{n} + \frac{r}{2} (u_{j+1}^{n} - 2 u_{j}^{n} + u_{j-1}^{n})
    \]
    $\Rightarrow$ クランク・ニコルソン法 [$O(\Delta t^2) + O(\Delta x^2)$]
  \end{itemize}
\end{frame}


% \section{対角化 (復習)}
% \begin{frame}[t,fragile]{シュレディンガー方程式の行列表示}
  \begin{itemize}
    %\setlength{\itemsep}{1em}
  \item シュレディンガー方程式
    \[
    [-\frac{d^2}{dx^2}+V(x)]\psi(x) = E \psi(x)
    \]
  \item 連立差分方程式を行列の形で表す($\psi(x_0)=\psi(x_n)=0$)
    \[\begin{small}\hspace*{-4em}
    \begin{pmatrix}
      \frac{2}{h^2}+V(x_1) & -\frac{1}{h^2} \\
      -\frac{1}{h^2} & \frac{2}{h^2}+V(x_2) & -\frac{1}{h^2} \\
      & -\frac{1}{h^2} & \frac{2}{h^2}+V(x_3) & -\frac{1}{h^2} \\
      & & \ddots & \ddots \\
      & & & -\frac{1}{h^2} & \frac{2}{h^2}+V(x_{n-1}) \\
    \end{pmatrix}
    \begin{pmatrix}
      \psi(x_1) \\
      \psi(x_2) \\
      \psi(x_3) \\
      \vdots \\
      \psi(x_{n-1}) \\
    \end{pmatrix}
    = \cdots % E
    %% \begin{pmatrix}
    %%   \psi(x_1) \\
    %%   \psi(x_2) \\
    %%   \psi(x_3) \\
    %%   \vdots \\
    %%   \psi(x_{n-1}) \\
    %% \end{pmatrix}
    \end{small}
    \]
  \item $(n-1) \times (n-1)$の疎行列の固有値問題
    \begin{itemize}
    \item 固有値: 固有エネルギー
    \item 固有ベクトル: 波動関数
    \end{itemize}
  \end{itemize}
\end{frame}

% \begin{frame}[t,fragile]{固体物理・量子統計物理に現れる行列}
  \begin{itemize}
    %\setlength{\itemsep}{1em}
  \item 強束縛近似(tight-binding approx.)のもとでの第二量子化表示
    \[
    H = -t \sum_{\langle i,j \rangle \sigma} (c_{i,\sigma}^\dagger c_{j,\sigma} + h.c.) + \text{(相互作用)}
    \]
  \item 局所スピン模型(ハイゼンベルグ模型)
    \[
    H = -J\sum_{\langle i,j \rangle} S_i \cdot S_j
    = -J\sum_{\langle i,j \rangle} [S_i^z S_j^z +\frac{1}{2} (S_i^+ S_j^- + S_i^- S_j^+) ]
    \]
  \item 格子点の数を$n$とすると、ハミルトニアンはそれぞれ$4^n \times 4^n$、$2^n \times 2^n$の(疎)行列で表される。
  \item $n$が大きくなると、行列の次元は指数関数的に増加
  \item 量子多体系に共通する困難
  \end{itemize}
\end{frame}

% \begin{frame}[t,fragile]{行列の数値対角化}
  \begin{itemize}
    %\setlength{\itemsep}{1em}
  \item 一般的に次元が5以上の行列の固有値は、あらかじめ定まる有限回の手続きでは求まらない
    \begin{itemize}
    \item 必ず何らかの反復法(+収束判定)が必要となる
    \end{itemize}
  \item 密行列向きの方法
    \begin{itemize}
    \item Jacobi法
    \item Givens変換・Householder法(三重対角化) + QR法など
    \end{itemize}
  \item 疎行列向きの方法
    \begin{itemize}
    \item べき乗法
    \item Lanczos法(三重対角化) + QR法など
    \end{itemize}
  \item 固有ベクトル
    \begin{itemize}
    \item QR法で求めたものを逆変換
    \item 逆反復法で精度改善
    \end{itemize}
  \end{itemize}
\end{frame}


\section{横磁場イジング模型}

\begin{frame}[t,fragile]{横磁場イジング模型}
  \begin{itemize}
    %\setlength{\itemsep}{1em}
  \item ハミルトニアン($2^N \times 2^N$行列)
    \[
      H = H_z + H_x = - J \sum_{\langle i,j \rangle} \sigma_i^z \sigma_j^z - h \sum_i \sigma_i^z - \Gamma \sum_i \sigma_i^x
    \]
  \item $\sigma_i^x$、$\sigma_i^z$: パウリ行列($2 \times 2$行列)
    \begin{align*}
      \big(\sigma_i^z\big)^2 &= \big(\sigma_i^x\big)^2 = I \\
      [ \sigma_i^z, \sigma_i^x ] &\ne 0
    \end{align*}
  \item $J$: スピン間の相互作用($J>0$: 強磁性、$J<0$: 反強磁性)
  \item $h$: 縦磁場(準位間のエネルギー差$=2h$)
  \item $\Gamma$: 横磁場(トンネリング)
  \item 以降、$\sigma_i^z$を対角化する基底($|\!\uparrow\rangle_i$, $|\!\downarrow\rangle_i$)で考える
  \end{itemize}
\end{frame}

\begin{frame}[t,fragile]{横磁場イジング模型}
  \begin{itemize}
    %\setlength{\itemsep}{1em}
  \item 2サイト系
    \[
      H = -J \sigma_1^z \sigma_2^z - h (\sigma_1^z + \sigma_2^z) - \Gamma (\sigma_1^x + \sigma_2^x)
    \]
  \item 行列要素
    \begin{align*}
      \langle \uparrow \uparrow \!| H |\! \uparrow \uparrow \rangle &= -J - 2h \\
      \langle \uparrow \uparrow \!| H |\! \uparrow \downarrow \rangle &= -\Gamma \\
      \langle \uparrow \uparrow \!| H |\! \downarrow \downarrow \rangle &= 0 \\
      &\vdots
    \end{align*}
  \end{itemize}
\end{frame}

\begin{frame}[t,fragile]{量子相転移}
  \begin{itemize}
    %\setlength{\itemsep}{1em}
  \item $h=0$の場合
    \begin{itemize}
    \item $\Gamma \rightarrow 0$: $|\!\uparrow\uparrow\cdots\uparrow\rangle$、あるいは$|\!\downarrow\downarrow\cdots\downarrow\rangle$が基底状態(二重縮退)
    \item $J \rightarrow 0$: $\sigma_i^x$の固有状態($|\!\uparrow\rangle_i + |\!\downarrow\rangle_i$)の積が基底状態(全ての状態の重ね合わせ)
    \end{itemize}
  \item 一次元系
    \[
      H = - J \sum_{i} \sigma_i^z \sigma_{i+1}^z - \Gamma \sum_i \sigma_i^x
    \]
    $\Gamma = J$で量子相転移(熱ゆらぎではなく量子ゆらぎによる相転移)  
  \end{itemize}
\end{frame}


\section{多体量子系の時間発展}

\begin{frame}[t,fragile]{横磁場イジング模型の時間発展}
  \begin{itemize}
    %\setlength{\itemsep}{1em}
  \item 時間依存シュレディンガー方程式の形式解
    \[
    \Psi(t) = e^{-iHt} \Psi(0)
    \]
    \begin{itemize}
    \item 有限差分法、クランク・ニコルソン法
    \end{itemize}
  \item 鈴木・トロッター分解 ($\Delta t = t / M$)
    \begin{align*}
      e^{-iHt} &= \big[ e^{-iH\Delta t} \big]^M \approx \big[ e^{-iH_z\Delta t} e^{-iH_x\Delta t} \big]^M \\
      &= \big[ e^{i\Delta t J \sum_{\langle i,j \rangle} \sigma_i^z \sigma_j^z} e^{i\Gamma \sigma_1^x\Delta t} e^{i\Gamma \sigma_2^x\Delta t} \cdots e^{i\Gamma \sigma_N^x\Delta t} \big]^M
    \end{align*}
    (さらに$e^{-iH\Delta t} \approx e^{-iH_z\Delta t/2} e^{-iH_x\Delta t} e^{-iH_z\Delta t/2}$と対称に分解すると近似の次数が上がる)
  \item $[\sigma_i^x]^2 = I$より
    \[
    e^{i\Gamma \sigma_i^x\Delta t} = \cos (\Gamma\Delta t) + i \sigma_i^x \sin (\Gamma\Delta t)
    \]
  \end{itemize}
\end{frame}

\begin{frame}[t,fragile]{量子アニーリング}
  \begin{itemize}
    %\setlength{\itemsep}{1em}
  \item 離散最適化問題
    \[
    H = -J \sum_{i<j} \epsilon_{ij} \sigma_i^z \sigma_j^z
    \]
    の基底状態配位と基底状態エネルギーを求めたい
    \begin{center}
      \resizebox{0.6\textwidth}{!}{\includegraphics{image/spinglass.pdf}}
    \end{center}
  \end{itemize}
\end{frame}

\begin{frame}[t,fragile]{量子アニーリング}
  \begin{itemize}
    %\setlength{\itemsep}{1em}
  \item 横磁場を導入
    \[
    H = -J \sum_{i<j} \epsilon_{ij} \sigma_i^z \sigma_j^z - \Gamma \sum_i \sigma_i^x
    \]
  \item 古典極限 ($J=1$, $\Gamma=0$)
    \begin{itemize}
    \item 求めたい基底状態
    \end{itemize}
  \item 量子極限 ($J=0$, $\Gamma=1$)
    \begin{itemize}
    \item $2^N$個の全ての状態の重ね合わせ
    \end{itemize}
  \item 量子アニーリング
    \begin{itemize}
    \item $J+\Gamma=1$を保ったままで、$\Gamma=1$から$\Gamma=0$まで「ゆっくり」と減少させながら時間発展させる
      \[
      J = 1-\Gamma = t/T \qquad (0 \le t \le T)
      \]
    \item $T \rightarrow \infty$の極限で確率1で基底状態に収束
    \end{itemize}
  \end{itemize}
\end{frame}


\section{量子回路シミュレーション}

\begin{frame}[t,fragile]{量子コンピュータと量子ゲート}
  \begin{itemize}
    %\setlength{\itemsep}{1em}
  \item (ゲート型)量子コンピュータ

    $N$量子ビットに対して、量子ゲートにより状態を操作

    \begin{itemize}
    \item 「量子ビット」= 2準位系($S=1/2$スピン) \ $|0\rangle=|\!\uparrow\rangle, |1\rangle=|\!\downarrow\rangle$
    \item 「量子ゲート」= 少数量子ビットに対するユニタリ変換
    \end{itemize}
  \item 1量子ビットゲート
    \begin{itemize}
    \item Xゲート(量子NOT)
      \[
      X = \begin{pmatrix} 0 & 1 \\ 1 & 0 \end{pmatrix} = \sigma^x
      \]
    \item Rzゲート($z$軸まわりの回転)
      \[
      Rz(\theta) = \begin{pmatrix} e^{-i\theta/2} & 0 \\ 0 & e^{i\theta/2} \end{pmatrix} = e^{-i\theta\sigma^z/2}
      \]
    \item アダマールゲート
      \[
      H = \frac{1}{\sqrt{2}} \begin{pmatrix} 1 & 1 \\ 1 & -1 \end{pmatrix} = \frac{1}{\sqrt{2}} (\sigma^x + \sigma^z)
      \]
    \end{itemize}
  \end{itemize}
\end{frame}

\begin{frame}[t,fragile]{量子ビットゲート}
  \begin{itemize}
    %\setlength{\itemsep}{1em}
  \item 2量子ビットゲート
    \begin{itemize}
    \item CXゲート(制御NOT)
      \[
      CX = \begin{pmatrix} 1 & 0 & 0 & 0 \\ 0 & 1 & 0 & 0 \\ 0 & 0 & 0 & 1 \\ 0 & 0 & 1 & 0 \end{pmatrix}
      \]
    \end{itemize}
  \item 3量子ビットゲート
    \begin{itemize}
    \item CCXゲート(トフォリゲート)
      \[
      CCX = \begin{pmatrix}
        1 & 0 & 0 & 0 & 0 & 0 & 0 & 0 \\
        0 & 1 & 0 & 0 & 0 & 0 & 0 & 0 \\
        0 & 0 & 1 & 0 & 0 & 0 & 0 & 0 \\
        0 & 0 & 0 & 1 & 0 & 0 & 0 & 0 \\
        0 & 0 & 0 & 0 & 1 & 0 & 0 & 0 \\
        0 & 0 & 0 & 0 & 0 & 1 & 0 & 0 \\
        0 & 0 & 0 & 0 & 0 & 0 & 0 & 1 \\
        0 & 0 & 0 & 0 & 0 & 0 & 1 & 0 \end{pmatrix}
      \]
    \end{itemize}
  \end{itemize}
\end{frame}

\begin{frame}[t,fragile]{量子加算器}
  \begin{itemize}
    %\setlength{\itemsep}{1em}
  \item 1量子ビットの加算 $\Rightarrow$ CCXゲートとCXゲートで実現できる
    \begin{center}
      \resizebox{0.5\textwidth}{!}{\includegraphics{image/adder1.pdf}}
    \end{center}
  \item CCXゲートは、Hゲート、Rzゲート、CXゲートの組み合わせで表現できる
  \item 任意のユニタリ変換は、Hゲート、Rzゲート、CXゲートの組み合わせで表現できる (万能量子ゲート)
  \end{itemize}
\end{frame}

\begin{frame}[t,fragile]{量子加算器}
  \begin{itemize}
    %\setlength{\itemsep}{1em}
  \item 3量子ビット加算器
    \begin{center}
      \resizebox{0.6\textwidth}{!}{\includegraphics{image/adder3.pdf}}
      \resizebox{0.35\textwidth}{!}{\includegraphics{image/adder2.pdf}}
    \end{center}
  \end{itemize}
\end{frame}

\begin{frame}[t,fragile]{疎行列ベクトル積による実装}
  \begin{itemize}
    %\setlength{\itemsep}{1em}
    \item 状態ベクトル
      \begin{itemize}
        \item 長さ\(2^N\)の複素ベクトル (\(N\): スピン数、量子ビット数)
      \end{itemize}
    \item 1量子ビット演算(例: \(\sigma^x_2\))
      \begin{itemize}
        \item \(\sigma^x_2\)自体は\(2\times2\)行列
        \item \(2^N\)次元のベクトルに作用するのは、\(2^N\times2^N\)行列
          \[
            {I}_2 \otimes \sigma^x \otimes {I}_2 \otimes \cdots \otimes {I}_2
          \]
        \item 次元は大きいが、非ゼロ要素は各行各列に1個だけ
        \item CXやCCXゲートも同様
      \end{itemize}
    \item 疎行列ベクトル積
      \begin{itemize}
        \item \(2^N\times2^N\)行列を陽に構成するのは非効率
        
          メモリ、計算時間\(\sim 2^{2N}\)
        \item 非ゼロ要素の場所も値も規則的なので、その場で簡単に計算できる
        \item 要素の移る先を計算し、値を入れ替えるだけでよい
        
          メモリ\(=O(1)\)、計算時間\(\sim 2^{N}\)
      \end{itemize}
  
  \end{itemize}
\end{frame}


\section{}
\begin{frame}[t]{講義日程 (予定)}
  \begin{itemize}
    % \setlength{\itemsep}{1em}
  \item 全8回 (金曜5限 16:50-18:35)
    \begin{itemize}
    \item 10月4日(金) 講義1: 多体系の統計力学とモンテカルロ法
    \item {\color{gray} 10月11日(金) 休講 (物理学教室コロキウム)}
    \item 10月18日(金) 実習1
    \item {\color{gray} 10月25日(金) 休講}
    \item 11月1日(金) 講義2: 偏微分方程式と多体系の量子力学
    \item 11月8日(金) 実習2
    \item {\color{gray} 11月15日(金) 休講}
    \item 11月29日(金) 講義3: 少数多体系・分子動力学
    \item 12月6日(金) 実習3
    \item {\color{gray} 12月13日(金) 休講 (物理学教室コロキウム)}
    \item {\color{gray} 12月20日(金) 休講 (ニュートン祭)}
    \item 12月27日(金) 講義4: 最適化問題
    \item 1月10日(金) 実習4
    \item {\color{gray} 1月24日(金) 休講 (物理学教室コロキウム)}
    \end{itemize}
  \end{itemize}
\end{frame}


\end{document}
