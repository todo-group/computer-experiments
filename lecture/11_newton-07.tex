\begin{frame}[t,fragile]{反復計算}
  \begin{itemize}
    %\setlength{\itemsep}{1em}
  \item {\tt while}による反復 (ハンドブック3.2.3節)
\begin{lstlisting}
double residual = 1;    /* 残差 */
double error = 1;       /* 誤差 */
double delta = 1.0e-12; // 欲しい精度
while (residual > delta && error > delta) {
  /* ニュートン法の漸化式 */
  /* residual と error を計算 */
}
\end{lstlisting}
残差と誤差のどちらかが欲しい精度に達したら計算を終了
\item {\tt break}を使う例 (ハンドブック3.2.4節)
\begin{lstlisting}
for (;;) {
  /* ニュートン法の漸化式 */
  /* residual と error を計算 */
  if (residual < delta || error < delta) break;
}
\end{lstlisting}
  \end{itemize}
\end{frame}
