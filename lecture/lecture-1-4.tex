% -*- coding: utf-8 -*-

\documentclass[10pt,dvipdfmx]{beamer}
\usepackage{tutorial}

\title{計算機実験I (講義4)}
\date{2025/06/18}

\begin{document}

\begin{frame}
  \titlepage
  \tableofcontents
  出席: 本日の15時までにUTOLでアンケートに回答
\end{frame}

\section{密行列の対角化}
\input{1_basics/matrix.tex}
\input{4_eigenvalue_problem/intro-01.tex}
%\input{4_eigenvalue_problem/intro-02.tex}
\begin{frame}[t,fragile]{ポアソン方程式の境界値問題}
  \begin{itemize}
    %\setlength{\itemsep}{1em}
  \item ノイマン型境界条件の場合
    \begin{itemize}
    \item 境界上で$u(x,y)$の微分が定義される。
    \item 例) $\partial u(0,y) / \partial x = h(0,y)$
    \end{itemize}
  \item 境界条件を差分近似で表す
    \[
    \frac{u_{1j} - u_{0j}}{h} = h_{0j} \qquad j=1 \cdots (n-1)
    \]
    $(n+1)^2-4$個の未知数に対して、ポアソン方程式の差分近似とあわせて、合計$(n-1)^2+4(n-1)=(n+1)^2-4$個の連立一次方程式
  \item (全ての$u_ij$を定数だけシフトしても方程式は変わらない → 独立な方程式は\((n+1)^2-3\)個 → 追加で1点の値を与える必要がある) \\[2em]
  \item 二次元グリッド上の点$(i,j)$と長さ$(n+1)^2$のベクトルの要素との対応関係をきちんと定義することが大事
  \end{itemize}
\end{frame}

%\input{4_eigenvalue_problem/intro-04.tex}
\input{4_eigenvalue_problem/intro-05.tex}
\input{4_eigenvalue_problem/intro-06.tex}
\input{4_eigenvalue_problem/intro-07.tex}

% \section{密行列の対角化}

\input{4_eigenvalue_problem/dense-01.tex}
\input{4_eigenvalue_problem/dense-02.tex}
\input{4_eigenvalue_problem/dense-03.tex}
\input{4_eigenvalue_problem/dense-04.tex}
\input{4_eigenvalue_problem/dense-05.tex}
\input{4_eigenvalue_problem/dense-09.tex}
%\input{4_eigenvalue_problem/dense-06.tex}
\input{4_eigenvalue_problem/dense-07.tex}

\section{疎行列に対する反復法}
\input{4_eigenvalue_problem/sparse-01.tex}
%\input{4_eigenvalue_problem/intro-04.tex}
\input{4_eigenvalue_problem/sparse-02.tex}
\input{4_eigenvalue_problem/sparse-03.tex}
\input{4_eigenvalue_problem/sparse-04.tex}
\input{4_eigenvalue_problem/sparse-05.tex}
\input{4_eigenvalue_problem/sparse-06.tex}
\input{4_eigenvalue_problem/sparse-07.tex}
\input{4_eigenvalue_problem/sparse-08.tex}
\input{4_eigenvalue_problem/sparse-09.tex}
\input{4_eigenvalue_problem/sparse-10.tex}
\input{4_eigenvalue_problem/sparse-11.tex}

% \input{4_eigenvalue_problem/inverse.tex}
% \input{4_eigenvalue_problem/variational-method.tex}

\input{4_eigenvalue_problem/svd.tex}

\input{5_linear_regression/least-square.tex}

\section{}
\begin{frame}[t]{講義日程(予定)}
  \begin{itemize}
    % \setlength{\itemsep}{1em}
  \item 全8回 (水曜2限 10:25-12:10)
    \begin{itemize}
    \item 4月9日 講義1: 講義の概要・基本的なアルゴリズム
    \item 4月16日 実習1 (グループ1): 環境整備・C言語プログラミング
    \item 4月23日 実習1 (グループ2): 環境整備・C言語プログラミング
    \item 5月7日 講義2: 常微分方程式
    \item 5月14日 実習2 (グループ1): 基本的なアルゴリズム・常微分方程式
    \item 5月21日 実習2 (グループ2): 基本的なアルゴリズム・常微分方程式
    \item 5月28日 講義3: 連立方程式
    \item 6月4日 実習3 (グループ1): 連立方程式
    \item 6月11日 実習3 (グループ2): 連立方程式
    \item 6月18日 講義4: 行列の対角化
    \item 6月25日 実習4 (グループ1): 行列の対角化
    \item 7月2日 実習4 (グループ2): 行列の対角化
    \item {\color{gray} 7月9日 休講}
    \item {\color{gray} 7月16日 休講}
    \end{itemize}
  \item 実習はクラスを2グループに分けて実施(グループ1: 学生証番号が奇数、グループ2: 偶数)
  \end{itemize}
\end{frame}

%% \begin{frame}[t]{もくもく会}
%%   \begin{itemize}
%%     % \setlength{\itemsep}{1em}
%%   \item 出席は自由
%%   \item 90分間、各自テーマを決めてもくもくと作業する
%%     \begin{itemize}
%%     \item Zoomに接続しっぱなしにする (できればビデオONで)
%%     \item 最初にZoomチャットでその時間に自分がやることを宣言
%%     \end{itemize}
%%   \item 質問自由
%%     \begin{itemize}
%%     \item ZoomでマイクをONにして / Zoom チャット / Slack
%%     \end{itemize}
%%   \item 作業内容の例
%%     \begin{itemize}
%%     \item C言語をマスターする(計算機ハンドブックの例を端から試す)
%%     \item レポート課題のどれかに取り組む
%%     \item 自分のプログラムを10倍速くする、等
%%     \end{itemize}
%%   \end{itemize}
%% \end{frame}


% \section{今後について}

\begin{frame}[t,fragile]{計算機環境}
  \begin{itemize}
    %\setlength{\itemsep}{1em}
  \item 教育用計算機システム
  \item 知の物理学クラスタ(ai.phys.s.u-tokyo.ac.jp)
    \begin{itemize}
    \item 卒業まで利用可 (希望すれば大学院でも)
    \item バッチシステムを使えば、かなり大規模な計算も可能
    \end{itemize}
  \item 計算機利用・シミュレーションに関する質問は今後も歓迎
    \begin{itemize}
      \item Slack
      \item メール {\tt computer@phys.s.u-tokyo.ac.jp}
      \item X (Twitter)...
    \end{itemize}
  \end{itemize}
\end{frame}

\begin{frame}[t,fragile]{計算機実験II}
  \begin{itemize}
    %\setlength{\itemsep}{1em}
  \item 3年冬学期、金曜5限
  \item 全8回程度
  \item 計算機実験Iで身に付けた知識をもとに、より高度な数値計算手法・アルゴリズムを学び、物理学における具体的な問題への応用を通して実践的な知識と経験を身につける。
    \begin{itemize}
      \item モンテカルロ法 (乱択アルゴリズム、モンテカルロ積分、マルコフ連鎖モンテカルロ)
      \item 偏微分方程式の初期値問題
      \item 多体系の量子力学 (横磁場イジング模型、時間発展、量子コンピュータ)
      \item 少数多体系・分子動力学
      \item 最適化とその応用 (共役勾配法、シミュレーテッドアニーリング、他)
      \item $\cdots$
    \end{itemize}
  \end{itemize}
\end{frame}

\end{document}
