% -*- coding: utf-8 -*-

\documentclass[10pt,dvipdfmx]{beamer}
\usepackage{tutorial}

\title{計算機実験I (第4回)}
\date{2020/05/20}

\begin{document}

\begin{frame}
  \titlepage
  \tableofcontents
\end{frame}

\begin{frame}[t]{本日の課題}
  \begin{itemize}
    %\setlength{\itemsep}{1em}
  \item まだ完了していない人は「\href{https://utphys-comp.github.io}{計算機実験のための環境整備}」({\small \href{https://utphys-comp.github.io}{https://utphys-comp.github.io}})を参考に、必要な環境を引き続き整備
  \item 実習
    \begin{itemize}
    \item 講義資料の中の {\tt random.c}, {\tt complex.c}, {\tt pointer.c}, {\tt pointer-matrix.c}, {\tt matrix-example.c}, {\tt dlange.c}, {\tt dgesv.c}をコンパイル・実行。ソースコードの中身を確認 \\
      サンプルコード一式: \href{https://github.com/todo-group/ComputerExperiments/releases/tag/2020s-computer1}{example-1-4.zip}
    \item 実習課題一覧\href{https://github.com/todo-group/ComputerExperiments/releases/tag/2020s-computer1}{exercise-1.pdf}の課題1〜10, 16, 17, 23-25の中から適宜選び実習
    \end{itemize}
  \item レポートNo.1: 提出締切2020/5/29(金) 18:00 (レポート内容・提出方法についてはITC-LMSを参照のこと)
  \item 質問は、Slackの「\#5 対角化」あるいは他の適当と思われるチャンネルで
  \end{itemize}
\end{frame}

\section{行列演算}
% -*- coding: utf-8 -*-

\documentclass[10pt,dvipdfmx]{beamer}
\usepackage{tutorial}

\begin{document}
\section{行列とLAPACK}
\begin{frame}[t,fragile]{二次元配列}
  \begin{itemize}
    \setlength{\itemsep}{1em}
  \item C言語では、二次元配列は一次元配列の先頭をさす(ポインタ)の配列として表される(と理解しておけば良い)
  \item \verb+a[i]+は、要素\verb+a[i][0]+を指すポインタ
    \begin{itemize}
    \item \verb+a+ と \verb+&a[0]+ は等価 (\verb+&a[0][0]+ ではない)
    \item \verb+a[0]+ と \verb+&a[0][0]+ は等価
    \item \verb+a[2]+ と \verb+&a[2][0]+ は等価
    \item \verb^(a+2)^ と \verb^&a[2]^ は等価
    \item \verb^(*(a+2))[3]^ と \verb^*(*(a+2)+3)^ と \verb^a[2][3]^ は等価
    \item \verb^*(a+2)[3]^ と \verb^*((a+2)[3])^ と \verb^*(a[5])^ と\verb^a[5][0]^ は等価
    \item \verb^[]^は\verb^*^よりも強い
    \end{itemize}
  \item ポインタ確認プログラム: \href{https://github.com/todo-group/computer-experiments/blob/master/exercise/matrix/pointer-matrix.c}{pointer-matrix.c}
  \end{itemize}
\end{frame}

\begin{frame}[t,fragile]{動的二次元配列の確保}
  \begin{itemize}
    \setlength{\itemsep}{1em}
  \item 各行を表す配列とそれぞれの先頭アドレスを保持する配列の二種類が必要
\begin{lstlisting}
double **a;
m = 10;  
n = 10;  
a = (double**)malloc((size_t)(m * sizeof(double*));
for (int i = 0; i < m; ++i)
  a[i] = (double*)malloc((size_t)(n * sizeof(double));
\end{lstlisting}
\item 各行を保持する配列が、メモリ上で連続に確保される保証はない
\item 行列用のライブラリ(LAPACK等)を使うときに問題となる
  \end{itemize}
\end{frame}

\begin{frame}[t,fragile]{BLASライブラリ}
  \begin{itemize}
    \setlength{\itemsep}{1em}
  \item 行列・行列積、行列・ベクトル積などを高速に行う最適化された関数群
  \item 行列・行列積を計算するサブルーチン {\tt dgemm} \\
    \url{http://www.netlib.org/lapack/explore-html/d7/d2b/dgemm_8f.html}
    \begin{itemize}
    \item $C = \alpha A \times B + \beta C$ を計算
    \item BLASもFortranで書かれている
    \end{itemize}
  \item 例: \href{https://github.com/todo-group/computer-experiments/blob/master/exercise/matrix/multiply.c}{multiply.c}, \href{https://github.com/todo-group/computer-experiments/blob/master/exercise/matrix/multiply_dgemm.c}{multiply\_dgemm.c}
  \end{itemize}
\end{frame}

\begin{frame}[t,fragile]{LAPACK (Linear Algebra PACKage)}
  \begin{itemize}
    %\setlength{\itemsep}{1em}
  \item 線形計算のための高品質な数値計算ライブラリ
    \begin{itemize}
    \item \url{http://www.netlib.org/lapack}
    \item 線形方程式、固有値問題、特異値問題、線形最小二乗問題など
    \item (FFT 高速フーリエ変換は入っていない)
    \item LAPACK自体はFortranで書かれている
    \end{itemize}
  \item ほぼ全てのPC、ワークステーション、スーパーコンピュータで利用可 (インストール済)
  \item Netlibでソースが公開されているリファレンス実装は遅いが、それぞれのベンダー(Intel、Fujitsu、etc)による最適化されたLAPACKが用意されている場合が多い(MKL、SSL2、etc)
  \item LAPACKを使うことにより、高速で信頼性が高く、ポータブルなコードを書くことが可能になる
  \end{itemize}
\end{frame}

\begin{frame}[t,fragile]{LAPACKによる連立一次方程式の求解}
  \begin{itemize}
    \setlength{\itemsep}{1em}
  \item LU分解を行うサブルーチン {\tt dgetrf} \\
    \url{http://www.netlib.org/lapack/explore-html/d3/d6a/dgetrf_8f.html}
  \item Fortranによる関数宣言
\begin{lstlisting}
subroutine dgetrf(integer M, integer N,
         double precision, dimension(lda, *) A,
         integer LDA, integer, dimension(*) IPIV,
         integer INFO)
\end{lstlisting}
\item {\tt A}: 左辺の行列、{\tt M,N}: 次元、{\tt IPIV}: 選択されたピボット行のリスト、{\tt lda}: 通常{\tt M} (行数)と同じで良い
  \end{itemize}
\end{frame}

\begin{frame}[t,fragile]{CからBLAS/LAPACKを呼び出す際の注意事項}
  \begin{itemize}
    %\setlength{\itemsep}{1em}
  \item (もともとFortran言語で書かれていたことによる制限)
  \item 関数名はすべて小文字、最後に \verb+_+ (下線)を付ける
  \item スカラー、ベクトル、行列は全て「ポインタ渡し」とする
  \item ベクトルや行列は最初の要素へのポインタを渡す (サイズは別に渡す)
  \item 行列の要素は(0,0) $\rightarrow$ (1,0) $\rightarrow$ (2,0) $\rightarrow\cdots\rightarrow$ $(m-1,0)$ $\rightarrow$ (0,1) $\rightarrow$ (1,1) $\rightarrow\cdots\rightarrow$ $(m-1,n-1)$の順で連続して並んでいなければならない(column-major)
    \begin{itemize}
    \item C言語の二次元配列では \verb+a[i][j]+ の次には \verb%a[i][j+1]%が入っている(row-major)
    \item 行列が転置されて解釈されてしまう!
    \end{itemize}
  \item コンパイル時には{\tt -llapack -lblas}オプションを指定し、LAPACKライブラリとBLASライブラリをリンクする(ハンドブック2.1.6節)
  \end{itemize}
\end{frame}

\begin{frame}[t,fragile]{cmatrix.hライブラリ}
  \begin{itemize}
    %\setlength{\itemsep}{1em}
  \item Column-major形式の二次元配列の確保({\tt alloc\_dmatrix})、開放({\tt free\_dmatrix})、出力({\tt print\_dmatrix})、読み込み({\tt read\_dmatrix})を行うためのユーティリティ関数、(i,j)成分にアクセスするためのマクロ({\tt mat\_elem})他を準備
  \item ソースコード: \href{https://github.com/todo-group/computer-experiments/blob/master/exercise/include/cmatrix.h}{cmatrix.h}
  \item 使用例
\begin{lstlisting}
#include "cmatrix.h"
...
double **mat;
mat = alloc_dmatrix(m, n);
mat_elem(mat, 1, 3) = 5.0;
...
free_dmatrix(mat);
\end{lstlisting}
  \item サンプルコード: \href{https://github.com/todo-group/computer-experiments/blob/master/exercise/matrix/matrix_example.c}{matrix\_example.c}
  \end{itemize}
\end{frame}

\begin{frame}[t,fragile]{alloc\_dmatrixでの動的二次元配列の確保}
  \begin{itemize}
    %\setlength{\itemsep}{1em}
  \item 長さ$m \times n$の一次元配列を用意し、各列(それぞれ$m$要素)の先頭アドレスを長さ$n$のポインター配列に格納する
\begin{lstlisting}
double **a;
m = 10;  
n = 10;  
a = (double**)malloc((size_t)(n * sizeof(double*));
a[0] = (double*)malloc((size_t)(m*n * sizeof(double));
for (int i = 1; i < n; ++i)
  a[i] = a[i-1] + m;
\end{lstlisting}
\item 行列の(i,j)成分を\verb+a[j][i]+に格納することにする (column-major)
  \end{itemize}
\end{frame}

\begin{frame}[t,fragile]{要素アクセス・先頭アドレス}
  \begin{itemize}
    % \setlength{\itemsep}{1em}
  \item 行列の(i,j)成分を\verb+a[j][i]+に格納することにする(column-major)
    \begin{itemize}
      \item \href{https://github.com/todo-group/computer-experiments/blob/master/exercise/matrix/cmatrix.h}{cmatrix.h}ではマクロ(\verb+mat_elem+)を準備
\begin{lstlisting}
#define mat_elem(mat, i, j) (mat)[j][i]
\end{lstlisting}
\item このマクロを使うと、例えば(i,j)成分への代入は以下のように書ける
\begin{lstlisting}
mat_elem(a, i, j) = 1;
\end{lstlisting}
\end{itemize}
  \item LAPACKにベクトルや行列の最初の要素へのポインタを渡す
    \begin{itemize}
      \item ベクトルの最初の要素(0)へのポインタ: \verb+&v[0]+
      \item 行列の最初の要素(0,0)へのポインタ: \verb+&a[0][0]+
      \item \href{https://github.com/todo-group/computer-experiments/blob/master/exercise/matrix/cmatrix.h}{cmatrix.h}にマクロ({\tt vec\_ptr}、{\tt mat\_ptr})が準備されているのでそれぞれ、{\tt vec\_ptr(v)}、{\tt mat\_ptr(a)}と書ける
    \end{itemize}
  \end{itemize}
\end{frame}

\begin{frame}[t,fragile]{LAPACKによる連立一次方程式の求解}
  \begin{itemize}
    \setlength{\itemsep}{1em}
  \item C言語から呼び出すための関数宣言を作成 (ハンドブック3.6.4節)
\begin{lstlisting}
void dgetrf_(int *M, int *N, double *A,
             int *LDA, int*IPIV, int *INFO);
\end{lstlisting}
関数名は全て小文字。関数名の最後に {\tt \_} (下線)を付ける
\item LU分解の例
\begin{lstlisting}
m = 10;
n = 10;
a = alloc_dmatrix(m, n);
...
dgetrf_(&m, &n, mat_ptr(a), &m, vec_ptr(ipiv), &info);
\end{lstlisting}
完全なソースコード: \href{https://github.com/todo-group/computer-experiments/blob/master/exercise/linear_system/lu_decomp.c}{lu\_decomp.c}
  \end{itemize}
\end{frame}

\end{document}


\section{疑似乱数}
%-*- coding:utf-8 -*-

\begin{frame}[t,fragile]{Box-Muller法}
  \begin{itemize}
    %\setlength{\itemsep}{1em}
  \item 一様分布乱数から正規分布乱数を生成する方法
    \begin{itemize}
    \item 2次元の(標準)ガウス分布を考える
      \[
      f(x,y)\,dx\,dy= \frac{1}{2\pi} e^{-(x^2+y^2)/2} \,dx\,dy
      \]
    \item 極座標$(r,\theta)$に変換 ($x=r\cos\theta$, $y=r\sin\theta$)
      \[
      \frac{1}{2\pi} e^{-(x^2+y^2)/2} \,dx\,dy = \frac{1}{2\pi} r \, e^{-r^2/2} \,dr\,d\theta
      \]
    \item $\theta$は$(0,2\pi)$の一様分布
    \item $r$は$f(r) = r \, e^{-r^2/2}$に従う
      \begin{align*}
        F(r) = \int_0^r f(r) \, dr = 1 - e^{-r^2/2}, \qquad r = F^{-1}(q) = \sqrt{- 2 \log(1-q)}
      \end{align*}
    \item 二つの一様乱数から二つの独立な正規分布乱数が生成される
    \end{itemize}
  \end{itemize}
\end{frame}


\section{複素数}
\begin{frame}[t,fragile]{複素数}
  \begin{itemize}
    %\setlength{\itemsep}{1em}
  \item {\tt complex.h}をincludeすることで、double complex 型 (float complex 型)が使えるようになる
  \item 複素数の値は、{\tt CMPLX} (あるいは{\tt CMPLXF})で設定する
  \item {\tt cexp}, {\tt csin}, {\tt clog}等の初等関数が使える
  \item 実部は{\tt creal}, 虚部は{\tt cimag}で取り出せる
  \item プログラム例: \href{https://github.com/todo-group/computer-experiments/blob/master/exercise/basics/complex.c}{complex.c}
\begin{lstlisting}
#include <complex.h>
...
double complex x, y;
x = CMPLX(0, 1); /* 虚数単位 */
y = cexp(x * M_PI);
printf("i = (%lf,%lf)\n", creal(x), cimag(x));
printf("e^{i*pi} = (%lf,%lf)\n", creal(y), cimag(y));
\end{lstlisting}
  \end{itemize}
\end{frame}


\section{C言語における行列・LAPACKの利用}

\begin{frame}[t,fragile]{C言語におけるポインタ}
  \begin{itemize}
    \setlength{\itemsep}{1em}
  \item 変数はメモリ上のどこかに格納されている
    \begin{itemize}
    \item 変数の値: メモリに格納されている数値
    \item アドレス: 変数の値が格納されているメモリ上の番地
    \end{itemize}
  \item ポインタ変数
    \begin{itemize}
    \item 値としてアドレスを格納する変数のこと
    \item ポインタ変数の値(アドレス)とポインタ変数のアドレスは異なるものであることに注意
    \end{itemize}
  \item ポインタ変数の宣言、代入、実体へのアクセス
    \begin{itemize}
    \item 整数型ポインタ変数の宣言: {\color{red} \verb+int *p;+}
    \item 整数型変数の宣言: \verb+int q;+
    \item 変数\verb+q+のアドレスをポインタ変数\verb+p+に代入: {\color{red} \verb+p = &q;+}
    \item ポインタ変数\verb+p+に格納されているアドレスに格納されている値の参照(間接参照): {\color{red} \verb+*p+}
    \end{itemize}
  \end{itemize}
\end{frame}

\begin{frame}[t,fragile]{ポインタの例(1)}
  \begin{itemize}
    %\setlength{\itemsep}{1em}
  \item 例2.5.1 (ハンドブック2.5節)
\begin{lstlisting}
#include <stdio.h>
int main() {
  int q = 200;
  int* p = &q;
  printf("q is %d and *p is %d.\n", q, *p);
  return 0;
}
\end{lstlisting}
\begin{itemize}
\item \verb+q+のアドレスを\verb+p+に代入
\item \verb+q+と\verb+*p+の値を出力 → 両者とも200
\end{itemize}
  \end{itemize}
\end{frame}

\begin{frame}[t,fragile]{ポインタの例(2)}
  \begin{itemize}
    %\setlength{\itemsep}{1em}
  \item 例2.5.2 (ハンドブック2.5節)
\begin{lstlisting}
#include <stdio.h>
int main() {
  int q;
  int* p = &q;
  *p = 300;
  printf("q is %d and *p is %d.\n", q, *p);
  return 0;
}
\end{lstlisting}
\begin{itemize}
\item \verb+q+のアドレスを\verb+p+に代入
\item \verb+*p+に300を代入 (ここで\verb+q=300;+と書いても等価)
\item \verb+q+と\verb+*p+の値を出力 → 両者とも300
\end{itemize}
  \end{itemize}
\end{frame}

\begin{frame}[t,fragile]{関数呼び出し(ポインタ渡し)}
  \begin{itemize}
    %\setlength{\itemsep}{1em}
  \item 例2.6.4 (ハンドブック2.6節)
\begin{lstlisting}
#include <stdio.h>
void division(int divident, int divisor, int *quotient,
              int *residual) {
  *quotient = divident / divisor;
  *residual = divident % divisor;
}
int main() {
  int josuu = 3;
  int hi_josuu = 13;
  int shou, amari;
  division(hi_josuu, josuu, &shou, &amari);
  printf("%d / %d = %d ... %d\n", hi_josuu, josuu,
         shou, amari);
}
\end{lstlisting}
  \end{itemize}
\end{frame}

\begin{frame}[t,fragile]{間違った例(値渡し)}
  \begin{itemize}
    %\setlength{\itemsep}{1em}
  \item 例2.6.5 (ハンドブック2.6節)
\begin{lstlisting}
#include <stdio.h>
void division(int divident, int divisor, int quotient,
  int residual) {
  quotient = divident / divisor;
  residual = divident % divisor;
}
int main() {
  int josuu = 3;
  int hi_josuu = 13;
  int shou, amari;
  division(hi_josuu, josuu, shou, amari);
  printf("%d / %d = %d ... %d\n", hi_josuu, josuu,
         shou, amari);
}
\end{lstlisting}
\begin{itemize}
\item 誤った答えが出力される。なぜ?
\end{itemize}
  \end{itemize}
\end{frame}

\begin{frame}[t,fragile]{一次元配列}
  \begin{itemize}
    %\setlength{\itemsep}{1em}
  \item (静的)一次元配列 (ハンドブック2.3.1節)
\begin{lstlisting}
double v[10];
v[0] = 1.0;
v[1] = 2.0;
...
\end{lstlisting}
    要素数はコンパイル時にすでに決まっている定数でなければならない
  \item (動的)一次元配列 (ハンドブック2.12節)
\begin{lstlisting}
double *v; /* ポインタ */
v = (double*)malloc((size_t)(10 * sizeof(double));
...
free(v); /* 確保した領域を開放 */
\end{lstlisting}
実行時に要素数を指定可能
  \end{itemize}
\end{frame}

\begin{frame}[t,fragile]{ポインタと一次元配列}
  \begin{itemize}
    \setlength{\itemsep}{1em}
  \item 一次元配列を表す変数は、(実は)最初の要素を指すポインタ  (ハンドブック3.5.3節)
    \begin{itemize}
    \item \verb+v+ と \verb+&v[0]+ は等価
    \item \verb^(v+2)^ と \verb^&v[2]^ は等価
    \item \verb+*v+ と \verb+v[0]+ は等価
    \item \verb^*(v+2)^ と \verb^v[2]^ は等価
    \item \verb^(v+2)[3]^ は?
    \end{itemize}
  \item C言語では配列の添字は0から始まることに注意
  \item \verb^double v[10];^ と宣言した場合、\verb^v[0]^ 〜 \verb^v[9]^ の10個の要素を持つ配列が作られる。\verb^v[10]^ は存在しない。値を代入したり参照しようとするとエラーとなる
  \item ポインタ確認プログラム: \href{https://github.com/todo-group/computer-experiments/blob/master/exercise/matrix/pointer.c}{pointer.c}
  \end{itemize}
\end{frame}

\begin{frame}[t,fragile]{二次元配列}
  \begin{itemize}
    \setlength{\itemsep}{1em}
  \item C言語では、二次元配列は一次元配列の先頭をさす(ポインタ)の配列として表される(と理解しておけば良い)
  \item \verb+a[i]+は、要素\verb+a[i][0]+を指すポインタ
    \begin{itemize}
    \item \verb+a+ と \verb+&a[0]+ は等価 (\verb+&a[0][0]+ ではない)
    \item \verb+a[0]+ と \verb+&a[0][0]+ は等価
    \item \verb+a[2]+ と \verb+&a[2][0]+ は等価
    \item \verb^(a+2)^ と \verb^&a[2]^ は等価
    \item \verb^(*(a+2))[3]^ と \verb^*(*(a+2)+3)^ と \verb^a[2][3]^ は等価
    \item \verb^*(a+2)[3]^ と \verb^*((a+2)[3])^ と \verb^*(a[5])^ と\verb^a[5][0]^ は等価
    \item \verb^[]^は\verb^*^よりも強い
    \end{itemize}
  \item ポインタ確認プログラム: \href{https://github.com/todo-group/computer-experiments/blob/master/exercise/matrix/pointer-matrix.c}{pointer-matrix.c}
  \end{itemize}
\end{frame}

\begin{frame}[t,fragile]{動的二次元配列の確保}
  \begin{itemize}
    \setlength{\itemsep}{1em}
  \item 各行を表す配列とそれぞれの先頭アドレスを保持する配列の二種類が必要
\begin{lstlisting}
double **a;
m = 10;  
n = 10;  
a = (double**)malloc((size_t)(m * sizeof(double*));
for (int i = 0; i < m; ++i)
  a[i] = (double*)malloc((size_t)(n * sizeof(double));
\end{lstlisting}
\item 各行を保持する配列が、メモリ上で連続に確保される保証はない
\item 行列用のライブラリ(LAPACK等)を使うときに問題となる
  \end{itemize}
\end{frame}

\begin{frame}[t,fragile]{BLASライブラリ}
  \begin{itemize}
    \setlength{\itemsep}{1em}
  \item 行列・行列積、行列・ベクトル積などを高速に行う最適化された関数群
  \item 行列・行列積を計算するサブルーチン {\tt dgemm} \\
    \url{http://www.netlib.org/lapack/explore-html/d7/d2b/dgemm_8f.html}
    \begin{itemize}
    \item $C = \alpha A \times B + \beta C$ を計算
    \item BLASもFortranで書かれている
    \end{itemize}
  \item 例: \href{https://github.com/todo-group/computer-experiments/blob/master/exercise/matrix/multiply.c}{multiply.c}, \href{https://github.com/todo-group/computer-experiments/blob/master/exercise/matrix/multiply_dgemm.c}{multiply\_dgemm.c}
  \end{itemize}
\end{frame}

\begin{frame}[t,fragile]{LAPACK (Linear Algebra PACKage)}
  \begin{itemize}
    %\setlength{\itemsep}{1em}
  \item 線形計算のための高品質な数値計算ライブラリ
    \begin{itemize}
    \item \url{http://www.netlib.org/lapack}
    \item 線形方程式、固有値問題、特異値問題、線形最小二乗問題など
    \item (FFT 高速フーリエ変換は入っていない)
    \item LAPACK自体はFortranで書かれている
    \end{itemize}
  \item ほぼ全てのPC、ワークステーション、スーパーコンピュータで利用可 (インストール済)
  \item Netlibでソースが公開されているリファレンス実装は遅いが、それぞれのベンダー(Intel、Fujitsu、etc)による最適化されたLAPACKが用意されている場合が多い(MKL、SSL2、etc)
  \item LAPACKを使うことにより、高速で信頼性が高く、ポータブルなコードを書くことが可能になる
  \end{itemize}
\end{frame}

%\begin{frame}[t,fragile]{LAPACKによる連立一次方程式の求解}
  \begin{itemize}
    \setlength{\itemsep}{1em}
  \item LU分解を行うサブルーチン {\tt dgetrf} \\
    \url{http://www.netlib.org/lapack/explore-html/d3/d6a/dgetrf_8f.html}
  \item Fortranによる関数宣言
\begin{lstlisting}
subroutine dgetrf(integer M, integer N,
         double precision, dimension(lda, *) A,
         integer LDA, integer, dimension(*) IPIV,
         integer INFO)
\end{lstlisting}
\item {\tt A}: 左辺の行列、{\tt M,N}: 次元、{\tt IPIV}: 選択されたピボット行のリスト、{\tt lda}: 通常{\tt M} (行数)と同じで良い
  \end{itemize}
\end{frame}

\begin{frame}[t,fragile]{CからBLAS/LAPACKを呼び出す際の注意事項}
  \begin{itemize}
    %\setlength{\itemsep}{1em}
  \item (もともとFortran言語で書かれていたことによる制限)
  \item 関数名はすべて小文字、最後に \verb+_+ (下線)を付ける
  \item スカラー、ベクトル、行列は全て「ポインタ渡し」とする
  \item ベクトルや行列は最初の要素へのポインタを渡す (サイズは別に渡す)
  \item 行列の要素は(0,0) $\rightarrow$ (1,0) $\rightarrow$ (2,0) $\rightarrow\cdots\rightarrow$ $(m-1,0)$ $\rightarrow$ (0,1) $\rightarrow$ (1,1) $\rightarrow\cdots\rightarrow$ $(m-1,n-1)$の順で連続して並んでいなければならない(column-major)
    \begin{itemize}
    \item C言語の二次元配列では \verb+a[i][j]+ の次には \verb%a[i][j+1]%が入っている(row-major)
    \item 行列が転置されて解釈されてしまう!
    \end{itemize}
  \item コンパイル時には{\tt -llapack -lblas}オプションを指定し、LAPACKライブラリとBLASライブラリをリンクする(ハンドブック2.1.6節)
  \end{itemize}
\end{frame}

\begin{frame}[t,fragile]{cmatrix.hライブラリ}
  \begin{itemize}
    %\setlength{\itemsep}{1em}
  \item Column-major形式の二次元配列の確保({\tt alloc\_dmatrix})、開放({\tt free\_dmatrix})、出力({\tt print\_dmatrix})、読み込み({\tt read\_dmatrix})を行うためのユーティリティ関数、(i,j)成分にアクセスするためのマクロ({\tt mat\_elem})他を準備
  \item ソースコード: \href{https://github.com/todo-group/computer-experiments/blob/master/exercise/include/cmatrix.h}{cmatrix.h}
  \item 使用例
\begin{lstlisting}
#include "cmatrix.h"
...
double **mat;
mat = alloc_dmatrix(m, n);
mat_elem(mat, 1, 3) = 5.0;
...
free_dmatrix(mat);
\end{lstlisting}
  \item サンプルコード: \href{https://github.com/todo-group/computer-experiments/blob/master/exercise/matrix/matrix_example.c}{matrix\_example.c}
  \end{itemize}
\end{frame}

\begin{frame}[t,fragile]{alloc\_dmatrixでの動的二次元配列の確保}
  \begin{itemize}
    %\setlength{\itemsep}{1em}
  \item 長さ$m \times n$の一次元配列を用意し、各列(それぞれ$m$要素)の先頭アドレスを長さ$n$のポインター配列に格納する
\begin{lstlisting}
double **a;
m = 10;  
n = 10;  
a = (double**)malloc((size_t)(n * sizeof(double*));
a[0] = (double*)malloc((size_t)(m*n * sizeof(double));
for (int i = 1; i < n; ++i)
  a[i] = a[i-1] + m;
\end{lstlisting}
\item 行列の(i,j)成分を\verb+a[j][i]+に格納することにする (column-major)
  \end{itemize}
\end{frame}

\begin{frame}[t,fragile]{要素アクセス・先頭アドレス}
  \begin{itemize}
    % \setlength{\itemsep}{1em}
  \item 行列の(i,j)成分を\verb+a[j][i]+に格納することにする(column-major)
    \begin{itemize}
      \item \href{https://github.com/todo-group/computer-experiments/blob/master/exercise/matrix/cmatrix.h}{cmatrix.h}ではマクロ(\verb+mat_elem+)を準備
\begin{lstlisting}
#define mat_elem(mat, i, j) (mat)[j][i]
\end{lstlisting}
\item このマクロを使うと、例えば(i,j)成分への代入は以下のように書ける
\begin{lstlisting}
mat_elem(a, i, j) = 1;
\end{lstlisting}
\end{itemize}
  \item LAPACKにベクトルや行列の最初の要素へのポインタを渡す
    \begin{itemize}
      \item ベクトルの最初の要素(0)へのポインタ: \verb+&v[0]+
      \item 行列の最初の要素(0,0)へのポインタ: \verb+&a[0][0]+
      \item \href{https://github.com/todo-group/computer-experiments/blob/master/exercise/matrix/cmatrix.h}{cmatrix.h}にマクロ({\tt vec\_ptr}、{\tt mat\_ptr})が準備されているのでそれぞれ、{\tt vec\_ptr(v)}、{\tt mat\_ptr(a)}と書ける
    \end{itemize}
  \end{itemize}
\end{frame}

%\begin{frame}[t,fragile]{LAPACKによる連立一次方程式の求解}
  \begin{itemize}
    \setlength{\itemsep}{1em}
  \item C言語から呼び出すための関数宣言を作成 (ハンドブック3.6.4節)
\begin{lstlisting}
void dgetrf_(int *M, int *N, double *A,
             int *LDA, int*IPIV, int *INFO);
\end{lstlisting}
関数名は全て小文字。関数名の最後に {\tt \_} (下線)を付ける
\item LU分解の例
\begin{lstlisting}
m = 10;
n = 10;
a = alloc_dmatrix(m, n);
...
dgetrf_(&m, &n, mat_ptr(a), &m, vec_ptr(ipiv), &info);
\end{lstlisting}
完全なソースコード: \href{https://github.com/todo-group/computer-experiments/blob/master/exercise/linear_system/lu_decomp.c}{lu\_decomp.c}
  \end{itemize}
\end{frame}

\begin{frame}[t,fragile]{行列のノルム}
  \begin{itemize}
    %\setlength{\itemsep}{1em}
  \item 行列のノルム: $\| A \|$
    \begin{itemize}
    \item 正定値性: $\| A \| \ge 0$. $\| A \| = 0$となるのは$A=O$のときのみ
    \item 斉次性: $\| \alpha A \| = |\alpha| \| A \|$
    \item 劣加法性: $\| A+B \| \le \| A \| + \| B \|$
      % \\ (加えて、正方行列に対して、劣乗法性、*-性)
    \end{itemize}
  \item さまざまなノルム
    \begin{itemize}
    \item 1-ノルム: $\displaystyle \| A \|_1 = \max_{1 \le j \le n} \sum_{i=1}^m | a_{ij} |$
    \item $\infty$-ノルム: $\displaystyle \| A \|_\infty = \max_{1 \le j \le m} \sum_{j=1}^n | a_{ij} |$
    \item スペクトルノルム: $A^* A$の最大固有値の平方根 (固有値分解が必要)
    \item フロベニウスノルム: $\displaystyle \| A \|_{\rm F} = \Big[ \sum_{i=1}^m \sum_{j=1}^n |a_{ij}|^2 \Big]^{1/2}$
    \item 最大ノルム: $\displaystyle \| A \|_{\rm max} = \max \{ | a_{ij} | \}$ \\ (劣乗法性$\| AB \| \le \| A \| \| B \|$を満たさない)
    \end{itemize}
  \item 例題: 適当な$m \times n$行列についてノルムを計算するプログラムを作成せよ
  \end{itemize}
\end{frame}

\begin{frame}[t,fragile]{LAPACKによる行列のノルムの計算}
  \begin{itemize}
    %\setlength{\itemsep}{1em}
  \item 倍精度(単精度)実行列の場合{\tt dlange} ({\tt flange})関数、倍精度(単精度)複素行列の場合{\tt zlange} ({\tt clange})関数を使う \\
    \url{http://www.netlib.org/lapack/explore-html/dc/d09/dlange_8f.html}
  \item C言語から呼び出すための関数宣言: \href{https://github.com/todo-group/computer-experiments/blob/master/exercise/matrix/dlange.h}{dlange.h}
\begin{lstlisting}
double dlange_(char *NORM, int *M, int *N, double *A, int *LDA, double *WORK);
\end{lstlisting}
\item {\tt dlange}関数の利用例: \href{https://github.com/todo-group/computer-experiments/blob/master/exercise/matrix/dlange.c}{dlange.c}
\begin{lstlisting}
norm = 'F';
m = 10;
n = 10;
a = alloc_dmatrix(m, n);
...
res = dlange_(&norm, &m, &n, mat_ptr(a), &m, vec_ptr(work));
\end{lstlisting}
  \end{itemize}
\end{frame}

\begin{frame}[t,fragile]{LAPACKによる連立一次方程式の求解}
  \begin{itemize}
    %\setlength{\itemsep}{1em}
  \item 連立一次方程式: $Ax=b$
  \item C言語から呼び出すための関数宣言: \href{https://github.com/todo-group/computer-experiments/blob/master/exercise/linear_system/dgesv.h}{dgesv.h}
\begin{lstlisting}
void dgesv_(int *N, int *NRHS, double *A, int *LDA, int *IPIV, double *B, int *LDB, int *INFO);
\end{lstlisting}
  \item {\tt dgesv}の利用例: \href{https://github.com/todo-group/computer-experiments/blob/master/exercise/linear_system/dgesv.c}{dgesv.c}
\begin{lstlisting}
n = 3;
a = alloc_dmatrix(n, n);
b = alloc_dvector(n);
...
dgesv_(&n, &nrhs, mat_ptr(a), &n, vec_ptr(ipiv), vec_ptr(b), &n, &info);
\end{lstlisting}
配列{\tt b}は解で上書きされる
  \item {\tt dgesv}の中ではまず行列をLU分解し、その結果を利用して解を求めている
  \end{itemize}
\end{frame}


\begin{frame}[t]{本日の課題}
  \begin{itemize}
    %\setlength{\itemsep}{1em}
  \item まだ完了していない人は「\href{https://utphys-comp.github.io}{計算機実験のための環境整備}」({\small \href{https://utphys-comp.github.io}{https://utphys-comp.github.io}})を参考に、必要な環境を引き続き整備
  \item 実習
    \begin{itemize}
    \item 講義資料の中の {\tt random.c}, {\tt complex.c}, {\tt pointer.c}, {\tt pointer-matrix.c}, {\tt matrix-example.c}, {\tt dlange.c}, {\tt dgesv.c}をコンパイル・実行。ソースコードの中身を確認 \\
      サンプルコード一式: \href{https://github.com/todo-group/ComputerExperiments/releases/tag/2020s-computer1}{example-1-4.zip}
    \item 実習課題一覧\href{https://github.com/todo-group/ComputerExperiments/releases/tag/2020s-computer1}{exercise-1.pdf}の課題1〜10, 16, 17, 23-25の中から適宜選び実習
    \end{itemize}
  \item レポートNo.1: 提出締切2020/5/29(金) 18:00 (レポート内容・提出方法についてはITC-LMSを参照のこと)
  \item 質問は、Slackの「\#5 対角化」あるいは他の適当と思われるチャンネルで
  \end{itemize}
\end{frame}

\end{document}
