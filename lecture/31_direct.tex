\section{連立一次方程式の直接解法}

\begin{frame}[t,fragile]{逆行列の「間違った」求め方}
  \begin{itemize}
    \setlength{\itemsep}{1em}
  \item 線形代数の教科書に載っている公式
    \[
    A^{-1} = \frac{\tilde{A}}{|A|}
    \]
    $|A|$: $A$の行列式、$\tilde{A}$: $A$の余因子行列
  \item $n \times n$行列の行列式を定義通り計算すると、計算量〜$O(n!)$
  \item したがって、上の方法で逆行列を計算すると、計算量〜$O(n!)$
  \item $n=100$の場合: $n! \approx 9.3 \times 10^{157}$
  \end{itemize}
\end{frame}

\begin{frame}[t,fragile]{逆行列の「正しい」求め方}
  \begin{itemize}
    \setlength{\itemsep}{1em}
  \item 連立一次方程式 $A {\bf x} = {\bf e}_j$ を全ての${\bf e}_j$について解く
  \item Gaussの消去法による連立一次方程式の解法: 計算量〜$O(n^3)$
  \item Gaussの消去法の途中で出てくる下三角行列(L)と上三角行列(U)行列を再利用(LU分解)すれば、逆行列全体を求めるための計算量も$O(n^3)$
  \item 行列式も$O(n^3)$で計算可
  \item $n=100$の場合: $n^3 = 10^6 \ll 9.3 \times 10^{157}$
  \end{itemize}
\end{frame}

\begin{frame}[t,fragile]{ガウスの消去法}
  \begin{itemize}
    \setlength{\itemsep}{1em}
  \item 解くべき連立方程式
    \begin{align*}
    a_{11}^{(1)} x_1 + a_{12}^{(1)} x_2 + a_{13}^{(1)} x_3 + \cdots + a_{1n}^{(1)} x_n &= b_{1}^{(1)} \\
    a_{21}^{(1)} x_1 + a_{22}^{(1)} x_2 + a_{23}^{(1)} x_3 + \cdots + a_{2n}^{(1)} x_n &= b_{2}^{(1)} \\
    a_{31}^{(1)} x_1 + a_{32}^{(1)} x_2 + a_{33}^{(1)} x_3 + \cdots + a_{3n}^{(1)} x_n &= b_{3}^{(1)} \\
    \cdots \\
    a_{n1}^{(1)} x_1 + a_{n2}^{(1)} x_2 + a_{n3}^{(1)} x_3 + \cdots + a_{nn}^{(1)} x_n &= b_{n}^{(1)}
    \end{align*}
  \item ある行を定数倍しても、方程式の解は変わらない
  \item ある行の定数倍を他の行から引いても、方程式の解は変わらない
  \end{itemize}
\end{frame}

\begin{frame}[t,fragile]{ガウスの消去法}
  \begin{itemize}
    \setlength{\itemsep}{1em}
  \item 1行目を$m_{i1} = a_{i1}^{(1)}/a_{11}^{(1)}$倍して、$i$行目($i \ge 2$)から引く
    \begin{align*}
    a_{11}^{(1)} x_1 + a_{12}^{(1)} x_2 + a_{13}^{(1)} x_3 + \cdots + a_{1n}^{(1)} x_n &= b_{1}^{(1)} \\
    a_{22}^{(2)} x_2 + a_{23}^{(2)} x_3 + \cdots + a_{2n}^{(2)} x_n &= b_{2}^{(2)} \\
    a_{32}^{(2)} x_2 + a_{33}^{(2)} x_3 + \cdots + a_{3n}^{(2)} x_n &= b_{3}^{(2)} \\
    \cdots \\
    a_{n2}^{(2)} x_2 + a_{n3}^{(2)} x_3 + \cdots + a_{nn}^{(2)} x_n &= b_{n}^{(2)}
    \end{align*}
  \item ここで
    \begin{align*}
      a_{ij}^{(2)} &= a_{ij}^{(1)} - m_{i1} a_{1j}^{(1)} \qquad i \ge 2, j \ge 2 \\
      b_{i}^{(2)} &= b_{i}^{(1)} - m_{i1} b_{1}^{(1)} \qquad i \ge 2
    \end{align*}
  \end{itemize}
\end{frame}

\begin{frame}[t,fragile]{ガウスの消去法}
  \begin{itemize}
    \setlength{\itemsep}{1em}
  \item 2行目を$m_{i2} = a_{i2}^{(2)}/a_{22}^{(2)}$倍して、$i$行目($i \ge 3$)から引く
    \begin{align*}
    a_{11}^{(1)} x_1 + a_{12}^{(1)} x_2 + a_{13}^{(1)} x_3 + \cdots + a_{1n}^{(1)} x_n &= b_{1}^{(1)} \\
    a_{22}^{(2)} x_2 + a_{23}^{(2)} x_3 + \cdots + a_{2n}^{(2)} x_n &= b_{2}^{(2)} \\
    a_{33}^{(3)} x_3 + \cdots + a_{3n}^{(3)} x_n &= b_{3}^{(3)} \\
    \cdots \\
    a_{n3}^{(3)} x_3 + \cdots + a_{nn}^{(3)} x_n &= b_{n}^{(3)}
    \end{align*}
  \item ここで
    \begin{align*}
      a_{ij}^{(3)} &= a_{ij}^{(2)} - m_{i2} a_{2j}^{(2)} \qquad i \ge 3, j \ge 3 \\
      b_{i}^{(3)} &= b_{i}^{(2)} - m_{i2} b_{2}^{(2)} \qquad i \ge 3
    \end{align*}
  \end{itemize}
\end{frame}

\begin{frame}[t,fragile]{ガウスの消去法}
  \begin{itemize}
    \setlength{\itemsep}{1em}
  \item 最終的には、左辺が右上三角形をした連立方程式となる
    \begin{align*}
    a_{11}^{(1)} x_1 + a_{12}^{(1)} x_2 + a_{13}^{(1)} x_3 + \cdots + a_{1n}^{(1)} x_n &= b_{1}^{(1)} \\
    a_{22}^{(2)} x_2 + a_{23}^{(2)} x_3 + \cdots + a_{2n}^{(2)} x_n &= b_{2}^{(2)} \\
    a_{33}^{(3)} x_3 + \cdots + a_{3n}^{(3)} x_n &= b_{3}^{(3)} \\
    \cdots \\
    a_{n-1,n-1}^{(n-1)} x_{n-1} + a_{n-1,n}^{(n-1)} x_n &= b_{n-1}^{(n-1)} \\
    a_{nn}^{(n)} x_n &= b_{n}^{(n)}
    \end{align*}
  \item これを下から順に解いていけばよい(後退代入)
  \end{itemize}
\end{frame}

\begin{frame}[t,fragile]{練習問題}
  \begin{itemize}
    \setlength{\itemsep}{1em}
  \item 次の連立方程式をガウスの消去法で(手で)解け
    \begin{align*}
      \begin{pmatrix} 1 & 4 & 7 \\ 2 & 5 & 8 \\ 3 & 6 & 10 \end{pmatrix} \begin{pmatrix} x_1 \\ x_2 \\ x_3 \end{pmatrix} = \begin{pmatrix} 18 \\ 24 \\ 31 \end{pmatrix}
    \end{align*}
  \end{itemize}
\end{frame}

\begin{frame}[t,fragile]{ガウスの消去法のコード}
\begin{lstlisting}
for (k = 0; k < n; ++k) {
  for (i = k + 1; i < n; ++i) {
    for (j = k + 1; j < n; ++j) {
      a[i][j] -= a[k][j] * a[i][k] / a[k][k];
    }
    b[i] -= b[k] * a[i][k] / a[k][k];
  }
}
for (k = n-1; k >= 0; --k) {
  for (j = k + 1; j < n; ++j) {
    b[k] -= a[k][j] * b[j];
  }
  b[k] /= a[k][k];
}
\end{lstlisting}
\begin{itemize}
\item C言語では配列の添字が0から始まることに注意
\end{itemize}
\end{frame}

\begin{frame}[t,fragile]{ピボット選択}
  \begin{itemize}
    \setlength{\itemsep}{1em}
  \item ガウスの消去法の途中で$a_{kk}^{(k)}$が零になると、計算を先に進めることができなくなる
  \item 行を入れ替えても、方程式の解は変わらない $\Rightarrow$ $k$行以降で、$a_{ik}^{(k)}$が非零の行と入れ替える (ピボット選択)
  \item 実際のコードでは、情報落ちを防ぐため、$a_{kk}^{(k)}$が零でない場合でも、$a_{ik}^{(k)}$の絶対値が最大の行と入れ替える
  \item ピボット選択が必要となる例
    \begin{align*}
      \begin{pmatrix} 1 & 2 & 3 \\ 3 & 6 & 4 \\ 4 & 6 & 7 \end{pmatrix} \begin{pmatrix} x_1 \\ x_2 \\ x_3 \end{pmatrix} = \begin{pmatrix} 8 \\ 19 \\ 23 \end{pmatrix}
    \end{align*}
  \end{itemize}
\end{frame}

\begin{frame}[t,fragile]{ガウスの消去法の行列表示}
  \begin{itemize}
    \setlength{\itemsep}{1em}
  \item $a_{kk}^{(k)}$を用いた$a_{ik}^{(k)}$ ($i>k$)の消去は、方程式の両辺に
    \begin{align*}
      M_k = 
      \begin{pmatrix}
        1 & \\
        0 & 1 \\
        0 & 0 & \ddots \\
        \vdots & \vdots & & 1 \\
        \vdots & \vdots & & -m_{k+1,k} & 1 & \\
        \vdots & \vdots & & -m_{k+2,k} & 0 & \ddots \\
        \vdots & \vdots & & \vdots & \vdots & & 1 & \\
        0 & 0 & \hdots & -m_{nk} & 0 & \hdots & 0 & 1
      \end{pmatrix}
    \end{align*}
    を掛けるのと等価: $M_k A^{(k)} = A^{(k+1)}$、$M_k {\bf b}^{(k)} = {\bf b}^{(k+1)}$
  \end{itemize}
\end{frame}

\begin{frame}[t,fragile]{LU分解}
  \begin{itemize}
    \setlength{\itemsep}{1em}
  \item $M_k$の逆行列
    \begin{align*}
      L_k = M_k^{-1} = 
      \begin{pmatrix}
        1 & \\
        0 & 1 \\
        0 & 0 & \ddots \\
        \vdots & \vdots & & 1 \\
        \vdots & \vdots & & m_{k+1,k} & 1 & \\
        \vdots & \vdots & & m_{k+2,k} & 0 & \ddots \\
        \vdots & \vdots & & \vdots & \vdots & & 1 & \\
        0 & 0 & \hdots & m_{nk} & 0 & \hdots & 0 & 1
      \end{pmatrix}
    \end{align*}
    から$L=L_1L_2\cdots L_n$を定義すると、$L$は下三角行列、また$U = A^{(n)}$ (上三角行列)とすると、$A = LU$
  \end{itemize}
\end{frame}

\begin{frame}[t,fragile]{LU分解}
  \begin{itemize}
    \setlength{\itemsep}{1em}
  \item LU分解による連立一次方程式の解法
    \begin{itemize}
    \item 方程式は$A{\bf x} = LU{\bf x} = {\bf b}$と書ける
    \item まず、$L{\bf y} = {\bf b}$を解いて、${\bf y}$を求める(前進代入)
    \item 次に、$U{\bf x} = {\bf y}$を解いて、${\bf x}$を求める(後退代入)
    \end{itemize}
  \item 計算量はガウスの消去法と変わらない
  \item 一度LU分解をしておけば、異なる${\bf b}$に対する解も簡単に求められる
  \item 行列式は$U$の対角成分の積で与えられる
  \end{itemize}
\end{frame}
