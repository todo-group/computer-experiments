%-*- coding:utf-8 -*-

\begin{frame}[t,fragile]{温度の制御}
  \begin{itemize}
    %\setlength{\itemsep}{1em}
  \item ランジュバン(Langevin)法
    \begin{itemize}
    \item 乱数を使う方法
    \item 摩擦項と揺動項(ランダム力)を付け加える
      \begin{align*}
        \frac{dp_i}{dt} = - \frac{\partial H}{\partial q_i} - \gamma_i p_i + R_i(t)
      \end{align*}
    \item $\gamma_i$: 摩擦係数
    \item $R_i(T)$: 平均零の白色ノイズ
      \begin{align*}
        \langle R_i(t) R_i(t') \rangle = 2 m \gamma_i k_B T \delta(t-t')
      \end{align*}
    \item 摩擦項と揺動項がつりあうところで、温度$T$のカノニカル分布が実現する
    \end{itemize}
  \item Andersen法
    \begin{itemize}
    \item 各粒子と熱浴の結合係数: \(\nu\)
    \item 時間幅\(h\)の時間後、各粒子に対して確率\(\nu h\)でボルツマン分布から速度を再設定する。
    \end{itemize}
  \end{itemize}
\end{frame}
