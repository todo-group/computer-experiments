%-*- coding:utf-8 -*-

\begin{frame}[t,fragile]{Perron-Frobeniusの定理}
  \begin{itemize}
    %\setlength{\itemsep}{1em}
  \item 正の正方行列$A$(すべての要素が正)について以下が成り立つ
    \begin{itemize}
      \item 他の全ての固有値よりも絶対値の大きな正の固有値$r$が存在する
      \item 固有値$r$は単純固有値である(縮退していない)
      \item 固有値$r$に対する右(左)固有ベクトル$v$ ($w$) は正のベクトルである
      \item 固有値 $r$ は $\displaystyle \min_i \sum_j a_{ij} \le r \le \max_i \sum_j a_{ij}$ を満たす
    \end{itemize}
  \item $A$が零の要素を持つ場合でも$A$が原始的(primitive = エルゴード的)である限り、上の結果は成り立つ
  \item 遷移行列は上の条件を満たす
    \begin{itemize}
    \item カノニカル分布は絶対値最大の固有ベクトル
    \item 遷移行列を掛けていくとカノニカル分布に収束
    \end{itemize}
  \end{itemize}
\end{frame}
