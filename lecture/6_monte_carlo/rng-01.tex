\begin{frame}[t,fragile]{乱数}
  \begin{itemize}
    %\setlength{\itemsep}{1em}
  \item 自然乱数 (ハードウェア乱数)
    \begin{itemize}
    \item さいころ、コイン、ルーレット、核分裂反応、熱雑音、ショット雑音 ...
    \end{itemize}
  \item 疑似乱数とは
    \begin{itemize}
    \item 計算機でプログラムに従って生成する乱数(のようなもの)
    \item 乱数は何に役立つか?
      \begin{itemize}
      \item 等式のチェック、例外の発見
      \item 初期値にランダムネスを入れることで最悪の場合を避ける
      \item サンプリングを使ったシミュレーション (→計算機実験II)
      \end{itemize}
    \item 疑似乱数に必要な条件
      \begin{itemize}
      \item 多数の乱数が生成可能
      \item ポータビリティ
      \item 生成速度
      \item 再現性
      \item 統計的性質
      \end{itemize}
  %% \item 擬似乱数 (pseudo random number)
  %%   \begin{itemize}
  %%   \item 計算機でプログラムに従って生成
  %%   \item 分布の一様性、相関、周期に注意する必要あり
    \end{itemize}
  \end{itemize}
\end{frame}
