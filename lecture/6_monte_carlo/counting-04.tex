%-*- coding:utf-8 -*-

\begin{frame}[t,fragile]{ビット表現}
  \begin{itemize}
    %\setlength{\itemsep}{1em}
  \item 特定のビットの取り出し
    \begin{itemize}
    \item シフト演算(\verb+>>+)とAND演算(\verb+&+)を利用 \ 例)

      3番目($i=3$)のビット(0 / 1)を取り出す: \verb+((s >> 3) & 1)+

      3番目($i=3$)のスピン状態($\sigma_3 = \pm 1$)を取り出す: \verb+(1-2*((s >> 3) & 1))+

      $\sigma_i \sigma_j$の計算: \verb+(1-2*((s >> i) & 1)) * (1-2*((s >> j) & 1))+
    \item {\color{red} $i$は0から数える}ことに注意($i=0,1,\ldots,N-1$)
    \item ビットAND (\verb+&+)と論理AND (\verb+&&+)との違いに注意
    \end{itemize}
  \item 特定のビットのフリップ(反転)
    \begin{itemize}
    \item シフト演算(\verb+<<+)とXOR(排他的論理和)演算(\verb+^+)を利用 \ 例)
      
      3番目($i=3$)のビットを反転: \verb+s^(1<<3)+
    \end{itemize}
  \item $N$ビット全てが1の状態を作る
    \begin{itemize}
    \item \verb+(1 << N)-1+
    \end{itemize}
  \item $N$重のforループを書く代わりに、状態を1つの$N$ビットの整数($s=0,\cdots,2^N-1$)で表し、ひとつのループに
  \end{itemize}
\end{frame}
