%-*- coding:utf-8 -*-

\begin{frame}[t,fragile]{分子動力学法}
  \begin{itemize}
    \setlength{\itemsep}{1em}
  \item 適当な初期条件から、運動方程式に従って位置と運動量を時間発展させる
    \begin{itemize}
    \item Euler法、Runge-Kutta法、リープ・フロッグ法、速度ベルレ法など
    \item $6N$次元の連立微分方程式
    \end{itemize}
  \item 時間発展に関する物理量の時間平均から平均を評価
    \begin{align*}
      \langle A(p,x) \rangle &= \frac{1}{Z(E)} \int A(p,x) \, \delta(H(p,x)-E) \, dp \, dx \\
      &\simeq \frac{1}{t_{\rm max}} \int_0^{t_{\rm max}} A(p(t),x(t)) \, dt
    \end{align*}
    \begin{itemize}
    \item ハミルトニアンが時間依存しない場合は全エネルギーが保存する→ミクロカノニカル分布
    \end{itemize}
  \item 平衡状態における平均値だけでなく、熱や電荷の輸送などの動的現象、非平衡状態からの緩和現象などもシミュレーションできる
  \end{itemize}
\end{frame}
