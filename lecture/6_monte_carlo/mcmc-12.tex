%-*- coding:utf-8 -*-

\begin{frame}[t,fragile]{イジング模型に対するモンテカルロ法}
  \begin{itemize}
    %\setlength{\itemsep}{1em}
  \item 更新の単位は一つのスピンとするのが一番自然
  \item メトロポリス法に必要なのは更新前後のエネルギー差だけなので、全エネルギーを計算しなおす必要なし
\begin{verbatim}
for (s = 0; s < num_sites; ++s) {
  delta = 0.0;
  for (j = 0; j < num_neighbors; ++j) {
    v = mat_elem(neighbor, s, j);
    delta += 2 * J * spin[s] * spin[v];
  }
  if (random() < exp(-beta * delta))
    spin[s] = -1 * spin[s];
}  
\end{verbatim}
  \end{itemize}
\end{frame}
