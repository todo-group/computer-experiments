\begin{frame}[t,fragile]{統計物理における平衡状態}
  \begin{itemize}
    \setlength{\itemsep}{1em}
  \item Boltzmann分布 ($\beta \equiv 1/k_B T$)
    \[
    \pi(s) = \exp[-\beta {\cal H}(s)] \Big / \sum_s \exp[-\beta {\cal H}(s)]
    \]
  \item 物理量の期待値
    \[
    \langle A \rangle = \sum_s A(s) \exp[-\beta {\cal H}(s)] \Big/ \sum_s \exp[-\beta {\cal H}(s)]
    \]
  \item $\sum_s$は全ての状態に関する和 (系の体積に対して指数関数的に増加)
  \item 全ての状態について和をとるかわりに、Boltzmann重みが大きい(=$\cal H$が小さい)ところだけをモンテカルロ・サンプリング
  \end{itemize}
\end{frame}
