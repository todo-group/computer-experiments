%-*- coding:utf-8 -*-

\begin{frame}[t,fragile]{物理量の計算}
  \begin{itemize}
    %\setlength{\itemsep}{1em}
  \item 内部エネルギー$E$
    \begin{itemize}
    \item 初期状態のスピンを全て上向き(1)に取ると$E=-J \times \text{ボンド数}$
    \item モンテカルロステップ毎にエネルギーの変化分を計算しているので、採択された場合にはその値を足し込む
    \end{itemize}
  \item 比熱$C$: 内部エネルギーのゆらぎから計算できる
    \[
    C = \frac{1}{N} \frac{\partial E}{\partial T} = \frac{1}{NT^2} (\langle E^2 \rangle - \langle E \rangle^2)
    \]
  \item 磁化$m$: スピンの値の平均値 $m = \frac{1}{N} \sum_i \sigma_i$
    \begin{itemize}
    \item 外部磁場がない場合、対称性から$m$の長時間平均は厳密には零になる
    \item 熱力学極限では対称性が自発的に破れて、低温で有限の$m$
    \item シミュレーションでは$m$ではなく$m^2$を見るとよい
    \end{itemize}
  \end{itemize}
\end{frame}
