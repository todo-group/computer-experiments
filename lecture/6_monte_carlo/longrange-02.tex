%-*- coding:utf-8 -*-

\begin{frame}[t,fragile]{エバルト法}
  \begin{itemize}
    %\setlength{\itemsep}{1em}
  \item 第2項目(長距離項)の計算
    \begin{itemize}
    \item 変数変換($s=t/r$)
      \[
      \begin{split}
        U_2 &= \frac{1}{2} \sum_{i,j} \sum_{\mathbf{n}} \frac{q_i q_j}{r} \frac{2}{\sqrt{\pi}} \int_0^{\alpha r} \exp(-t^2) \, dt \\
        &= \frac{1}{2} \sum_{i,j} q_i q_j  \frac{2}{\sqrt{\pi}} \int_0^{\alpha} \sum_{\mathbf{n}} \exp(-r^2 s^2) \, ds
        \end{split}
      \]
    \item 被積分関数は$\mathbf{r}_i$に関して周期$L$の周期関数$\rightarrow$フーリエ変換を導入
      \[
      U_2 = \frac{2\pi}{L^3} \sum_{\mathbf{h} \ne \mathbf{0}} \frac{e^{-|\mathbf{h}|^2/4\alpha^2}}{|\mathbf{h}|^2} \Big| \sum_i q_i e^{i\mathbf{h} \cdot \mathbf{r}_i} \Big|^2 \qquad \mathbf{h} = \frac{2\pi}{L}(h_x, h_y, h_z)
      \]
    \item $|\mathbf{h}|$が大きくなると急速に小さくなる
    \item $\mathbf{h} = \mathbf{0}$の項は電気的中性条件から消える
    \end{itemize}
  \end{itemize}
\end{frame}

