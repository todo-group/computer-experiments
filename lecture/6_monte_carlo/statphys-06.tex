%-*- coding:utf-8 -*-

\begin{frame}[t,fragile]{数値積分}
  \begin{itemize}
    %\setlength{\itemsep}{1em}
  \item 一次元の場合
    \begin{itemize}
    \item 台形公式: 積分区間$(a,b)$を$M$個の幅$\Delta=(b-a)/M$の区間に分けて線形関数で近似
      \begin{align*}
        \int_a^b f(x) \, dx &= \sum_{i=1}^M \int_{a+\Delta (i-1)}^{a+\Delta i} f(x) \, dx \\
        &\simeq \sum_{i=1}^M \Delta \frac{f({a+\Delta (i-1)}) + f({a+\Delta i})}{2}
      \end{align*}
    \item より高次の公式: シンプソンの公式(区間を二次式で近似)など
    \end{itemize}
    \item 高次元になると区間の数が指数関数的に増える (次元の呪い)
  \end{itemize}
\end{frame}
