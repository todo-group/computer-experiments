%-*- coding:utf-8 -*-

\begin{frame}[t,fragile]{ハミルトニアン・モンテカルロ法が有効な例}
  \begin{itemize}
    %\setlength{\itemsep}{1em}
  \item 状態変数が連続変数の場合のみ使用可能
  \begin{itemize}
    \item 結合した調和振動子系
      \begin{align*}
        E(x_1,x_2,\cdots,x_N) = \sum x_i^2 + \sum_{i,j} C_{i,j} (x_i - x_j)^2
      \end{align*}
    \item 連続古典スピン系 (ハイゼンベルグ模型)
      \begin{align*}
        E(\vec{S}_1, \vec{S}_2, \cdots, \vec{S}_N) = \sum_{i,j} J_{i,j} \vec{S}_i \cdot \vec{S}_j
      \end{align*}
  \end{itemize}

  \item エネルギーの状態変数に関する微分の情報(=力)を利用する
  \item 現実の系の時間発展と同一である必要はない - 仮想的な時間発展
\end{itemize}
\end{frame}
