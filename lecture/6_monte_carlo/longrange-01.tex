%-*- coding:utf-8 -*-

\begin{frame}[t,fragile]{エバルト法}
  \begin{itemize}
    %\setlength{\itemsep}{1em}
  \item 周期的なクーロンポテンシャル(粒子数$N$)
    \[
    U = \frac{1}{2} \sum_{i,j} \sum_{\mathbf{n}} \frac{q_i q_j}{r} \qquad r = |\mathbf{r}| = |\mathbf{r}_i - \mathbf{r}_j + L \mathbf{n}|
    \]
  \item 無限個のミラーイメージに関する和をいかに計算するか
    \begin{itemize}
    \item 誤差関数(erf)と相補誤差関数(erfc)を用いて、ポテンシャルを2つの部分に分ける($\alpha$はある定数)
      \[
      U = \frac{1}{2} \sum_{i,j} \sum_{\mathbf{n}} \frac{q_i q_j \, \mathrm{erfc}(\alpha r)}{r} + \frac{1}{2} \sum_{i,j} \sum_{\mathbf{n}} \frac{q_i q_j \, \mathrm{erf}(\alpha r)}{r}
      \]
    \item 第1項目は$r$が大きくなると急速に小さくなる
    \end{itemize}
  \end{itemize}
\end{frame}
