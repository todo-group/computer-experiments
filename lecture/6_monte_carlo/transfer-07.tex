\begin{frame}[t,fragile]{疎行列}
  \begin{itemize}
    \setlength{\itemsep}{1em}
  \item 疎行列: 非零の要素の数が非常に少ない行列
    \begin{itemize}
    \item 全ての要素($N^2$)を保存しておくのはメモリの無駄
    \item 行列ベクトル積の計算量(通常$O(N^2)$)も削減の余地あり
    \end{itemize}
  \item 疎行列の格納
    \begin{itemize}
    \item 三重対角行列(一次元ラプラシアンなど): 例) \href{https://github.com/todo-group/computer-experiments/blob/master/exercise/eigenvalue_problem/tridiagonal.c}{\texttt{tridiagonal.c}}

      対角成分+副対角成分を3本のベクトルに保存しておけばよい

      対称(エルミート)行列の場合は副対角はどちらか1本だけでよい

      LAPACKの三重対角行列用のソルバー(\href{https://www.netlib.org/lapack/explore-html/db/daa/group__stev_ga774cb10a577ccdd97546ef1c5599f8ee.html}{\texttt{DSTEV}}, \href{https://www.netlib.org/lapack/explore-html/d6/d46/group__gttrf_ga8d1e46216e6c861c89bd4328b8be52a1.html#ga8d1e46216e6c861c89bd4328b8be52a1}{\texttt{DGTTRF}}など)にはこの形式の行列を渡す

    \item 一般の疎行列: 例) \href{https://github.com/todo-group/computer-experiments/blob/master/exercise/eigenvalue_problem/sparse.c}{\texttt{sparse.c}}

      各行で非零の要素の場所(列)とその値をベクトルに保存する

      CRS (Compressed Row Storage)形式とも呼ばれる

    \item matfree形式: 例) \href{https://github.com/todo-group/computer-experiments/blob/master/exercise/eigenvalue_problem/matfree.c}{\texttt{matfree.c}}

      要素は保存せず、その場で非零の場所と要素を計算する

      FTCS法を行列形式を使わずに素直に実装するのと同じ

      横磁場イジング模型のハミルトニアンの掛け算などでも使える
    \end{itemize}
  \end{itemize}
\end{frame}
