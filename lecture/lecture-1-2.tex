% -*- coding: utf-8 -*-

\documentclass[10pt,dvipdfmx]{beamer}
\usepackage{tutorial}

\title{計算機実験I (第2回)}
\date{2021/04/21}

\begin{document}

\begin{frame}
  \titlepage
  \tableofcontents
\end{frame}

\begin{frame}[t]{本日の課題}
  \begin{itemize}
    %\setlength{\itemsep}{1em}
  \item (まだ完了していない人は「\href{https://utphys-comp.github.io}{計算機実験のための環境整備}」({\small \href{https://utphys-comp.github.io}{https://utphys-comp.github.io}})を参考に、各自必要な環境を引き続き整備する)
  \item 4月20日までに「計算機実験SSH公開鍵登録フォーム」から登録を行った人にはaiのアカウント作成・公開鍵の登録が完了済
    \begin{itemize}
    \item ユーザ名: 「u」の後に共通ID10桁 \\ 例: u1234567890
    \item aiのホームディレクトリに{\tt random}という名前のファイルが作成されている。その中の6桁の数字を「作業レポート」で報告
    \end{itemize}
  \item 「計算機実験I」実習課題: [数値課題・アルゴリズム基礎] に取り組む
  \item 講義終了後(18:00まで)にITC-LMSで「作業レポート(2021/04/21)」を提出
  \end{itemize}
\end{frame}

% -*- coding: utf-8 -*-

\section{ニュートン法}

\begin{frame}[t,fragile]{ニュートン法}
  \begin{itemize}
    \setlength{\itemsep}{1em}
  \item 反復法により方程式$f(x)=0$の解を求める
  \item 真の解を$x_0$、現在の解の候補を$x_n=x_0+\epsilon$とすると
    \[
    0 = f(x_0) = f(x_0+\epsilon-\epsilon) = f(x_n) - f'(x_n) \epsilon + O(\epsilon^2)
    \]
  \item 次の解の候補 (反復法、逐次近似法)
    \[
    \epsilon \approx \frac{f(x_n)}{f'(x_n)} \quad\quad x_{n+1} = x_n - \frac{f(x_n)}{f'(x_n)}
    \]
  \item 複素変数の複素関数や多変数の場合にも自然に拡張可
  \end{itemize}
\end{frame}

\input{newton-02.tex}
\begin{frame}[t,fragile]{多次元の場合}
  \begin{itemize}
    \setlength{\itemsep}{1em}
  \item $f(x)=0$: $d$次元(非線形)連立方程式
  \item $x$は$d$次元のベクトル: $x = {}^t(x_1,x_2,\cdots,x_d)$
  \item $f(x)$も$d$次元のベクトル: $f(x) = {}^t(f_1(x), f_2(x),\cdots,f_d(x))$
  \item 真の解のまわりでの展開 ($x_n = x_0 + \epsilon$)
    \[
    0 = f(x_0) = f(x_0+\epsilon-\epsilon) = f(x_n) - \frac{\partial f(x_n)}{\partial x} \cdot \epsilon + O(|\epsilon|^2)
    \]
  \item ヤコビ行列($d\times d$): $\displaystyle \Big[\frac{\partial f(x)}{\partial x}\Big]_{ij} = \frac{\partial f_i(x)}{\partial x_j}$
  \item 次の解の候補: $\displaystyle x_{n+1} = x_n - \Big[\frac{\partial f(x_n)}{\partial x}\Big]^{-1} f(x_n)$
  \end{itemize}
\end{frame}

\begin{frame}[t,fragile]{ニュートン法による最適化}
  \begin{itemize}
    \setlength{\itemsep}{1em}
  \item $x$は$d$次元のベクトル: $x = {}^t(x_1,x_2,\cdots,x_d)$、目的関数$f(x)$はスカラー
  \item 勾配ベクトル: $\displaystyle [\nabla f(x)]_i = \frac{\partial f(x)}{\partial x_i}$
  \item 極小値(最小値)となる条件: $\nabla f(x)=0$
  \item ニュートン法で$f(x)$を$\nabla f(x)$で置き換えればよい
  \item 次の解の候補: $\displaystyle x_{n+1} = x_n - H^{-1}(x_n) \nabla f(x_n)$
  \item ヘッセ行列(Hessian): $\displaystyle H_{ij}(x) = \frac{\partial^2 f}{\partial x_i \partial x_j}(x)$
  \end{itemize}
\end{frame}

\begin{frame}[t,fragile]{準ニュートン法}
  \begin{itemize}
    %\setlength{\itemsep}{1em}
  % \item ニュートン法では、ヘッセ行列の計算・保存が必要
  \item 準ニュートン法: それまでの反復で計算した勾配ベクトルから、ヘッセ行列の逆行列を近似($C_n \approx H^{-1}$)
  \begin{itemize}
    \item 極小点の近傍では
      \begin{align*}
          x-x_n &= -H^{-1} \cdot \nabla f(x_n) \\
          x-x_{n+1} &= -H^{-1} \cdot \nabla f(x_{n+1})
      \end{align*}
    \item 差をとると($s_n = x_{n+1} - x_n$, $y_n = \nabla f(x_{n+1}) - \nabla f(x_n)$)
      \begin{align*}
        s_n &= H^{-1} \cdot y_n
      \end{align*}
    \item この式が満たされるように、$C_{n+1}$を$C_{n}+(\text{補正})$の形で構成(DFP法: Davidon-Fletcher-Powell)
      \[
        C_{n+1} = C_{n} + \frac{s_n s_n^T}{s_n^T y_n} - \frac{C_{n} y_n (C_{n} y_n)^T}{y_n^T C_n y_n}
      \]
    \item 他にも、BFGS法など
    \end{itemize}
  \end{itemize}
\end{frame}

\begin{frame}[t,fragile]{計算をいつやめるか?}
  \begin{itemize}
    %\setlength{\itemsep}{1em}
  \item 残差による判定
    \[
    |f(x)| < \delta
    \]
  \item 誤差による判定
    \[
    | x_{n+1} - x_{n} | < \epsilon
    \]
  \item 解$x=x_0$が$m$重解の場合、$x=x_0$のまわりで展開すると
    \[
    f(x) \simeq \alpha (x-x_0)^m
    \]
    残差が$\delta$程度になったときの誤差は、$\delta^{1/m}$程度

    逆に$|x-x_0|$が$\delta^{1/m}$以下になると、$f(x)$の値がそれ以上変化しない $\Rightarrow$ $m$重解の精度は計算精度の$1/m$桁程度しかない

  \item 残差による判定と誤差による判定を併用するのがよい
  \end{itemize}
\end{frame}

\begin{frame}[t,fragile]{反復計算}
  \begin{itemize}
    %\setlength{\itemsep}{1em}
  \item {\tt while}による反復 (ハンドブック2.2.3節)
\begin{lstlisting}
double residual = 1;    /* 残差: 適当に大きな値 */
double error = 1;       /* 誤差: 適当に大きな値 */
double delta = 1.0e-12; /* 欲しい精度 */
while (residual > delta && error > delta) {
  /* ニュートン法の漸化式 */
  /* residual と error を計算 */
}
\end{lstlisting}
残差と誤差のどちらかが欲しい精度に達したら計算を終了
\item {\tt break}を使う例 (ハンドブック2.2.4節)
\begin{lstlisting}
for (;;) {
  /* ニュートン法の漸化式 */
  /* residual と error を計算 */
  if (residual < delta || error < delta) break;
}
\end{lstlisting}
  \end{itemize}
\end{frame}

\begin{frame}[t,fragile]{初期段階における収束の改善}
  \begin{itemize}
    %\setlength{\itemsep}{1em}
  \item Newton法は初期値によっては収束しない
  \item 発散や振動を抑える方法として「減速」が有効な場合も
  \item 減速
    \begin{itemize}
    \item 反復式を少し修正する
      \[
      x_{n+1} = x_n - \mu_n \frac{f(x_n)}{f'(x_n)}
      \]
    \item まずは$\mu_n=1$として計算
    \item $|f(x_{n+1})| < |f(x_{n})|$が成り立たないようであれば、$\mu_n$を半分にして再計算
    \item $\mu_n$が十分に小さくなれば、$|f(x)|$は必ず減少する
    \end{itemize}
  \end{itemize}
\end{frame}


% -*- coding: utf-8 -*-

\section{二分法}

\input{1_basics/bisection-01.tex}

% -*- coding: utf-8 -*-

\section{囲い込み法}

\begin{frame}[t,fragile]{囲い込み法(一次元の最適化)}
  \begin{itemize}
    %\setlength{\itemsep}{1em}
  \item 関数$f(x)$の極小点を求める
  \item $f(a) > f(b) < f(c)$を満たす3点の組$a < b < c$の領域を狭めていく
  \item $[a,b]$、$[b,c]$の広い方(例えば後者)を$b$から見て、黄金比
    [$1:(1+\sqrt{5})/2 \approx 0.382:0.618$]に内分する点を$x$とする
    \begin{itemize}
    \item $f(b) > f(x)$の場合: $[b,c]$を新しい領域にとる
    \item $f(b) < f(x)$の場合: $[a,x]$を新しい領域にとる
    \end{itemize}
  \item もともとの$b$が$[a,c]$を$0.382:0.618$に内分する点だった場合、
    新しい領域の幅は、どちらの場合も0.618
  \item 最初の比率が黄金比からずれていたとしても、黄金比に収束
  \item 黄金分割法(golden section)とも呼ばれる
  \end{itemize}
\end{frame}

\begin{frame}[t,fragile]{最初の囲い込み}
  \begin{itemize}
    % \setlength{\itemsep}{1em}
  \item 1点を選び、適当な$\Delta x$を取る
  \item 左右に$\Delta x$動かしてみて、関数値が小さくなる方へ動く
  \item どちらに進んでも関数値が大きくなる場合には、囲い込み完了
  \item 小さくなった場合、その方向へ再び増えるまで$\Delta x$を倍々に増やしながら進む
  \item 最後の3点で極小値を囲い込むことができる
  \item 囲い込み法のプログラムの例: \href{https://github.com/todo-group/computer-experiments/blob/master/exercise/optimization/golden_section.c}{golden\_section.c}
  \end{itemize}
\end{frame}



% -*- coding: utf-8 -*-

\section{常微分方程式の初期値問題}

\begin{frame}[t,fragile]{準備: 微分方程式の書き換え}
  \begin{itemize}
    %\setlength{\itemsep}{1em}
  \item 2階の常微分方程式の一般形
    \[
    \frac{d^2y}{dt^2} + p(t)\frac{dy}{dt} + q(t)y = r(t)
    \]
  \item $y_1 \equiv y$, $y_2 \equiv \frac{dy}{dt}$とおくと
    \[
    \left\{
    \begin{array}{ccl}
      \frac{dy_1}{dt} & = & y_2 \\
      \frac{dy_2}{dt} & = & r(t) - p(t) y_2 - q(t) y_1
    \end{array}
    \right.
    \]
  \item さらに$\bm{y}\equiv(y_1, y_2)$, $\bm{f}(t, \bm{y})\equiv \left(y_2, r(t)-p(t)y_2 - q(t)y_1\right)$とおくと
    \[
    \frac{d\bm{y}}{dt} = \bm{f}(t, \bm{y})
    \]
  \item $n$階常微分方程式 $\Rightarrow$ $n$次元の1階常微分方程式
  \end{itemize}
\end{frame}

\begin{frame}[t,fragile]{初期値問題と境界値問題}
  \begin{itemize}
    \setlength{\itemsep}{1em}
  \item 初期値問題
    \begin{itemize}
    \item 微分方程式において、ある1点に関する全ての境界条件(初期値)が与えられているもの
    \item 質点の運動など(時系列の問題)
  \end{itemize}
  \item 境界値問題
    \begin{itemize}
    \item 複数の点に関する境界条件が与えられているもの
    \item 物体のゆがみの計算や静電場の計算など(空間的に解く問題)
  \end{itemize}
  \item 初期値問題は初期値から逐次的に解くことが可能
  \item 境界値問題は初期値問題に比べて計算法が複雑
  \end{itemize}
\end{frame}

\begin{frame}[t,fragile]{初期値問題の解法 (Euler法)}
  \begin{itemize}
    %\setlength{\itemsep}{1em}
  \item $h$を微小量として微分を差分で近似する(前進差分)
    \[
    \frac{dy}{dt} \approx \frac{y(t+h) - y(t)}{h} = f(t, y)
    \]
  \item $t=0$における$y(t)$の初期値を$y_0$、$t_n \equiv nh$、$y_n$を$y(t_n)$の近似値とおくと、
    \[
    y_{n+1}-y_n = h f( t_n, y_n)
    \]
  \item Euler法
    \begin{itemize}
    \item $y_0$からはじめて、$y_1,y_2,\cdots$を順次求めていく
    \end{itemize}
  \end{itemize}
\end{frame}

\begin{frame}[t,fragile]{Euler法の精度}
  \begin{itemize}
    %\setlength{\itemsep}{1em}
  \item 微分方程式の両辺を$t_n$から$t_{n+1}$まで積分(積分方程式)
    \[
    y(t_{n+1}) - y(t_n) = \int^{t_{n+1}}_{t_n} \!\! f(t, y(t)) dt = h \int^1_0 \! f(t_n+h\tau, y(t_n+h\tau)) d\tau
    \]
  \item Euler法は、被積分関数を定数で近似することに対応
    \[
    f(t_n+h\tau, y(t_n+h\tau)) = f(t_n, y(t_n)) + O(h)
    \]
  \item $t=0$からある$t_f$まで積分すると、反復回数$N = t_f / h$
  \item $t=t_f$における誤差 $\sim N \times h \times O(h) = O(h)$
  \end{itemize}
\end{frame}

\begin{frame}[t,fragile]{Euler法の改良}
  \begin{itemize}
    %\setlength{\itemsep}{1em}
  \item 積分方程式の被積分関数をもう1次高次まで展開
    \[
    f(t_n+h\tau, y(t_n+h\tau)) = f(t_n, y(t_n)) +
    \tau h
    \left\{
    \frac{\partial f}{\partial t}
    + f \frac{\partial f}{\partial y}
    \right\}_{t=t_n, y=y_n}
    \!\!\!\!\!\!\!\!\!\!\!\! + O(h^2)
    \]
  \item 積分を実行すると
    \[
    y(t_{n+1}) = y(t_n) + h f(t_n, y_n) + \frac{1}{2}h^2
    \left\{
    \frac{\partial f}{\partial t}
    + f \frac{\partial f}{\partial y}
    \right\}_{t=t_n, y=y_n}
    \!\!\!\!\!\!\!\!\!\!\!\! + O(h^3)
    \]
  \end{itemize}
\end{frame}

\begin{frame}[t,fragile]{中点法(2次Runge-Kutta法)}
  \begin{itemize}
    %\setlength{\itemsep}{1em}
  \item 2次公式
    \[
    \begin{array}{rcl}
      k_1 & = & h f(t_n, y_n) \\
      k_2 & = & h f(t_n + \frac{1}{2}h, y_n + \frac{1}{2}k_1) \\
      y_{n+1} & = & y_n + k_2
    \end{array}
    \]
  \item このとき
    \[
    k_2 = h 
    \left\{
    f(t_n, y_n)
    + \frac{1}{2}h \frac{\partial f}{\partial t}
    + \frac{1}{2}k_1 \frac{\partial f}{\partial y}
    + O(h^2)
    \right\}
    \]
  \item したがって
    \[
    y_{n+1} = y_n + h f(t_n, y_n) + \frac{1}{2}h^2
    \left\{
    \frac{\partial f}{\partial t}
    + f \frac{\partial f}{\partial y}
    \right\}_{t=t_n, y=y_n}
    \!\!\!\!\!\!\!\!\!\!\!\!+ O(h^3)
    \]
  \end{itemize}
\end{frame}

\begin{frame}[t,fragile]{高次のRunge-Kutta法}
  \begin{itemize}
    %\setlength{\itemsep}{1em}
  \item 3次Runge-Kutta法
    \[
    \begin{array}{rcl}
      k_1 & = & h f(t_n, y_n) \\
      k_2 & = & h f(t_n + \frac{2}{3}h, y_n + \frac{2}{3}k_1) \\
      k_3 & = & h f(t_n + \frac{2}{3}h, y_n + \frac{2}{3}k_2) \\
      y_{n+1} & = & y_n + \frac{1}{4}k_1 + \frac{3}{8}k_2
      + \frac{3}{8}k_3
    \end{array}
    \]
  \item 4次Runge-Kutta法
    \[
    \begin{array}{rcl}
      k_1 & = & h f(t_n, y_n) \\
      k_2 & = & h f(t_n + \frac{1}{2}h, y_n + \frac{1}{2}k_1) \\
      k_3 & = & h f(t_n + \frac{1}{2}h, y_n + \frac{1}{2}k_2) \\
      k_4 & = & h f(t_n + h, y_n + k_3) \\
      y_{n+1} & = & y_n + \frac{1}{6}k_1 + \frac{1}{3}k_2
      + \frac{1}{3}k_3 + \frac{1}{6}k_4
    \end{array}
    \]
  \item 4次までは次数と$f$の計算回数が等しい
  \end{itemize}
\end{frame}

\begin{frame}[t,fragile]{計算コストと精度}
  \begin{itemize}
    \setlength{\itemsep}{1em}
  \item 実際の計算では$f(t,y)$の計算にほとんどのコストがかかる
  \item 計算回数と計算精度の関係
    \begin{center}
      \begin{tabular}[h]{c|cccc}
        & 1次(Euler法) & 2次(中点法) & 3次 & 4次 \\
        \hline
        計算精度 & $O(h)$ & $O(h^2)$ & $O(h^3)$ & $O(h^4)$ \\
        計算回数 & $N$ & $2N$ & $3N$ & $4N$
      \end{tabular}
    \end{center}
  \item 高次のRunge-Kuttaを使う方が効率的
  \item どれくらい小さな$h$が必要となるか、前もっては分からない
  \item 刻み幅を変えて($h,h/2,h/4,\dots$)計算してみることが大事
    \begin{itemize}
    \item 誤差の評価
    \item 公式の間違いの発見
    \end{itemize}
  \end{itemize}
\end{frame}


\section{陽解法と陰解法}

\begin{frame}[t,fragile]{陽解法と陰解法}
  \begin{itemize}
    %\setlength{\itemsep}{1em}
  \item 陽解法: 右辺が既知の変数のみで書かれる(例: Euler法)
    \begin{itemize}
    \item プログラムがシンプル
    \end{itemize}
  \item 陰解法: 右辺にも未知変数が含まれる
    \begin{itemize}
    \item 例: 逆Euler法
      \begin{align*}
        y(t) &= y(t+h-h) = y(t+h) - h f(t+h,y(t+h)) + O(h^2) \\
        y_{n+1} &= y_n + h f(t+h,{\color{red}y_{n+1}})
      \end{align*}
    \item 数値的により安定な場合が多い
    \item 一般的には、Newton法などを使って非線形方程式を解く必要がある
    \end{itemize}
  \end{itemize}
\end{frame}

\begin{frame}[t,fragile]{Euler法の安定性}
  \begin{itemize}
    %\setlength{\itemsep}{1em}
  \item 方程式$\displaystyle \frac{dy}{dt} = k y(t)$を初期条件$y(0)=1$のもとで解くと$y(t)=e^{kt}$
    \begin{itemize}
      \item ${\rm Re}\, k < 0$であれば、$\displaystyle \lim_{t\rightarrow \infty} y(t) = 0$となる
    \end{itemize}
  \item (陽的) Euler法
    \[
    y_{n+1} = y_n + h f(t_n,y_n) = y_n + h k y_n = (1+hk)y_n
    \]
    \begin{itemize}
    \item $\displaystyle \lim_{t\rightarrow \infty} y(t) = 0$となるための条件
      \[
      |  1 + hk | < 1
      \]
    \item $k$が負の実数であっても、$h > 2 / |k|$では発散 $\Rightarrow$ 不安定
    \end{itemize}
  \end{itemize}
\end{frame}

\begin{frame}[t,fragile]{陰解法の安定性}
  \begin{itemize}
    %\setlength{\itemsep}{1em}
  \item (陰的) 逆Euler法
    \begin{align*}
    y_{n+1} &= y_n + h f(t_n,y_{n+1}) = y_n + h k y_{n+1} \\
    y_{n+1} &= \frac{1}{1-hk} y_n
    \end{align*}
    \begin{itemize}
    \item $\displaystyle \lim_{t\rightarrow \infty} y(t) = 0$となるための条件
      \[
      |  1 - hk | > 1
      \]
    \item $k$の実部が負であれば、常に$\displaystyle \lim_{t\rightarrow \infty} y(t) = 0$
    \item 真の解がゼロに収束する$k$の全領域において数値解も収束

      $\Rightarrow$ 「A安定」という
    \end{itemize}
  \end{itemize}
\end{frame}



\begin{frame}[t]{今後の予定}
  \begin{itemize}
    % \setlength{\itemsep}{1em}
  \item 全8回 (水曜2限 10:25-11:55)
    \begin{itemize}
    \item {\color{gray} 4月7日 第1回: 環境整備・数値誤差}
    \item {\color{gray} 4月14日 もくもく会}
    \item {\color{gray} 4月21日 第2回: ニュートン法・二分法・常微分方程式}
    \item 4月28日 第3回: 固有値問題・シンプレクティック積分法
    \item 5月12日 もくもく会
    \item 5月19日 第4回: 行列演算・複素数・ライブラリ [レポートNo.1締切]
    \item 5月26日 第5回: 連立一次方程式・直接解法・反復解法
    \item 6月2日 もくもく会
    \item 6月9日 第6回: 行列の対角化 [レポートNo.2締切]
    \item 6月16日 第7回: 疎行列に対する反復法・変分法
    \item 6月23日 第8回: 特異値分解・最小二乗法
    \item 7月7日 [レポートNo.3締切]
     \end{itemize}
  \end{itemize}
\end{frame}

\end{document}
