% -*- coding: utf-8 -*-

\documentclass[10pt,dvipdfmx]{beamer}
\usepackage{tutorial}

\title{計算機実験I (講義2)}
\date{2022/04/20}

\begin{document}

\begin{frame}
  \titlepage
  \tableofcontents
  出席: 本日の18時までにITC-LMSでアンケートに回答
\end{frame}

% -*- coding: utf-8 -*-

\section{常微分方程式の初期値問題}

\begin{frame}[t,fragile]{準備: 微分方程式の書き換え}
  \begin{itemize}
    %\setlength{\itemsep}{1em}
  \item 2階の常微分方程式の一般形
    \[
    \frac{d^2y}{dt^2} + p(t)\frac{dy}{dt} + q(t)y = r(t)
    \]
  \item $y_1 \equiv y$, $y_2 \equiv \frac{dy}{dt}$とおくと
    \[
    \left\{
    \begin{array}{ccl}
      \frac{dy_1}{dt} & = & y_2 \\
      \frac{dy_2}{dt} & = & r(t) - p(t) y_2 - q(t) y_1
    \end{array}
    \right.
    \]
  \item さらに$\bm{y}\equiv(y_1, y_2)$, $\bm{f}(t, \bm{y})\equiv \left(y_2, r(t)-p(t)y_2 - q(t)y_1\right)$とおくと
    \[
    \frac{d\bm{y}}{dt} = \bm{f}(t, \bm{y})
    \]
  \item $n$階常微分方程式 $\Rightarrow$ $n$次元の1階常微分方程式
  \end{itemize}
\end{frame}

\begin{frame}[t,fragile]{初期値問題と境界値問題}
  \begin{itemize}
    \setlength{\itemsep}{1em}
  \item 初期値問題
    \begin{itemize}
    \item 微分方程式において、ある1点に関する全ての境界条件(初期値)が与えられているもの
    \item 質点の運動など(時系列の問題)
  \end{itemize}
  \item 境界値問題
    \begin{itemize}
    \item 複数の点に関する境界条件が与えられているもの
    \item 物体のゆがみの計算や静電場の計算など(空間的に解く問題)
  \end{itemize}
  \item 初期値問題は初期値から逐次的に解くことが可能
  \item 境界値問題は初期値問題に比べて計算法が複雑
  \end{itemize}
\end{frame}

\begin{frame}[t,fragile]{初期値問題の解法 (Euler法)}
  \begin{itemize}
    %\setlength{\itemsep}{1em}
  \item $h$を微小量として微分を差分で近似する(前進差分)
    \[
    \frac{dy}{dt} \approx \frac{y(t+h) - y(t)}{h} = f(t, y)
    \]
  \item $t=0$における$y(t)$の初期値を$y_0$、$t_n \equiv nh$、$y_n$を$y(t_n)$の近似値とおくと、
    \[
    y_{n+1}-y_n = h f( t_n, y_n)
    \]
  \item Euler法
    \begin{itemize}
    \item $y_0$からはじめて、$y_1,y_2,\cdots$を順次求めていく
    \end{itemize}
  \end{itemize}
\end{frame}

\begin{frame}[t,fragile]{Euler法の精度}
  \begin{itemize}
    %\setlength{\itemsep}{1em}
  \item 微分方程式の両辺を$t_n$から$t_{n+1}$まで積分(積分方程式)
    \[
    y(t_{n+1}) - y(t_n) = \int^{t_{n+1}}_{t_n} \!\! f(t, y(t)) dt = h \int^1_0 \! f(t_n+h\tau, y(t_n+h\tau)) d\tau
    \]
  \item Euler法は、被積分関数を定数で近似することに対応
    \[
    f(t_n+h\tau, y(t_n+h\tau)) = f(t_n, y(t_n)) + O(h)
    \]
  \item $t=0$からある$t_f$まで積分すると、反復回数$N = t_f / h$
  \item $t=t_f$における誤差 $\sim N \times h \times O(h) = O(h)$
  \end{itemize}
\end{frame}

\begin{frame}[t,fragile]{Euler法の改良}
  \begin{itemize}
    %\setlength{\itemsep}{1em}
  \item 積分方程式の被積分関数をもう1次高次まで展開
    \[
    f(t_n+h\tau, y(t_n+h\tau)) = f(t_n, y(t_n)) +
    \tau h
    \left\{
    \frac{\partial f}{\partial t}
    + f \frac{\partial f}{\partial y}
    \right\}_{t=t_n, y=y_n}
    \!\!\!\!\!\!\!\!\!\!\!\! + O(h^2)
    \]
  \item 積分を実行すると
    \[
    y(t_{n+1}) = y(t_n) + h f(t_n, y_n) + \frac{1}{2}h^2
    \left\{
    \frac{\partial f}{\partial t}
    + f \frac{\partial f}{\partial y}
    \right\}_{t=t_n, y=y_n}
    \!\!\!\!\!\!\!\!\!\!\!\! + O(h^3)
    \]
  \end{itemize}
\end{frame}

\begin{frame}[t,fragile]{中点法(2次Runge-Kutta法)}
  \begin{itemize}
    %\setlength{\itemsep}{1em}
  \item 2次公式
    \[
    \begin{array}{rcl}
      k_1 & = & h f(t_n, y_n) \\
      k_2 & = & h f(t_n + \frac{1}{2}h, y_n + \frac{1}{2}k_1) \\
      y_{n+1} & = & y_n + k_2
    \end{array}
    \]
  \item このとき
    \[
    k_2 = h 
    \left\{
    f(t_n, y_n)
    + \frac{1}{2}h \frac{\partial f}{\partial t}
    + \frac{1}{2}k_1 \frac{\partial f}{\partial y}
    + O(h^2)
    \right\}
    \]
  \item したがって
    \[
    y_{n+1} = y_n + h f(t_n, y_n) + \frac{1}{2}h^2
    \left\{
    \frac{\partial f}{\partial t}
    + f \frac{\partial f}{\partial y}
    \right\}_{t=t_n, y=y_n}
    \!\!\!\!\!\!\!\!\!\!\!\!+ O(h^3)
    \]
  \end{itemize}
\end{frame}

\begin{frame}[t,fragile]{高次のRunge-Kutta法}
  \begin{itemize}
    %\setlength{\itemsep}{1em}
  \item 3次Runge-Kutta法
    \[
    \begin{array}{rcl}
      k_1 & = & h f(t_n, y_n) \\
      k_2 & = & h f(t_n + \frac{2}{3}h, y_n + \frac{2}{3}k_1) \\
      k_3 & = & h f(t_n + \frac{2}{3}h, y_n + \frac{2}{3}k_2) \\
      y_{n+1} & = & y_n + \frac{1}{4}k_1 + \frac{3}{8}k_2
      + \frac{3}{8}k_3
    \end{array}
    \]
  \item 4次Runge-Kutta法
    \[
    \begin{array}{rcl}
      k_1 & = & h f(t_n, y_n) \\
      k_2 & = & h f(t_n + \frac{1}{2}h, y_n + \frac{1}{2}k_1) \\
      k_3 & = & h f(t_n + \frac{1}{2}h, y_n + \frac{1}{2}k_2) \\
      k_4 & = & h f(t_n + h, y_n + k_3) \\
      y_{n+1} & = & y_n + \frac{1}{6}k_1 + \frac{1}{3}k_2
      + \frac{1}{3}k_3 + \frac{1}{6}k_4
    \end{array}
    \]
  \item 4次までは次数と$f$の計算回数が等しい
  \end{itemize}
\end{frame}

\begin{frame}[t,fragile]{計算コストと精度}
  \begin{itemize}
    \setlength{\itemsep}{1em}
  \item 実際の計算では$f(t,y)$の計算にほとんどのコストがかかる
  \item 計算回数と計算精度の関係
    \begin{center}
      \begin{tabular}[h]{c|cccc}
        & 1次(Euler法) & 2次(中点法) & 3次 & 4次 \\
        \hline
        計算精度 & $O(h)$ & $O(h^2)$ & $O(h^3)$ & $O(h^4)$ \\
        計算回数 & $N$ & $2N$ & $3N$ & $4N$
      \end{tabular}
    \end{center}
  \item 高次のRunge-Kuttaを使う方が効率的
  \item どれくらい小さな$h$が必要となるか、前もっては分からない
  \item 刻み幅を変えて($h,h/2,h/4,\dots$)計算してみることが大事
    \begin{itemize}
    \item 誤差の評価
    \item 公式の間違いの発見
    \end{itemize}
  \end{itemize}
\end{frame}


\section{陽解法と陰解法}

\begin{frame}[t,fragile]{陽解法と陰解法}
  \begin{itemize}
    %\setlength{\itemsep}{1em}
  \item 陽解法: 右辺が既知の変数のみで書かれる(例: Euler法)
    \begin{itemize}
    \item プログラムがシンプル
    \end{itemize}
  \item 陰解法: 右辺にも未知変数が含まれる
    \begin{itemize}
    \item 例: 逆Euler法
      \begin{align*}
        y(t) &= y(t+h-h) = y(t+h) - h f(t+h,y(t+h)) + O(h^2) \\
        y_{n+1} &= y_n + h f(t+h,{\color{red}y_{n+1}})
      \end{align*}
    \item 数値的により安定な場合が多い
    \item 一般的には、Newton法などを使って非線形方程式を解く必要がある
    \end{itemize}
  \end{itemize}
\end{frame}

\begin{frame}[t,fragile]{Euler法の安定性}
  \begin{itemize}
    %\setlength{\itemsep}{1em}
  \item 方程式$\displaystyle \frac{dy}{dt} = k y(t)$を初期条件$y(0)=1$のもとで解くと$y(t)=e^{kt}$
    \begin{itemize}
      \item ${\rm Re}\, k < 0$であれば、$\displaystyle \lim_{t\rightarrow \infty} y(t) = 0$となる
    \end{itemize}
  \item (陽的) Euler法
    \[
    y_{n+1} = y_n + h f(t_n,y_n) = y_n + h k y_n = (1+hk)y_n
    \]
    \begin{itemize}
    \item $\displaystyle \lim_{t\rightarrow \infty} y(t) = 0$となるための条件
      \[
      |  1 + hk | < 1
      \]
    \item $k$が負の実数であっても、$h > 2 / |k|$では発散 $\Rightarrow$ 不安定
    \end{itemize}
  \end{itemize}
\end{frame}

\begin{frame}[t,fragile]{陰解法の安定性}
  \begin{itemize}
    %\setlength{\itemsep}{1em}
  \item (陰的) 逆Euler法
    \begin{align*}
    y_{n+1} &= y_n + h f(t_n,y_{n+1}) = y_n + h k y_{n+1} \\
    y_{n+1} &= \frac{1}{1-hk} y_n
    \end{align*}
    \begin{itemize}
    \item $\displaystyle \lim_{t\rightarrow \infty} y(t) = 0$となるための条件
      \[
      |  1 - hk | > 1
      \]
    \item $k$の実部が負であれば、常に$\displaystyle \lim_{t\rightarrow \infty} y(t) = 0$
    \item 真の解がゼロに収束する$k$の全領域において数値解も収束

      $\Rightarrow$ 「A安定」という
    \end{itemize}
  \end{itemize}
\end{frame}



\section{Numerov法}

\begin{frame}[t,fragile]{Numerov法}
  \begin{itemize}
    %\setlength{\itemsep}{1em}
  \item Numerov法
    \begin{itemize}
    \item 二階の常微分方程式で一階の項がない場合に使える
    \item 連立微分方程式に直さずに直接二階微分方程式を解く
    \item 4次の陰解法
    \item 方程式が線形の場合は陽解法に書き直せる
    \end{itemize}
  \item 微分方程式
    \[
    \frac{d^2y}{dt^2} = f(t,y)
    \]
  $y=y(t)$を$t=t_n$のまわりでテイラー展開する。$t_{n \pm 1} = t_n \pm h$での表式は
      \[
      y(t_{n \pm 1}) = y(t_n) \pm h y'(t_n) + \frac{h^2}{2} y''(t_n) \pm \frac{h^3}{6} y'''(t_n) + \frac{h^4}{24} y''''(t_n)  + O(h^5)
      \]
  \end{itemize}
\end{frame}

\begin{frame}[t,fragile]{Numerov法}
  \begin{itemize}
    %\setlength{\itemsep}{1em}
  \item 二階微分の差分近似 ($y_i \equiv y(x_i)$等と書く)
    \[
    \frac{y_{i+1} - 2 y_i + y_{i-1}}{h^2} = y''_{i} + \frac{h^2}{12} y''''_{i} + O(h^4)
    \]
  一方で、微分方程式より
    \[
    y''''_i = \frac{d^2f}{dx^2}\Big|_{x=x_i} = \frac{f_{i+1}-2f_i+f_{i-1}}{h^2} + O(h^2)
    \]
    組み合わせると
    \[
    y_{i+1} = 2y_i - y_{i-1} + \frac{h^2}{12} (f_{i+1} + 10f_{i} + f_{i-1}) + O(h^6)
    \]
  \end{itemize}
\end{frame}

\begin{frame}[t,fragile]{Numerov法}
  \begin{itemize}
    %\setlength{\itemsep}{1em}
  \item 方程式が線形の場合、$f(t,y) = -a(t) y(t)$を代入すると
    \[
    y_{n+1} = 2y_n - y_{n-1} - \frac{h^2}{12} (a_{n+1}y_{n+1} + 10a_{n}y_{n} + a_{n-1}y_{n-1}) + O(h^6)
    \]
  $y_{n+1}$を左辺に集めると、陽解法となる
    \[
    y_{n+1} = \frac{2 (1-\frac{5h^2}{12} a_n)y_n - (1 + \frac{h^2}{12} a_{n-1}) y_{n-1}}{1 + \frac{h^2}{12} a_{n+1}} + O(h^6)
    \]
  \end{itemize}
\end{frame}


\section{固有値問題}

\begin{frame}[t,fragile]{時間依存しないシュレディンガー方程式}
  \begin{itemize}
    %\setlength{\itemsep}{1em}
  \item 井戸型ポテンシャル中の一粒子問題
    \begin{align*}
      \big[ -\frac{\hbar^2}{2m}\frac{d^2}{dx^2} + V(x) \big] \psi(x) = E \psi(x) \\
      V(x) = \begin{cases}
        0 & \text{$a \le x \le b$} \\ \infty & \text{otherwise}
      \end{cases}
    \end{align*}
  \item $\hbar^2/2m = 1$、$a=0$、$b=1$となるように変数変換して
    \begin{align*}
      \big( \frac{d^2}{dx^2} + E \big) \psi(x) = 0 \qquad 0 \le x \le 1
    \end{align*}
    を境界条件$\psi(0) = \psi(1) = 0$のもとで解けば良い
  \end{itemize}
\end{frame}

\begin{frame}[t,fragile]{固有値問題の解法(シューティング)}
  \begin{itemize}
    %\setlength{\itemsep}{1em}
  \item $x_i=h \times i$ ($h=1/n$)、$x_0=0$、$x_n=1$とする
  \item $\psi(x_0)=0$、$\psi(x_1) = 1$を仮定 ($\psi'(x_0)=1/h$と与えたことに相当)
  \item $E = 0$とおく
  \item Runge-Kutta法、Numerov法などを用いて$x=x_n$まで積分
  \item $\psi(x_n)$の符号がかわるまで、$E$を少しずつ増やす
  \item 符号が変わったら、$E$の区間を半分ずつに狭めていき、$\psi(x_n)=0$となる$E$ (固有エネルギー)と$\psi(x)$ (波動関数)を得る
  \end{itemize}
\end{frame}


\section{シンプレクティック積分法}

\begin{frame}[t,fragile]{ハミルトン力学系}
  \begin{itemize}
    % \setlength{\itemsep}{1em}
  \item 時間をあらわに含まない場合のハミルトン方程式
    \[
    \frac{dq}{dt} = \frac{\partial H}{\partial p}, \ \frac{dp}{dt} = -\frac{\partial H}{\partial q}
    \]
    \begin{itemize}
    \item エネルギー保存則
      \[
      \frac{dH}{dt} = \frac{\partial H}{\partial q} \frac{dq}{dt} + \frac{\partial H}{\partial p} \frac{dp}{dt} = 0
      \]
    \item 位相空間の体積が保存(Liouvilleの定理)

      位相空間上の流れの場$\bm{v} = (\frac{dq}{dt},\frac{dp}{dt})$について
      \[
      \text{div} \bm{v} = \frac{\partial}{\partial q} \frac{dq}{dt} + \frac{\partial}{\partial p} \frac{dp}{dt} = 0
      \]
    \end{itemize}
  \item Euler法、Runge-Kutta法などはいずれの性質も満たさない
  \end{itemize}
\end{frame}

\begin{frame}[t,fragile]{シンプレクティック数値積分法(Symplectic Integrator)}
  \begin{itemize}
    %\setlength{\itemsep}{1em}
  \item 体積保存を満たす解法
  \item 例: 調和振動子$H=\frac{1}{2}(p^2+q^2)$の運動方程式
    \[
    \frac{dq}{dt} = p, \ \frac{dp}{dt} = -q
    \]
    の一方をEuler法で、他方を逆オイラー法で解く
    \begin{align*}
      q_{n+1} &= q_n + h p_n \\
      p_{n+1} &= p_n - h q_{n+1} = (1-h^2) p_n - h q_n \\
      \begin{pmatrix} q_{n+1} \\ p_{n+1} \end{pmatrix} &= \begin{pmatrix} 1 & h \\ -h & 1-h^2 \end{pmatrix} \begin{pmatrix} q_{n} \\ p_{n} \end{pmatrix}
    \end{align*}
  \end{itemize}
\end{frame}

\begin{frame}[t,fragile]{体積・エネルギーの保存}
  \begin{itemize}
    %\setlength{\itemsep}{1em}
  \item 体積保存
    \begin{align*}
      \det \begin{pmatrix} 1 & h \\ -h & 1-h^2 \end{pmatrix} = 1
    \end{align*}
  \item エネルギーの保存
    \begin{align*}
      \frac{1}{2}(p_{n+1}^2+q_{n+1}^2) + {\color{red}\frac{h}{2} p_{n+1} q_{n+1}} = \frac{1}{2}(p_{n}^2+q_{n}^2) + {\color{red}\frac{h}{2} p_{n} q_{n}}
    \end{align*}
  \item 位相空間の体積は厳密に保存
  \item エネルギーは$O(h)$の範囲で保存し続ける
  \end{itemize}
\end{frame}

\begin{frame}[t,fragile]{2次のシンプレクティック積分法}
  \begin{itemize}
    \setlength{\itemsep}{1em}
  \item ハミルトニアンが$H(p,q) = T(p) + V(q)$の形で書けるとする
  \item リープ・フロッグ法
    \begin{align*}
      {\color{red} p(t+h/2)} &= p(t) - \frac{h}{2} \frac{\partial V(q)}{\partial q}|_{q=q(t)} \\
      {\color{blue} q(t+h)} &= q(t) + h {\color{red}p(t+h/2)} \\
      p(t+h) &= {\color{red}p(t+h/2}) - \frac{h}{2} \frac{\partial V(q)}{\partial q}|_{q=q(t+h)}
    \end{align*}
  \end{itemize}
\end{frame}

\begin{frame}[t,fragile]{シンプレクティック積分法}
  \begin{itemize}
    %\setlength{\itemsep}{1em}
  \item ハミルトン力学系の満たすべき特性(位相空間の体積保存)を満たす
  \item 一般的には陰解法
  \item ハミルトニアンが$H(p,q) = T(p) + V(q)$の形で書ける場合は陽的なシンプレクティック積分法が存在する
  \item エネルギーは近似的に保存する
  \item $n$次のシンプレクティック積分法では、エネルギーは$O(h^n)$の範囲で振動(発散しない)
  \item より高次のシンプレクティック積分法についても、システマティックに構成できる(ただし係数を解析的に求められるのは4次まで)。
    参考文献: H. Yoshida, Phys. Lett. A {\bf 150}, 262 (1990)
  \end{itemize}
  \begin{itemize}
    \item 系の満たすべき保存則を満たすように差分法を構成することが重要
    \item 参考: 計算科学フォーラムでの河合宗司先生(東北大)の講演 \url{https://hpcic-kkf.com/forum/2024/kkf_01/}
  \end{itemize}
\end{frame}

\begin{frame}[t,fragile]{ベルレ(Verlet)法}
  \begin{itemize}
    %\setlength{\itemsep}{1em}
  \item 微分方程式が$\ddot{q}(t) = f[t,q(t)]$の形で書かれる場合を考える
  \item $q(t)$のテイラー展開
    \[
    q(t \pm h) = q(t) \pm h \dot{q}(t) + \frac{h^2}{2} \ddot{q}(t) \pm \cdots
    \]
    から、$q(t + h)$と$q(t - h)$の表式を足し合わせて整理
    \[
    q(t+h) = 2q(t) - q(t-h)+h^2 f[t,q(t)] + O(h^4)
    \]
  \item 初期条件: $q(0)$と$q(0)-q(-1) \simeq hp(0)$から$q(h)$を求める
    \[
    q(h) = q(0) + h p(0) + \frac{h^2}{2} f[0, q(0)]
    \]
  \item 運動量は中心差分で計算
    \[
    p(t) = \frac{q(t+h) - q(t-h)}{2h}
    \]
  \end{itemize}
\end{frame}

\begin{frame}[t,fragile]{ベルレ法の変形}
  \begin{itemize}
    %\setlength{\itemsep}{1em}
  \item 半整数ステップにおける運動量
    \[
    p(t + h/2) = \frac{q(t+h)-q(t)}{h}
    \]
    を導入すると
    \begin{align*}
      \begin{split}
        p(t + h/2) &= p(t - h/2) + \frac{q(t+h)-2q(t)+q(t-h)}{h} \\
        &= p(t - h/2) + h f[t,q(t)]
      \end{split}
    \end{align*}
    一方
    \[
    q(t+h) = q(t) + hp(t+h/2)
    \]
  \item $\Rightarrow$ リープ・フロッグ法

    座標についてはベルレ法と数学的に等価だが、丸め誤差に強い
  \end{itemize}
\end{frame}

\begin{frame}[t,fragile]{速度ベルレ法(Velocity Verlet)}
  \begin{itemize}
    %\setlength{\itemsep}{1em}
  \item 運動量に対する中心差分
    \[
    p(t) = \frac{q(t+h)-q(t-h)}{2h}
    \]
    から
    \[
    q(t+h) = 2hp(t) + q(t-h)
    \]
    ベルレ法の式と和をとると
    \begin{align*}
      q(t + h) &= q(t) + hp(t) + h^2 f[t,q(t)] / 2 \\
      p(t + h) &= p(t) + h \Big( f[t+h,q(t+h)]  + f[t,q(t)] \Big) / 2
    \end{align*}
  \item $\Rightarrow$ 速度ベルレ法

    2次のシンプレクティック積分法、丸め誤差にも安定
  \end{itemize}
\end{frame}

\begin{frame}[t,fragile]{シンプレクティック法の一般論}
  \begin{itemize}
    %\setlength{\itemsep}{1em}
  \item $z=\begin{bmatrix} q \\ p \end{bmatrix}$と書き、$f(z)$に作用する演算子$\hat{D}(h)$をポアソン括弧を用いて定義($h=h(z)$はパラメータ)
    \[
    \hat{D}(h)f = \{ f, h \} = \frac{\partial f}{\partial q} \frac{\partial h}{\partial p} - \frac{\partial h}{\partial q} \frac{\partial f}{\partial p}
    \]
  \item ハミルトニアン$H=T(p)+V(q)$に対する正準方程式
    \[
    \dot{z} = \hat{D}(H) z
    \]
    に対する形式解
    \[
    z(t+h) = e^{h \hat{D}(H)} z(t)
    \]
  \item $e^{h \hat{D}(H)}$: 時間発展演算子 (正準変換)
  \end{itemize}
\end{frame}

\begin{frame}[t,fragile]{シンプレクティック法の一般論}
  \begin{itemize}
    %\setlength{\itemsep}{1em}
  \item $H=T(p)$の時
    \begin{align*}
      \hat{D}(T)z &= \frac{\partial z}{\partial q} \frac{\partial T}{\partial p} - \frac{\partial T}{\partial q} \frac{\partial z}{\partial p} = \begin{bmatrix} \frac{dT}{dp} \\ 0 \end{bmatrix} \\
      \hat{D}(T)^n z &= 0 \qquad n \ge 2 \\
      \therefore e^{h\hat{D}(T)} z &= [ 1+h\hat{D}(T) ] z = \begin{bmatrix} q + h \frac{dT}{dp} \\ p \end{bmatrix}
    \end{align*}
  \item $H=V(q)$の時も同様に
    \begin{align*}
      e^{h\hat{D}(V)} z &= [ 1+h\hat{D}(V) ] z = \begin{bmatrix} q \\ p - h \frac{dV}{dq} \end{bmatrix}
    \end{align*}
  \item $e^{h\hat{D}(T)}$, $e^{h\hat{D}(V)}$とも正準変換
  \item 一般的には、$e^{h\hat{D}(T+V)} z$は厳密には計算できない
  \end{itemize}
\end{frame}

\begin{frame}[t,fragile]{シンプレクティック法の一般論}
  \begin{itemize}
    %\setlength{\itemsep}{1em}
  \item $e^{h\hat{D}(T+V)}$を$e^{a_i h\hat{D}(T)}$, $e^{b_i h\hat{D}(V)}$の積で近似する
  \item 演算子(行列) $A,B$について
    \[
    e^{h (A+B)} = e^{hA} e^{hB} + O(h^2)
    \]
  \item 一次のシンプレクティック法
    \begin{align*}
      \hat{S}_1 &= e^{h\hat{D}(V)} e^{h\hat{D}(T)} \\
      q(t+h) &= q(t) + h \frac{dT(p(t))}{dp} \\
      p(t+h) &= p(t) - h \frac{dV(q({\color{red}t+h}))}{dq}
    \end{align*}
    $\Rightarrow$ オイラー法+逆オイラー法
  \end{itemize}
\end{frame}

\begin{frame}[t,fragile]{シンプレクティック法の一般論}
  \begin{itemize}
    %\setlength{\itemsep}{1em}
  \item 二次のシンプレクティック法
    \[
    e^{h (A+B)} = e^{\frac{h}{2}B} e^{hA} e^{\frac{h}{2}B} + O(h^3)
    \]
    から
    \begin{align*}
      \hat{S}_2 &= e^{\frac{h}{2}\hat{D}(T)} e^{h\hat{D}(V)} e^{\frac{h}{2}\hat{D}(T)}
    \end{align*}
    $\Rightarrow$ リープ・フロッグ法
  \item 四次のシンプレクティック法 (吉田の方法)
    \begin{align*}
      \hat{S}_4 &= e^{c_1h\hat{D}(T)} e^{d_1 h\hat{D}(V)} e^{c_2 h\hat{D}(T)} e^{d_2 h\hat{D}(V)} e^{c_3h\hat{D}(T)} e^{d_3 h\hat{D}(V)} e^{c_4h\hat{D}(T)} \\
      & c_1 = c_4 = \frac{1}{2(2-2^{1/3})}, \ c_2 = c_3 = \frac{1-2^{1/3}}{2(2-2^{1/3})}, \\
      & d_1 = d_3 = \frac{1}{2-2^{1/3}}, \ d_2 = \frac{2^{1/3}}{2-2^{1/3}}
    \end{align*}
  \end{itemize}
\end{frame}


\begin{frame}[t]{講義日程(予定)}
  \begin{itemize}
    % \setlength{\itemsep}{1em}
  \item 全8回 (水曜2限 10:25-11:55)
    \begin{itemize}
    \item 4月6日 講義1: 講義の概要・基本的なアルゴリズム
    \item 4月13日 演習1: 環境整備・C言語プログラミング・図のプロット
    \item 4月20日 講義2: 常微分方程式
    \item {\color{red} 4月27日 演習2 (グループ1):} 基本的なアルゴリズム・常微分方程式
    \item {\color{red} 5月11日 演習2 (グループ2):} 基本的なアルゴリズム・常微分方程式
    \item {\color{gray} 5月18日 休講}
    \item 5月25日 講義3: 連立方程式
    \item 6月1日 演習3 (グループ1): 連立方程式
    \item 6月8日 演習3 (グループ2): 連立方程式
    \item {\color{gray} 6月15日 休講}
    \item {\color{gray} 6月22日 休講}
    \item 6月29日 講義4: 行列の対角化
    \item 7月6日 演習4 (グループ1): 行列の対角化
    \item 7月13日 演習4 (グループ2): 行列の対角化
    \end{itemize}
  \item 2回目以降の演習はクラスを2グループに分けて実施(グループ1: 学生証番号が奇
数、グループ2: 偶数)
  \end{itemize}
\end{frame}

\end{document}
