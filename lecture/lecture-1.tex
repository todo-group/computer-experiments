\documentclass[dvipdfmx]{beamer}
\usepackage{tutorial}

\title{計算機実験(L1) --- 講義・実習の概要}
\date{2016/04/06}

\begin{document}

\begin{frame}
  \titlepage
  \tableofcontents
\end{frame}

\section{講義・実習の概要}

\begin{frame}[t]{講義・実習の目的}
  \begin{itemize}
    %\setlength{\itemsep}{1em}
  \item 理論・実験を問わず、学部〜大学院〜で必要となる現代的かつ普遍的な計算機の素養を身につける
  \item UNIX環境に慣れる(シェル、ファイル操作、エディタ)
  \item ネットワークの活用 (リモートログイン、バージョン管理、共同作業)
  \item プログラムの作成(C言語、コンパイラ、プログラム実行)
  \item 基本的な数値計算アルゴリズム・数値計算の常識を学ぶ
  \item 科学技術文書作成に慣れる(\LaTeX, グラフ作成)
  \item (Mathematica (数式処理)、Python (スクリプト言語)等の利用)
  \end{itemize}
\end{frame}

\begin{frame}[t]{身に付けて欲しいこと}
  \begin{itemize}
    %\setlength{\itemsep}{1em}
  \item ツールとしてないものは自分で作る (物理の伝統)
  \item すでにあるものは積極的に再利用する (車輪の再発明をしない)
  \item 数学公式と数値計算アルゴリズムは別物
  \item 刻み幅・近似度合いを変えて何度か計算を行う
  \item グラフ化して目で見てみる
  \item 計算量(コスト)のスケーリング(次数)に気をつける
  \item 記録に残す・再現性を確保する
  \end{itemize}
\end{frame}

\begin{frame}[t]{講義・実習内容}
  \begin{itemize}
    % \setlength{\itemsep}{1em}
  \item UNIX操作・ネットワーク
  \item プログラミング: C言語、数値計算ライブラリの利用
  \item ツール: エディタ、コンパイラ、\LaTeX、Gnuplot、バージョン管理システム
  \item 数値計算の基礎
  \item 常微分方程式の解法
  \item 連立一次方程式の解法
  \item 行列の対角化
  \item 線形回帰・カーネル法
  \item モンテカルロ法
  \item 最適化問題
  \end{itemize}
\end{frame}

\begin{frame}[t,fragile]{講義と実習}
  \begin{itemize}
    %\setlength{\itemsep}{1em}
  \item 講義 (2グループに分けて交互に)
    \begin{itemize}
    \item ツールなどの基礎知識: 実習の準備
    \item アルゴリズムの基本概念: 実習の準備
    \item より高度なアルゴリズムの紹介
    \end{itemize}
  \item 実習 (2グループに分けて交互に)
    \begin{itemize}
    \item 練習課題: 各自で取り組む
    \item レポート課題: 個別レポートとして提出
    \item 応用課題: 余裕のある人は積極的に取り組む。グループワークの課題例
    \end{itemize}
  \item スタッフ \href{mailto:computer@exa.phys.s.u-tokyo.ac.jp}{computer@exa.phys.s.u-tokyo.ac.jp}
    \begin{itemize}
    \item 講義: 藤堂
    \item 実習: 諏訪(藤堂研)、鈴木(早野研)
    \item 実習TA: 安藤(早野研M2)、島垣(藤堂研M2)
    \end{itemize}
  \end{itemize}    
\end{frame}

\begin{frame}[t,fragile]{評価方法・レポート}
  \begin{itemize}
    %\setlength{\itemsep}{1em}
  \item 評価
    \begin{itemize}
    \item 出席(講義・実習)
    \item 個別レポート
    \item グループワーク (発表会・グループレポート)
    \end{itemize}    
  \item 個別レポート
    \begin{itemize}
    \item 各自が \LaTeX で作成の上提出 (計3回)
    \item 提出方法・締切については、後日指示
    \end{itemize}
  \item グループワーク (全18グループ)
    \begin{itemize}
    \item ワークグループ(X1〜X9, Y1〜Y9)毎にテーマを決めて共同作業
    \item グループ分けは『「計算機実験」グループ・日程表』を参照
    \item テーマの例は講義・実習時に提示 (それ以外のテーマも歓迎)
    \item 6月中旬〜下旬、ワークグループ毎に面談
    \item 最終回(7/13)に発表会(プレビュー + ポスターセッション)
    \end{itemize}
  \end{itemize}    
\end{frame}

\begin{frame}[t]{講義資料}
  \begin{itemize}
    \setlength{\itemsep}{1em}
  \item 「計算機実験」ハンドブック (配布済)
    \begin{itemize}
    \item UNIX入門
    \item C言語入門
    \item \LaTeX 入門
    \item バージョン管理システム
    \end{itemize}
  \item 講義資料、実習資料、追加資料、参考書
    \begin{itemize}
    \item ITC-LMSで配布・提示 \url{https://itc-lms.ecc.u-tokyo.ac.jp/lms/course/view.php?id=95540}
    \item 公開資料については、\url{http://exa.phys.s.u-tokyo.ac.jp/ja/lectures/2016S-computer}でも配布
    \item 主な資料の \LaTeX ソースコードもGitHubで公開 \\
    {\footnotesize \url{https://github.com/todo-group/computer-experiments}}
    \end{itemize}
  \end{itemize}
\end{frame}

\begin{frame}[t]{質問がある場合には、、、}
  \begin{enumerate}
    %\setlength{\itemsep}{1em}
  \item ITC-LMS の掲示板を見る
  \item ハンドブック、講義資料を確認
  \item まわりの人に質問してみる
  \item ネットで検索
  \item 計算機実験担当者(\href{mailto:computer@exa.phys.s.u-tokyo.ac.jp}{computer@exa.phys.s.u-tokyo.ac.jp}) に相談
  \end{enumerate}
  メールで質問するときに注意すべきこと
  \begin{itemize}
  \item (メールの)標題をきちんとつける、きちんと名乗る
  \item 実行環境を明示する
  \item 問題を再現する手順を明記する
  \item 関連するファイル(Cや \LaTeX のソースコード等)を添付する
  \item エラーメッセージを添付する
  \end{itemize}
\end{frame}

\begin{frame}[t,fragile]{実習環境}
  \begin{itemize}
    \setlength{\itemsep}{1em}
  \item 情報基盤センター大演習室 (iMac端末)
    \begin{itemize}
    \item Cプログラミング、\LaTeX、Gnuplot、などに利用
    \end{itemize}
  \item 計算機端末室
    \begin{itemize}
    \item 理学部4号館1215室 (iMac 16台)
    \item 月曜10:00-金曜19:00に利用可
    \end{itemize}
  \item 物理学教室ワークステーションクラスタ photon
    \begin{itemize}
    \item SSHでリモートログインして使用する(ハンドブック2.2節)
    \item あらかじめ公開鍵の登録が必要(実習EX0 準備練習2,3)
    \end{itemize}
  \item MateriApps LIVE! (USBメモリで配布)
    \begin{itemize}
    \item Mac, Windows PC 上で動作する仮想UNIX環境
  \item インストール方法はUSBメモリ内の\href{https://github.com/cmsi/MateriAppsLive/wiki/MateriAppsLive-ltx}{README.html}、\href{https://github.com/cmsi/MateriAppsLive-setup/blob/master/ova/setup.pdf}{setup.pdf}を参照のこと
    \end{itemize}
  \end{itemize}
\end{frame}

\section{SSH (Secure Shell)}

\begin{frame}[t,fragile]{SSHによるリモートログイン}
  \begin{itemize}
    \setlength{\itemsep}{1em}
  \item UNIX (Macも含む)では、SSH (Secure Shell)を使うことで、別のコンピュータに遠隔ログインして作業することができる

    例)
    
    {\tt \$ \underline{ssh -X remote.phys.s.u-tokyo.ac.jp -l {\it username}}}
  \item 二種類の認証方式
    \begin{itemize}
    \item パスワード認証: ハンドブック2.2節ではこちらを説明
    \item 公開鍵認証方式: よりセキュリティーの高い方法

      近年はこちらが主流 (photon や ECCS SSHサーバも公開鍵認証)
    \end{itemize}
  \end{itemize}
\end{frame}

\begin{frame}[t,fragile]{SSHの公開鍵認証}
  \begin{itemize}
    \setlength{\itemsep}{1em}
  \item あらかじめクライアント(接続元)側で、「秘密鍵」と「公開鍵」のペアを生成し、「公開鍵」をサーバ(接続先)に置いておく
    \begin{itemize}
    \item 生成には {\tt ssh-keygen} コマンドを使う(準備練習EX0-2)
    \item クライアント側に「秘密鍵」、サーバ側に「公開鍵」の両者が揃ってはじめて、クライアントからサーバにリモートログインできる
    \item たとえ「公開鍵」が盗まれてしまっても、それだけではリモートログインできないので安心
    \item 「秘密鍵」は絶対に人に見られてはならない
    \end{itemize}
  \item 鍵の場所
    \begin{itemize}
    \item 秘密鍵: {\tt \$HOME/.ssh/id\_rsa} に生成される
    \item 公開鍵: {\tt \$HOME/.ssh/id\_rsa.pub} に生成される $\Rightarrow$
      サーバの {\tt \$HOME/.ssh/authorized\_keys} にコピーする
    \end{itemize}
  \end{itemize}
\end{frame}

\section{数値誤差}

\begin{frame}[t,fragile]{数値誤差の原因}
  \begin{itemize}
    \setlength{\itemsep}{1em}
  \item 丸め誤差: 無理数や10進数を有限のビットの2進数で表現することによる誤差
    (例: 0.1 が 0.0999999999998 になる)
  \item 打ち切り誤差: テイラー展開による近似を有限項で打ち切ることによる誤差
    (例: 数値微分)
  \item 桁落ち: 非常に近い数の引き算により生じる
  \item 情報落ち: 非常に大きな数に小さな数を足し込む場合に生じる
    (例: 数値積分や常微分方程式の初期値問題で刻み幅を小さくしすぎると生じる)
  \item オーバーフロー(桁あふれ): 表現できる値を超えてしまう
  \end{itemize}
\end{frame}

\begin{frame}[t,fragile]{桁落ち}
  \begin{itemize}
    \setlength{\itemsep}{1em}
  \item 2次方程式 $ax^2+bx+c=0$の解の公式
    \[
    x_{\pm} = \frac{-b \pm \sqrt{b^2-4ac}}{2a}
    \]
    $b^2 \gg |ac|$の時、桁落ちが生じる
  \item 例) $2.718282x^2 - 684.4566x+0.3161592=0$ の解を7桁の精度で計算してみる(伊理・藤野1985)
    \begin{align*}
      \sqrt{D} &= \sqrt{(684.4566)^2 - 4 \times 2.718282 \times 0.3161592} = 684.4541 \\
      x_+ &= \frac{684.4566+684.4541}{2 \times 2.718282} = \frac{1368.911}{5.436564} = 251.7970 \\
      x_- &= \frac{684.4566-684.4541}{2 \times 2.718282} = \frac{0.0025}{5.436564} = 0.00045\underline{98493}
    \end{align*}
  \end{itemize}
\end{frame}

\begin{frame}[t,fragile]{桁落ちを防ぐ方法}
  \begin{itemize}
    \setlength{\itemsep}{1em}
  \item $b$の符号に応じて、一方を求める(この例では$x_+$)
  \item 他方は解と係数の関係を使って求める
    \[
    x_- = \frac{c/a}{x_+} = \frac{0.3161592 / 2.718282}{251.7970} = 0.000461913\underline{8}
    \]
  \item 回避できない例: 重解に近い場合 $2.718282x^2 - 1.854089x + 0.3161592=0$
    \begin{align*}
      \sqrt{D} &= \sqrt{(1.854089)^2 - 4 \times 2.718282 \times 0.3161592} \\ &= 0.002\underline{64575} \\
      x_\pm &= 1.854089 \pm 0.002\underline{64575} = 1.856\underline{737}, 1.851\underline{445}
    \end{align*}
  \end{itemize}
\end{frame}

\begin{frame}[t,fragile]{数値微分}
  \begin{itemize}
    \setlength{\itemsep}{1em}
  \item 関数のテイラー展開
    \[
    f(x+h) = f(x) + h f'(x) + h^2 f''(x)/2 + h^3 f'''(x)/6 + \cdots
    \]
  \item 数値微分の最低次近似
    \[
    f_1(x,h) \equiv \frac{f(x+h)-f(x)}{h} = f'(x) + h f''(x)/2 + O(h^2)
    \]
  \item より高次の近似
    \[
    f_2(x,h) \equiv \frac{f(x+h)-f(x-h)}{2h} = f'(x) + h^2 f'''(x)/6 + O(h^3)
    \]
  \item 刻み$h$を小さくすると打ち切り誤差は減少するが、小さすぎると今度は桁落ちが大きくなる
  \end{itemize}
\end{frame}

\begin{frame}[t,fragile]{刻み幅を変えた計算}
  \begin{itemize}
    \setlength{\itemsep}{1em}
  \item 刻み幅を変えて何度か計算を行い、収束の様子をみる
  \item グラフ化して目で見てみる
  \item 理論式と比較
    \begin{itemize}
    \item 計算式の正しさの確認
    \item 近似の改良 (収束の加速・補外)
    \end{itemize}
  \item 桁落ち・情報落ちの影響の有無
  \end{itemize}
\end{frame}

\section{ニュートン法}

\begin{frame}[t,fragile]{ニュートン法}
  \begin{itemize}
    \setlength{\itemsep}{1em}
  \item 反復法により方程式$f(x)=0$の解を求める
  \item 真の解を$x_0$、適当な解の候補を$x'=x_0+\epsilon$とすると
    \[
    0 = f(x_0) = f(x_0+\epsilon-\epsilon) = f(x') - \epsilon f'(x') + O(\epsilon^2)
    \]
  \item 次の解の候補 (反復法、逐次近似法)
    \[
    x'' = x'-\epsilon \approx x' - \frac{f(x')}{f'(x')}
    \]
  \item 複素変数の複素関数や多変数の場合にも自然に拡張可
  \end{itemize}
\end{frame}

\begin{frame}[t,fragile]{ニュートン法の収束}
  \begin{itemize}
    \setlength{\itemsep}{1em}
  \item $x'$が$x_0$に十分近い時
    \begin{align*}
      f(x') &\approx (x'-x_0) f'(x_0) + \frac{(x' - x_0)^2}{2} f''(x_0) \\
      f'(x') &\approx f'(x_0) + (x' - x_0) f''(x_0)
    \end{align*}
  \item ニュートン法で一回反復すると
    \begin{align*}
      x'' =  x' - \frac{f(x')}{f'(x')} &\approx x' - (x'-x_0)(1-\frac{(x'-x_0)}{2}\frac{f''}{f'}) \\
      (x''-x_0) &\approx \frac{f''}{f'} (x' - x_0)^2
    \end{align*}
    \item 一回の反復で誤差が2乗で減る(正しい桁数が倍に増える) ⇒ 二次収束
  \end{itemize}
\end{frame}

\end{document}
