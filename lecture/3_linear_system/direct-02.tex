\begin{frame}[t,fragile]{逆行列の「正しい」求め方}
  \begin{itemize}
    %\setlength{\itemsep}{1em}
  \item 連立一次方程式 $A {\bf x} = {\bf e}_j$ を全ての${\bf e}_j$について解く
    \begin{itemize}
    \item Gaussの消去法による連立一次方程式の解法: 計算量〜$O(n^3)$
    \item Gaussの消去法の途中で出てくる下三角行列(L)と上三角行列(U)行列を再利用(LU分解)すれば、逆行列全体を求めるための計算量も$O(n^3)$
    \item $n=100$の場合: $n^3 = 10^6 \ll 9.3 \times 10^{157}$
    \end{itemize}
  \item $A$の行列式も$O(n^3)$で計算可 \\[2em]
  \item 逆行列をベクトルに掛ける必要がある場合には、逆行列を明示的に求めるのではなく連立一次方程式を解く
    \[ {\bf x} = A^{-1} {\bf b} \ \ \Rightarrow \ \ A {\bf x} = {\bf b} \]
  \end{itemize}
\end{frame}
