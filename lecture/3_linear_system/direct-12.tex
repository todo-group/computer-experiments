\begin{frame}[t,fragile]{LU分解}
  \begin{itemize}
    %\setlength{\itemsep}{1em}
  \item LU分解による連立一次方程式の解法
    \begin{itemize}
    \item 方程式は$A{\bf x} = LU{\bf x} = {\bf b}$と書ける
    \item まず、$L{\bf y} = {\bf b}$を解いて、${\bf y}$を求める(前進代入)
    \item 次に、$U{\bf x} = {\bf y}$を解いて、${\bf x}$を求める(後退代入)
    \end{itemize}
  \item 計算量はガウスの消去法と変わらない
  \item 一度LU分解をしておけば、異なる${\bf b}$に対する解も簡単に求められる \\[2em]
  \item 行列式は$U$の対角成分の積で与えられる (ピボット選択する場合は、行の入れ替えにより符号が変わることに注意)
  \end{itemize}
\end{frame}
