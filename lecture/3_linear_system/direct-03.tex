\begin{frame}[t,fragile]{ガウスの消去法}
  \begin{itemize}
    %\setlength{\itemsep}{1em}
  \item 解くべき連立方程式
    \begin{align*}
    a_{11}^{(1)} x_1 + a_{12}^{(1)} x_2 + a_{13}^{(1)} x_3 + \cdots + a_{1n}^{(1)} x_n &= b_{1}^{(1)} \\
    a_{21}^{(1)} x_1 + a_{22}^{(1)} x_2 + a_{23}^{(1)} x_3 + \cdots + a_{2n}^{(1)} x_n &= b_{2}^{(1)} \\
    a_{31}^{(1)} x_1 + a_{32}^{(1)} x_2 + a_{33}^{(1)} x_3 + \cdots + a_{3n}^{(1)} x_n &= b_{3}^{(1)} \\
    \cdots \\
    a_{n1}^{(1)} x_1 + a_{n2}^{(1)} x_2 + a_{n3}^{(1)} x_3 + \cdots + a_{nn}^{(1)} x_n &= b_{n}^{(1)}
    \end{align*}
  \item ある行を定数倍しても、方程式の解は変わらない
  \item ある行の定数倍を他の行から引いても、方程式の解は変わらない
  \end{itemize}
\end{frame}
