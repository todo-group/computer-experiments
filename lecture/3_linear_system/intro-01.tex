\begin{frame}[t,fragile]{連立一次方程式のあらわれる例}
  \begin{itemize}
    %\setlength{\itemsep}{1em}
  \item 偏微分方程式の境界値問題の差分法による求解
  \item 非線形連立方程式に対するニュートン法
    \[ {\bf x}' = {\bf x} - \Big( \frac{\partial {\bf f}({\bf x})}{\partial {\bf x}} \Big)^{-1} {\bf f}({\bf x}) \]
    \begin{itemize}
    \item 微分方程式の初期値問題の陰解法など
    \item 逆行列を求めベクトルに掛ける代わりに連立一次方程式を解く
      \[ {\bf x} = A^{-1} {\bf b} \ \ \Rightarrow \ \ A {\bf x} = {\bf b} \]
    \end{itemize}
  \end{itemize}
\end{frame}
