%-*- coding:utf-8 -*-

\documentclass[10pt,dvipdfmx]{beamer}
\usepackage{tutorial}

\title{計算機実験II (L5) --- 少数多体系・分子動力学}
\date{2021/11/19}

\begin{document}

\begin{frame}
  \titlepage
  \tableofcontents
\end{frame}

\begin{frame}[t]{講義日程 (予定)}
  \begin{itemize}
    % \setlength{\itemsep}{1em}
  \item 全8回 (金曜5限 16:50-18:35)
    \begin{itemize}
    \item 10月4日(金) 講義1: 多体系の統計力学とモンテカルロ法
    \item {\color{gray} 10月11日(金) 休講 (物理学教室コロキウム)}
    \item 10月18日(金) 実習1
    \item {\color{gray} 10月25日(金) 休講}
    \item 11月1日(金) 講義2: 偏微分方程式と多体系の量子力学
    \item 11月8日(金) 実習2
    \item {\color{gray} 11月15日(金) 休講}
    \item 11月29日(金) 講義3: 少数多体系・分子動力学
    \item 12月6日(金) 実習3
    \item {\color{gray} 12月13日(金) 休講 (物理学教室コロキウム)}
    \item {\color{gray} 12月20日(金) 休講 (ニュートン祭)}
    \item 12月27日(金) 講義4: 最適化問題
    \item 1月10日(金) 実習4
    \item {\color{gray} 1月24日(金) 休講 (物理学教室コロキウム)}
    \end{itemize}
  \end{itemize}
\end{frame}


\section{少数多体系・分子動力学}

%-*- coding:utf-8 -*-

\begin{frame}[t,fragile]{古典多粒子系}
  \begin{itemize}
    %\setlength{\itemsep}{1em}
  \item ハミルトニアン
    \[
    H = \sum_i \frac{p_i^2}{2m} + U(q_1, q_2, \cdots, q_{3N})
    \]
  \item 相互作用・ポテンシャル (化学の分野では「力場」と呼ばれる)
    \begin{itemize}
    \item 短距離相互作用: 剛体球ポテンシャル、Lennard-Jonesポテンシャル、3体ポテンシャル、第一原理(量子力学)ポテンシャル、粗視化モデル$\cdots$
    \item 長距離相互作用: 重力、静電相互作用
    \end{itemize}
  \item 運動方程式(常微分方程式)
    \[
    \frac{dq_i}{dt} = \frac{\partial H}{\partial p_i} = \frac{p_i}{m}, \ \ \frac{dp_i}{dt} = - \frac{\partial H}{\partial q_i} = - \frac{\partial U}{\partial q_i}
    \]
  \end{itemize}
\end{frame}

%-*- coding:utf-8 -*-

\begin{frame}[t,fragile]{分子動力学法}
  \begin{itemize}
    \setlength{\itemsep}{1em}
  \item 適当な初期条件から、運動方程式に従って位置と運動量を時間発展させる
    \begin{itemize}
    \item Euler法、Runge-Kutta法、リープ・フロッグ法、速度ベルレ法など
    \item $6N$次元の連立微分方程式
    \end{itemize}
  \item 時間発展に関する物理量の時間平均から平均を評価
    \begin{align*}
      \langle A(p,x) \rangle &= \frac{1}{Z(E)} \int A(p,x) \, \delta(H(p,x)-E) \, dp \, dx \\
      &\simeq \frac{1}{t_{\rm max}} \int_0^{t_{\rm max}} A(p(t),x(t)) \, dt
    \end{align*}
    \begin{itemize}
    \item ハミルトニアンが時間依存しない場合は全エネルギーが保存する→ミクロカノニカル分布
    \end{itemize}
  \item 平衡状態における平均値だけでなく、熱や電荷の輸送などの動的現象、非平衡状態からの緩和現象などもシミュレーションできる
  \end{itemize}
\end{frame}

%-*- coding:utf-8 -*-

\begin{frame}[t,fragile]{フローチャート}
  \begin{center}
    \resizebox{0.45\textwidth}{!}{\includegraphics{image/md-flowchart.pdf}}
  \end{center}
\end{frame}

%-*- coding:utf-8 -*-

\begin{frame}[t,fragile]{境界条件と力の計算}
  \begin{itemize}
    \setlength{\itemsep}{1em}
  \item 気体・液体・固体などの熱力学的極限を調べたい場合
    \begin{itemize}
    \item 端の効果を取り除くために周期境界条件を採用
    \item 周期的に同じパターンが続く
    \end{itemize}
  \item ポテンシャルの計算
    \begin{align*}
      U = \frac{1}{2} \sum_i \sum_{j \ne i} \sum_{n_x=-\infty}^{\infty} \sum_{n_y=-\infty}^{\infty} \sum_{n_z=-\infty}^{\infty} U[\mathbf{r}_i - \mathbf{r}_j + L(n_x,n_y,n_z)]
    \end{align*}
  \item 短距離力
    \begin{itemize}
    \item カットオフを入れる
    \item 最も近いイメージだけ考慮(minimum image convention)
    \end{itemize}
  \item 長距離力
    \begin{itemize}
    \item カットオフを入れると物理が変わる
    \item エバルト法、ツリー法、高速多重極展開
    \end{itemize}
  \end{itemize}
\end{frame}


\section{常微分方程式の初期値問題(復習)}
% \begin{frame}[t,fragile]{準備: 微分方程式の書き換え}
  \begin{itemize}
    %\setlength{\itemsep}{1em}
  \item 2階の常微分方程式の一般形
    \[
    \frac{d^2y}{dt^2} + p(t)\frac{dy}{dt} + q(t)y = r(t)
    \]
  \item $y_1 \equiv y$, $y_2 \equiv \frac{dy}{dt}$とおくと
    \[
    \left\{
    \begin{array}{ccl}
      \frac{dy_1}{dt} & = & y_2 \\
      \frac{dy_2}{dt} & = & r(t) - p(t) y_2 - q(t) y_1
    \end{array}
    \right.
    \]
  \item さらに$\bm{y}\equiv(y_1, y_2)$, $\bm{f}(t, \bm{y})\equiv \left(y_2, r(t)-p(t)y_2 - q(t)y_1\right)$とおくと
    \[
    \frac{d\bm{y}}{dt} = \bm{f}(t, \bm{y})
    \]
  \item $n$階常微分方程式 $\Rightarrow$ $n$次元の1階常微分方程式
  \end{itemize}
\end{frame}

\begin{frame}[t,fragile]{初期値問題の解法 (Euler法)}
  \begin{itemize}
    %\setlength{\itemsep}{1em}
  \item $h$を微小量として微分を差分で近似する(前進差分)
    \[
    \frac{dy}{dt} \approx \frac{y(t+h) - y(t)}{h} = f(t, y)
    \]
  \item $t=0$における$y(t)$の初期値を$y_0$、$t_n \equiv nh$、$y_n$を$y(t_n)$の近似値とおくと、
    \[
    y_{n+1}-y_n = h f( t_n, y_n)
    \]
  \item Euler法
    \begin{itemize}
    \item $y_0$からはじめて、$y_1,y_2,\cdots$を順次求めていく
    \end{itemize}
  \end{itemize}
\end{frame}

\begin{frame}[t,fragile]{高次のRunge-Kutta法}
  \begin{itemize}
    %\setlength{\itemsep}{1em}
  \item 3次Runge-Kutta法
    \[
    \begin{array}{rcl}
      k_1 & = & h f(t_n, y_n) \\
      k_2 & = & h f(t_n + \frac{2}{3}h, y_n + \frac{2}{3}k_1) \\
      k_3 & = & h f(t_n + \frac{2}{3}h, y_n + \frac{2}{3}k_2) \\
      y_{n+1} & = & y_n + \frac{1}{4}k_1 + \frac{3}{8}k_2
      + \frac{3}{8}k_3
    \end{array}
    \]
  \item 4次Runge-Kutta法
    \[
    \begin{array}{rcl}
      k_1 & = & h f(t_n, y_n) \\
      k_2 & = & h f(t_n + \frac{1}{2}h, y_n + \frac{1}{2}k_1) \\
      k_3 & = & h f(t_n + \frac{1}{2}h, y_n + \frac{1}{2}k_2) \\
      k_4 & = & h f(t_n + h, y_n + k_3) \\
      y_{n+1} & = & y_n + \frac{1}{6}k_1 + \frac{1}{3}k_2
      + \frac{1}{3}k_3 + \frac{1}{6}k_4
    \end{array}
    \]
  \item 4次までは次数と$f$の計算回数が等しい
  \end{itemize}
\end{frame}

\begin{frame}[t,fragile]{計算コストと精度}
  \begin{itemize}
    \setlength{\itemsep}{1em}
  \item 実際の計算では$f(t,y)$の計算にほとんどのコストがかかる
  \item 計算回数と計算精度の関係
    \begin{center}
      \begin{tabular}[h]{c|cccc}
        & 1次(Euler法) & 2次(中点法) & 3次 & 4次 \\
        \hline
        計算精度 & $O(h)$ & $O(h^2)$ & $O(h^3)$ & $O(h^4)$ \\
        計算回数 & $N$ & $2N$ & $3N$ & $4N$
      \end{tabular}
    \end{center}
  \item 高次のRunge-Kuttaを使う方が効率的
  \item どれくらい小さな$h$が必要となるか、前もっては分からない
  \item 刻み幅を変えて($h,h/2,h/4,\dots$)計算してみることが大事
    \begin{itemize}
    \item 誤差の評価
    \item 公式の間違いの発見
    \end{itemize}
  \end{itemize}
\end{frame}


\section{シンプレクティック積分法(復習)}
\begin{frame}[t,fragile]{ハミルトン力学系}
  \begin{itemize}
    % \setlength{\itemsep}{1em}
  \item 時間をあらわに含まない場合のハミルトン方程式
    \[
    \frac{dq}{dt} = \frac{\partial H}{\partial p}, \ \frac{dp}{dt} = -\frac{\partial H}{\partial q}
    \]
    \begin{itemize}
    \item エネルギー保存則
      \[
      \frac{dH}{dt} = \frac{\partial H}{\partial q} \frac{dq}{dt} + \frac{\partial H}{\partial p} \frac{dp}{dt} = 0
      \]
    \item 位相空間の体積が保存(Liouvilleの定理)

      位相空間上の流れの場$\bm{v} = (\frac{dq}{dt},\frac{dp}{dt})$について
      \[
      \text{div} \bm{v} = \frac{\partial}{\partial q} \frac{dq}{dt} + \frac{\partial}{\partial p} \frac{dp}{dt} = 0
      \]
    \end{itemize}
  \item Euler法、Runge-Kutta法などはいずれの性質も満たさない
  \end{itemize}
\end{frame}

\begin{frame}[t,fragile]{シンプレクティック数値積分法(Symplectic Integrator)}
  \begin{itemize}
    %\setlength{\itemsep}{1em}
  \item 体積保存を満たす解法
  \item 例: 調和振動子$H=\frac{1}{2}(p^2+q^2)$の運動方程式
    \[
    \frac{dq}{dt} = p, \ \frac{dp}{dt} = -q
    \]
    の一方をEuler法で、他方を逆オイラー法で解く
    \begin{align*}
      q_{n+1} &= q_n + h p_n \\
      p_{n+1} &= p_n - h q_{n+1} = (1-h^2) p_n - h q_n \\
      \begin{pmatrix} q_{n+1} \\ p_{n+1} \end{pmatrix} &= \begin{pmatrix} 1 & h \\ -h & 1-h^2 \end{pmatrix} \begin{pmatrix} q_{n} \\ p_{n} \end{pmatrix}
    \end{align*}
  \end{itemize}
\end{frame}

\begin{frame}[t,fragile]{体積・エネルギーの保存}
  \begin{itemize}
    %\setlength{\itemsep}{1em}
  \item 体積保存
    \begin{align*}
      \det \begin{pmatrix} 1 & h \\ -h & 1-h^2 \end{pmatrix} = 1
    \end{align*}
  \item エネルギーの保存
    \begin{align*}
      \frac{1}{2}(p_{n+1}^2+q_{n+1}^2) + {\color{red}\frac{h}{2} p_{n+1} q_{n+1}} = \frac{1}{2}(p_{n}^2+q_{n}^2) + {\color{red}\frac{h}{2} p_{n} q_{n}}
    \end{align*}
  \item 位相空間の体積は厳密に保存
  \item エネルギーは$O(h)$の範囲で保存し続ける
  \end{itemize}
\end{frame}

\begin{frame}[t,fragile]{2次のシンプレクティック積分法}
  \begin{itemize}
    \setlength{\itemsep}{1em}
  \item ハミルトニアンが$H(p,q) = T(p) + V(q)$の形で書けるとする
  \item リープ・フロッグ法
    \begin{align*}
      {\color{red} p(t+h/2)} &= p(t) - \frac{h}{2} \frac{\partial V(q)}{\partial q}|_{q=q(t)} \\
      {\color{blue} q(t+h)} &= q(t) + h {\color{red}p(t+h/2)} \\
      p(t+h) &= {\color{red}p(t+h/2}) - \frac{h}{2} \frac{\partial V(q)}{\partial q}|_{q=q(t+h)}
    \end{align*}
  \end{itemize}
\end{frame}


\section{ベルレ法}
\begin{frame}[t,fragile]{ベルレ(Verlet)法}
  \begin{itemize}
    %\setlength{\itemsep}{1em}
  \item 微分方程式が$\ddot{q}(t) = f[t,q(t)]$の形で書かれる場合を考える
  \item $q(t)$のテイラー展開
    \[
    q(t \pm h) = q(t) \pm h \dot{q}(t) + \frac{h^2}{2} \ddot{q}(t) \pm \cdots
    \]
    から、$q(t + h)$と$q(t - h)$の表式を足し合わせて整理
    \[
    q(t+h) = 2q(t) - q(t-h)+h^2 f[t,q(t)] + O(h^4)
    \]
  \item 初期条件: $q(0)$と$q(0)-q(-1) \simeq hp(0)$から$q(h)$を求める
    \[
    q(h) = q(0) + h p(0) + \frac{h^2}{2} f[0, q(0)]
    \]
  \item 運動量は中心差分で計算
    \[
    p(t) = \frac{q(t+h) - q(t-h)}{2h}
    \]
  \end{itemize}
\end{frame}

\begin{frame}[t,fragile]{ベルレ法の変形}
  \begin{itemize}
    %\setlength{\itemsep}{1em}
  \item 半整数ステップにおける運動量
    \[
    p(t + h/2) = \frac{q(t+h)-q(t)}{h}
    \]
    を導入すると
    \begin{align*}
      \begin{split}
        p(t + h/2) &= p(t - h/2) + \frac{q(t+h)-2q(t)+q(t-h)}{h} \\
        &= p(t - h/2) + h f[t,q(t)]
      \end{split}
    \end{align*}
    一方
    \[
    q(t+h) = q(t) + hp(t+h/2)
    \]
  \item $\Rightarrow$ リープ・フロッグ法

    座標についてはベルレ法と数学的に等価だが、丸め誤差に強い
  \end{itemize}
\end{frame}

\begin{frame}[t,fragile]{速度ベルレ法(Velocity Verlet)}
  \begin{itemize}
    %\setlength{\itemsep}{1em}
  \item 運動量に対する中心差分
    \[
    p(t) = \frac{q(t+h)-q(t-h)}{2h}
    \]
    から
    \[
    q(t+h) = 2hp(t) + q(t-h)
    \]
    ベルレ法の式と和をとると
    \begin{align*}
      q(t + h) &= q(t) + hp(t) + h^2 f[t,q(t)] / 2 \\
      p(t + h) &= p(t) + h \Big( f[t+h,q(t+h)]  + f[t,q(t)] \Big) / 2
    \end{align*}
  \item $\Rightarrow$ 速度ベルレ法

    2次のシンプレクティック積分法、丸め誤差にも安定
  \end{itemize}
\end{frame}


\section{シンプレクティック積分法の一般論}
\begin{frame}[t,fragile]{シンプレクティック法の一般論}
  \begin{itemize}
    %\setlength{\itemsep}{1em}
  \item $z=\begin{bmatrix} q \\ p \end{bmatrix}$と書き、$f(z)$に作用する演算子$\hat{D}(h)$をポアソン括弧を用いて定義($h=h(z)$はパラメータ)
    \[
    \hat{D}(h)f = \{ f, h \} = \frac{\partial f}{\partial q} \frac{\partial h}{\partial p} - \frac{\partial h}{\partial q} \frac{\partial f}{\partial p}
    \]
  \item ハミルトニアン$H=T(p)+V(q)$に対する正準方程式
    \[
    \dot{z} = \hat{D}(H) z
    \]
    に対する形式解
    \[
    z(t+h) = e^{h \hat{D}(H)} z(t)
    \]
  \item $e^{h \hat{D}(H)}$: 時間発展演算子 (正準変換)
  \end{itemize}
\end{frame}

\begin{frame}[t,fragile]{シンプレクティック法の一般論}
  \begin{itemize}
    %\setlength{\itemsep}{1em}
  \item $H=T(p)$の時
    \begin{align*}
      \hat{D}(T)z &= \frac{\partial z}{\partial q} \frac{\partial T}{\partial p} - \frac{\partial T}{\partial q} \frac{\partial z}{\partial p} = \begin{bmatrix} \frac{dT}{dp} \\ 0 \end{bmatrix} \\
      \hat{D}(T)^n z &= 0 \qquad n \ge 2 \\
      \therefore e^{h\hat{D}(T)} z &= [ 1+h\hat{D}(T) ] z = \begin{bmatrix} q + h \frac{dT}{dp} \\ p \end{bmatrix}
    \end{align*}
  \item $H=V(q)$の時も同様に
    \begin{align*}
      e^{h\hat{D}(V)} z &= [ 1+h\hat{D}(V) ] z = \begin{bmatrix} q \\ p - h \frac{dV}{dq} \end{bmatrix}
    \end{align*}
  \item $e^{h\hat{D}(T)}$, $e^{h\hat{D}(V)}$とも正準変換
  \item 一般的には、$e^{h\hat{D}(T+V)} z$は厳密には計算できない
  \end{itemize}
\end{frame}

\begin{frame}[t,fragile]{シンプレクティック法の一般論}
  \begin{itemize}
    %\setlength{\itemsep}{1em}
  \item $e^{h\hat{D}(T+V)}$を$e^{a_i h\hat{D}(T)}$, $e^{b_i h\hat{D}(V)}$の積で近似する
  \item 演算子(行列) $A,B$について
    \[
    e^{h (A+B)} = e^{hA} e^{hB} + O(h^2)
    \]
  \item 一次のシンプレクティック法
    \begin{align*}
      \hat{S}_1 &= e^{h\hat{D}(V)} e^{h\hat{D}(T)} \\
      q(t+h) &= q(t) + h \frac{dT(p(t))}{dp} \\
      p(t+h) &= p(t) - h \frac{dV(q({\color{red}t+h}))}{dq}
    \end{align*}
    $\Rightarrow$ オイラー法+逆オイラー法
  \end{itemize}
\end{frame}

\begin{frame}[t,fragile]{シンプレクティック法の一般論}
  \begin{itemize}
    %\setlength{\itemsep}{1em}
  \item 二次のシンプレクティック法
    \[
    e^{h (A+B)} = e^{\frac{h}{2}B} e^{hA} e^{\frac{h}{2}B} + O(h^3)
    \]
    から
    \begin{align*}
      \hat{S}_2 &= e^{\frac{h}{2}\hat{D}(T)} e^{h\hat{D}(V)} e^{\frac{h}{2}\hat{D}(T)}
    \end{align*}
    $\Rightarrow$ リープ・フロッグ法
  \item 四次のシンプレクティック法 (吉田の方法)
    \begin{align*}
      \hat{S}_4 &= e^{c_1h\hat{D}(T)} e^{d_1 h\hat{D}(V)} e^{c_2 h\hat{D}(T)} e^{d_2 h\hat{D}(V)} e^{c_3h\hat{D}(T)} e^{d_3 h\hat{D}(V)} e^{c_4h\hat{D}(T)} \\
      & c_1 = c_4 = \frac{1}{2(2-2^{1/3})}, \ c_2 = c_3 = \frac{1-2^{1/3}}{2(2-2^{1/3})}, \\
      & d_1 = d_3 = \frac{1}{2-2^{1/3}}, \ d_2 = \frac{2^{1/3}}{2-2^{1/3}}
    \end{align*}
  \end{itemize}
\end{frame}

\begin{frame}[t,fragile]{シンプレクティック積分法}
  \begin{itemize}
    %\setlength{\itemsep}{1em}
  \item ハミルトン力学系の満たすべき特性(位相空間の体積保存)を満たす
  \item 一般的には陰解法
  \item ハミルトニアンが$H(p,q) = T(p) + V(q)$の形で書ける場合は陽的なシンプレクティック積分法が存在する
  \item エネルギーは近似的に保存する
  \item $n$次のシンプレクティック積分法では、エネルギーは$O(h^n)$の範囲で振動(発散しない)
  \item より高次のシンプレクティック積分法についても、システマティックに構成できる(ただし係数を解析的に求められるのは4次まで)。
    参考文献: H. Yoshida, Phys. Lett. A {\bf 150}, 262 (1990)
  \end{itemize}
  \begin{itemize}
    \item 系の満たすべき保存則を満たすように差分法を構成することが重要
    \item 参考: 計算科学フォーラムでの河合宗司先生(東北大)の講演 \url{https://hpcic-kkf.com/forum/2024/kkf_01/}
  \end{itemize}
\end{frame}


\section{}
\begin{frame}[t]{本日の課題}
  \begin{itemize}
    %\setlength{\itemsep}{1em}
  \item 実習
    \begin{itemize}
    \item 実習課題一覧\href{https://github.com/todo-group/ComputerExperiments/releases/tag/2021a-computer2}{exercise-2.pdf}から常微分方程式(あるいは別の)課題を選び実習
    \end{itemize}
  \item 質問はSlackの「\# 3\_常微分方程式」あるいは他の適当と思われるチャンネルで
  \item 本日24時までにITC-LMSのアンケート「第5回(11/19)作業レポート」に回答(出席のかわり)
  \end{itemize}
\end{frame}

\end{document}
