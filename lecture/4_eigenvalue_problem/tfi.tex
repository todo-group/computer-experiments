\section{横磁場イジング模型}

\begin{frame}[t,fragile]{横磁場イジング模型}
  \begin{itemize}
    %\setlength{\itemsep}{1em}
  \item ハミルトニアン($2^N \times 2^N$行列)
    \[
      H = H_z + H_x = - J \sum_{\langle i,j \rangle} \sigma_i^z \sigma_j^z - h \sum_i \sigma_i^z - \Gamma \sum_i \sigma_i^x
    \]
  \item $\sigma_i^x$、$\sigma_i^z$: パウリ行列($2 \times 2$行列)
    \begin{align*}
      \big(\sigma_i^z\big)^2 &= \big(\sigma_i^x\big)^2 = I \\
      [ \sigma_i^z, \sigma_i^x ] &\ne 0
    \end{align*}
  \item $J$: スピン間の相互作用($J>0$: 強磁性、$J<0$: 反強磁性)
  \item $h$: 縦磁場(準位間のエネルギー差$=2h$)
  \item $\Gamma$: 横磁場(トンネリング)
  \item 以降、$\sigma_i^z$を対角化する基底($|\!\uparrow\rangle_i$, $|\!\downarrow\rangle_i$)で考える
  \end{itemize}
\end{frame}

\begin{frame}[t,fragile]{横磁場イジング模型}
  \begin{itemize}
    %\setlength{\itemsep}{1em}
  \item 2サイト系
    \[
      H = -J \sigma_1^z \sigma_2^z - h (\sigma_1^z + \sigma_2^z) - \Gamma (\sigma_1^x + \sigma_2^x)
    \]
  \item 行列要素
    \begin{align*}
      \langle \uparrow \uparrow \!| H |\! \uparrow \uparrow \rangle &= -J - 2h \\
      \langle \uparrow \uparrow \!| H |\! \uparrow \downarrow \rangle &= -\Gamma \\
      \langle \uparrow \uparrow \!| H |\! \downarrow \downarrow \rangle &= 0 \\
      &\vdots
    \end{align*}
  \end{itemize}
\end{frame}

\begin{frame}[t,fragile]{量子相転移}
  \begin{itemize}
    %\setlength{\itemsep}{1em}
  \item $h=0$の場合
    \begin{itemize}
    \item $\Gamma \rightarrow 0$: $|\!\uparrow\uparrow\cdots\uparrow\rangle$、あるいは$|\!\downarrow\downarrow\cdots\downarrow\rangle$が基底状態(二重縮退)
    \item $J \rightarrow 0$: $\sigma_i^x$の固有状態($|\!\uparrow\rangle_i + |\!\downarrow\rangle_i$)の積が基底状態(全ての状態の重ね合わせ)
    \end{itemize}
  \item 一次元系
    \[
      H = - J \sum_{i} \sigma_i^z \sigma_{i+1}^z - \Gamma \sum_i \sigma_i^x
    \]
    $\Gamma = J$で量子相転移(熱ゆらぎではなく量子ゆらぎによる相転移)  
  \end{itemize}
\end{frame}

\section{多体量子系の時間発展}

\begin{frame}[t,fragile]{横磁場イジング模型の時間発展}
  \begin{itemize}
    %\setlength{\itemsep}{1em}
  \item 時間依存シュレディンガー方程式の形式解
    \[
    \Psi(t) = e^{-iHt} \Psi(0)
    \]
    \begin{itemize}
    \item 有限差分法、クランク・ニコルソン法
    \end{itemize}
  \item 鈴木・トロッター分解 ($\Delta t = t / M$)
    \begin{align*}
      e^{-iHt} &= \big[ e^{-iH\Delta t} \big]^M \approx \big[ e^{-iH_z\Delta t} e^{-iH_x\Delta t} \big]^M \\
      &= \big[ e^{i\Delta t J \sum_{\langle i,j \rangle} \sigma_i^z \sigma_j^z} e^{i\Gamma \sigma_1^x\Delta t} e^{i\Gamma \sigma_2^x\Delta t} \cdots e^{i\Gamma \sigma_N^x\Delta t} \big]^M
    \end{align*}
    (さらに$e^{-iH\Delta t} \approx e^{-iH_z\Delta t/2} e^{-iH_x\Delta t} e^{-iH_z\Delta t/2}$と対称に分解すると近似の次数が上がる)
  \item $[\sigma_i^x]^2 = I$より
    \[
    e^{i\Gamma \sigma_i^x\Delta t} = \cos (\Gamma\Delta t) + i \sigma_i^x \sin (\Gamma\Delta t)
    \]
  \end{itemize}
\end{frame}

\begin{frame}[t,fragile]{量子アニーリング}
  \begin{itemize}
    %\setlength{\itemsep}{1em}
  \item 離散最適化問題
    \[
    H = -J \sum_{i<j} \epsilon_{ij} \sigma_i^z \sigma_j^z
    \]
    の基底状態配位と基底状態エネルギーを求めたい
    \begin{center}
      \resizebox{0.6\textwidth}{!}{\includegraphics{image/spinglass.pdf}}
    \end{center}
  \end{itemize}
\end{frame}

\begin{frame}[t,fragile]{量子アニーリング}
  \begin{itemize}
    %\setlength{\itemsep}{1em}
  \item 横磁場を導入
    \[
    H = -J \sum_{i<j} \epsilon_{ij} \sigma_i^z \sigma_j^z - \Gamma \sum_i \sigma_i^x
    \]
  \item 古典極限 ($J=1$, $\Gamma=0$)
    \begin{itemize}
    \item 求めたい基底状態
    \end{itemize}
  \item 量子極限 ($J=0$, $\Gamma=1$)
    \begin{itemize}
    \item $2^N$個の全ての状態の重ね合わせ
    \end{itemize}
  \item 量子アニーリング
    \begin{itemize}
    \item $J+\Gamma=1$を保ったままで、$\Gamma=1$から$\Gamma=0$まで「ゆっくり」と減少させながら時間発展させる
      \[
      J = 1-\Gamma = t/T \qquad (0 \le t \le T)
      \]
    \item $T \rightarrow \infty$の極限で確率1で基底状態に収束
    \end{itemize}
  \end{itemize}
\end{frame}

\section{量子コンピュータ}

\begin{frame}[t,fragile]{量子コンピュータと量子ゲート}
  \begin{itemize}
    %\setlength{\itemsep}{1em}
  \item (ゲート型)量子コンピュータ

    $N$量子ビットに対して、量子ゲートにより状態を操作

    \begin{itemize}
    \item 「量子ビット」= 2準位系($S=1/2$スピン) \ $|0\rangle=|\!\uparrow\rangle, |1\rangle=|\!\downarrow\rangle$
    \item 「量子ゲート」= 少数量子ビットに対するユニタリ変換
    \end{itemize}
  \item 1量子ビットゲート
    \begin{itemize}
    \item Xゲート(量子NOT)
      \[
      X = \begin{pmatrix} 0 & 1 \\ 1 & 0 \end{pmatrix} = \sigma^x
      \]
    \item Rzゲート($z$軸まわりの回転)
      \[
      Rz(\theta) = \begin{pmatrix} e^{-i\theta/2} & 0 \\ 0 & e^{i\theta/2} \end{pmatrix} = e^{-i\theta\sigma^z}
      \]
    \item アダマールゲート
      \[
      H = \frac{1}{\sqrt{2}} \begin{pmatrix} 1 & 1 \\ 1 & -1 \end{pmatrix} = \frac{1}{\sqrt{2}} (\sigma^x + \sigma^z)
      \]
    \end{itemize}
  \end{itemize}
\end{frame}

\begin{frame}[t,fragile]{量子ビットゲート}
  \begin{itemize}
    %\setlength{\itemsep}{1em}
  \item 2量子ビットゲート
    \begin{itemize}
    \item CXゲート(制御NOT)
      \[
      CX = \begin{pmatrix} 1 & 0 & 0 & 0 \\ 0 & 1 & 0 & 0 \\ 0 & 0 & 0 & 1 \\ 0 & 0 & 1 & 0 \end{pmatrix}
      \]
    \end{itemize}
  \item 3量子ビットゲート
    \begin{itemize}
    \item CCXゲート(トフォリゲート)
      \[
      CCX = \begin{pmatrix}
        1 & 0 & 0 & 0 & 0 & 0 & 0 & 0 \\
        0 & 1 & 0 & 0 & 0 & 0 & 0 & 0 \\
        0 & 0 & 1 & 0 & 0 & 0 & 0 & 0 \\
        0 & 0 & 0 & 1 & 0 & 0 & 0 & 0 \\
        0 & 0 & 0 & 0 & 1 & 0 & 0 & 0 \\
        0 & 0 & 0 & 0 & 0 & 1 & 0 & 0 \\
        0 & 0 & 0 & 0 & 0 & 0 & 0 & 1 \\
        0 & 0 & 0 & 0 & 0 & 0 & 1 & 0 \end{pmatrix}
      \]
    \end{itemize}
  \end{itemize}
\end{frame}

\begin{frame}[t,fragile]{量子加算器}
  \begin{itemize}
    %\setlength{\itemsep}{1em}
  \item 1量子ビットの加算 $\Rightarrow$ CCXゲートとCXゲートで実現できる
    \begin{center}
      \resizebox{0.5\textwidth}{!}{\includegraphics{image/adder1.pdf}}
    \end{center}
  \item CCXゲートは、Hゲート、Rzゲート、CXゲートの組み合わせで表現できる
  \item 任意の量子回路は、Hゲート、Rzゲート、CXゲートの組み合わせで表現できる (万能量子ゲート)
  \end{itemize}
\end{frame}

\begin{frame}[t,fragile]{量子加算器}
  \begin{itemize}
    %\setlength{\itemsep}{1em}
  \item 3量子ビット加算器
    \begin{center}
      \resizebox{0.6\textwidth}{!}{\includegraphics{image/adder3.pdf}}
      \resizebox{0.35\textwidth}{!}{\includegraphics{image/adder2.pdf}}
    \end{center}
  \end{itemize}
\end{frame}
