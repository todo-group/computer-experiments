\begin{frame}[t,fragile]{Lanczos法}
  \begin{itemize}
    \setlength{\itemsep}{1em}
  \item 行列で表現すると
    \begin{align*}
      \hspace*{-2em}
      A
      (v_1v_2\cdots v_m)
      &=
      (v_1v_2\cdots v_m v_{m+1})
      \begin{pmatrix}
        \alpha_1 & \beta_1\\
        \beta_1 & \alpha_2 & \beta_2 \\
        & \beta_2 & \alpha_3 & \beta_3 \\
        & & \beta_3 & \alpha_4 & \beta_4 \\
        & & & \ddots & \ddots & \ddots \\
        & & & & \beta_{m-1} & \alpha_m \\
        & & & & & \beta_m \\
      \end{pmatrix}
    \end{align*}
    両辺に左から$(v_1v_2\cdots v_m)^T$をかけると
    \[
    (v_1v_2\cdots v_m)^T A (v_1v_2\cdots v_m)
    \]
    は3重対角行列となることがわかる
  \end{itemize}
\end{frame}
