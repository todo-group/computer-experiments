\begin{frame}[t,fragile]{行列の数値対角化}
  \begin{itemize}
    %\setlength{\itemsep}{1em}
  \item 一般的に次元が5以上の行列の固有値は、あらかじめ定まる有限回の手続きでは求まらない
    \begin{itemize}
    \item 必ず何らかの反復法(+収束判定)が必要となる
    \end{itemize}
  \item 密行列向きの方法
    \begin{itemize}
    \item Jacobi法
    \item Givens変換・Householder法(三重対角化) + QR法など
    \end{itemize}
  \item 疎行列向きの方法
    \begin{itemize}
    \item べき乗法
    \item Lanczos法(三重対角化) + QR法など
    \end{itemize}
  \item 固有ベクトル
    \begin{itemize}
    \item QR法で求めたものを逆変換
    \item 逆反復法で精度改善
    \end{itemize}
  \end{itemize}
\end{frame}
