\section{逆反復法}

\begin{frame}[t,fragile]{逆反復法による固有ベクトルの精度向上}
  \begin{itemize}
    %\setlength{\itemsep}{1em}
  \item 逆反復法
    \begin{itemize}
    \item 近似固有値を$\mu$とするとき、行列$(A - \mu E)^{-1}$を考えると、固有ベクトルは$A$と同じ、固有値は$(\lambda-\mu)^{-1}$。
    \item $\mu$が十分に正確であれば、$(\lambda-\mu)^{-1}$は絶対値最大の固有値。行列$(A - \mu E)^{-1}$を適当な初期ベクトルにかけ続けると$\lambda$に対応する固有ベクトルに収束 (c.f. べき乗法)
    \item 実際には$(A-\mu E) x' = x$という連立方程式を繰り返し解く
    \end{itemize}
  \end{itemize}
\end{frame}
