\begin{frame}[t,fragile]{LAPACKの対角化ルーチン}
  \begin{itemize}
    %\setlength{\itemsep}{1em}
  \item 様々な対角化ルーチンが準備されている
    \begin{itemize}
    \item 倍精度実対称行列の対角化 {\tt dsyev}
      \url{http://www.netlib.org/lapack/explore-html/dd/d4c/dsyev_8f.html}
    \item Fortranによる関数宣言
\begin{lstlisting}
subroutine dsyev(character JOBZ, character UPLO,
  integer N, double precision, dimension(lda, *) A,
  integer LDA, double precision, dimension(*) W,
  double precision, dimension(*) WORK,
  integer LWORK, integer INFO)		
\end{lstlisting}
    \end{itemize}
  \item 他にも{\tt dsyevd}、{\tt dsyevr}、{\tt dsyevx}などがある \\
    3重対角化までは同じ。3重対角行列の対角化が異なる
  \item 単精度版の{\tt ssyev}、複素(エルミート行列)版の{\tt zheev}など
  \item {\tt dsyev}の使用例: \href{https://github.com/todo-group/computer-experiments/blob/master/exercise/eigenvalue_problem/diag.c}{diag.c}
  \end{itemize}
\end{frame}
