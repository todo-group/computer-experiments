\begin{frame}[t,fragile]{固体物理・量子統計物理に現れる行列}
  \begin{itemize}
    %\setlength{\itemsep}{1em}
  \item 強束縛近似(tight-binding approx.)のもとでの第二量子化表示
    \[
    H = -t \sum_{\langle i,j \rangle \sigma} (c_{i,\sigma}^\dagger c_{j,\sigma} + h.c.) + \text{(相互作用)}
    \]
  \item 局所スピン模型(ハイゼンベルグ模型)
    \[
    H = -J\sum_{\langle i,j \rangle} S_i \cdot S_j
    = -J\sum_{\langle i,j \rangle} [S_i^z S_j^z +\frac{1}{2} (S_i^+ S_j^- + S_i^- S_j^+) ]
    \]
  \item 格子点の数を$n$とすると、ハミルトニアンはそれぞれ$4^n \times 4^n$、$2^n \times 2^n$の(疎)行列で表される。
  \item $n$が大きくなると、行列の次元は指数関数的に増加
  \item 量子多体系に共通する困難
  \end{itemize}
\end{frame}
