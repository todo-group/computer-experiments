\begin{frame}[t,fragile]{疎行列ベクトル積}
  \begin{itemize}
    %\setlength{\itemsep}{1em}
  \item 密行列ベクトル積
    \begin{itemize}
    \item メモリ: $O(N^2)$, 計算量: $O(N^2)$
    \end{itemize}
  \item 疎行列の場合
    \begin{itemize}
    \item メモリ: $O(N)$ or $O(1)$, 計算量: $O(N)$ まで削減可能
    \item 三重対角行列

      対角成分・副対角成分をベクトルで記憶

      例) \href{https://github.com/todo-group/computer-experiments/blob/master/exercise/eigenvalue_problem/tridiagonal.c}{tridiagonal.c} (一次元一粒子系)

    \item 一般の疎行列

      各行について、非ゼロの要素の場所(列)と値のみを記憶 (CRS, CCS形式)

      例) \href{https://github.com/todo-group/computer-experiments/blob/master/exercise/eigenvalue_problem/sparse.c}{sparse.c} (一次元横磁場イジング模型)

    \item 非ゼロの要素の場所(列)と値が簡単に計算できる場合

      その場で(行列ベクトル積の実行時に)計算

      例) \href{https://github.com/todo-group/computer-experiments/blob/master/exercise/eigenvalue_problem/matfree.c}{matfree.c} (一次元横磁場イジング模型)
    \end{itemize}
  \end{itemize}
\end{frame}
