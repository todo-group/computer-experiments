\begin{frame}[t,fragile]{量子アニーリング}
  \begin{itemize}
    %\setlength{\itemsep}{1em}
  \item 横磁場を導入
    \[
    H = -J \sum_{i<j} \epsilon_{ij} \sigma_i^z \sigma_j^z - \Gamma \sum_i \sigma_i^x
    \]
  \item 古典極限 ($J=1$, $\Gamma=0$)
    \begin{itemize}
    \item 求めたい基底状態
    \end{itemize}
  \item 量子極限 ($J=0$, $\Gamma=1$)
    \begin{itemize}
    \item $2^N$個の全ての状態の重ね合わせ
    \end{itemize}
  \item 量子アニーリング
    \begin{itemize}
    \item $J+\Gamma=1$を保ったままで、$\Gamma=1$から$\Gamma=0$まで「ゆっくり」と減少させながら時間発展させる
      \[
      J = 1-\Gamma = t/T \qquad (0 \le t \le T)
      \]
    \item $T \rightarrow \infty$の極限で確率1で基底状態に収束
    \end{itemize}
  \end{itemize}
\end{frame}
