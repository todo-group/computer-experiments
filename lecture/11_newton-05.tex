\begin{frame}[t,fragile]{準ニュートン法}
  \begin{itemize}
    %\setlength{\itemsep}{1em}
  \item ニュートン法では、ヘッセ行列の計算・保存が必要
  \item 準ニュートン法: それまでの反復で計算した勾配ベクトルから、ヘッセ行列を近似($B_n$)
  \item BFGS法(Broyden-Fletcher-Goldfarb-Shanno)
    \[
    B_{n+1} = B_{n} + \frac{y_n y_n^T}{y_n^T s_n} - \frac{B_{n} s_n (B_{n} s_n)^T}{s_n^T B_n s_n}
    \]
  \item $s_n = x_{n+1} - x_n$、$y_n = \nabla g(x_{n+1}) - \nabla g(x_n)$
  \item 直接$B_{n}$の逆行列$C_{n}$を更新することも可能
    \[
    C_{n+1} = B_{n+1}^{-1} = C_n + \Big( 1 + \frac{y_n^T C_n y_n}{y_n^T s_n} \Big)
    \frac{s_n s_n^T}{y_n^T s_n} - \frac{C_n y_n s_n^T + s_n y_n^T C_n^T}{y_n^T s_n} \]
  \item 他にも、SR1法、BHHH法、記憶制限BFGS法
  \end{itemize}
\end{frame}
