%-*- coding:utf-8 -*-

\documentclass[dvipdfmx]{beamer}
\usepackage{tutorial}

\title{計算機実験II (L1) --- 対角化と量子力学}
\date{2017/09/29}

\begin{document}

\begin{frame}
  \titlepage
  \tableofcontents
\end{frame}

\section{講義・実習の概要}

\begin{frame}[t]{講義・実習の目的}
  \begin{itemize}
    %\setlength{\itemsep}{1em}
  \item 理論・実験を問わず、学部〜大学院〜で必要となる現代的かつ普遍的な計算機の素養を身につける
  \item {\color{gray}UNIX環境に慣れる(シェル、ファイル操作、エディタ)}
  \item {\color{gray}ネットワークの活用 (リモートログイン、バージョン管理、共同作業)}
  \item {\color{gray}プログラムの作成(C言語、コンパイラ、プログラム実行)}
  \item 基本的な数値計算アルゴリズム・数値計算の常識を学ぶ
  \item {\color{gray}科学技術文書作成に慣れる(\LaTeX, グラフ作成)}
  \item {\color{gray}(Mathematica (数式処理)、Python (スクリプト言語)等の利用)}
  \item {\color{red}物理学における具体的な問題を通して実践的な知識と経験を身につける}
  \end{itemize}
\end{frame}

\begin{frame}[t]{身に付けて欲しいこと}
  \begin{itemize}
    %\setlength{\itemsep}{1em}
  \item ツールとしてないものは自分で作る (物理の伝統)
  \item すでにあるものは積極的に再利用する (車輪の再発明をしない)
  \item 数学公式と数値計算アルゴリズムは別物
  \item 刻み幅・近似度合いを変えて何度か計算を行う
  \item グラフ化して目で見てみる
  \item 計算量(コスト)のスケーリング(次数)に気をつける
  \item 記録に残す・再現性を確保する
  \item {\color{red}問題の解き方は一通りではない}
  \end{itemize}
\end{frame}

\begin{frame}[t]{「計算機実験I」で学んだ項目}
  \begin{itemize}
    %\setlength{\itemsep}{1em}
  \item UNIX、ネットワーク、\LaTeX、gnuplot
  \item プログラミング: C言語、配列、数値計算ライブラリ
  \item 数値計算の基礎: 数値誤差、ニュートン法
  \item 常微分方程式: Euler法・Runge-Kutta法、Numerov法、シンプレクティック積分法、陰解法
  \item 連立一次方程式: ガウスの消去法、LU分解、ヤコビ法・ガウスザイデル法・SOR法
  \item 行列の対角化: Jacobi法、Householder法、べき乗法、Lancos法、特異値分解
  \item 線形回帰の基礎
  \end{itemize}
\end{frame}

\begin{frame}[t]{講義・実習内容}
  問題解決型
  \begin{itemize}
    \setlength{\itemsep}{1em}
  \item (数値対角化と)量子力学
  \item (モンテカルロ法と)統計物理
  \item (最適化と)実験データ解析
  \item {\color{gray}スパコンと並列計算}
  \end{itemize}
\end{frame}

\begin{frame}[t,fragile]{講義と実習}
  \begin{itemize}
    \setlength{\itemsep}{1em}
  \item スタッフ \href{mailto:computer@exa.phys.s.u-tokyo.ac.jp}{computer@exa.phys.s.u-tokyo.ac.jp}
    \begin{itemize}
    \item 講義: 藤堂
    \item 実習: 鈴木(早野研)、斉藤(古澤研)
    \item 実習TA: 鈴木(藤堂研M1)、井坂(古澤研M1)
    \end{itemize}
  \item 評価
    \begin{itemize}
    \item 出席(講義・実習)
    \item レポート
    \end{itemize}    
  \end{itemize}    
\end{frame}

\begin{frame}[t]{質問がある場合には、、、}
  \begin{enumerate}
    %\setlength{\itemsep}{1em}
  \item ITC-LMS の掲示板を見る
  \item ハンドブック、講義資料を確認
  \item まわりの人に質問してみる
  \item ネットで検索
  \item 計算機実験担当者(\href{mailto:computer@exa.phys.s.u-tokyo.ac.jp}{computer@exa.phys.s.u-tokyo.ac.jp}) に相談
  \end{enumerate}
  メールで質問するときに注意すべきこと
  \begin{itemize}
  \item (メールの)標題をきちんとつける、きちんと名乗る
  \item 実行環境を明示する
  \item 問題を再現する手順を明記する
  \item 関連するファイル(Cや \LaTeX のソースコード等)を添付する
  \item エラーメッセージを添付する
  \end{itemize}
\end{frame}

\begin{frame}[t,fragile]{実習環境}
  \begin{itemize}
    \setlength{\itemsep}{1em}
  \item 情報基盤センター大演習室 (iMac端末)
    \begin{itemize}
    \item Cプログラミング、\LaTeX、Gnuplot、などに利用
    \end{itemize}
  \item 計算機端末室
    \begin{itemize}
    \item 理学部4号館1215室 (iMac 16台)
    \item {\color{red}(長期休暇を除き)週7日24時間利用可}
    \end{itemize}
  \item 物理学教室ワークステーションクラスタ photon
    \begin{itemize}
    \item SSHでリモートログインして使用する(ハンドブック2.2節)
    \item あらかじめ公開鍵の登録が必要(計算機実験I 実習EX0 準備練習2,3)
    \end{itemize}
  \item MateriApps LIVE! (USBメモリで配布)
    \begin{itemize}
    \item Mac, Windows PC 上で動作する仮想UNIX環境
    \item 再度インストールを行いたい人はUSBを貸与するので申し出ること
    \end{itemize}
  \end{itemize}
\end{frame}

\section{二重井戸ポテンシャル}

\begin{frame}[t,fragile]{二重井戸ポテンシャル中の粒子}
  \begin{itemize}
    %\setlength{\itemsep}{1em}
  \item 時間依存しないシュレディンガー方程式
    \begin{align*}
      \big[ -\frac{d^2}{dx^2} + V(x) \big] \psi(x) = E \psi(x)
    \end{align*}
    ($\hbar^2/2m = 1$となるように単位をとった)
  \item 二重井戸ポテンシャル
    \begin{align*}
      V(x) = \begin{cases}
        \infty & \text{$x < 0$, $x > 1$} \\
        0 & \text{$0 < x < a$, $b < x < 1$} \\
        v & \text{$a < x < b$}
      \end{cases}
    \end{align*}
    ただし、$0<a<b<1$とする
  \item 境界条件: $\psi(0) = \psi(1) = 0$、$0 < x < 1$で$\psi(x)$とその導関数が連続
  \end{itemize}
\end{frame}

\begin{frame}[t,fragile]{シュレディンガー方程式の解法}
  \begin{itemize}
    \setlength{\itemsep}{1em}
  \item シューティング
    \begin{itemize}
    \item 計算機実験I (L2)
    \item シューティングに用いる積分法: 2階常微分方程式の2次元1階連立微分方程式への書き換え、オイラー法とその改良、Numerov法
    \end{itemize}
  \item ハミルトニアンの対角化
    \begin{itemize}
    \item 計算機実験I (L4)
    \item 対角化手法: ハウスホルダー法(LAPACK)、べき乗法、Lanczos法
    \end{itemize}
  \item その他の方法: 手で解けるところはあらかじめ解いて次元を減らす
  \item それぞれのコスト(=計算時間・メモリ)は?
  \end{itemize}
\end{frame}


\begin{frame}[t,fragile]{シューティング・対角化}
  \begin{itemize}
    \setlength{\itemsep}{1em}
  \item シューティング
    \begin{itemize}
    \item 計算機実験I (L1) p.35
    \item 2階常微分方程式の2次元1階連立微分方程式への書き換え[計算機実験I (L1) p.2]
    \item オイラー法とその改良[計算機実験I (L1) p.4]
    \item Numerov法[計算機実験I (L1) p.12]
    \end{itemize}
  \item ハミルトニアンの対角化
    \begin{itemize}
    \item 計算機実験I (L3) p.17
    \item ハウスホルダー法(LAPACK) [計算機実験I (L3) p.26]
    \item べき乗法[計算機実験I (L3) p.30]
    \item Lanczos法[計算機実験I (L3) p.34]
    \end{itemize}
  \end{itemize}
\end{frame}

\section{二分法}

\begin{frame}[t,fragile]{二分法}
  \begin{itemize}
    % \setlength{\itemsep}{1em}
  \item 反復法により一次元の方程式$f(x)=0$の解を求める
  \item 導関数を使わず関数値のみを利用 (c.f. ニュートン法)
  \item 初期条件として、$f(a) \times f(b) < 0$を満たす2点の組($a<b$)で解をはさみ込み、領域を狭めていく
  \item $a$と$b$の中点$x=(a+b)/2$を考える
    \begin{itemize}
    \item $|f(x)|$が十分小さい場合: $x$が解
    \item $f(a) \times f(x) < 0$の場合: $[a,x]$を新しい領域にとる
    \item $f(x) \times f(b) < 0$の場合: $[x,b]$を新しい領域にとる
    \end{itemize}
  \item 領域$[a,b]$の幅が十分小さくなったら終了
  \item 反復のたびに領域の幅は半分になる
  \item 全ての解を得られる保証はない
  \item 二分法の例: \href{https://github.com/todo-group/computer-experiments/blob/master/exercise/basics/bisection.c}{example-2-L1/bisection.c}
  \end{itemize}
\end{frame}


\end{document}
