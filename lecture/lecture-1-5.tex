% -*- coding: utf-8 -*-

\documentclass[10pt,dvipdfmx]{beamer}
\usepackage{tutorial}

\title{計算機実験I (第5回)}
\date{2020/05/27}

\begin{document}

\begin{frame}
  \titlepage
  \tableofcontents
\end{frame}

\begin{frame}[t]{講義予定}
  \begin{itemize}
    %\setlength{\itemsep}{1em}
  {\color{gray} \item 2020-04-22: 第1回 環境整備}
  {\color{gray} \item 2020-05-07: 第2回 計算機実験の基礎}
  {\color{gray} \item 2020-05-13: 第3回 常微分方程式の解法 \\
    \hspace*{5em} (レポートNo.1出題 5/29締切)}
  {\color{gray} \item 2020-05-20: 第4回 行列演算とライブラリ}
\item 2020-05-27: 第5回 連立一次方程式の解法 \\
  \hspace*{5em} (レポートNo.2出題 6/12締切)
\item 2020-06-03: 第6回 行列の対角化
\item 2020-06-10: 第7回 疎行列に対する反復解法
\item 2020-06-17: 第8回 特異値分解と最小二乗法 \\
  \hspace*{5em} (レポートNo.3出題 7/3締切)
  \end{itemize}
\end{frame}

\begin{frame}[t]{本日の課題}
  \begin{itemize}
    %\setlength{\itemsep}{1em}
  \item 「\href{https://utphys-comp.github.io}{計算機実験のための環境整備}」({\small \href{https://utphys-comp.github.io}{https://utphys-comp.github.io}})が完了していない人は至急連絡を!
  \item 実習
    \begin{itemize}
    \item 講義資料の中の{\tt gauss.c}, {\tt lu\_decomp.c}をコンパイル・実行。ソースコードの中身を確認 \\
      サンプルコード一式: \href{https://github.com/todo-group/ComputerExperiments/releases/tag/2020s-computer1}{example-1-5.zip} (前回のものも含む)
    \item 実習課題一覧\href{https://github.com/todo-group/ComputerExperiments/releases/tag/2020s-computer1}{exercise-1.pdf}の課題11〜15 (あるいはそれ以外)から適宜選び実習
    \end{itemize}
  \item 質問はSlackの「\#4 連立方程式」あるいは他の適当と思われるチャンネルで \\[2em]
    
  \item レポートNo.2: 提出締切2020/6/12(金) 18:00 (レポート内容・提出方法についてはITC-LMSを参照のこと)
  \end{itemize}
\end{frame}

\section{物理にあらわれる連立一次方程式}
% -*- coding: utf-8 -*-

\documentclass[10pt,dvipdfmx]{beamer}
\usepackage{tutorial}

\begin{document}
\section{行列とLAPACK}
\begin{frame}[t,fragile]{二次元配列}
  \begin{itemize}
    \setlength{\itemsep}{1em}
  \item C言語では、二次元配列は一次元配列の先頭をさす(ポインタ)の配列として表される(と理解しておけば良い)
  \item \verb+a[i]+は、要素\verb+a[i][0]+を指すポインタ
    \begin{itemize}
    \item \verb+a+ と \verb+&a[0]+ は等価 (\verb+&a[0][0]+ ではない)
    \item \verb+a[0]+ と \verb+&a[0][0]+ は等価
    \item \verb+a[2]+ と \verb+&a[2][0]+ は等価
    \item \verb^(a+2)^ と \verb^&a[2]^ は等価
    \item \verb^(*(a+2))[3]^ と \verb^*(*(a+2)+3)^ と \verb^a[2][3]^ は等価
    \item \verb^*(a+2)[3]^ と \verb^*((a+2)[3])^ と \verb^*(a[5])^ と\verb^a[5][0]^ は等価
    \item \verb^[]^は\verb^*^よりも強い
    \end{itemize}
  \item ポインタ確認プログラム: \href{https://github.com/todo-group/computer-experiments/blob/master/exercise/matrix/pointer-matrix.c}{pointer-matrix.c}
  \end{itemize}
\end{frame}

\begin{frame}[t,fragile]{動的二次元配列の確保}
  \begin{itemize}
    \setlength{\itemsep}{1em}
  \item 各行を表す配列とそれぞれの先頭アドレスを保持する配列の二種類が必要
\begin{lstlisting}
double **a;
m = 10;  
n = 10;  
a = (double**)malloc((size_t)(m * sizeof(double*));
for (int i = 0; i < m; ++i)
  a[i] = (double*)malloc((size_t)(n * sizeof(double));
\end{lstlisting}
\item 各行を保持する配列が、メモリ上で連続に確保される保証はない
\item 行列用のライブラリ(LAPACK等)を使うときに問題となる
  \end{itemize}
\end{frame}

\begin{frame}[t,fragile]{BLASライブラリ}
  \begin{itemize}
    \setlength{\itemsep}{1em}
  \item 行列・行列積、行列・ベクトル積などを高速に行う最適化された関数群
  \item 行列・行列積を計算するサブルーチン {\tt dgemm} \\
    \url{http://www.netlib.org/lapack/explore-html/d7/d2b/dgemm_8f.html}
    \begin{itemize}
    \item $C = \alpha A \times B + \beta C$ を計算
    \item BLASもFortranで書かれている
    \end{itemize}
  \item 例: \href{https://github.com/todo-group/computer-experiments/blob/master/exercise/matrix/multiply.c}{multiply.c}, \href{https://github.com/todo-group/computer-experiments/blob/master/exercise/matrix/multiply_dgemm.c}{multiply\_dgemm.c}
  \end{itemize}
\end{frame}

\begin{frame}[t,fragile]{LAPACK (Linear Algebra PACKage)}
  \begin{itemize}
    %\setlength{\itemsep}{1em}
  \item 線形計算のための高品質な数値計算ライブラリ
    \begin{itemize}
    \item \url{http://www.netlib.org/lapack}
    \item 線形方程式、固有値問題、特異値問題、線形最小二乗問題など
    \item (FFT 高速フーリエ変換は入っていない)
    \item LAPACK自体はFortranで書かれている
    \end{itemize}
  \item ほぼ全てのPC、ワークステーション、スーパーコンピュータで利用可 (インストール済)
  \item Netlibでソースが公開されているリファレンス実装は遅いが、それぞれのベンダー(Intel、Fujitsu、etc)による最適化されたLAPACKが用意されている場合が多い(MKL、SSL2、etc)
  \item LAPACKを使うことにより、高速で信頼性が高く、ポータブルなコードを書くことが可能になる
  \end{itemize}
\end{frame}

\begin{frame}[t,fragile]{LAPACKによる連立一次方程式の求解}
  \begin{itemize}
    \setlength{\itemsep}{1em}
  \item LU分解を行うサブルーチン {\tt dgetrf} \\
    \url{http://www.netlib.org/lapack/explore-html/d3/d6a/dgetrf_8f.html}
  \item Fortranによる関数宣言
\begin{lstlisting}
subroutine dgetrf(integer M, integer N,
         double precision, dimension(lda, *) A,
         integer LDA, integer, dimension(*) IPIV,
         integer INFO)
\end{lstlisting}
\item {\tt A}: 左辺の行列、{\tt M,N}: 次元、{\tt IPIV}: 選択されたピボット行のリスト、{\tt lda}: 通常{\tt M} (行数)と同じで良い
  \end{itemize}
\end{frame}

\begin{frame}[t,fragile]{CからBLAS/LAPACKを呼び出す際の注意事項}
  \begin{itemize}
    %\setlength{\itemsep}{1em}
  \item (もともとFortran言語で書かれていたことによる制限)
  \item 関数名はすべて小文字、最後に \verb+_+ (下線)を付ける
  \item スカラー、ベクトル、行列は全て「ポインタ渡し」とする
  \item ベクトルや行列は最初の要素へのポインタを渡す (サイズは別に渡す)
  \item 行列の要素は(0,0) $\rightarrow$ (1,0) $\rightarrow$ (2,0) $\rightarrow\cdots\rightarrow$ $(m-1,0)$ $\rightarrow$ (0,1) $\rightarrow$ (1,1) $\rightarrow\cdots\rightarrow$ $(m-1,n-1)$の順で連続して並んでいなければならない(column-major)
    \begin{itemize}
    \item C言語の二次元配列では \verb+a[i][j]+ の次には \verb%a[i][j+1]%が入っている(row-major)
    \item 行列が転置されて解釈されてしまう!
    \end{itemize}
  \item コンパイル時には{\tt -llapack -lblas}オプションを指定し、LAPACKライブラリとBLASライブラリをリンクする(ハンドブック2.1.6節)
  \end{itemize}
\end{frame}

\begin{frame}[t,fragile]{cmatrix.hライブラリ}
  \begin{itemize}
    %\setlength{\itemsep}{1em}
  \item Column-major形式の二次元配列の確保({\tt alloc\_dmatrix})、開放({\tt free\_dmatrix})、出力({\tt print\_dmatrix})、読み込み({\tt read\_dmatrix})を行うためのユーティリティ関数、(i,j)成分にアクセスするためのマクロ({\tt mat\_elem})他を準備
  \item ソースコード: \href{https://github.com/todo-group/computer-experiments/blob/master/exercise/include/cmatrix.h}{cmatrix.h}
  \item 使用例
\begin{lstlisting}
#include "cmatrix.h"
...
double **mat;
mat = alloc_dmatrix(m, n);
mat_elem(mat, 1, 3) = 5.0;
...
free_dmatrix(mat);
\end{lstlisting}
  \item サンプルコード: \href{https://github.com/todo-group/computer-experiments/blob/master/exercise/matrix/matrix_example.c}{matrix\_example.c}
  \end{itemize}
\end{frame}

\begin{frame}[t,fragile]{alloc\_dmatrixでの動的二次元配列の確保}
  \begin{itemize}
    %\setlength{\itemsep}{1em}
  \item 長さ$m \times n$の一次元配列を用意し、各列(それぞれ$m$要素)の先頭アドレスを長さ$n$のポインター配列に格納する
\begin{lstlisting}
double **a;
m = 10;  
n = 10;  
a = (double**)malloc((size_t)(n * sizeof(double*));
a[0] = (double*)malloc((size_t)(m*n * sizeof(double));
for (int i = 1; i < n; ++i)
  a[i] = a[i-1] + m;
\end{lstlisting}
\item 行列の(i,j)成分を\verb+a[j][i]+に格納することにする (column-major)
  \end{itemize}
\end{frame}

\begin{frame}[t,fragile]{要素アクセス・先頭アドレス}
  \begin{itemize}
    % \setlength{\itemsep}{1em}
  \item 行列の(i,j)成分を\verb+a[j][i]+に格納することにする(column-major)
    \begin{itemize}
      \item \href{https://github.com/todo-group/computer-experiments/blob/master/exercise/matrix/cmatrix.h}{cmatrix.h}ではマクロ(\verb+mat_elem+)を準備
\begin{lstlisting}
#define mat_elem(mat, i, j) (mat)[j][i]
\end{lstlisting}
\item このマクロを使うと、例えば(i,j)成分への代入は以下のように書ける
\begin{lstlisting}
mat_elem(a, i, j) = 1;
\end{lstlisting}
\end{itemize}
  \item LAPACKにベクトルや行列の最初の要素へのポインタを渡す
    \begin{itemize}
      \item ベクトルの最初の要素(0)へのポインタ: \verb+&v[0]+
      \item 行列の最初の要素(0,0)へのポインタ: \verb+&a[0][0]+
      \item \href{https://github.com/todo-group/computer-experiments/blob/master/exercise/matrix/cmatrix.h}{cmatrix.h}にマクロ({\tt vec\_ptr}、{\tt mat\_ptr})が準備されているのでそれぞれ、{\tt vec\_ptr(v)}、{\tt mat\_ptr(a)}と書ける
    \end{itemize}
  \end{itemize}
\end{frame}

\begin{frame}[t,fragile]{LAPACKによる連立一次方程式の求解}
  \begin{itemize}
    \setlength{\itemsep}{1em}
  \item C言語から呼び出すための関数宣言を作成 (ハンドブック3.6.4節)
\begin{lstlisting}
void dgetrf_(int *M, int *N, double *A,
             int *LDA, int*IPIV, int *INFO);
\end{lstlisting}
関数名は全て小文字。関数名の最後に {\tt \_} (下線)を付ける
\item LU分解の例
\begin{lstlisting}
m = 10;
n = 10;
a = alloc_dmatrix(m, n);
...
dgetrf_(&m, &n, mat_ptr(a), &m, vec_ptr(ipiv), &info);
\end{lstlisting}
完全なソースコード: \href{https://github.com/todo-group/computer-experiments/blob/master/exercise/linear_system/lu_decomp.c}{lu\_decomp.c}
  \end{itemize}
\end{frame}

\end{document}

\begin{frame}[t,fragile]{ポアソン方程式の境界値問題}
  \begin{itemize}
    %\setlength{\itemsep}{1em}
  \item 二次元ポアソン方程式
    \[ \frac{\partial^2 u(x,y)}{\partial x^2} + \frac{\partial^2 u(x,y)}{\partial y^2} = f(x,y) \qquad 0 \le x \le 1, \ 0 \le y \le 1\]
  \item ディリクレ型境界条件: $u(x,y) = g(x,y)$ on $\partial \Omega$
  \item 有限差分法により離散化
    \begin{itemize}
    \item $x$方向、$y$方向をそれぞれ$n$等分: $(x_i,y_j) = (i/n, j/n)$
    \item $(n+1)^2$個の格子点の上で$u(x_i,y_j)=u_{ij}$が定義される
    \item そのうち$4n$個の値は境界条件で定まる
    \item ポアソン方程式を中心差分で近似 ($h=1/n$)
      \[
      \frac{u_{i+1,j}-2u_{ij}+u_{i-1,j}}{h^2} + \frac{u_{i,j+1}-2u_{ij}+u_{i,j-1}}{h^2} = f_{ij}
      \]
      残り$(n-1)^2$個の未知数に対する連立一次方程式
    \end{itemize}
  \end{itemize}
\end{frame}

\begin{frame}[t,fragile]{シュレディンガー方程式の行列表示}
  \begin{itemize}
    %\setlength{\itemsep}{1em}
  \item シュレディンガー方程式
    \[
    [-\frac{d^2}{dx^2}+V(x)]\psi(x) = E \psi(x)
    \]
  \item 連立差分方程式を行列の形で表す($\psi(x_0)=\psi(x_n)=0$)
    \[\begin{small}\hspace*{-4em}
    \begin{pmatrix}
      \frac{2}{h^2}+V(x_1) & -\frac{1}{h^2} \\
      -\frac{1}{h^2} & \frac{2}{h^2}+V(x_2) & -\frac{1}{h^2} \\
      & -\frac{1}{h^2} & \frac{2}{h^2}+V(x_3) & -\frac{1}{h^2} \\
      & & \ddots & \ddots \\
      & & & -\frac{1}{h^2} & \frac{2}{h^2}+V(x_{n-1}) \\
    \end{pmatrix}
    \begin{pmatrix}
      \psi(x_1) \\
      \psi(x_2) \\
      \psi(x_3) \\
      \vdots \\
      \psi(x_{n-1}) \\
    \end{pmatrix}
    = \cdots % E
    %% \begin{pmatrix}
    %%   \psi(x_1) \\
    %%   \psi(x_2) \\
    %%   \psi(x_3) \\
    %%   \vdots \\
    %%   \psi(x_{n-1}) \\
    %% \end{pmatrix}
    \end{small}
    \]
  \item $(n-1) \times (n-1)$の疎行列の固有値問題
    \begin{itemize}
    \item 固有値: 固有エネルギー
    \item 固有ベクトル: 波動関数
    \end{itemize}
  \end{itemize}
\end{frame}


% -*- coding: utf-8 -*-

\section{連立一次方程式の直接解法}

\begin{frame}[t,fragile]{逆行列の「間違った」求め方}
  \begin{itemize}
    \setlength{\itemsep}{1em}
  \item 線形代数の教科書に載っている公式
    \[
    A^{-1} = \frac{\tilde{A}}{|A|}
    \]
    $|A|$: $A$の行列式、$\tilde{A}$: $A$の余因子行列
  \item $n \times n$行列の行列式を定義通り計算すると、計算量〜$O(n!)$
  \item したがって、上の方法で逆行列を計算すると、計算量〜$O(n!)$
  \item $n=100$の場合: $n! \approx 9.3 \times 10^{157}$
  \end{itemize}
\end{frame}

\begin{frame}[t,fragile]{逆行列の「正しい」求め方}
  \begin{itemize}
    \setlength{\itemsep}{1em}
  \item 連立一次方程式 $A {\bf x} = {\bf e}_j$ を全ての${\bf e}_j$について解く
  \item Gaussの消去法による連立一次方程式の解法: 計算量〜$O(n^3)$
  \item Gaussの消去法の途中で出てくる下三角行列(L)と上三角行列(U)行列を再利用(LU分解)すれば、逆行列全体を求めるための計算量も$O(n^3)$
  \item 行列式も$O(n^3)$で計算可
  \item $n=100$の場合: $n^3 = 10^6 \ll 9.3 \times 10^{157}$
  \end{itemize}
\end{frame}

\begin{frame}[t,fragile]{ガウスの消去法}
  \begin{itemize}
    %\setlength{\itemsep}{1em}
  \item 解くべき連立方程式
    \begin{align*}
    a_{11}^{(1)} x_1 + a_{12}^{(1)} x_2 + a_{13}^{(1)} x_3 + \cdots + a_{1n}^{(1)} x_n &= b_{1}^{(1)} \\
    a_{21}^{(1)} x_1 + a_{22}^{(1)} x_2 + a_{23}^{(1)} x_3 + \cdots + a_{2n}^{(1)} x_n &= b_{2}^{(1)} \\
    a_{31}^{(1)} x_1 + a_{32}^{(1)} x_2 + a_{33}^{(1)} x_3 + \cdots + a_{3n}^{(1)} x_n &= b_{3}^{(1)} \\
    \cdots \\
    a_{n1}^{(1)} x_1 + a_{n2}^{(1)} x_2 + a_{n3}^{(1)} x_3 + \cdots + a_{nn}^{(1)} x_n &= b_{n}^{(1)}
    \end{align*}
  \item ある行を定数倍しても、方程式の解は変わらない
  \item ある行の定数倍を他の行から引いても、方程式の解は変わらない
  \end{itemize}
\end{frame}

\begin{frame}[t,fragile]{ガウスの消去法}
  \begin{itemize}
    \setlength{\itemsep}{1em}
  \item 1行目を$m_{i1} = a_{i1}^{(1)}/a_{11}^{(1)}$倍して、$i$行目($i \ge 2$)から引く
    \begin{align*}
    a_{11}^{(1)} x_1 + a_{12}^{(1)} x_2 + a_{13}^{(1)} x_3 + \cdots + a_{1n}^{(1)} x_n &= b_{1}^{(1)} \\
    a_{22}^{(2)} x_2 + a_{23}^{(2)} x_3 + \cdots + a_{2n}^{(2)} x_n &= b_{2}^{(2)} \\
    a_{32}^{(2)} x_2 + a_{33}^{(2)} x_3 + \cdots + a_{3n}^{(2)} x_n &= b_{3}^{(2)} \\
    \cdots \\
    a_{n2}^{(2)} x_2 + a_{n3}^{(2)} x_3 + \cdots + a_{nn}^{(2)} x_n &= b_{n}^{(2)}
    \end{align*}
  \item ここで
    \begin{align*}
      a_{ij}^{(2)} &= a_{ij}^{(1)} - m_{i1} a_{1j}^{(1)} \qquad i \ge 2, j \ge 2 \\
      b_{i}^{(2)} &= b_{i}^{(1)} - m_{i1} b_{1}^{(1)} \qquad i \ge 2
    \end{align*}
  \end{itemize}
\end{frame}

\begin{frame}[t,fragile]{ガウスの消去法}
  \begin{itemize}
    %\setlength{\itemsep}{1em}
  \item 2行目を$m_{i2} = a_{i2}^{(2)}/a_{22}^{(2)}$倍して、$i$行目($i \ge 3$)から引く
    \begin{align*}
    a_{11}^{(1)} x_1 + a_{12}^{(1)} x_2 + a_{13}^{(1)} x_3 + \cdots + a_{1n}^{(1)} x_n &= b_{1}^{(1)} \\
    a_{22}^{(2)} x_2 + a_{23}^{(2)} x_3 + \cdots + a_{2n}^{(2)} x_n &= b_{2}^{(2)} \\
    a_{33}^{(3)} x_3 + \cdots + a_{3n}^{(3)} x_n &= b_{3}^{(3)} \\
    \cdots \\
    a_{n3}^{(3)} x_3 + \cdots + a_{nn}^{(3)} x_n &= b_{n}^{(3)}
    \end{align*}
  \item ここで
    \begin{align*}
      a_{ij}^{(3)} &= a_{ij}^{(2)} - m_{i2} a_{2j}^{(2)} \qquad i \ge 3, j \ge 3 \\
      b_{i}^{(3)} &= b_{i}^{(2)} - m_{i2} b_{2}^{(2)} \qquad i \ge 3
    \end{align*}
  \end{itemize}
\end{frame}

\begin{frame}[t,fragile]{ガウスの消去法}
  \begin{itemize}
    %\setlength{\itemsep}{1em}
  \item 最終的には、左辺が右上三角形をした連立方程式となる
    \begin{align*}
    a_{11}^{(1)} x_1 + a_{12}^{(1)} x_2 + a_{13}^{(1)} x_3 + \cdots + a_{1n}^{(1)} x_n &= b_{1}^{(1)} \\
    a_{22}^{(2)} x_2 + a_{23}^{(2)} x_3 + \cdots + a_{2n}^{(2)} x_n &= b_{2}^{(2)} \\
    a_{33}^{(3)} x_3 + \cdots + a_{3n}^{(3)} x_n &= b_{3}^{(3)} \\
    \cdots \\
    a_{n-1,n-1}^{(n-1)} x_{n-1} + a_{n-1,n}^{(n-1)} x_n &= b_{n-1}^{(n-1)} \\
    a_{nn}^{(n)} x_n &= b_{n}^{(n)}
    \end{align*}
  \item これを下から順に解いていけばよい(後退代入)
  \end{itemize}
\end{frame}

\begin{frame}[t,fragile]{練習問題}
  \begin{itemize}
    \setlength{\itemsep}{1em}
  \item 次の連立方程式をガウスの消去法で(手で)解け
    \begin{align*}
      \begin{pmatrix} 1 & 4 & 7 \\ 2 & 5 & 8 \\ 3 & 6 & 10 \end{pmatrix} \begin{pmatrix} x_1 \\ x_2 \\ x_3 \end{pmatrix} = \begin{pmatrix} 18 \\ 24 \\ 31 \end{pmatrix}
    \end{align*}
  \end{itemize}
\end{frame}

\begin{frame}[t,fragile]{ガウスの消去法のコード}
\begin{lstlisting}
for (k = 0; k < n; ++k) {
  for (i = k + 1; i < n; ++i) {
    for (j = k + 1; j < n; ++j) {
      mat_elem(a,i,j) -= mat_elem(a,k,j) * mat_elem(a,i,k) / mat_elem(a,k,k);
    }
    b[i] -= b[k] * mat_elem(a,i,k) / mat_elem(a,k,k);
  }
}
for (k = n-1; k >= 0; --k) {
  for (j = k + 1; j < n; ++j) {
    b[k] -= mat_elem(a,k,j) * b[j];
  }
  b[k] /= mat_elem(a,k,k);
}
\end{lstlisting}
\begin{itemize}
\item C言語では配列の添字が0から始まることに注意
\item \verb+mat_elem(a,i,j)+は行列aの(i,j)成分を表す({\tt cmatrix.h}中で定義)
\item サンプルコード: \href{https://github.com/todo-group/computer-experiments/blob/master/exercise/linear_system/gauss.c}{gauss.c}
\item 入力ファイル: \href{https://github.com/todo-group/computer-experiments/blob/master/exercise/linear_system/input1.dat}{input1.dat}
\end{itemize}
\end{frame}

\begin{frame}[t,fragile]{ピボット選択}
  \begin{itemize}
    %\setlength{\itemsep}{1em}
  \item ガウスの消去法の途中で$a_{kk}^{(k)}$が零になると、計算を先に進めることができなくなる
  \item 行を入れ替えても、方程式の解は変わらない $\Rightarrow$ $k$行以降で、$a_{ik}^{(k)}$が非零の行と入れ替える (ピボット選択)
  \item 実際のコードでは、情報落ちを防ぐため、$a_{kk}^{(k)}$が零でない場合でも、$a_{ik}^{(k)}$の絶対値が最大の行と入れ替える
  \item ピボット選択が必要となる例
    \begin{align*}
      \begin{pmatrix} 1 & 2 & 3 \\ 3 & 6 & 4 \\ 4 & 6 & 7 \end{pmatrix} \begin{pmatrix} x_1 \\ x_2 \\ x_3 \end{pmatrix} = \begin{pmatrix} 8 \\ 19 \\ 23 \end{pmatrix}
    \end{align*}
    \item 行列がrank落ちしている場合は、ピボット選択を行っても途中で0になる (cf. 特異値分解を用いた最小二乗解)
  \end{itemize}
\end{frame}

\begin{frame}[t,fragile]{ガウスの消去法の行列表示}
  \begin{itemize}
    \setlength{\itemsep}{1em}
  \item $a_{kk}^{(k)}$を用いた$a_{ik}^{(k)}$ ($i>k$)の消去は、方程式の両辺に
    \begin{align*}
      M_k = 
      \begin{pmatrix}
        1 & \\
        0 & 1 \\
        0 & 0 & \ddots \\
        \vdots & \vdots & & 1 \\
        \vdots & \vdots & & -m_{k+1,k} & 1 & \\
        \vdots & \vdots & & -m_{k+2,k} & 0 & \ddots \\
        \vdots & \vdots & & \vdots & \vdots & & 1 & \\
        0 & 0 & \hdots & -m_{nk} & 0 & \hdots & 0 & 1
      \end{pmatrix}
    \end{align*}
    を掛けるのと等価: $M_k A^{(k)} = A^{(k+1)}$、$M_k {\bf b}^{(k)} = {\bf b}^{(k+1)}$
  \end{itemize}
\end{frame}

\begin{frame}[t,fragile]{LU分解}
  \begin{itemize}
    %\setlength{\itemsep}{1em}
  \item $M_k$の逆行列
    \begin{align*}
      L_k = M_k^{-1} = 
      \begin{bmatrix}
        1 & \\
        0 & 1 \\
        0 & 0 & \ddots \\
        \vdots & \vdots & & 1 \\
        \vdots & \vdots & & m_{k+1,k} & 1 & \\
        \vdots & \vdots & & m_{k+2,k} & 0 & \ddots \\
        \vdots & \vdots & & \vdots & \vdots & & 1 & \\
        0 & 0 & \hdots & m_{nk} & 0 & \hdots & 0 & 1
      \end{bmatrix}
    \end{align*}
    から$L=L_1L_2\cdots L_{n-1}$を定義すると、$L$は下三角行列、また$U = A^{(n)}$ (上三角行列)とすると、$A = LU$
  \end{itemize}
\end{frame}

\begin{frame}[t,fragile]{LU分解}
  \begin{itemize}
    %\setlength{\itemsep}{1em}
  \item LU分解による連立一次方程式の解法
    \begin{itemize}
    \item 方程式は$A{\bf x} = LU{\bf x} = {\bf b}$と書ける
    \item まず、$L{\bf y} = {\bf b}$を解いて、${\bf y}$を求める(前進代入)
    \item 次に、$U{\bf x} = {\bf y}$を解いて、${\bf x}$を求める(後退代入)
    \end{itemize}
  \item 計算量はガウスの消去法と変わらない
  \item 一度LU分解をしておけば、異なる${\bf b}$に対する解も簡単に求められる \\[2em]
  \item 行列式は$U$の対角成分の積で与えられる (ピボット選択する場合は、行の入れ替えにより符号が変わることに注意)
  \end{itemize}
\end{frame}



\section{連立一次方程式の反復解法}

\begin{frame}[t,fragile]{直接法と反復法}
  \begin{itemize}
    %\setlength{\itemsep}{1em}
  \item 直接法: 連立方程式を有限回数($\sim n^3$)の手間で直接解く
  \item 反復法: $A{\bf x}={\bf b}$を、等価な${\bf x} = \phi({\bf x}) = M{\bf x} + {\bf c}$の形に変形し、適当な初期値${\bf x}_0$から出発して、${\bf x}^{(k+1)} = \phi({\bf x}^{(k)})$を繰り返して解を求める
    \begin{itemize}
    \item 欠点: 有限回数では終わらない (あらかじめ定めた収束条件が満たされるまで反復)
    \item 利点: 行列ベクトル積$M{\bf x}$が計算できさえすればよい。特に$M$が疎行列の場合には、$M{\bf x}$は非常に高速に計算できる可能性がある。メモリの点でも有利
    \item 利点: 直接法に比べて、プログラムも比較的単純になる場合が多い
    \end{itemize}
  \end{itemize}
\end{frame}

\begin{frame}[t,fragile]{反復法}
  \begin{itemize}
    %\setlength{\itemsep}{1em}
  \item 行列$A$を対角行列$D$、左下三角行列$E$、右上三角行列$F$の和に分解
    \[
    A{\bf x} = (D + E + F){\bf x} = {\bf b}
    \]
  \item ヤコビ法: 対角成分以外を右辺に移す
    \[
      {\bf x}^{(k+1)} = D^{-1} ({\bf b} - (E+F) {\bf x}^{(k)}) = -D^{-1}(E+F){\bf x}^{(k)} + D^{-1} {\bf b}
      \]
    \item ガウスザイデル法: ヤコビ法で右辺の${\bf x}$の値として、各段階ですでに得られている最新のものを使う
    \begin{align*}
      {\bf x}^{(k+1)} &= D^{-1} ({\bf b} - E{\bf x}^{(k+1)} - F{\bf x}^{(k)}) \\
      {\bf x}^{(k+1)} &= -(D+E)^{-1} F{\bf x}^{(k)} + (D+E)^{-1}{\bf b}
    \end{align*}
  \end{itemize}
\end{frame}

\begin{frame}[t,fragile]{反復法}
  \begin{itemize}
    %\setlength{\itemsep}{1em}
  \item SOR (Successive Over-Relaxation)法: ガウスザイデル法における修正量に1より大きな値($\omega$)を掛け、補正を加速
    \begin{align*}
      {\bf \xi}^{(k+1)} &= D^{-1} ({\bf b} - E{\bf x}^{(k+1)} - F{\bf x}^{(k)}) \\
      {\bf x}^{(k+1)} &= {\bf x}^{(k)} + \omega({\bf \xi}^{(k+1)} - {\bf x}^{(k)})
    \end{align*}
    ${\bf \xi}^{(k+1)}$を消去すると
    \begin{align*}
      {\bf x}^{(k+1)} = &(I+\omega D^{-1}E)^{-1} \{(1-\omega)I - \omega D^{-1} F\}{\bf x}^{(k)} \\ &+ \omega(D+\omega E)^{-1}{\bf b}
    \end{align*}
    \item 反復法は常に収束するとは限らない
    \item 行列$A$が対角優位、あるいは正定値対称行列の場合には収束が保証される
  \end{itemize}
\end{frame}


\begin{frame}[t]{本日の課題}
  \begin{itemize}
    %\setlength{\itemsep}{1em}
  \item 「\href{https://utphys-comp.github.io}{計算機実験のための環境整備}」({\small \href{https://utphys-comp.github.io}{https://utphys-comp.github.io}})が完了していない人は至急連絡を!
  \item 実習
    \begin{itemize}
    \item 講義資料の中の{\tt gauss.c}, {\tt lu\_decomp.c}をコンパイル・実行。ソースコードの中身を確認 \\
      サンプルコード一式: \href{https://github.com/todo-group/ComputerExperiments/releases/tag/2020s-computer1}{example-1-5.zip} (前回のものも含む)
    \item 実習課題一覧\href{https://github.com/todo-group/ComputerExperiments/releases/tag/2020s-computer1}{exercise-1.pdf}の課題11〜15 (あるいはそれ以外)から適宜選び実習
    \end{itemize}
  \item 質問はSlackの「\#4 連立方程式」あるいは他の適当と思われるチャンネルで \\[2em]
    
  \item レポートNo.2: 提出締切2020/6/12(金) 18:00 (レポート内容・提出方法についてはITC-LMSを参照のこと)
  \end{itemize}
\end{frame}

\end{document}
