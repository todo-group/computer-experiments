% -*- coding: utf-8 -*-

\documentclass[10pt,dvipdfmx]{beamer}
\usepackage{tutorial}

\title{計算機実験I (第1回)}
\date{2025/04/09}

\begin{document}

\begin{frame}
  \titlepage
  \tableofcontents
\end{frame}

\section{講義・実習の概要}

\begin{frame}[t]{講義・実習の目的}
  \begin{itemize}
    %\setlength{\itemsep}{1em}
  \item 理論・実験を問わず、学部〜大学院〜で必要となる現代的かつ普遍的な計算機の素養を身につける
  \item UNIX環境に慣れる(シェル、ファイル操作、エディタ)
  \item ネットワークの活用 (リモートログイン、共同作業)
  \item プログラムの作成(C言語、コンパイラ、プログラム実行)
  \item 基本的な数値計算アルゴリズム・数値計算の常識を学ぶ
  \item 科学技術文書作成に慣れる(\LaTeX, グラフ作成)
  \end{itemize}
\end{frame}

\begin{frame}[t]{身に付けて欲しいこと}
  \begin{itemize}
    %\setlength{\itemsep}{1em}
  \item ツールとしてないものは自分で作る (物理の伝統)
  \item すでにあるものは積極的に再利用する (車輪の再発明をしない)
  \item 数学公式と数値計算アルゴリズムは別物
  \item 刻み幅・近似度合いを変えて何度か計算を行う
  \item グラフ化して目で見てみる
  \item 計算量(コスト)のスケーリング(次数)に気をつける
  \item (計算機は指示したことを指示したようにしかやってくれないということを認識する)
  \end{itemize}
\end{frame}

\begin{frame}[t]{講義・実習内容}
  \begin{itemize}
    % \setlength{\itemsep}{1em}
  \item UNIX操作・ネットワーク
  \item プログラミング: C言語、数値計算ライブラリの利用
  \item ツール: エディタ、コンパイラ、\LaTeX、gnuplot
  \item 数値計算の基礎
  \item 常微分方程式の解法
  \item 連立一次方程式の解法
  \item 行列の対角化
  \item 線形回帰
  \end{itemize}
\end{frame}

\begin{frame}[t,fragile]{講義と実習}
  \begin{itemize}
    %\setlength{\itemsep}{1em}
  \item スタッフ \href{mailto:computer@exa.phys.s.u-tokyo.ac.jp}{computer@exa.phys.s.u-tokyo.ac.jp}
    \begin{itemize}
    \item 講義: 藤堂
    \item 実習: 樫山助教、高橋助教
    \item 実習TA: 沓澤、室田
    \end{itemize}
  \item 講義・実習の進め方
    \begin{itemize}
    \item 毎回: 講義(座学)と実習の組み合わせ
    \end{itemize}
  \item 評価
    \begin{itemize}
    \item 出席(ITC-LMSでのアンケートに回答)

      講義当日11:00-18:00の間に回答
      
    \item レポート(計3回)
    \end{itemize}    
  \end{itemize}    
\end{frame}

\begin{frame}[t]{講義資料}
  \begin{itemize}
    % \setlength{\itemsep}{1em}
  \item 講義資料置き場: \href{https://github.com/todo-group/ComputerExperiments/releases/tag/2021s-computer1}{https://github.com/todo-group/ComputerExperiments/releases/tag/2021s-computer1}

    短縮URL: \href{https://bit.ly/39MtgVO}{https://bit.ly/39MtgVO}
    
  \item 計算機実験ハンドブック
    \begin{itemize}
    \item UNIX入門
    \item gnuplot入門
    \item C言語入門
    \item \LaTeX 入門
    \end{itemize}
  \item \href{https://utphys-comp.github.io}{計算機実験のための環境整備}({\small \href{https://utphys-comp.github.io}{https://utphys-comp.github.io}})
  \item 講義資料
  \item 実習課題・サンプルコード

    \begin{itemize}
    \item この中から課題を選択してレポートを作成・提出
    \item 課題は順次追加・修正の可能性あり
    \end{itemize}
  \item 昨年度の講義動画
    \begin{itemize}
      \item \href{http://www.mi.u-tokyo.ac.jp/teaching_material.html}{http://www.mi.u-tokyo.ac.jp/teaching\_material.html}
    \end{itemize}
  \end{itemize}
\end{frame}

\begin{frame}[t]{質問がある場合には…}
  \begin{itemize}
    %\setlength{\itemsep}{1em}
  \item 講義時間外・講義時間中
    \begin{enumerate}
    \item 計算機実験Slack
    \item ITC-LMS「担当教員へのメッセージ」
    \item メール(\href{mailto:computer@exa.phys.s.u-tokyo.ac.jp}{computer@exa.phys.s.u-tokyo.ac.jp})
    \end{enumerate}
  \item 講義時間中
    \begin{enumerate}
    \item slido (匿名で質問可)
    \item オンライン授業(Zoom)のチャット
    \end{enumerate}
  \item 質問するときに注意すべきこと
    \begin{itemize}
    \item (メールの場合) Subjectをきちんとつける、きちんと名乗る
    \item 実行環境を明示する
    \item 問題を再現する手順を明記する
    \item 関連するファイル(Cや \LaTeX のソースコード等)を添付する
    \item エラーメッセージを添付する
    \end{itemize}
  \end{itemize}
\end{frame}

\begin{frame}[t,fragile]{レポート(予定)}
  \begin{itemize}
    %\setlength{\itemsep}{1em}
  \item レポート
    \begin{itemize}
    \item 各自が \LaTeX で作成の上提出 (計3回)
    \item 提出方法: ITC-LMSでPDFを提出
    \item レポート課題(基本課題)以外に独自に取り組んだ場合には加点
    \end{itemize}
  \item レポートNo.1
    \begin{itemize}
    \item \mbox{} [数値誤差・アルゴリズム基礎] 2題、[常微分方程式] うちから1題の合計3題を選択
    \item 締切: 5月19日
    \end{itemize}
  \item レポートNo.2
    \begin{itemize}
    \item \mbox{} [連立一次方程式]から2題を選択
    \item 締切: 6月9日
    \end{itemize}
  \item レポートNo.3
    \begin{itemize}
    \item \mbox{} [対角化]から2題を選択
    \item 締切: 7月7日
    \end{itemize}
  \end{itemize}    
\end{frame}
  

\section{環境整備}

\begin{frame}[t,fragile]{計算機実験に必要な環境整備}
  \begin{itemize}
    % \setlength{\itemsep}{1em}
  \item 「\href{https://github.com/utphys-comp/handbook/releases/download/handbook-2021/handbook.pdf}{計算機実験ハンドブック}」に書いてあることが、自宅のPCでも一通り試せるような環境を準備する
    \begin{itemize}
      % \setlength{\itemsep}{1em}
    \item プログラミング (オフライン・リモート利用)

      エディタ、コンパイラ(C, C++, Fortran, BLAS/LAPACK, MPI/OpenMP)
    \item 計算結果のプロット (オフライン利用)

      gnuplot (python/matplotlib, MATLAB でも可)
    \item 文書(レポート、論文)の作成 (オフライン・クラウド利用)

      \LaTeX (TeX Live あるいは Overleaf)
    \item ネットワークの利用 (ECCSなどの大学の計算機にリモートアクセスできる環境)

      ターミナル、SSH
    \item インタプリタ環境 (オフライン・クラウド利用)

      MATLAB、Python 2/3
    \end{itemize}
  \item 「\href{https://utphys-comp.github.io}{計算機実験のための環境整備}」({\small \href{https://utphys-comp.github.io}{https://utphys-comp.github.io}})を参考に、各自必要な環境を整備する
  \end{itemize}
\end{frame}


\section{数値誤差}

\begin{frame}[t,fragile]{数値誤差の原因}
  \begin{itemize}
    \setlength{\itemsep}{1em}
  \item 打ち切り誤差: テイラー展開による近似を有限項で打ち切ることによる誤差
    (例: 数値微分)
  \item 丸め誤差: 無理数や10進数を有限のビットの2進数で表現することによる誤差
    (例: 0.1 が 0.100000000000000006 になる)
  \item 桁落ち: 非常に近い数の引き算により生じる
  \item 情報落ち: 非常に大きな数に小さな数を足し込む場合に生じる
    (例: 数値積分や常微分方程式の初期値問題で刻み幅を小さくしすぎると生じる)
  \item オーバーフロー(桁あふれ): 表現できる最大値を超えてしまう
  \end{itemize}
\end{frame}

\begin{frame}[t,fragile]{桁落ち}
  \begin{itemize}
    \setlength{\itemsep}{1em}
  \item 2次方程式 $ax^2+bx+c=0$の解の公式
    \[
    x_{\pm} = \frac{-b \pm \sqrt{b^2-4ac}}{2a}
    \]
    $b^2 \gg |ac|$の時、桁落ちが生じる
  \item 例) $2.718282x^2 - 684.4566x+0.3161592=0$ の解を7桁の精度で計算してみる(伊理・藤野1985)
    \begin{align*}
      \sqrt{D} &= \sqrt{(684.4566)^2 - 4 \times 2.718282 \times 0.3161592} = 684.4541 \\
      x_+ &= \frac{684.4566+684.4541}{2 \times 2.718282} = \frac{1368.911}{5.436564} = 251.7970 \\
      x_- &= \frac{684.4566-684.4541}{2 \times 2.718282} = \frac{0.0025}{5.436564} = 0.00045\underline{98493}
    \end{align*}
  \end{itemize}
\end{frame}

\begin{frame}[t,fragile]{MATLABでの計算}
  \begin{itemize}
    \setlength{\itemsep}{1em}
  \item MATLABはシンボリック変数を含む式で小数を使うと、中で分数に直して(厳密に)計算する
    \begin{lstlisting}
>> syms x
>> solve(2.718282*x^2-684.4566*x+0.3161592,x)
ans =
 256876540725939712/2040342300381321 - 65985072983579570978585834230192105^(1/2)/2040342300381321
 65985072983579570978585834230192105^(1/2)/2040342300381321 + 256876540725939712/2040342300381321
    \end{lstlisting}
    \item 10桁の精度で解を評価してみる
    \begin{lstlisting}
>> vpa(solve(2.718282*x^2-684.4566*x+0.3161592,x),10)
ans =
 0.000461913553
    251.7970337
    \end{lstlisting}
  \end{itemize}
\end{frame}

\begin{frame}[t,fragile]{桁落ちを防ぐ方法}
  \begin{itemize}
    \setlength{\itemsep}{1em}
  \item $b$の符号に応じて、一方を求める(この例では$x_+$)
  \item 他方は解と係数の関係を使って求める
    \[
    x_- = \frac{c/a}{x_+} = \frac{0.3161592 / 2.718282}{251.7970} = 0.000461913\underline{8}
    \]
  \item 回避できない例: 重解に近い場合 $2.718282x^2 - 1.854089x + 0.3161592=0$
    \begin{align*}
      \sqrt{D} &= \sqrt{(1.854089)^2 - 4 \times 2.718282 \times 0.3161592} \\ &= 0.002\underline{64575} \\
      x_\pm &= 1.854089 \pm 0.002\underline{64575} = 1.856\underline{737}, 1.851\underline{445}
    \end{align*}
  \end{itemize}
\end{frame}

\begin{frame}[t,fragile]{数値微分(差分)}
  \begin{itemize}
    \setlength{\itemsep}{1em}
  \item 関数のテイラー展開
    \[
    f(x+h) = f(x) + h f'(x) + h^2 f''(x)/2 + h^3 f'''(x)/6 + \cdots
    \]
  \item 数値微分の最低次近似(前進差分)
    \[
    f_1(x,h) \equiv \frac{f(x+h)-f(x)}{h} = f'(x) + h f''(x)/2 + O(h^2)
    \]
  \item より高次の近似(中心差分)
    \[
    f_2(x,h) \equiv \frac{f(x+h)-f(x-h)}{2h} = f'(x) + h^2 f'''(x)/6 + O(h^3)
    \]
  \item 刻み$h$を小さくすると打ち切り誤差は減少するが、小さすぎると今度は桁落ちが大きくなる
  \end{itemize}
\end{frame}

\begin{frame}[t,fragile]{差分の一般式}
  \begin{itemize}
    %\setlength{\itemsep}{1em}
  \item 関数のテイラー展開: $\displaystyle f(x+h) = \sum_{k} \frac{h^k}{k!} f^{(k)}(x)$
  \item $f^{(m)}(x)$を$n$個の$f(x+h_j)$の線形結合で表す($n \ge m+1$)
    \begin{align*}
      f^{(m)}(x) &\approx \sum_j a_j f(x+h_j) = \sum_j a_j \sum_{k} \frac{h_j^k}{k!} f^{(k)}(x) \\
      & = \sum_{k} C_k f^{(k)}(x) \qquad (C_k \equiv \sum_j a_j \frac{h_j^k}{k!})
    \end{align*}
  \item $C_k = \delta_{k,m}$ ($k = 0 \cdots n-1$)となるように$a_0 \cdots a_{n-1}$を決める
  \item 行列$\displaystyle G_{kj} = \frac{h_j^k}{k!}$と列ベクトル$a_j$と$b_k = \delta_{k,m}$を導入すると、条件式は$G a = b$と書ける (連立一次方程式)
  \end{itemize}
\end{frame}

\begin{frame}[t,fragile]{刻み幅を変えた計算}
  \begin{itemize}
    \setlength{\itemsep}{1em}
  \item 刻み幅を変えて何度か計算を行い、収束の様子をみる
  \item グラフ化して目で見てみる
  \item 理論式と比較
    \begin{itemize}
    \item 計算式の正しさの確認
    \item 近似の改良 (収束の加速・補外)
    \end{itemize}
  \item 桁落ち・情報落ちの影響の有無
  \end{itemize}
\end{frame}

\begin{frame}[t,fragile]{数値微分(差分)}
  \begin{itemize}
    \setlength{\itemsep}{1em}
  \item 関数のテイラー展開
    \[
    f(x+h) = f(x) + h f'(x) + h^2 f''(x)/2 + h^3 f'''(x)/6 + \cdots
    \]
  \item 数値微分の最低次近似(前進差分)
    \[
    f_1(x,h) \equiv \frac{f(x+h)-f(x)}{h} = f'(x) + h f''(x)/2 + O(h^2)
    \]
  \item より高次の近似(中心差分)
    \[
    f_2(x,h) \equiv \frac{f(x+h)-f(x-h)}{2h} = f'(x) + h^2 f'''(x)/6 + O(h^3)
    \]
  \item 刻み$h$を小さくすると打ち切り誤差は減少するが、小さすぎると今度は桁落ちが大きくなる
  \end{itemize}
\end{frame}

\begin{frame}[t,fragile]{差分の一般式}
  \begin{itemize}
    %\setlength{\itemsep}{1em}
  \item 関数のテイラー展開: $\displaystyle f(x+h) = \sum_{k} \frac{h^k}{k!} f^{(k)}(x)$
  \item $f^{(m)}(x)$を$n$個の$f(x+h_j)$の線形結合で表す($n \ge m+1$)
    \begin{align*}
      f^{(m)}(x) &\approx \sum_j a_j f(x+h_j) = \sum_j a_j \sum_{k} \frac{h_j^k}{k!} f^{(k)}(x) \\
      & = \sum_{k} C_k f^{(k)}(x) \qquad (C_k \equiv \sum_j a_j \frac{h_j^k}{k!})
    \end{align*}
  \item $C_k = \delta_{k,m}$ ($k = 0 \cdots n-1$)となるように$a_0 \cdots a_{n-1}$を決める
  \item 行列$\displaystyle G_{kj} = \frac{h_j^k}{k!}$と列ベクトル$a_j$と$b_k = \delta_{k,m}$を導入すると、条件式は$G a = b$と書ける (連立一次方程式)
  \end{itemize}
\end{frame}

\begin{frame}[t,fragile]{複素差分}
  \begin{itemize}
    %\setlength{\itemsep}{1em}
  \item 虚軸方向のテイラー展開を考える
    \[
    f(x+ih) = f(x) + ihf'(x) - h^2 f''(x)/2  - ih^3 f'''(x)/6 + \cdots
    \]
  \item テイラー展開の虚部を取ることで微分が求まる
    \[
    f'(x) = \frac{\mathrm{Im} (f(x+ih))}{h} + O(h^2)
    \]
  \item 引き算がないので桁落ちに強い
  \item 関数の値が複素数に対して正しく計算される必要がある
  \end{itemize}
\end{frame}


\section{ニュートン法}
\begin{frame}[t,fragile]{ニュートン法}
  \begin{itemize}
    \setlength{\itemsep}{1em}
  \item 反復法により方程式$f(x)=0$の解を求める
  \item 真の解を$x_0$、現在の解の候補を$x_n=x_0+\epsilon$とすると
    \[
    0 = f(x_0) = f(x_0+\epsilon-\epsilon) = f(x_n) - f'(x_n) \epsilon + O(\epsilon^2)
    \]
  \item 次の解の候補 (反復法、逐次近似法)
    \[
    \epsilon \approx \frac{f(x_n)}{f'(x_n)} \quad\quad x_{n+1} = x_n - \frac{f(x_n)}{f'(x_n)}
    \]
  \item 複素変数の複素関数や多変数の場合にも自然に拡張可
  \end{itemize}
\end{frame}

\input{1_basics/newton-02.tex}
\begin{frame}[t,fragile]{多次元の場合}
  \begin{itemize}
    \setlength{\itemsep}{1em}
  \item $f(x)=0$: $d$次元(非線形)連立方程式
  \item $x$は$d$次元のベクトル: $x = {}^t(x_1,x_2,\cdots,x_d)$
  \item $f(x)$も$d$次元のベクトル: $f(x) = {}^t(f_1(x), f_2(x),\cdots,f_d(x))$
  \item 真の解のまわりでの展開 ($x_n = x_0 + \epsilon$)
    \[
    0 = f(x_0) = f(x_0+\epsilon-\epsilon) = f(x_n) - \frac{\partial f(x_n)}{\partial x} \cdot \epsilon + O(|\epsilon|^2)
    \]
  \item ヤコビ行列($d\times d$): $\displaystyle \Big[\frac{\partial f(x)}{\partial x}\Big]_{ij} = \frac{\partial f_i(x)}{\partial x_j}$
  \item 次の解の候補: $\displaystyle x_{n+1} = x_n - \Big[\frac{\partial f(x_n)}{\partial x}\Big]^{-1} f(x_n)$
  \end{itemize}
\end{frame}

\begin{frame}[t,fragile]{ニュートン法による最適化}
  \begin{itemize}
    \setlength{\itemsep}{1em}
  \item $x$は$d$次元のベクトル: $x = {}^t(x_1,x_2,\cdots,x_d)$、目的関数$f(x)$はスカラー
  \item 勾配ベクトル: $\displaystyle [\nabla f(x)]_i = \frac{\partial f(x)}{\partial x_i}$
  \item 極小値(最小値)となる条件: $\nabla f(x)=0$
  \item ニュートン法で$f(x)$を$\nabla f(x)$で置き換えればよい
  \item 次の解の候補: $\displaystyle x_{n+1} = x_n - H^{-1}(x_n) \nabla f(x_n)$
  \item ヘッセ行列(Hessian): $\displaystyle H_{ij}(x) = \frac{\partial^2 f}{\partial x_i \partial x_j}(x)$
  \end{itemize}
\end{frame}

\begin{frame}[t,fragile]{準ニュートン法}
  \begin{itemize}
    %\setlength{\itemsep}{1em}
  % \item ニュートン法では、ヘッセ行列の計算・保存が必要
  \item 準ニュートン法: それまでの反復で計算した勾配ベクトルから、ヘッセ行列の逆行列を近似($C_n \approx H^{-1}$)
  \begin{itemize}
    \item 極小点の近傍では
      \begin{align*}
          x-x_n &= -H^{-1} \cdot \nabla f(x_n) \\
          x-x_{n+1} &= -H^{-1} \cdot \nabla f(x_{n+1})
      \end{align*}
    \item 差をとると($s_n = x_{n+1} - x_n$, $y_n = \nabla f(x_{n+1}) - \nabla f(x_n)$)
      \begin{align*}
        s_n &= H^{-1} \cdot y_n
      \end{align*}
    \item この式が満たされるように、$C_{n+1}$を$C_{n}+(\text{補正})$の形で構成(DFP法: Davidon-Fletcher-Powell)
      \[
        C_{n+1} = C_{n} + \frac{s_n s_n^T}{s_n^T y_n} - \frac{C_{n} y_n (C_{n} y_n)^T}{y_n^T C_n y_n}
      \]
    \item 他にも、BFGS法など
    \end{itemize}
  \end{itemize}
\end{frame}

\begin{frame}[t,fragile]{最急降下法(steepest descent)}
  \begin{itemize}
    %\setlength{\itemsep}{1em}
  \item 関数の微分の情報を使う
  \item 現在の点$x$における勾配を計算
    \[
    -\nabla f|_i = -\frac{\partial f}{\partial x_i}
    \]
  \item 坂を下る方向にそって、一次元最適化 (ニュートン法、囲い込み法)
  \item 動いた先の勾配の方向でさらに最適化を繰り返す
  \item 関数値は単調減少 $\Rightarrow$ 極小値に収束
  \end{itemize}
\end{frame}

\begin{frame}[t,fragile]{勾配降下法(gradient descent)}
  \begin{itemize}
    %\setlength{\itemsep}{1em}
  \item 勾配方向に一次元最適化を行うかわりに、あらかじめ決めた一定量($\epsilon$)だけ坂を下る
    \[
    x_{n+1} = x_n - \epsilon \, \nabla f
    \]
  \item あらかじめ最適な$\epsilon$を知るのは困難
  \item 機械学習の分野では、(なぜか) $\epsilon=0.1$が良いとされている
  \item この方法を「最急降下法」、一次元最適化を行う勾配法を「最適降下法(optimum descent)」と呼ぶ場合も
    % \item c.f.) 確率的勾配降下法(stochastic gradient descent)
  \item $\epsilon$を自動的に調整する手法も提案されている: ADAM, Adagrad, etc
  \end{itemize}
\end{frame}

\begin{frame}[t,fragile]{計算をいつやめるか?}
  \begin{itemize}
    %\setlength{\itemsep}{1em}
  \item 残差による判定
    \[
    |f(x)| < \delta
    \]
  \item 誤差による判定
    \[
    | x_{n+1} - x_{n} | < \epsilon
    \]
  \item 解$x=x_0$が$m$重解の場合、$x=x_0$のまわりで展開すると
    \[
    f(x) \simeq \alpha (x-x_0)^m
    \]
    残差が$\delta$程度になったときの誤差は、$\delta^{1/m}$程度

    逆に$|x-x_0|$が$\delta^{1/m}$以下になると、$f(x)$の値がそれ以上変化しない $\Rightarrow$ $m$重解の精度は計算精度の$1/m$桁程度しかない

  \item 残差による判定と誤差による判定を併用するのがよい
  \end{itemize}
\end{frame}

\begin{frame}[t,fragile]{反復計算}
  \begin{itemize}
    %\setlength{\itemsep}{1em}
  \item {\tt while}による反復 (ハンドブック2.2.3節)
\begin{lstlisting}
double residual = 1;    /* 残差: 適当に大きな値 */
double error = 1;       /* 誤差: 適当に大きな値 */
double delta = 1.0e-12; /* 欲しい精度 */
while (residual > delta && error > delta) {
  /* ニュートン法の漸化式 */
  /* residual と error を計算 */
}
\end{lstlisting}
残差と誤差のどちらかが欲しい精度に達したら計算を終了
\item {\tt break}を使う例 (ハンドブック2.2.4節)
\begin{lstlisting}
for (;;) {
  /* ニュートン法の漸化式 */
  /* residual と error を計算 */
  if (residual < delta || error < delta) break;
}
\end{lstlisting}
  \end{itemize}
\end{frame}

\begin{frame}[t,fragile]{初期段階における収束の改善}
  \begin{itemize}
    %\setlength{\itemsep}{1em}
  \item Newton法は初期値によっては収束しない
  \item 発散や振動を抑える方法として「減速」が有効な場合も
  \item 減速
    \begin{itemize}
    \item 反復式を少し修正する
      \[
      x_{n+1} = x_n - \mu_n \frac{f(x_n)}{f'(x_n)}
      \]
    \item まずは$\mu_n=1$として計算
    \item $|f(x_{n+1})| < |f(x_{n})|$が成り立たないようであれば、$\mu_n$を半分にして再計算
    \item $\mu_n$が十分に小さくなれば、$|f(x)|$は必ず減少する
    \end{itemize}
  \end{itemize}
\end{frame}


% -*- coding: utf-8 -*-

\section{二分法}

\input{1_basics/bisection-01.tex}

% -*- coding: utf-8 -*-

\section{囲い込み法}

\begin{frame}[t,fragile]{囲い込み法(一次元の最適化)}
  \begin{itemize}
    %\setlength{\itemsep}{1em}
  \item 関数$f(x)$の極小点を求める
  \item $f(a) > f(b) < f(c)$を満たす3点の組$a < b < c$の領域を狭めていく
  \item $[a,b]$、$[b,c]$の広い方(例えば後者)を$b$から見て、黄金比
    [$1:(1+\sqrt{5})/2 \approx 0.382:0.618$]に内分する点を$x$とする
    \begin{itemize}
    \item $f(b) > f(x)$の場合: $[b,c]$を新しい領域にとる
    \item $f(b) < f(x)$の場合: $[a,x]$を新しい領域にとる
    \end{itemize}
  \item もともとの$b$が$[a,c]$を$0.382:0.618$に内分する点だった場合、
    新しい領域の幅は、どちらの場合も0.618
  \item 最初の比率が黄金比からずれていたとしても、黄金比に収束
  \item 黄金分割法(golden section)とも呼ばれる
  \end{itemize}
\end{frame}

\begin{frame}[t,fragile]{最初の囲い込み}
  \begin{itemize}
    % \setlength{\itemsep}{1em}
  \item 1点を選び、適当な$\Delta x$を取る
  \item 左右に$\Delta x$動かしてみて、関数値が小さくなる方へ動く
  \item どちらに進んでも関数値が大きくなる場合には、囲い込み完了
  \item 小さくなった場合、その方向へ再び増えるまで$\Delta x$を倍々に増やしながら進む
  \item 最後の3点で極小値を囲い込むことができる
  \item 囲い込み法のプログラムの例: \href{https://github.com/todo-group/computer-experiments/blob/master/exercise/optimization/golden_section.c}{golden\_section.c}
  \end{itemize}
\end{frame}



\section{}
\begin{frame}[t]{おわりに}
  \begin{itemize}
    % \setlength{\itemsep}{1em}
  \item 出席
  \begin{itemize}
    \item UTOLでのアンケートに回答してください(期限は本日15:00)
  \end{itemize}
  \item 次回
    \begin{itemize}
    \item 4月16日 実習1 (グループ1): 環境整備・C言語プログラミング
    \item 4月23日 実習1 (グループ2): 環境整備・C言語プログラミング
    \item グループ1: 学生証番号が奇数、グループ2: 偶数
    \item ノートPCを持参してください
    \end{itemize}
  \end{itemize}
\end{frame}

\end{document}
