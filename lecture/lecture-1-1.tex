% -*- coding: utf-8 -*-

\documentclass[10pt,dvipdfmx]{beamer}
\usepackage{tutorial}

\title{計算機実験I (第1回)}
\date{2025/04/09}

\begin{document}

\begin{frame}
  \titlepage
  \tableofcontents
\end{frame}

\section{講義・実習の概要}

\begin{frame}[t]{講義・実習の目的}
  \begin{itemize}
    %\setlength{\itemsep}{1em}
  \item 理論・実験を問わず、学部〜大学院〜で必要となる現代的かつ普遍的な計算機の素養を身につける
  \item UNIX環境に慣れる(シェル、ファイル操作、エディタ)
  \item ネットワークの活用 (リモートログイン、共同作業)
  \item プログラムの作成(C言語、コンパイラ、プログラム実行)
  \item 基本的な数値計算アルゴリズム・数値計算の常識を学ぶ
  \item 科学技術文書作成に慣れる(\LaTeX, グラフ作成)
  \end{itemize}
\end{frame}

\begin{frame}[t]{身に付けて欲しいこと}
  \begin{itemize}
    %\setlength{\itemsep}{1em}
  \item ツールとしてないものは自分で作る (物理の伝統)
  \item すでにあるものは積極的に再利用する (車輪の再発明をしない)
  \item 数学公式と数値計算アルゴリズムは別物
  \item 刻み幅・近似度合いを変えて何度か計算を行う
  \item グラフ化して目で見てみる
  \item 計算量(コスト)のスケーリング(次数)に気をつける
  \item (計算機は指示したことを指示したようにしかやってくれないということを理解する)
  \end{itemize}
\end{frame}

\begin{frame}[t]{講義・実習内容}
  \begin{itemize}
    % \setlength{\itemsep}{1em}
  \item UNIX操作、ネットワークの活用
  \item プログラミング: C言語、数値計算ライブラリの利用
  \item ツール: エディタ、コンパイラ、\LaTeX、gnuplot
  \item 数値計算の基礎
  \item 常微分方程式の解法
  \item 連立一次方程式の解法
  \item 行列の対角化
  \item 線形回帰
  \end{itemize}
\end{frame}

\begin{frame}[t,fragile]{スタッフ・進め方・評価}
  \begin{itemize}
    %\setlength{\itemsep}{1em}
  \item スタッフ \href{mailto:computer@.phys.s.u-tokyo.ac.jp}{computer@phys.s.u-tokyo.ac.jp}
    \begin{itemize}
    \item 講義: 藤堂
    \item 実習: 石河特任助教(常行研)、諏訪助教(藤堂研)
    \item 実習TA: 久保(常行研)、齊藤(藤堂研)
    \end{itemize}
  \item 講義・実習の進め方
    \begin{itemize}
    \item 講義(座学)と実習を交互に実施
    \item 実習の回には各自PCを持参すること
    \end{itemize}
  \item 評価
    \begin{itemize}
    \item 出席(UTOLでのアンケートに回答)

      講義・実習当日10:30-15:00の間に回答
      
    \item レポート(計3回)
    \end{itemize}    
  \end{itemize}    
\end{frame}

\begin{frame}[t]{講義日程(予定)}
  \begin{itemize}
    % \setlength{\itemsep}{1em}
  \item 全8回 (水曜2限 10:25-12:10)
    \begin{itemize}
    \item 4月9日 講義1: 講義の概要・基本的なアルゴリズム
    \item 4月16日 実習1 (グループ1): 環境整備・C言語プログラミング
    \item 4月23日 実習1 (グループ2): 環境整備・C言語プログラミング
    \item 5月7日 講義2: 常微分方程式
    \item 5月14日 実習2 (グループ1): 基本的なアルゴリズム・常微分方程式
    \item 5月21日 実習2 (グループ2): 基本的なアルゴリズム・常微分方程式
    \item 5月28日 講義3: 連立方程式
    \item 6月4日 実習3 (グループ1): 連立方程式
    \item 6月11日 実習3 (グループ2): 連立方程式
    \item 6月18日 講義4: 行列の対角化
    \item 6月25日 実習4 (グループ1): 行列の対角化
    \item 7月2日 実習4 (グループ2): 行列の対角化
    \item {\color{gray} 7月9日 休講}
    \item {\color{gray} 7月16日 休講}
    \end{itemize}
  \item 実習はクラスを2グループに分けて実施(グループ1: 学生証番号が奇数、グループ2: 偶数)
  \end{itemize}
\end{frame}

\begin{frame}[t,fragile]{レポート}
  \begin{itemize}
    %\setlength{\itemsep}{1em}
  \item レポート
    \begin{itemize}
    \item 各自が \LaTeX で作成の上提出 (計3回)
    \item 提出方法: UTOLでPDFファイルを提出
    \end{itemize}
  \item レポートNo.1
    \begin{itemize}
    \item \mbox{} [数値誤差・アルゴリズム基礎] 2題、[常微分方程式] うちから1題の合計3題を選択
    \item 締切: 6月4日(予定)
    \end{itemize}
  \item レポートNo.2
    \begin{itemize}
    \item \mbox{} [連立一次方程式]から2題を選択
    \item 締切: 6月25日(予定)
    \end{itemize}
  \item レポートNo.3
    \begin{itemize}
    \item \mbox{} [対角化]から2題を選択
    \item 締切: 7月16日(予定)
    \end{itemize}
  \item 詳細はUTOLに掲示予定
  \end{itemize}    
\end{frame}

\begin{frame}[t]{講義資料}
  \begin{itemize}
    % \setlength{\itemsep}{1em}
  \item 講義資料置き場:
  
    {\footnotesize \href{https://github.com/todo-group/ComputerExperiments/tree/2025s-computer1}{https://github.com/todo-group/ComputerExperiments/tree/2025s-computer1}}
    \begin{itemize}
    \item 計算機実験ハンドブック
      \begin{itemize}
      \item UNIX入門
      \item gnuplot入門
      \item C言語入門
      \item \LaTeX 入門
      \end{itemize}
    \item \href{https://utphys-comp.github.io}{計算機実験のための環境整備}({\small \href{https://utphys-comp.github.io}{https://utphys-comp.github.io}})
    \item \href{https://github.com/cmsi/malive-tutorial/blob/master/introduction/shell.md}{UNIX入門}({\small \href{https://github.com/cmsi/malive-tutorial/blob/master/introduction/shell.md}{https://github.com/cmsi/malive-tutorial/blob/master/introduction/shell.md}})
    \item 講義資料
    \item 実習課題
    \item サンプルコード
    \end{itemize}
  \end{itemize}
\end{frame}

\begin{frame}[t]{質問がある場合には…}
  \begin{itemize}
    %\setlength{\itemsep}{1em}
  \item 講義・実習時間中は自由に質問してください
    \begin{itemize}
    \item 計算機実験Slack
    \end{itemize}
  \item 講義時間外も質問を受け付けます
    \begin{itemize}
    \item 計算機実験Slack
      \begin{enumerate}
        \item 情報セキュリティ教育の受講を完了していることを確認
        \item UTokyo Accountの多要素認証を有効化していることを確認
        \item \href{https://utokyo.enterprise.slack.com/}{https://utokyo.enterprise.slack.com} にアクセス・ログイン
        \item ワークスペース「理学部物理学科: 計算機実験2025」を検索し、「詳細を表示」→「参加をリクエストする」
      \end{enumerate}
    \item UTOL「担当教員へのメッセージ」
    \item メール(\href{mailto:computer@phys.s.u-tokyo.ac.jp}{computer@phys.s.u-tokyo.ac.jp})
    \end{itemize}
  \item メール・Slackなどで質問するときに注意すべきこと
    \begin{itemize}
    \item (メールの場合) Subjectをきちんとつける、きちんと名乗る
    \item 実行環境を明示する
    \item 問題を再現する手順を明記する
    \item 関連するファイル(Cや \LaTeX のソースコード等)を添付する
    \item エラーメッセージを添付する
    \end{itemize}
  \end{itemize}
\end{frame}

\section{環境整備}

\begin{frame}[t,fragile]{計算機実験に必要な環境整備}
  \begin{itemize}
    % \setlength{\itemsep}{1em}
  \item 「\href{https://github.com/utphys-comp/handbook/releases/download/handbook-2024/handbook.pdf}{計算機実験ハンドブック}」に書いてあることが、自分のPCで一通り試せるような環境を準備する
    \begin{itemize}
      % \setlength{\itemsep}{1em}
    \item プログラミング (オフライン・リモート利用)

      エディタ、コンパイラ(C, C++, Fortran, BLAS/LAPACK, MPI/OpenMP)
    \item 計算結果のプロット (オフライン利用)

      gnuplot (python/matplotlib, MATLAB でも可)
    \item 文書(レポート、論文)の作成 (オフライン・クラウド利用)

      \LaTeX (TeX Live あるいは Overleaf)
    \item ネットワークの利用 (ECCSなどの大学の計算機にリモートアクセスできる環境)

      ターミナル、SSH
    \item インタプリタ環境 (オフライン・クラウド利用)

      MATLAB、Python
    \end{itemize}
  \item 「\href{https://utphys-comp.github.io}{計算機実験のための環境整備}」({\small \href{https://utphys-comp.github.io}{https://utphys-comp.github.io}})を参考に、環境を整備する $\Rightarrow$ 実習1
  \end{itemize}
\end{frame}

\input{1_basics/error.tex}
\input{1_basics/diff-01.tex}
\input{1_basics/diff-02.tex}
\input{1_basics/diff-03.tex}

\section{ニュートン法}
\input{1_basics/newton-01.tex}
\input{1_basics/newton-02.tex}
\input{1_basics/newton-03.tex}
\input{1_basics/newton-04.tex}
\input{1_basics/newton-05.tex}
\input{7_optimization/descent-01.tex}
\input{7_optimization/descent-02.tex}
\input{1_basics/newton-06.tex}
\input{1_basics/newton-07.tex}
\input{1_basics/newton-08.tex}

\input{1_basics/bisection.tex}
\input{7_optimization/bracketing.tex}

\section{}
\begin{frame}[t]{おわりに}
  \begin{itemize}
    % \setlength{\itemsep}{1em}
  \item 出席
  \begin{itemize}
    \item UTOLでのアンケートに回答してください(期限は本日15:00)
  \end{itemize}
  \item 次回
    \begin{itemize}
    \item 4月16日 実習1 (グループ1): 環境整備・C言語プログラミング
    \item 4月23日 実習1 (グループ2): 環境整備・C言語プログラミング
    \item グループ1: 学生証番号が奇数、グループ2: 偶数
    \item ノートPCを持参してください
    \end{itemize}
  \end{itemize}
\end{frame}

\end{document}
