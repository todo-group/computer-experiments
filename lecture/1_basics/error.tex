\section{数値誤差}

\begin{frame}[t,fragile]{数値誤差の原因}
  \begin{itemize}
    \setlength{\itemsep}{1em}
  \item 打ち切り誤差: テイラー展開による近似を有限項で打ち切ることによる誤差
    (例: 数値微分)
  \item 丸め誤差: 無理数や10進数を有限のビットの2進数で表現することによる誤差
    (例: 0.1 が 0.100000000000000006 になる)
  \item 桁落ち: 非常に近い数の引き算により生じる
  \item 情報落ち: 非常に大きな数に小さな数を足し込む場合に生じる
    (例: 数値積分や常微分方程式の初期値問題で刻み幅を小さくしすぎると生じる)
  \item オーバーフロー(桁あふれ): 表現できる最大値を超えてしまう
  \end{itemize}
\end{frame}

\begin{frame}[t,fragile]{桁落ち}
  \begin{itemize}
    \setlength{\itemsep}{1em}
  \item 2次方程式 $ax^2+bx+c=0$の解の公式
    \[
    x_{\pm} = \frac{-b \pm \sqrt{b^2-4ac}}{2a}
    \]
    $b^2 \gg |ac|$の時、桁落ちが生じる
  \item 例) $2.718282x^2 - 684.4566x+0.3161592=0$ の解を7桁の精度で計算してみる(伊理・藤野1985)
    \begin{align*}
      \sqrt{D} &= \sqrt{(684.4566)^2 - 4 \times 2.718282 \times 0.3161592} = 684.4541 \\
      x_+ &= \frac{684.4566+684.4541}{2 \times 2.718282} = \frac{1368.911}{5.436564} = 251.7970 \\
      x_- &= \frac{684.4566-684.4541}{2 \times 2.718282} = \frac{0.0025}{5.436564} = 0.00045\underline{98493}
    \end{align*}
  \end{itemize}
\end{frame}

\begin{frame}[t,fragile]{MATLABでの計算}
  \begin{itemize}
    \setlength{\itemsep}{1em}
  \item MATLABはシンボリック変数を含む式で小数を使うと、中で分数に直して(厳密に)計算する
    \begin{lstlisting}
>> syms x
>> solve(2.718282*x^2-684.4566*x+0.3161592,x)
ans =
 256876540725939712/2040342300381321 - 65985072983579570978585834230192105^(1/2)/2040342300381321
 65985072983579570978585834230192105^(1/2)/2040342300381321 + 256876540725939712/2040342300381321
    \end{lstlisting}
    \item 10桁の精度で解を評価してみる
    \begin{lstlisting}
>> vpa(solve(2.718282*x^2-684.4566*x+0.3161592,x),10)
ans =
 0.000461913553
    251.7970337
    \end{lstlisting}
  \end{itemize}
\end{frame}

\begin{frame}[t,fragile]{桁落ちを防ぐ方法}
  \begin{itemize}
    \setlength{\itemsep}{1em}
  \item $b$の符号に応じて、一方を求める(この例では$x_+$)
  \item 他方は解と係数の関係を使って求める
    \[
    x_- = \frac{c/a}{x_+} = \frac{0.3161592 / 2.718282}{251.7970} = 0.000461913\underline{8}
    \]
  \item 回避できない例: 重解に近い場合 $2.718282x^2 - 1.854089x + 0.3161592=0$
    \begin{align*}
      \sqrt{D} &= \sqrt{(1.854089)^2 - 4 \times 2.718282 \times 0.3161592} \\ &= 0.002\underline{64575} \\
      x_\pm &= 1.854089 \pm 0.002\underline{64575} = 1.856\underline{737}, 1.851\underline{445}
    \end{align*}
  \end{itemize}
\end{frame}

\begin{frame}[t,fragile]{数値微分(差分)}
  \begin{itemize}
    \setlength{\itemsep}{1em}
  \item 関数のテイラー展開
    \[
    f(x+h) = f(x) + h f'(x) + h^2 f''(x)/2 + h^3 f'''(x)/6 + \cdots
    \]
  \item 数値微分の最低次近似(前進差分)
    \[
    f_1(x,h) \equiv \frac{f(x+h)-f(x)}{h} = f'(x) + h f''(x)/2 + O(h^2)
    \]
  \item より高次の近似(中心差分)
    \[
    f_2(x,h) \equiv \frac{f(x+h)-f(x-h)}{2h} = f'(x) + h^2 f'''(x)/6 + O(h^3)
    \]
  \item 刻み$h$を小さくすると打ち切り誤差は減少するが、小さすぎると今度は桁落ちが大きくなる
  \end{itemize}
\end{frame}

\begin{frame}[t,fragile]{差分の一般式}
  \begin{itemize}
    %\setlength{\itemsep}{1em}
  \item 関数のテイラー展開: $\displaystyle f(x+h) = \sum_{k} \frac{h^k}{k!} f^{(k)}(x)$
  \item $f^{(m)}(x)$を$n$個の$f(x+h_j)$の線形結合で表す($n \ge m+1$)
    \begin{align*}
      f^{(m)}(x) &\approx \sum_j a_j f(x+h_j) = \sum_j a_j \sum_{k} \frac{h_j^k}{k!} f^{(k)}(x) \\
      & = \sum_{k} C_k f^{(k)}(x) \qquad (C_k \equiv \sum_j a_j \frac{h_j^k}{k!})
    \end{align*}
  \item $C_k = \delta_{k,m}$ ($k = 0 \cdots n-1$)となるように$a_0 \cdots a_{n-1}$を決める
  \item 行列$\displaystyle G_{kj} = \frac{h_j^k}{k!}$と列ベクトル$a_j$と$b_k = \delta_{k,m}$を導入すると、条件式は$G a = b$と書ける (連立一次方程式)
  \end{itemize}
\end{frame}

\begin{frame}[t,fragile]{刻み幅を変えた計算}
  \begin{itemize}
    \setlength{\itemsep}{1em}
  \item 刻み幅を変えて何度か計算を行い、収束の様子をみる
  \item グラフ化して目で見てみる
  \item 理論式と比較
    \begin{itemize}
    \item 計算式の正しさの確認
    \item 近似の改良 (収束の加速・補外)
    \end{itemize}
  \item 桁落ち・情報落ちの影響の有無
  \end{itemize}
\end{frame}
