\begin{frame}[t,fragile]{準ニュートン法}
  \begin{itemize}
    %\setlength{\itemsep}{1em}
  % \item ニュートン法では、ヘッセ行列の計算・保存が必要
  \item 準ニュートン法: それまでの反復で計算した勾配ベクトルから、ヘッセ行列の逆行列を近似($C_n \approx H^{-1}$)
  \begin{itemize}
    \item 極小点の近傍では
      \begin{align*}
          x-x_n &= -H^{-1} \cdot \nabla f(x_n) \\
          x-x_{n+1} &= -H^{-1} \cdot \nabla f(x_{n+1})
      \end{align*}
    \item 差をとると($s_n = x_{n+1} - x_n$, $y_n = \nabla f(x_{n+1}) - \nabla f(x_n)$)
      \begin{align*}
        s_n &= H^{-1} \cdot y_n
      \end{align*}
    \item この式が満たされるように、$C_{n+1}$を$C_{n}+(\text{補正})$の形で構成(DFP法: Davidon-Fletcher-Powell)
      \[
        C_{n+1} = C_{n} + \frac{s_n s_n^T}{s_n^T y_n} - \frac{C_{n} y_n (C_{n} y_n)^T}{y_n^T C_n y_n}
      \]
    \item 他にも、BFGS法など
    \end{itemize}
  \end{itemize}
\end{frame}
