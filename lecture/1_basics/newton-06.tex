\begin{frame}[t,fragile]{計算をいつやめるか?}
  \begin{itemize}
    %\setlength{\itemsep}{1em}
  \item 残差による判定
    \[
    |f(x)| < \delta
    \]
  \item 誤差による判定
    \[
    | x_{n+1} - x_{n} | < \epsilon
    \]
  \item 解$x=x_0$が$m$重解の場合、$x=x_0$のまわりで展開すると
    \[
    f(x) \simeq \alpha (x-x_0)^m
    \]
    残差が$\delta$程度になったときの誤差は、$\delta^{1/m}$程度

    逆に$|x-x_0|$が$\delta^{1/m}$以下になると、$f(x)$の値がそれ以上変化しない $\Rightarrow$ $m$重解の精度は計算精度の$1/m$桁程度しかない

  \item 残差による判定と誤差による判定を併用するのがよい
  \end{itemize}
\end{frame}
