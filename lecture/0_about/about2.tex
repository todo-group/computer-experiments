\section{講義・実習の概要}

\begin{frame}[t]{講義・実習の目的}
  \begin{itemize}
    %\setlength{\itemsep}{1em}
  \item 理論・実験を問わず、学部〜大学院〜で必要となる現代的かつ普遍的な計算機の素養を身につける
  \item {\color{gray}UNIX環境に慣れる(シェル、ファイル操作、エディタ)}
  \item {\color{gray}ネットワークの活用 (リモートログイン、共同作業)}
  \item {\color{gray}プログラムの作成(C言語、コンパイラ、プログラム実行)}
  \item 基本的な数値計算アルゴリズム・数値計算の常識を学ぶ
  \item {\color{gray}科学技術文書作成に慣れる(\LaTeX, グラフ作成)}
  \item {\color{red}物理学における具体的な問題を通して実践的な知識と経験を身につける}
  \end{itemize}
\end{frame}

\begin{frame}[t]{身に付けて欲しいこと}
  \begin{itemize}
    %\setlength{\itemsep}{1em}
  \item ツールとしてないものは自分で作る (物理の伝統)
  \item すでにあるものは積極的に再利用する (車輪の再発明をしない)
  \item 数学公式と数値計算アルゴリズムは別物
  \item 刻み幅・近似度合いを変えて何度か計算を行う
  \item グラフ化して目で見てみる
  \item 計算量(コスト)のスケーリング(次数)に気をつける
  \item 記録に残す・再現性を確保する
  \item {\color{red}問題の解き方は一通りではない}
  \item {\color{red}いろいろな手法を組み合わせて使う}
  \end{itemize}
\end{frame}

\begin{frame}[t]{講義・実習内容}
  \begin{itemize}
    \setlength{\itemsep}{1em}
  \item 問題解決型: 計算機実験Iで身に付けた知識をもとに、より高度な数値計算手法・アルゴリズムを学び、物理学における具体的な問題への応用を通して実践的な知識と経験を身につける
    \begin{itemize}
    \item 数値対角化と量子力学
    \item モンテカルロ法・分子動力学と統計物理
    \item 最適化問題
    \end{itemize}
  \item スタッフ \href{mailto:computer@exa.phys.s.u-tokyo.ac.jp}{computer@exa.phys.s.u-tokyo.ac.jp}
    \begin{itemize}
    \item 講義: 藤堂
    \item 実習: 鈴木助教、斉藤助教
    \item 実習TA: 曹(藤堂研M2)、森下(藤堂研M1)
    \end{itemize}
  \end{itemize}    
\end{frame}

%% \begin{frame}[t]{質問がある場合には、、、}
%%   \begin{enumerate}
%%     %\setlength{\itemsep}{1em}
%%   \item ITC-LMS の掲示板を見る
%%   \item ハンドブック、講義資料を確認
%%   \item まわりの人に質問してみる
%%   \item ネットで検索
%%   \item 計算機実験担当者(\href{mailto:computer@exa.phys.s.u-tokyo.ac.jp}{computer@exa.phys.s.u-tokyo.ac.jp}) に相談
%%   \end{enumerate}
%%   メールで質問するときに注意すべきこと
%%   \begin{itemize}
%%   \item (メールの)標題をきちんとつける、きちんと名乗る
%%   \item 実行環境を明示する
%%   \item 問題を再現する手順を明記する
%%   \item 関連するファイル(Cや \LaTeX のソースコード等)を添付する
%%   \item エラーメッセージを添付する
%%   \end{itemize}
%% \end{frame}
