\section{環境整備}

\begin{frame}[t,fragile]{計算機実験に必要な環境整備}
  \begin{itemize}
    % \setlength{\itemsep}{1em}
  \item 「\href{https://github.com/utphys-comp/handbook/releases/download/handbook-2021/handbook.pdf}{計算機実験ハンドブック}」に書いてあることが、自宅のPCでも一通り試せるような環境を準備する
    \begin{itemize}
      % \setlength{\itemsep}{1em}
    \item プログラミング (オフライン・リモート利用)

      エディタ、コンパイラ(C, C++, Fortran, BLAS/LAPACK, MPI/OpenMP)
    \item 計算結果のプロット (オフライン利用)

      gnuplot (python/matplotlib, MATLAB でも可)
    \item 文書(レポート、論文)の作成 (オフライン・クラウド利用)

      \LaTeX (TeX Live あるいは Overleaf)
    \item ネットワークの利用 (ECCSなどの大学の計算機にリモートアクセスできる環境)

      ターミナル、SSH
    \item インタプリタ環境 (オフライン・クラウド利用)

      MATLAB、Python 2/3
    \end{itemize}
  \item 「\href{https://utphys-comp.github.io}{計算機実験のための環境整備}」({\small \href{https://utphys-comp.github.io}{https://utphys-comp.github.io}})を参考に、各自必要な環境を整備する
  \end{itemize}
\end{frame}

