\section{実習0}

\begin{frame}[t,fragile]{EX0-1: 実習環境の準備}
  \begin{itemize}
    \setlength{\itemsep}{1em}
  \item[0-1-1] ECCS端末(iMac)を利用する場合
    \begin{itemize}
    \item ログイン
    \item 「ターミナル」を開く
    \end{itemize}
  \item[0-1-2] MateriApps LIVE!を利用する場合
    \begin{itemize}
    \item MateriApps LIVE! のインストール・設定(\href{https://github.com/cmsi/MateriAppsLive/wiki/MateriAppsLive-ltx}{README.html}、\href{http://www.slideshare.net/cms_initiative/materiapps-live-52477264}{malive.pdf})
    \item スタートメニュー⇒「Accessories」⇒「LXTerminal」
    \end{itemize}
  \item[0-1-3] ハンドブック 2章 UNIX入門
    \begin{itemize}
    \item 2.1 UNIXのコマンド (2.1.1)
    \item 2.2 リモートログインとファイル転送
    \item 2.3 Emacsを使う (2.3.2) (viを使ってもよい)
    \item 2.4 Gnuplotを使う
    \end{itemize}
  \end{itemize}
\end{frame}

\begin{frame}[t,fragile]{EX0-2: C言語と\LaTeX の練習}
  \begin{itemize}
    %\setlength{\itemsep}{1em}
  \item[0-2-1] \href{https://github.com/todo-group/computer-experiments/blob/master/exercise/basics/hello.c}{exercise/basics/hello.c}をCコンパイラでコンパイルし、実行
  \item[0-2-2] ハンドブック 3章 C言語入門
    \begin{itemize}
    \item 3.1 C言語の基礎知識 (3.1.1, 3.1.2, 3.1.3)
    \item 3.2 制御文 (3.2.1, 3.2.2)
    \item 3.6 関数 (3.6.2)
    \end{itemize}
  \item[0-2-3] \href{https://github.com/todo-group/computer-experiments/blob/master/exercise/basics/hello.tex}{exercise/basics/hello.tex}を{\tt platex}コマンドと{\tt dvipdfmx}コマンドを使ってPDFファイルに変換
  \item[0-2-4] ハンドブック 4章 \LaTeX 入門
    \begin{itemize}
    \item 4.1 \LaTeX の実行
    \end{itemize}
  \item[0-2-5] Gnuplotを使い、$\sin$関数をプロットし、PostScript形式でファイル(*.eps)に出力(ハンドブック2.4節)。作成した図のファイルを \LaTeX に張り込み、PDFファイルを作成(ハンドブック4.8節)
  \end{itemize}
\end{frame}
