\begin{frame}[t]{講義日程(予定)}
  \begin{itemize}
    % \setlength{\itemsep}{1em}
  \item 全8回 (水曜2限 10:25-12:10)
    \begin{itemize}
    \item 4月5日 講義1: 講義の概要・基本的なアルゴリズム
    \item 4月19日 演習1: 環境整備・C言語プログラミング・図のプロット
    \item 4月26日 講義2: 常微分方程式
    \item 5月10日 演習2 (グループ1): 基本的なアルゴリズム・常微分方程式
    \item 5月17日 演習2 (グループ2): 基本的なアルゴリズム・常微分方程式
    \item {\color{gray} 5月24日 休講}
    \item {\color{gray} 5月31日 休講}
    \item 6月7日 講義3: 連立方程式
    \item 6月14日 演習3 (グループ1): 連立方程式
    \item 6月21日 演習3 (グループ2): 連立方程式
    \item {\color{gray} 6月28日 休講}
    \item 7月5日 講義4: 行列の対角化
    \item 7月12日 演習4 (グループ1): 行列の対角化
    \item 7月19日 演習4 (グループ2): 行列の対角化
    \end{itemize}
  \item 2回目以降の演習はクラスを2グループに分けて実施(グループ1: 学生証番号が奇数、グループ2: 偶数)
  \end{itemize}
\end{frame}

%% \begin{frame}[t]{もくもく会}
%%   \begin{itemize}
%%     % \setlength{\itemsep}{1em}
%%   \item 出席は自由
%%   \item 90分間、各自テーマを決めてもくもくと作業する
%%     \begin{itemize}
%%     \item Zoomに接続しっぱなしにする (できればビデオONで)
%%     \item 最初にZoomチャットでその時間に自分がやることを宣言
%%     \end{itemize}
%%   \item 質問自由
%%     \begin{itemize}
%%     \item ZoomでマイクをONにして / Zoom チャット / Slack
%%     \end{itemize}
%%   \item 作業内容の例
%%     \begin{itemize}
%%     \item C言語をマスターする(計算機ハンドブックの例を端から試す)
%%     \item レポート課題のどれかに取り組む
%%     \item 自分のプログラムを10倍速くする、等
%%     \end{itemize}
%%   \end{itemize}
%% \end{frame}
