\begin{frame}[t,fragile]{Nelder-Meadの滑降シンプレックス法}
  \begin{itemize}
    %\setlength{\itemsep}{1em}
  \item $d+1$個の点$x_0,x_1,\cdots,x_d$は$f(x_0) \le f(x_1) \le \cdots \le f(x_d)$の順に並べられているとする
  \item 最大値を取る点$x_d$を除く$d$点の重心を$x_g$とする
  \item Nelder-Mead法では以下のステップを繰り返す
    \begin{itemize}
    \item $x_d$を$x_g$に関する対称な点$x_r$に移動(反射)
      \[
      x_r = x_g + (x_g - x_d)
      \]
    \item $f(x_r)$が$f(x_0)$よりも小さい場合、さらに先に進む(拡大)
      \[
      x_e = x_g + 2(x_r - x_g)
      \]
    \item $f(x_r)$が$f(x_{d-1})$よりもまだ大きい場合には、$x_d$を$x_g$に近づける(縮小)
      \[
      x_c = x_g + (x_d-x_g)/2
      \]
    \item $f(x_c)$が$f(x_d)$よりまだ大きい場合には、$x_0$以外の点を$x_0$に一様に近づける(収縮)
      \[
      x_i \leftarrow x_0 + (x_i-x_0)/2 \qquad (i=1 \cdots d)
      \]
    \end{itemize}
  \end{itemize}
\end{frame}
