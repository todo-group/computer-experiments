\begin{frame}[t,fragile]{共役勾配法による一般の関数の最適化}
  \begin{itemize}
    \setlength{\itemsep}{1em}
  \item $f({\bf x})$は厳密な二次形式ではない
  \item 係数行列$A$も分からない
  \item $\alpha_n$は一次元最適化を使って反復法で求める
  \item $\beta_n$は、$A$を知らなくても計算可能
    \begin{align*}
      \beta_n &= \frac{{\bf r}_{n+1}^T {\bf r}_{n+1}}{{\bf r}_n^T {\bf r}_n} \qquad \text{(Fletcher-Reeves)} \\
      \beta_n &= \frac{{\bf r}_{n+1}^T ({\bf r}_{n+1}-{\bf r}_{n})}{{\bf r}_n^T {\bf r}_n} \qquad \text{(Polac-Ribi\`ere)}
    \end{align*}
    実際の計算では、$\beta \leftarrow \max(0,\beta)$として、直交性が崩れた場合には方向をリセットして、勾配の方向に下るようにする
  \end{itemize}
\end{frame}
