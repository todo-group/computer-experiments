% -*- coding: utf-8 -*-

\documentclass[10pt,dvipdfmx]{beamer}
\usepackage{tutorial}

\title{計算機実験I (第7回)}
\date{2021/06/16}

\begin{document}

\begin{frame}
  \titlepage
  \tableofcontents
\end{frame}

\begin{frame}[t]{本日の課題}
  \begin{itemize}
    %\setlength{\itemsep}{1em}
  \item 「計算機実験 I」実習課題: [対角化] に取り組む
  \item 講義終了後(18:00まで)にITC-LMSで「作業レポート(2021/06/16)」を提出
  \item 質問はSlackの「\# 5\_対角化」あるいは他の適当と思われるチャンネルで \\[2em]
  \end{itemize}
\end{frame}

\section{行列の対角化}
% -*- coding: utf-8 -*-

\documentclass[10pt,dvipdfmx]{beamer}
\usepackage{tutorial}

\begin{document}
\section{行列とLAPACK}
\begin{frame}[t,fragile]{二次元配列}
  \begin{itemize}
    \setlength{\itemsep}{1em}
  \item C言語では、二次元配列は一次元配列の先頭をさす(ポインタ)の配列として表される(と理解しておけば良い)
  \item \verb+a[i]+は、要素\verb+a[i][0]+を指すポインタ
    \begin{itemize}
    \item \verb+a+ と \verb+&a[0]+ は等価 (\verb+&a[0][0]+ ではない)
    \item \verb+a[0]+ と \verb+&a[0][0]+ は等価
    \item \verb+a[2]+ と \verb+&a[2][0]+ は等価
    \item \verb^(a+2)^ と \verb^&a[2]^ は等価
    \item \verb^(*(a+2))[3]^ と \verb^*(*(a+2)+3)^ と \verb^a[2][3]^ は等価
    \item \verb^*(a+2)[3]^ と \verb^*((a+2)[3])^ と \verb^*(a[5])^ と\verb^a[5][0]^ は等価
    \item \verb^[]^は\verb^*^よりも強い
    \end{itemize}
  \item ポインタ確認プログラム: \href{https://github.com/todo-group/computer-experiments/blob/master/exercise/matrix/pointer-matrix.c}{pointer-matrix.c}
  \end{itemize}
\end{frame}

\begin{frame}[t,fragile]{動的二次元配列の確保}
  \begin{itemize}
    \setlength{\itemsep}{1em}
  \item 各行を表す配列とそれぞれの先頭アドレスを保持する配列の二種類が必要
\begin{lstlisting}
double **a;
m = 10;  
n = 10;  
a = (double**)malloc((size_t)(m * sizeof(double*));
for (int i = 0; i < m; ++i)
  a[i] = (double*)malloc((size_t)(n * sizeof(double));
\end{lstlisting}
\item 各行を保持する配列が、メモリ上で連続に確保される保証はない
\item 行列用のライブラリ(LAPACK等)を使うときに問題となる
  \end{itemize}
\end{frame}

\begin{frame}[t,fragile]{BLASライブラリ}
  \begin{itemize}
    \setlength{\itemsep}{1em}
  \item 行列・行列積、行列・ベクトル積などを高速に行う最適化された関数群
  \item 行列・行列積を計算するサブルーチン {\tt dgemm} \\
    \url{http://www.netlib.org/lapack/explore-html/d7/d2b/dgemm_8f.html}
    \begin{itemize}
    \item $C = \alpha A \times B + \beta C$ を計算
    \item BLASもFortranで書かれている
    \end{itemize}
  \item 例: \href{https://github.com/todo-group/computer-experiments/blob/master/exercise/matrix/multiply.c}{multiply.c}, \href{https://github.com/todo-group/computer-experiments/blob/master/exercise/matrix/multiply_dgemm.c}{multiply\_dgemm.c}
  \end{itemize}
\end{frame}

\begin{frame}[t,fragile]{LAPACK (Linear Algebra PACKage)}
  \begin{itemize}
    %\setlength{\itemsep}{1em}
  \item 線形計算のための高品質な数値計算ライブラリ
    \begin{itemize}
    \item \url{http://www.netlib.org/lapack}
    \item 線形方程式、固有値問題、特異値問題、線形最小二乗問題など
    \item (FFT 高速フーリエ変換は入っていない)
    \item LAPACK自体はFortranで書かれている
    \end{itemize}
  \item ほぼ全てのPC、ワークステーション、スーパーコンピュータで利用可 (インストール済)
  \item Netlibでソースが公開されているリファレンス実装は遅いが、それぞれのベンダー(Intel、Fujitsu、etc)による最適化されたLAPACKが用意されている場合が多い(MKL、SSL2、etc)
  \item LAPACKを使うことにより、高速で信頼性が高く、ポータブルなコードを書くことが可能になる
  \end{itemize}
\end{frame}

\begin{frame}[t,fragile]{LAPACKによる連立一次方程式の求解}
  \begin{itemize}
    \setlength{\itemsep}{1em}
  \item LU分解を行うサブルーチン {\tt dgetrf} \\
    \url{http://www.netlib.org/lapack/explore-html/d3/d6a/dgetrf_8f.html}
  \item Fortranによる関数宣言
\begin{lstlisting}
subroutine dgetrf(integer M, integer N,
         double precision, dimension(lda, *) A,
         integer LDA, integer, dimension(*) IPIV,
         integer INFO)
\end{lstlisting}
\item {\tt A}: 左辺の行列、{\tt M,N}: 次元、{\tt IPIV}: 選択されたピボット行のリスト、{\tt lda}: 通常{\tt M} (行数)と同じで良い
  \end{itemize}
\end{frame}

\begin{frame}[t,fragile]{CからBLAS/LAPACKを呼び出す際の注意事項}
  \begin{itemize}
    %\setlength{\itemsep}{1em}
  \item (もともとFortran言語で書かれていたことによる制限)
  \item 関数名はすべて小文字、最後に \verb+_+ (下線)を付ける
  \item スカラー、ベクトル、行列は全て「ポインタ渡し」とする
  \item ベクトルや行列は最初の要素へのポインタを渡す (サイズは別に渡す)
  \item 行列の要素は(0,0) $\rightarrow$ (1,0) $\rightarrow$ (2,0) $\rightarrow\cdots\rightarrow$ $(m-1,0)$ $\rightarrow$ (0,1) $\rightarrow$ (1,1) $\rightarrow\cdots\rightarrow$ $(m-1,n-1)$の順で連続して並んでいなければならない(column-major)
    \begin{itemize}
    \item C言語の二次元配列では \verb+a[i][j]+ の次には \verb%a[i][j+1]%が入っている(row-major)
    \item 行列が転置されて解釈されてしまう!
    \end{itemize}
  \item コンパイル時には{\tt -llapack -lblas}オプションを指定し、LAPACKライブラリとBLASライブラリをリンクする(ハンドブック2.1.6節)
  \end{itemize}
\end{frame}

\begin{frame}[t,fragile]{cmatrix.hライブラリ}
  \begin{itemize}
    %\setlength{\itemsep}{1em}
  \item Column-major形式の二次元配列の確保({\tt alloc\_dmatrix})、開放({\tt free\_dmatrix})、出力({\tt print\_dmatrix})、読み込み({\tt read\_dmatrix})を行うためのユーティリティ関数、(i,j)成分にアクセスするためのマクロ({\tt mat\_elem})他を準備
  \item ソースコード: \href{https://github.com/todo-group/computer-experiments/blob/master/exercise/include/cmatrix.h}{cmatrix.h}
  \item 使用例
\begin{lstlisting}
#include "cmatrix.h"
...
double **mat;
mat = alloc_dmatrix(m, n);
mat_elem(mat, 1, 3) = 5.0;
...
free_dmatrix(mat);
\end{lstlisting}
  \item サンプルコード: \href{https://github.com/todo-group/computer-experiments/blob/master/exercise/matrix/matrix_example.c}{matrix\_example.c}
  \end{itemize}
\end{frame}

\begin{frame}[t,fragile]{alloc\_dmatrixでの動的二次元配列の確保}
  \begin{itemize}
    %\setlength{\itemsep}{1em}
  \item 長さ$m \times n$の一次元配列を用意し、各列(それぞれ$m$要素)の先頭アドレスを長さ$n$のポインター配列に格納する
\begin{lstlisting}
double **a;
m = 10;  
n = 10;  
a = (double**)malloc((size_t)(n * sizeof(double*));
a[0] = (double*)malloc((size_t)(m*n * sizeof(double));
for (int i = 1; i < n; ++i)
  a[i] = a[i-1] + m;
\end{lstlisting}
\item 行列の(i,j)成分を\verb+a[j][i]+に格納することにする (column-major)
  \end{itemize}
\end{frame}

\begin{frame}[t,fragile]{要素アクセス・先頭アドレス}
  \begin{itemize}
    % \setlength{\itemsep}{1em}
  \item 行列の(i,j)成分を\verb+a[j][i]+に格納することにする(column-major)
    \begin{itemize}
      \item \href{https://github.com/todo-group/computer-experiments/blob/master/exercise/matrix/cmatrix.h}{cmatrix.h}ではマクロ(\verb+mat_elem+)を準備
\begin{lstlisting}
#define mat_elem(mat, i, j) (mat)[j][i]
\end{lstlisting}
\item このマクロを使うと、例えば(i,j)成分への代入は以下のように書ける
\begin{lstlisting}
mat_elem(a, i, j) = 1;
\end{lstlisting}
\end{itemize}
  \item LAPACKにベクトルや行列の最初の要素へのポインタを渡す
    \begin{itemize}
      \item ベクトルの最初の要素(0)へのポインタ: \verb+&v[0]+
      \item 行列の最初の要素(0,0)へのポインタ: \verb+&a[0][0]+
      \item \href{https://github.com/todo-group/computer-experiments/blob/master/exercise/matrix/cmatrix.h}{cmatrix.h}にマクロ({\tt vec\_ptr}、{\tt mat\_ptr})が準備されているのでそれぞれ、{\tt vec\_ptr(v)}、{\tt mat\_ptr(a)}と書ける
    \end{itemize}
  \end{itemize}
\end{frame}

\begin{frame}[t,fragile]{LAPACKによる連立一次方程式の求解}
  \begin{itemize}
    \setlength{\itemsep}{1em}
  \item C言語から呼び出すための関数宣言を作成 (ハンドブック3.6.4節)
\begin{lstlisting}
void dgetrf_(int *M, int *N, double *A,
             int *LDA, int*IPIV, int *INFO);
\end{lstlisting}
関数名は全て小文字。関数名の最後に {\tt \_} (下線)を付ける
\item LU分解の例
\begin{lstlisting}
m = 10;
n = 10;
a = alloc_dmatrix(m, n);
...
dgetrf_(&m, &n, mat_ptr(a), &m, vec_ptr(ipiv), &info);
\end{lstlisting}
完全なソースコード: \href{https://github.com/todo-group/computer-experiments/blob/master/exercise/linear_system/lu_decomp.c}{lu\_decomp.c}
  \end{itemize}
\end{frame}

\end{document}

\begin{frame}[t,fragile]{行列の数値対角化}
  \begin{itemize}
    %\setlength{\itemsep}{1em}
  \item 一般的に次元が5以上の行列の固有値は、あらかじめ定まる有限回の手続きでは求まらない
    \begin{itemize}
    \item 必ず何らかの反復法(+収束判定)が必要となる
    \end{itemize}
  \item 密行列向きの方法
    \begin{itemize}
    \item Jacobi法
    \item Givens変換・Householder法(三重対角化) + QR法など
    \end{itemize}
  \item 疎行列向きの方法
    \begin{itemize}
    \item べき乗法
    \item Lanczos法(三重対角化) + QR法など
    \end{itemize}
  \item 固有ベクトル
    \begin{itemize}
    \item QR法で求めたものを逆変換
    \item 逆反復法で精度改善
    \end{itemize}
  \end{itemize}
\end{frame}


\section{疎行列に対する反復法}
\begin{frame}[t,fragile]{反復法}
  \begin{itemize}
    \setlength{\itemsep}{1em}
  \item 疎行列の場合、行列ベクトル積は高速に行える
  \item Givens変換、Householder変換などを行うと疎行列性が失われる
  \item 行列ベクトル積のみを用いる反復法が効果的
    \begin{itemize}
    \item べき乗法
    \item Lanczos法
    \end{itemize}
  \end{itemize}
\end{frame}

\begin{frame}[t,fragile]{固体物理・量子統計物理に現れる行列}
  \begin{itemize}
    %\setlength{\itemsep}{1em}
  \item 強束縛近似(tight-binding approx.)のもとでの第二量子化表示
    \[
    H = -t \sum_{\langle i,j \rangle \sigma} (c_{i,\sigma}^\dagger c_{j,\sigma} + h.c.) + \text{(相互作用)}
    \]
  \item 局所スピン模型(ハイゼンベルグ模型)
    \[
    H = -J\sum_{\langle i,j \rangle} S_i \cdot S_j
    = -J\sum_{\langle i,j \rangle} [S_i^z S_j^z +\frac{1}{2} (S_i^+ S_j^- + S_i^- S_j^+) ]
    \]
  \item 格子点の数を$n$とすると、ハミルトニアンはそれぞれ$4^n \times 4^n$、$2^n \times 2^n$の(疎)行列で表される。
  \item $n$が大きくなると、行列の次元は指数関数的に増加
  \item 量子多体系に共通する困難
  \end{itemize}
\end{frame}

\begin{frame}[t,fragile]{べき乗法(Power Method)}
  \begin{itemize}
    %\setlength{\itemsep}{1em}
  \item 適当なベクトル$v_1$から出発する
  \item $v_1$が最大固有ベクトル$\xi_1$と直交していないとすると
    \[
    v_1 = c_1 \xi_1 + c_2 \xi_2 + c_3 \xi_3 + \cdots + c_n \xi_n
    \]
    と展開できる($c_1 \ne 0$)。この両辺に$A$を次々掛けて行くと
    \begin{align*}
      v_2 = A v_1 &= c_1 \lambda_1 \xi_1 + c_2 \lambda_2 \xi_2 + c_3 \lambda_3 \xi_3 + \cdots + c_n \lambda_n \xi_n \\
      v_3 = A^2 v_1 &= c_1 \lambda_1^2 \xi_1 + c_2 \lambda_2^2 \xi_2 + c_3 \lambda_3^2 \xi_3 + \cdots + c_n \lambda_n^2 \xi_n \\
      \vdots \\
      v_{k+1} = A^k v_1 &= c_1 \lambda_1^k \xi_1 + c_2 \lambda_2^k \xi_2 + c_3 \lambda_3^k \xi_3 + \cdots + c_n \lambda_n^k \xi_n \\
      &= c_1 \lambda_1^k \Big[ \xi_1 + \sum_{i=2}^n \frac{c_i}{c_1} \big( \frac{\lambda_i}{\lambda_1}\big)^k \xi_i \Big] \approx c_1 \lambda_1^k \xi_1 \\
    \end{align*}
  \end{itemize}
\end{frame}

\begin{frame}[t,fragile]{べき乗法の収束}
  \begin{itemize}
    \setlength{\itemsep}{1em}
  \item べき乗法による固有値
    \[
    \frac{v_{n+1}^T v_{n+1}}{v_{n+1}^T v_n} = \lambda_1 + O\Big( \big(\frac{\lambda_2}{\lambda_1} \big)^{2n}\Big)
    \]
  \item 誤差の収束
    \[
    \frac{v_{n+1}^T v_{n+1}}{v_{n+1}^T v_n} \approx \lambda_1 + e^{-2n \ln (\lambda_1/\lambda_2)}
    \]
  \item $1 / \ln (\lambda_1/\lambda_2)$ 程度の反復が必要
  \item $\lambda_1$と$\lambda_2$が近い場合には、反復回数が非常に多くなる
  \end{itemize}
\end{frame}

\begin{frame}[t,fragile]{第2固有値・第3固有値$\cdots$}
  \begin{itemize}
    %\setlength{\itemsep}{1em}
  \item 第1固有ベクトル$\xi_1$の成分を行列から差し引く(減次)
    \[
    A_1 = A - \lambda_1 \xi_1 \xi_1^T
    \]
    この行列は、固有値 $0,\lambda_2,\lambda_3,\cdots,\lambda_n$を持つ
  \item 行列$A_1$に対してべき乗法を使うと、第2固有値$\lambda_2$と対応する固有ベクトル$\xi_2$が得られる
  \item 第$k$固有値まで求まっている場合
    \[
    A_k = A - \sum_{i=1}^k \lambda_i \xi_i \xi_i^T
    \]
  \item 実際には数値誤差のため、ベクトルの直交性は厳密ではない
  \item 大きい方から数個程度を求めるのが限界
  \end{itemize}
\end{frame}

\begin{frame}[t,fragile]{Rayleigh-Ritzの方法}
  \begin{itemize}
    %\setlength{\itemsep}{1em}
  \item $n \times n$行列$A$について、互いに正規直交するベクトル$v_1,v_2,\cdots,v_m$ ($m < n$)が張る部分空間の中で「最良の」固有ベクトルを求めたい
  \item $n \times m$行列
    \[
    V=\big[v_1 v_2 \cdots v_m\big]
    \]
    を定義すると、$V^TV=E_m$が成り立つ(ただし$VV^T \ne E_n$)
  \item 部分空間内のベクトルを$w = \sum_i a_i v_i$と表すと、$\displaystyle \frac{w^TAw}{w^Tw}$が極大値を取る(本当の固有ベクトルにできるだけ平行になる)条件は、
    \[
    \frac{\partial}{\partial a_i} \frac{w^TAw}{w^Tw} \sim \sum_j H_{ij}a_j - \lambda a_i = 0
    \]
  \end{itemize}
\end{frame}

\begin{frame}[t,fragile]{Rayleigh-Ritzの方法}
  \begin{itemize}
    \setlength{\itemsep}{1em}
  \item 行列
    \[
    H_{ij} = v_i^T A_{ij} v_j
    \]
    に対する固有値問題: $H a = \lambda a$
  \item $\lambda$: もとの行列の近似固有値(Ritz値)
  \item $Va$: もとの行列の近似固有ベクトル(Ritzベクトル)
  \item 最大固有値に対する良い近似固有値が欲しい場合 $\Rightarrow$ $v_1,v_2,\cdots,v_M$を最大固有ベクトルになるべく近い(しかし、互いに直交する)ベクトルに選べば良い
  \end{itemize}
\end{frame}

\begin{frame}[t,fragile]{Lanczos法}
  \begin{itemize}
    %\setlength{\itemsep}{1em}
  \item 初期(ランダム)ベクトル$v_1$に$A$を掛けて生成される
    \[
    v_1, Av_1, A^2v_1, \cdots A^{m-1}v_1
    \]
    を正規直交化して$v_1,v_2,\cdots,v_m$を作る(Krylov部分空間)
    \[
      {\cal K}_m(A,v_1) = \text{span} \{ v_1, Av_1, A^2v_1, \cdots A^{m-1}v_1 \}
      \]
  \item 部分空間でのRitz値を固有値の近似値とする
  \item $A^kv_1$はどんどん最大固有ベクトルに近づいていくので、$m \ll n$でも良い近似固有値が得られると期待される
  \end{itemize}
\end{frame}

\begin{frame}[t,fragile]{Lanczos法}
  \begin{itemize}
    %\setlength{\itemsep}{1em}
  \item 正規化された初期(ランダム)ベクトル$v_1$から出発する %($v_0=0$とする)
  \item $v_2,v_3,\cdots$を生成する
    \begin{align*}
      v_2 &= (Av_1 - \alpha_1 v_1)/\beta_1 \\
      v_3 &= (Av_2 - \beta_1 v_1 - \alpha_2 v_2)/\beta_2 \\
      \vdots
    \end{align*}
    ここで
    \begin{align*}
      \alpha_i &= v_i^T A v_i \\
      \beta_i &= | A v_i - \beta_{i-1} v_{i-1} - \alpha_i v_i |, \ \beta_0 = 0
    \end{align*}
    と選ぶ
  \end{itemize}
\end{frame}

\begin{frame}[t,fragile]{Lanczos法}
  \begin{itemize}
    \setlength{\itemsep}{1em}
  \item $v_1,v_2,v_3,\cdots,v_{M+1}$は正規直交
  \item 漸化式を書き換えると
    \begin{align*}
      Av_1 &= \alpha_1 v_1 + \beta_1 v_2 \\
      Av_2 &= \beta_1 v_1 + \alpha_2 v_2 + \beta_2 v_3 \\
      Av_3 &= \beta_2 v_2 + \alpha_3 v_3 + \beta_3 v_4 \\
      \vdots \\
      Av_{M} &= \beta_{M-1} v_{M-1} + \alpha_M v_M + \beta_M v_{M+1}
    \end{align*}
  \end{itemize}
\end{frame}

\begin{frame}[t,fragile]{Lanczos法}
  \begin{itemize}
    % \setlength{\itemsep}{1em}
  \item 行列で表現すると
    \begin{align*}
      \hspace*{-2em}
      A
      \big[ v_1v_2\cdots v_m \big]
      &=
      \big[ v_1v_2\cdots v_m v_{m+1} \big]
      \begin{bmatrix}
        \alpha_1 & \beta_1\\
        \beta_1 & \alpha_2 & \beta_2 \\
        & \beta_2 & \alpha_3 & \beta_3 \\
        & & \beta_3 & \alpha_4 & \beta_4 \\
        & & & \ddots & \ddots & \ddots \\
        & & & & \beta_{m-1} & \alpha_m \\
        & & & & & \beta_m \\
      \end{bmatrix}
    \end{align*}
    両辺に左から$\big[ v_1v_2\cdots v_m \big]^T$をかけると
    \[
    \big[ v_1v_2\cdots v_m \big]^T A \big[ v_1v_2\cdots v_m \big]
    \]
    は3重対角行列となることがわかる
  \end{itemize}
\end{frame}

\begin{frame}[t,fragile]{Lanczos法}
  \begin{itemize}
    %\setlength{\itemsep}{1em}
  \item 原理的には、$N$ステップ目で$\beta_N=0$となり、3重対角化が完了する
  \item 実際には、数値誤差のため$v_1,v_2,v_3\cdots$の直交性が崩れる
    \begin{itemize}
      \item $M$を大きくしすぎると、おかしな固有値が出てくる
      \item 全ての固有値が欲しい場合にはHouseholder法を使う
    \end{itemize}
  \item Lanczos法では、大きな固有値に対応する固有ベクトルにできるだけ近いものから部分空間を作っていく
    \begin{itemize}
      \item 100万次元以上の行列の場合でも$M=100 \sim 200$程度で最初の数個の固有値は精度良く求まる
    \end{itemize}
  \item 必要な操作は、行列とベクトルの積、ベクトルの内積・スケーリング・和のみ
    \begin{itemize}
      \item 疎行列の場合、非常に効率が良い
    \end{itemize}
  \end{itemize}
\end{frame}


\section{逆反復法}

\begin{frame}[t,fragile]{逆反復法による固有ベクトルの精度向上}
  \begin{itemize}
    %\setlength{\itemsep}{1em}
  \item 逆反復法
    \begin{itemize}
    \item 近似固有値を$\mu$とするとき、行列$(A - \mu E)^{-1}$を考えると、固有ベクトルは$A$と同じ、固有値は$(\lambda-\mu)^{-1}$。
    \item $\mu$が十分に正確であれば、$(\lambda-\mu)^{-1}$は絶対値最大の固有値。行列$(A - \mu E)^{-1}$を適当な初期ベクトルにかけ続けると$\lambda$に対応する固有ベクトルに収束 (c.f. べき乗法)
    \item 実際には$(A-\mu E) x' = x$という連立方程式を繰り返し解く
    \end{itemize}
  \end{itemize}
\end{frame}

%-*- coding:utf-8 -*-

\section{変分法}

%-*- coding:utf-8 -*-

\begin{frame}[t,fragile]{変分法}
  \begin{itemize}
    %\setlength{\itemsep}{1em}
  \item 波動関数を互いに直交する正規化された波動関数(基底関数)の線形結合で近似する (変分波動関数、試行関数)
    \begin{align*}
      | \psi \rangle = \sum_{p=1}^m C_p | \phi_p \rangle \qquad (\langle \phi_p | \phi_q \rangle = \delta_{pq})
    \end{align*}
  \item エネルギーの期待値
    \begin{align*}
      E &= \frac{\langle \psi | H | \psi \rangle}{\langle \psi | \psi \rangle} = \frac{\sum_{p,q} C_p^* H_{pq} C_q}{\sum_{p,q} C_p^* \delta_{pq} C_q} \\
      H_{pq} &= \langle \phi_p | H | \phi_q \rangle
    \end{align*}
  \item $E$ができるだけ小さくなるよう係数$C_p$を最適化 (変分原理)
  \end{itemize}
\end{frame}

%-*- coding:utf-8 -*-

\begin{frame}[t,fragile]{変分法}
  \begin{itemize}
    %\setlength{\itemsep}{1em}
  \item $\delta E = 0$から
    \begin{align*}
      \sum_{q} (H_{pq} - E \delta_{pq} ) C_q = 0 \qquad \text{for $^\forall p$}
    \end{align*}
  \item $H_{pq}$, $\delta_{pq}$を$m \times m$行列と考えると、固有値問題とみなせる
    \begin{align*}
      H C = E C
    \end{align*}
  \item $H$はエルミート行列
  \item $\{ \phi_p \}$の張る部分空間での最適化 (= Rayleigh-Ritzの方法)
  \item 変分波動関数と真の波動関数の差が$\epsilon$程度の時、$E$と真の固有エネルギーの差は$\epsilon^2$程度
  \end{itemize}
\end{frame}

%-*- coding:utf-8 -*-

\begin{frame}[t,fragile]{非直交基底関数による変分法}
  \begin{itemize}
    %\setlength{\itemsep}{1em}
  \item 重なり積分
    \begin{align*}
      S_{pq} = \langle \phi_p | \phi_q \rangle \ne \delta_{pq}
    \end{align*}
  \item 変分波動関数の正規化条件
    \begin{align*}
      \langle \psi | \psi \rangle = \sum_{p,q} C_p^* \langle \phi_p | \phi_q \rangle C_q = \sum_{p,q} C_p^* S_{pq} C_q = 1
    \end{align*}
  \item エネルギー期待値
    \begin{align*}
      E = \frac{\sum_{p,q} C_p^* H_{pq} C_q}{\sum_{p,q} C_p^* S_{pq} C_q}
    \end{align*}
  \item $\delta E = 0$から
    \begin{align*}
      \sum_q (H_{pq} - E S_{pq}) C_q = 0 \ \Rightarrow \ HC = ESC \ \text{(一般化固有値問題)}
    \end{align*}
  \end{itemize}
\end{frame}

%-*- coding:utf-8 -*-

\begin{frame}[t,fragile]{一般化固有値問題}
  \begin{itemize}
    %\setlength{\itemsep}{1em}
  \item 重なり行列 $S_{pq} = \langle \phi_p | \phi_q \rangle$
    \begin{itemize}
      \item エルミート行列: $S_{pq} = S_{qp}^*$
      \item 正定値 ($\{\phi_p\}$が線形独立の場合):
        \begin{align*}
          x^\dagger S x = \sum_{pq} \langle \phi_p | \phi_q \rangle x_p^* x_q = || \sum_p x_p | \phi_p \rangle ||^2 > 0
        \end{align*}
    \end{itemize}
  \item 一般化固有値問題 $\Rightarrow$ 2回の固有値分解により解くことができる
    \begin{itemize}
      \item $S$を固有値分解: $S = U D U^\dagger$
      \item $S$固有値は全て正 $\Rightarrow$ $D^{-1/2}$を定義可
      \item $HC=ESC$ $\Rightarrow$ $D^{-1/2} U^\dagger H U D^{-1/2} D^{1/2} U^\dagger C = E D^{1/2} U^\dagger C$
      \item $H' = D^{-1/2} U^\dagger H U D^{-1/2}$、$C'D^{1/2} U^\dagger C$とおくと
        \begin{align*}
          H'C' = EC' \qquad \text{(通常の)固有値問題}
        \end{align*}
      \item (1回目の固有値分解はコレスキー分解$A=L L^\dagger$ ($L$は下三角行列)を用いてもよい)
    \end{itemize}
  \end{itemize}
\end{frame}



\begin{frame}[t]{今後の予定}
  \begin{itemize}
    % \setlength{\itemsep}{1em}
  \item 全8回 (水曜2限 10:25-11:55)
    \begin{itemize}
    \item {\color{gray} 4月7日 第1回: 環境整備・数値誤差}
    \item {\color{gray} 4月14日 もくもく会}
    \item {\color{gray} 4月21日 第2回: ニュートン法・二分法・常微分方程式}
    \item {\color{gray} 4月28日 第3回: 固有値問題・シンプレクティック積分法}
    \item {\color{gray} 5月12日 もくもく会}
    \item {\color{gray} 5月19日 第4回: 行列演算・複素数・ライブラリ [レポートNo.1締切]}
    \item {\color{gray} 5月26日 第5回: 連立一次方程式・直接解法・反復解法}
    \item {\color{gray} 6月2日 もくもく会}
    \item {\color{gray} 6月9日 第6回: 行列の対角化 [レポートNo.2締切]}
    \item {\color{gray} 6月16日 第7回: 疎行列に対する反復法・変分法}
    \item 6月23日 第8回: 特異値分解・最小二乗法
    \item 7月7日 [レポートNo.3締切]
     \end{itemize}
  \end{itemize}
\end{frame}

\end{document}
