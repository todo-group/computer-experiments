\begin{frame}[t,fragile]{固有値問題の解法(シューティング)}
  \begin{itemize}
    %\setlength{\itemsep}{1em}
  \item $x_n=h \times n$ ($h=1/N$)、$x_0=0$、$x_N=1$とする
  \item $\psi(x_0)=0$、$\psi(x_1) = 1$を仮定 ($\psi'(x_0)=1/h$と与えたことに相当)
  \item $E = 0$とおく
  \item Runge-Kutta法、Numerov法などを用いて$x=x_N$まで積分
  \item $\psi(x_N)$の符号がかわるまで、$E$を少しずつ増やす
  \item 符号が変わったら、$E$の区間を半分ずつに狭めていき、$\psi(x_N)=0$となる$E$ (固有エネルギー)と$\psi(x)$ (波動関数)を得る
  \end{itemize}
\end{frame}
