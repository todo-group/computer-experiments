\begin{frame}[t]{拡散方程式に対するクランク・ニコルソン法}
  \begin{itemize}
  \item さらに、時間方向にきざみ幅$\Delta t/2$の中心差分を使うと
    \[
    \frac{\partial u}{\partial t} \Big|_{(j \Delta x, n \Delta t)} = \frac{u_j^{n+\frac{1}{2}} - u_j^{n-\frac{1}{2}}}{\Delta t} + {\cal O}(\Delta t^2)
    \]
  \item $x$に関する中心差分と組み合わせ、$n \rightarrow n+\frac{1}{2}$し、さらに$u_j^{n+\frac{1}{2}}$を$(u_j^{n+1}+u_j^{n})/2$で近似すると
    \[
    u_{j}^{n+1} = u_{j}^{n} + \frac{r}{2} (u_{j+1}^{n+1} - 2 u_{j}^{n+1}  +u_{j-1}^{n+1} + u_{j+1}^{n} - 2 u_{j}^{n} + u_{j-1}^{n})
    \]
    あるいは
    \[
    u_{j}^{n+1} - \frac{r}{2} (u_{j+1}^{n+1} - 2 u_{j}^{n+1} + u_{j-1}^{n+1}) = u_{j}^{n} + \frac{r}{2} (u_{j+1}^{n} - 2 u_{j}^{n} + u_{j-1}^{n})
    \]
    $\Rightarrow$ クランク・ニコルソン法 [$O(\Delta t^2) + O(\Delta x^2)$]
  \end{itemize}
\end{frame}
