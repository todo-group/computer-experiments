\begin{frame}[t,fragile]{ベルレ法の変形}
  \begin{itemize}
    %\setlength{\itemsep}{1em}
  \item 半整数ステップにおける運動量
    \[
    p(t + h/2) = \frac{q(t+h)-q(t)}{h}
    \]
    を導入すると
    \begin{align*}
      \begin{split}
        p(t + h/2) &= p(t - h/2) + \frac{q(t+h)-2q(t)+q(t-h)}{h} \\
        &= p(t - h/2) + h f[t,q(t)]
      \end{split}
    \end{align*}
    一方
    \[
    q(t+h) = q(t) + hp(t+h/2)
    \]
  \item $\Rightarrow$ リープ・フロッグ法

    座標についてはベルレ法と数学的に等価だが、丸め誤差に強い
  \end{itemize}
\end{frame}
