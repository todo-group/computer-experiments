\begin{frame}[t,fragile]{シンプレクティック積分法}
  \begin{itemize}
    %\setlength{\itemsep}{1em}
  \item ハミルトン力学系の満たすべき特性(位相空間の体積保存)を満たす
  \item 一般的には陰解法
  \item ハミルトニアンが$H(p,q) = T(p) + V(q)$の形で書ける場合は陽的なシンプレクティック積分法が存在する
  \item エネルギーは近似的に保存する
  \item $n$次のシンプレクティック積分法では、エネルギーは$O(h^n)$の範囲で振動(発散しない)
  \item より高次のシンプレクティック積分法についても、システマティックに構成できる(ただし係数を解析的に求められるのは4次まで)。参考文系: H. Yoshida, Phys. Lett. A {\bf 150}, 262 (1990)
  \end{itemize}
\end{frame}
