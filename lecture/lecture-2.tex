\documentclass[dvipdfmx]{beamer}
\usepackage{tutorial}

\title{計算機実験(L2) --- 常備分方程式の解法}
\date{2016/04/13}

\begin{document}

\begin{frame}
  \titlepage
  \tableofcontents
\end{frame}

\section{常微分方程式の初期値問題}

\begin{frame}[t,fragile]{準備: 微分方程式の書き換え}
  \begin{itemize}
    %\setlength{\itemsep}{1em}
  \item 2階の常微分方程式の一般形
    \[
    \frac{d^2y}{dx^2} + p(x)\frac{dy}{dx} + q(x)y = r(x)
    \]
  \item $y_1 \equiv y$, $y_2 \equiv \frac{dy}{dx}$とおくと
    \[
    \left\{
    \begin{array}{ccl}
      \frac{dy_1}{dx} & = & y_2 \\
      \frac{dy_2}{dx} & = & r(x) - p(x) y_2 - q(x) y_1
    \end{array}
    \right.
    \]
  \item さらに$\bm{y}\equiv(y_1, y_2)$, $\bm{f}(x, \bm{y})\equiv \left(y_2, r(x)-p(x)y_2 - q(x)y_1\right)$
    \[
    \frac{d\bm{y}}{dx} = \bm{f}(x, \bm{y})
    \]
  \item $n$階常微分方程式 $\Rightarrow$ $n$次元の1階常微分方程式
  \end{itemize}
\end{frame}

\begin{frame}[t,fragile]{初期値問題と境界値問題}
  \begin{itemize}
    \setlength{\itemsep}{1em}
  \item 初期値問題
    \begin{itemize}
    \item 微分方程式において、ある1点に関する全ての境界条件(初期値)が与えられているもの
    \item 質点の運動など(時系列の問題)
  \end{itemize}
  \item 境界値問題
    \begin{itemize}
    \item 複数の点に関する境界条件が与えられているもの
    \item 物体のゆがみの計算や静電場の計算など(空間的に解く問題)
  \end{itemize}
  \item 初期値問題は初期値から逐次的に解くことが可能
  \item 境界値問題は初期値問題に比べて計算法が複雑
  \end{itemize}
\end{frame}

\begin{frame}[t,fragile]{初期値問題の解法 (Euler法)}
  \begin{itemize}
    \setlength{\itemsep}{1em}
  \item 微分を差分で近似する(前進差分)
    \[
    \frac{dy}{dt} \approx \frac{y(t+\Delta t) - y(t)}{\Delta t} = f(t, y)
    \]
  \item $t=0$における$y(t)$の初期値を$y_0$、$h$を微少量、$t_n \equiv nh$、$y_n$を$y(t_n)$の近似値とおくと、
    \[
    y_{n+1}-y_n = h f( t_n, y_n)
    \]
  \item Euler法
    \begin{itemize}
    \item $y_0$からはじめて、$y_1,y_2,\cdots$を順次求めていく
    \end{itemize}
  \end{itemize}
\end{frame}

\begin{frame}[t,fragile]{Euler法の精度}
  \begin{itemize}
    \setlength{\itemsep}{1em}
  \item 微分方程式の両辺を$t_n$から$t_{n+1}$まで積分(積分方程式)
    \[
    y(t_{n+1}) - y(t_n) = \int^{t_{n+1}}_{t_n} \!\! f(t, y(t)) dt = h \int^1_0 \! f(t_n+h\tau, y(t_n+h\tau)) d\tau
    \]
  \item Euler法は、被積分関数を定数で近似することに対応
    \[
    f(t_n+h\tau, y(t_n+h\tau)) = f(t_n, y(t_n)) + O(h)
    \]
  \item $t=0$からある$t_f$まで積分すると、反復回数$N = t_f / h$
  \item $t=t_f$における誤差 $\sim N \times h \times O(h) = O(h)$
  \end{itemize}
\end{frame}

\begin{frame}[t,fragile]{Euler法の改良}
  \begin{itemize}
    \setlength{\itemsep}{1em}
  \item 積分方程式の被積分関数をもう1次高次まで展開
    \[
    f(t_n+h\tau, y(t_n+h\tau)) = f(t_n, y(t_n)) +
    \tau h
    \left\{
    \frac{\partial f}{\partial t}
    + f \frac{\partial f}{\partial y}
    \right\}_{t=t_n, y=y_n}
    \!\!\!\!\!\!\!\!\!\!\!\! + O(h^2)
    \]
  \item 積分を実行すると
    \[
    y(t_{n+1}) = y(t_n) + h f(t_n, y_n) + \frac{1}{2}h^2
    \left\{
    \frac{\partial f}{\partial t}
    + f \frac{\partial f}{\partial y}
    \right\}_{t=t_n, y=y_n}
    \!\!\!\!\!\!\!\!\!\!\!\! + O(h^3)
    \]
  \end{itemize}
\end{frame}

\begin{frame}[t,fragile]{中点法(2次Runge-Kutta法)}
  \begin{itemize}
    %\setlength{\itemsep}{1em}
  \item 2次公式
    \[
    \begin{array}{rcl}
      k_1 & = & h f(t_n, y_n) \\
      k_2 & = & h f(t_n + \frac{1}{2}h, y_n + \frac{1}{2}k_1) \\
      y_{n+1} & = & y_n + k_2
    \end{array}
    \]
  \item このとき
    \[
    k_2 = h 
    \left\{
    f(t_n, y_n)
    + \frac{1}{2}h \frac{\partial f}{\partial t}
    + \frac{1}{2}k_1 \frac{\partial f}{\partial y}
    + O(h^2)
    \right\}
    \]
  \item したがって
    \[
    y_{n+1} = y_n + h f(t_n, y_n) + \frac{1}{2}h^2
    \left\{
    \frac{\partial f}{\partial t}
    + f \frac{\partial f}{\partial y}
    \right\}_{t=t_n, y=y_n}
    \!\!\!\!\!\!\!\!\!\!\!\!+ O(h^3)
    \]
  \end{itemize}
\end{frame}

\begin{frame}[t,fragile]{高次のRunge-Kutta法}
  \begin{itemize}
    %\setlength{\itemsep}{1em}
  \item 3次Runge-Kutta法
    \[
    \begin{array}{rcl}
      k_1 & = & h f(t_n, y_n) \\
      k_2 & = & h f(t_n + \frac{2}{3}h, y_n + \frac{2}{3}k_1) \\
      k_3 & = & h f(t_n + \frac{2}{3}h, y_n + \frac{2}{3}k_2) \\
      y_{n+1} & = & y_n + \frac{1}{4}k_1 + \frac{3}{8}k_2
      + \frac{3}{8}k_3
    \end{array}
    \]
  \item 4次Runge-Kutta法
    \[
    \begin{array}{rcl}
      k_1 & = & h f(t_n, y_n) \\
      k_2 & = & h f(t_n + \frac{1}{2}h, y_n + \frac{1}{2}k_1) \\
      k_3 & = & h f(t_n + \frac{1}{2}h, y_n + \frac{1}{2}k_2) \\
      k_4 & = & h f(t_n + h, y_n + k_3) \\
      y_{n+1} & = & y_n + \frac{1}{6}k_1 + \frac{1}{3}k_2
      + \frac{1}{3}k_3 + \frac{1}{6}k_4
    \end{array}
    \]
  \item 4次までは次数と$f$の計算回数が等しい
  \end{itemize}
\end{frame}

\begin{frame}[t,fragile]{計算コストと精度}
  \begin{itemize}
    \setlength{\itemsep}{1em}
  \item 実際の計算では$f(t,y)$の計算にほとんどのコストがかかる
  \item 計算回数と計算精度の関係
    \begin{center}
      \begin{tabular}[h]{c|cccc}
        & 1次(Euler法) & 2次(中点法) & 3次 & 4次 \\
        \hline
        計算精度 & $O(h)$ & $O(h^2)$ & $O(h^3)$ & $O(h^4)$ \\
        計算回数 & $N$ & $2N$ & $3N$ & $4N$
      \end{tabular}
    \end{center}
  \item 高次のRunge-Kuttaを使う方が効率的
  \item どれくらい小さな$h$が必要となるか、前もっては分からない
  \item 刻み幅を変えて($h,h/2,h/4,\dots$)計算してみることが大事
    \begin{itemize}
    \item 誤差の評価
    \item 公式の間違いの発見
    \end{itemize}
  \end{itemize}
\end{frame}

\begin{frame}[t,fragile]{陽解法と陰解法}
  \begin{itemize}
    \setlength{\itemsep}{1em}
  \item 陽解法: 右辺が既知の変数のみで書かれる(例: Euler法)
    \begin{itemize}
    \item プログラムがシンプル
    \end{itemize}
  \item 陰解法: 右辺にも未知変数が含まれる
    \begin{itemize}
    \item 例: 逆Euler法
      \begin{align*}
        y(t) &= y(t+h-h) = y(t+h) - h f(t+h,y(t+h)) + O(h^2) \\
        y_{n+1} &= y_n + h f(t+h,{\color{red}y_{n+1}})
      \end{align*}
    \item 数値的により安定な場合が多い
    \item Newton法などを使って、非線形方程式を解く必要がある
    \end{itemize}
  \end{itemize}
\end{frame}

\section{シンプレクティック積分法}

\begin{frame}[t,fragile]{ハミルトン力学系}
  \begin{itemize}
    % \setlength{\itemsep}{1em}
  \item 時間をあらわに含まない場合のハミルトン方程式
    \[
    \frac{dq}{dt} = \frac{\partial H}{\partial p}, \ \frac{dp}{dt} = -\frac{\partial H}{\partial q}
    \]
    \begin{itemize}
    \item エネルギー保存則
      \[
      \frac{dH}{dt} = \frac{\partial H}{\partial q} \frac{dq}{dt} + \frac{\partial H}{\partial p} \frac{dp}{dt} = 0
      \]
    \item 位相空間の体積が保存(Liouvilleの定理)

      位相空間上の流れの場$\bm{v} = (\frac{dq}{dt},\frac{dp}{dt})$について
      \[
      \text{div} \bm{v} = \frac{\partial}{\partial q} \frac{dq}{dt} + \frac{\partial}{\partial p} \frac{dp}{dt} = 0
      \]
    \end{itemize}
  \item Euler法、Runge-Kutta法などはいずれの性質も満たさない
  \end{itemize}
\end{frame}

\begin{frame}[t,fragile]{シンプレクティック数値積分法(Symplectic Integrator)}
  \begin{itemize}
    %\setlength{\itemsep}{1em}
  \item 体積保存を満たす解法
  \item 例: 調和振動子$H=\frac{1}{2}(p^2+q^2)$の運動方程式
    \[
    \frac{dq}{dt} = p, \ \frac{dp}{dt} = -q
    \]
    の一方をEuler法で、他方を逆オイラー法で解く
    \begin{align*}
      q_{n+1} &= q_n + h p_n \\
      p_{n+1} &= p_n - h q_{n+1} = (1-h^2) p_n - h q_n \\
      \begin{pmatrix} q_{n+1} \\ p_{n+1} \end{pmatrix} &= \begin{pmatrix} 1 & h \\ -h & 1-h^2 \end{pmatrix} \begin{pmatrix} q_{n} \\ p_{n} \end{pmatrix}
    \end{align*}
  \end{itemize}
\end{frame}

\begin{frame}[t,fragile]{体積・エネルギーの保存}
  \begin{itemize}
    \setlength{\itemsep}{1em}
  \item 体積保存
    \begin{align*}
      \det \begin{pmatrix} 1 & h \\ -h & 1-h^2 \end{pmatrix} = 1
    \end{align*}
  \item エネルギーの保存
    \begin{align*}
      \frac{1}{2}(p_{n+1}^2+q_{n+1}^2) + {\color{red}\frac{h}{2} p_{n+1} q_{n+1}} = \frac{1}{2}(p_{n}^2+q_{n}^2) + {\color{red}\frac{h}{2} p_{n} q_{n}}
    \end{align*}
  \item 位相空間の体積は厳密に保存
  \item エネルギーは$O(h)$の範囲で保存し続ける
  \end{itemize}
\end{frame}

\begin{frame}[t,fragile]{2次のシンプレクティック積分法}
  \begin{itemize}
    \setlength{\itemsep}{1em}
  \item ハミルトニアンが$H(p,q) = T(p) + V(q)$の形で書けるとする
  \item リープ・フロッグ法
    \begin{align*}
      {\color{red} p(t+h/2)} &= p(t) - \frac{h}{2} \frac{\partial V(q)}{\partial q}|_{q=q(t)} \\
      {\color{blue} q(t+h)} &= q(t) + h {\color{red}p(t+h/2)} \\
      p(t+h) &= {\color{red}p(t+h/2}) - \frac{h}{2} \frac{\partial V(q)}{\partial q}|_{q=q(t+h)}
    \end{align*}
  \end{itemize}
\end{frame}

\begin{frame}[t,fragile]{シンプレクティック積分法}
  \begin{itemize}
    \setlength{\itemsep}{1em}
  \item ハミルトン力学系の満たすべき特性(位相空間の体積保存)を満たす
  \item 一般的には陰解法
  \item ハミルトニアンが$H(p,q) = T(p) + V(q)$の形で書ける場合は陽的なシンプレクティック積分法が存在する
  \item エネルギーは近似的に保存する
  \item $n$次のシンプレクティック積分法では、エネルギーは$O(h^n)$の範囲で振動(発散しない)
  \end{itemize}
\end{frame}

\section{バージョン管理システムとは?}

\begin{frame}[t,fragile]{バージョン管理システムの必要性}
  \begin{itemize}
    \setlength{\itemsep}{1em}
  \item 広く行われている「ファイル管理」
    \begin{itemize}
    \item ファイル名・ディレクトリ名による管理

      (日付、人名、バージョン番号など)
    \item 手書きのログファイルによる記録
    \end{itemize}
  \item 問題点
    \begin{itemize}
    \item 記録を付け忘れる、記録を間違う、不完全な記録
    \item 人により命名規則がばらばら
    \item コンピュータ間でコピーを繰り返すと、どれを修正したか、どれが新しいか分からなくなる
    \item 同じバージョンを元に、別の人が独立に修正を行ってしまう

      (バージョンの分岐)
    \end{itemize}
  \end{itemize}
\end{frame}

\begin{frame}[t,fragile]{ありがちなパターン}
  \vspace*{-1.8em}
  \begin{center}
    \resizebox{0.93\textwidth}{!}{\includegraphics{image/vcs-bad.pdf}}
  \end{center}
  \vspace*{-2em}
  {\footnotesize (渡辺2013)}
\end{frame}

\begin{frame}[t,fragile]{バージョン管理システム(Version Control System)とは?}
  \begin{itemize}
    \setlength{\itemsep}{1em}
  \item ファイルの履歴をデータベース(リポジトリ)で一括管理するシステム
    \begin{itemize}
    \item 全ての修正履歴(差分)を保存
    \item 更新毎に一意なバージョン番号(リビジョン)を付与
    \item 任意のバージョン間の比較が可能
    \item もともとはプログラムのソースコードのためのシステム
    \item それ以外のファイル(例えば TeX ファイル)管理にも使える
    \end{itemize}
  \item チーム・分散環境での作業をサポート
    \begin{itemize}
    \item ネットワーク経由でファイルを check out/check in
    \item 複数箇所から同時に更新した場合に衝突を回避するしくみ
    \item ブランチ・マージ・タグの管理
    \item 一人で使っても複数人で使っても超便利
    \item 超優秀な(かつ超まじめな)秘書のようなもの (しかもタダ)
    \end{itemize}
  \end{itemize}
\end{frame}

\begin{frame}[t,fragile]{バージョン・ブランチ・マージ}
  \begin{center}\resizebox{!}{0.8\textheight}{\includegraphics{image/revision.pdf}}\end{center}
\end{frame}

\begin{frame}[t,fragile]{バージョン管理を使うと}
  \vspace*{-1.8em}
  \begin{center}
    \resizebox{1.0\textwidth}{!}{\includegraphics{image/vcs-good.pdf}}
  \end{center}
  \vspace*{-2em}
  {\footnotesize (渡辺2013)}
\end{frame}

\begin{frame}[t,fragile]{diff と patch}
  \begin{itemize}
    \setlength{\itemsep}{1em}
  \item {\tt diff}: 2つのテキストファイルの差分を出力するコマンド
    \begin{itemize}
    \item ファイル全体を保存するよりコンパクト
    \item 変更点を確認しやすい
      
      {\tt \$ \underline{diff -u file1.txt file2.txt > file.diff}}
    \end{itemize}
  \item {\tt patch}: {\tt diff}コマンドが生成した差分をファイルに適用するユーティリティー
    \begin{itemize}
    \item もとのファイルと差分から変更後のファイルを生成できる

      {\tt \$ \underline{patch < file.diff}}
    \end{itemize}
  \end{itemize}
\end{frame}

\begin{frame}[t,fragile]
  \frametitle{実習: diff \& patch (1)}
  \begin{itemize}
    %\setlength{\itemsep}{1em}
  \item 単一ファイルの例
\begin{lstlisting}
$ cp prologue.txt prologue-orig.txt
$ vi prologue.txt # prologue.txtを編集

$ diff -u prologue-orig.txt prologue.txt > prologue.diff
$ less prologue.diff # prologue.diffの中身を見てみる

$ mv prologue-orig.txt prologue.txt
$ patch < prologue.diff
$ less prologue.txt # prologue.txtの中身を確認
\end{lstlisting}
  \end{itemize}
\end{frame}

\begin{frame}[t,fragile]
  \frametitle{実習: diff \& patch (2)}
  \begin{itemize}
    %\setlength{\itemsep}{1em}
  \item ディレクトリ全体を扱う例
\begin{lstlisting}
$ cp -r vcs vcs.orig
# vcsの中のファイルを編集(ファイルの削除や追加も可)

$ diff -urN vcs.orig vcs > vcs.diff
$ less vcs.diff # vcs.diffの中身を見てみる

$ rm -rf vcs
$ mv vcs.orig vcs
$ patch -p0 < vcs.diff
# vcsの中身を確認
\end{lstlisting}
  \item diff と patch で差分の管理は可能になるが, 履歴は別に管理してお
    かなければならない
  \end{itemize}
\end{frame}

\section{Subversion入門}

\begin{frame}
  \frametitle{主なバージョン管理システム}
  \begin{itemize}
  \item BitKeeper - かつて Linux のカーネルのソース管理に使われていた
  \item CVS (Concurrent Versions System) - ネットワークでの利用を考慮とした初めてのバージョン管理システム. 以前はよく使われていた
  \item Git - 現在 Linux の開発に使われている. 分散型リポジトリ
  \item Mercurial - Git のライバル. 分散型リポジトリ
  \item SCCS (Source Code Control System) - 70年代にベル研で開発された世界初のバージョン管理システム. 現在は使われない
  \item {\color{red}Subversion} - CVSの改良版として開発された. 現在最もポピュラー? Mac OS X や多くの Linux には最初からインストールされている
  \end{itemize}
\end{frame}

\begin{frame}[t,fragile]{Subversionに関する資料}
  \begin{itemize}
    %\setlength{\itemsep}{1em}
  \item CVS/Subversionを使ったバージョン管理(前編:バージョン管理の基礎)
    \begin{itemize}
    \item \url{http://sourceforge.jp/magazine/08/09/09/1038233}
    \end{itemize}
  \item CVS/Subversionを使ったバージョン管理(後編:SVNを使ったバージョン管理)
    \begin{itemize}
      \item \url{http://sourceforge.jp/magazine/08/09/24/113215}
    \end{itemize}
  \item Subversion によるバージョン管理
    \begin{itemize}
      \item \url{http://svnbook.red-bean.com/index.ja.html}
    \end{itemize}
  \item 「Subversion によるバージョン管理」の読み方
    \begin{itemize}
      \item \url{http://exa.phys.s.u-tokyo.ac.jp/ja/members/wistaria/log/subversion-intro}
    \end{itemize}
  \item Gitを使いたい場合には、、、
    \begin{itemize}
      \item \href{http://www.cms-initiative.jp/ja/research-support/develop-support/how-to-publish/develop-apps/dt0l33/manage-version}{CMSIハンズオン - バージョン管理システム}
    \end{itemize}
  \end{itemize}
\end{frame}

\begin{frame}[t,fragile]{Subversionリポジトリ}
  \begin{columns}[T]
    \begin{column}{.7\textwidth}
      \begin{itemize}
        \setlength{\itemsep}{1em}
      \item ソースコードの全ての履歴を保存する「データベース」
      \item リポジトリからソースコードのチェックアウト
      \item リポジトリの実体は、ディスク上のディレクトリ
      \item リポジトリへのアクセス方法
        \begin{itemize}
        \item ssh によるアクセス: {\tt svn+ssh://ユーザ名@ホスト名/リポジトリ名}
        \item ネットワーク越しにアクセス可能
        \end{itemize}
      \end{itemize}
    \end{column}
    \begin{column}{.25\textwidth}
      \includegraphics[width=\textwidth]{image/ch02dia1.pdf}
    \end{column}
  \end{columns}
\end{frame}

\begin{frame}[t,fragile]{URI (Uniform Resource Indicator)}
  \begin{itemize}
    \setlength{\itemsep}{1em}
  \item URI: (ネット上の)場所や名前を表す書式
    \begin{itemize}
    \item 例: URL (Uniform Resource Locator)

      {\footnotesize \url{https://wistaria@itc-lms.ecc.u-tokyo.ac.jp/lms/course/view.php?id=74564}}
    \item {\color{red}\tt https:} スキーム(scheme): プロトコル・アクセス方法を指定
    \item {\color{red}\tt wistaria@} ユーザ情報: ユーザ名(やパスワード)を指定
    \item {\color{red}\tt itc-lms.ecc.u-tokyo.ac.jp} ホスト名・サーバ名
    \item ``{\color{red}\tt //}''\ からホスト名までをオーソリティ(authority)と呼ぶ
    \item {\color{red}\tt /lms/course/view.php} パス名
    \item {\color{red}\tt ?iid=74564} クエリ(query): サーバへの指示や命令
  \end{itemize}
  \item SubversionのリポジトリのURIの例

    {\scriptsize \tt {\color{red} svn+ssh:}//{\color{blue} ce05151598@}cmp.phys.s.u-tokyo.ac.jp/home/ce05151598/svnroot}
  \end{itemize}
\end{frame}

\begin{frame}[t,fragile]{注意事項}
  \begin{itemize}
    %\setlength{\itemsep}{1em}
  \item 大きなバイナリファイル(pdf, exe, doc, tar.gz, zipなど)はなるべくsubversionで管理しない
    \begin{itemize}
      \item バイナリファイルはうまく差分が扱えない (マージできない)
    \end{itemize}
  \item チェックアウトしたディレクトリ「以外」でのファイルの編集は危い
    \begin{itemize}
      \item チェックアウト ⇒ ファイルをコピー ⇒ コピーしたファイルを変更 ⇒ チェックアウトしたディレクトリで svn update ⇒ 変更したファイルをチェックアウトしたディレクトリに戻す ⇒ svn commit ⇒ {\color{red}他の人の変更点を取り消してしまう!}
    \end{itemize}
  \item チェックアウトしたディレクトリには、管理用ディレクトリ'.svn'ができている
    \begin{itemize}
      \item チェックアウトしたオリジナルバージョンが保存されている
    \end{itemize}
  \item svn stat, svn diff はネットワークにつながってなくても使用可
  \end{itemize}
\end{frame}

\begin{frame}
  \frametitle{バージョン管理システムの欠点(面倒な点)}
  \begin{itemize}
  \item 修正前に最新の状態にアップデートしなければならない \\
   ⇒ 慣れると習慣になります
  \item 全ての修正を「コミット」しなければならない \\
    ⇒ 慣れると習慣になります
  \item 衝突(コンフリクト)が発生した時に対処しなければならない \\
    ⇒ 衝突に気づかずに修正してしまうほうが怖いです
  \item サーバのセットアップが面倒くさい \\
    ⇒ まずはホスティングサービス(github, sourceforge, bitbucket)を試してみましょう \\
    ⇒ まわりにいるプロ(?)に相談しましょう \\[.5em]
  \item バージョン管理システムを使うと作業効率が倍以上になる \\
    ⇒ {\color{red} 使わないと人生を半分損する}
  \end{itemize}
\end{frame}

\begin{frame}[t,fragile]{実習: Subversion (1)}
  \begin{itemize}
    %\setlength{\itemsep}{1em}
  \item 「ハンドブック 付録A バージョン管理システム」を読みながら以下の実習を行う
  \item リポジトリの作成 (photonにログインして作業)

    {\tt \$ \underline{svnadmin create \$HOME/svnroot}}
  \item リポジトリ内にフォルダを作成(iMacにもどって作業)

    {\tt \$~\underline{svn mkdir -m 'New folder' svn+ssh://.../svnroot/prog}}
  \item リポジトリの中をのぞいてみる

    {\tt \$ \underline{svn ls svn+ssh://.../svnroot}}

  \item リポジトリからのチェックアウト

    {\tt \$ cd \$HOME}
    
    {\tt \$ \underline{svn co svn+ssh://.../svnroot/prog}}

    ローカルに{\tt prog}ディレクトリが作成される(まだ中身は空)
  \end{itemize}
\end{frame}

\begin{frame}[t,fragile]{実習: Subversion (2)}
  \begin{itemize}
    %\setlength{\itemsep}{1em}
  \item ファイルの作成

    {\tt \$ \underline{cd prog}}
    
    {\tt \$ \underline{emacs main.c}}
  \item ファイルをsubverionの管理下に置く

    {\tt \$ \underline{svn add main.c}}
  \item 状態の確認

    {\tt \$ \underline{svn stat}}
  \item ファイルのコミット(サーバへの送信)

    {\tt \$ \underline{svn ci -m 'First version'}}
  \end{itemize}
\end{frame}

\begin{frame}[t,fragile]{実習: Subversion (3)}
  \begin{itemize}
    %\setlength{\itemsep}{1em}
  \item もう一つ別のディレクトリにチェックアウトしてみる

    {\tt \$ \underline{cd \$HOME}}
    
    {\tt \$ \underline{svn co svn+ssh://.../svnroot/prog other}}

    {\tt \$ \underline{cd other}}
    
  \item {\tt other}の下のファイルを修正、状態確認、コミット

    {\tt \$ \underline{emacs main.c}}

    {\tt \$ \underline{svn stat}}

    {\tt \$ \underline{svn ci -m 'Fixed a bug'}}
  \item {\tt \$HOME/prog}の下のファイルはそのまま
  \item 最新の状態にアップデート

    {\tt \$ \underline{cd \$HOME/prog}}

    {\tt \$ \underline{svn udpate}}
  \end{itemize}
\end{frame}

\begin{frame}[t,fragile]{実習: Subversion (4)}
  \begin{itemize}
    %\setlength{\itemsep}{1em}
  \item コミットしようとしたファイルがすでに他の人(or 他の場所)から更新されていたら
    \begin{itemize}
      \item {\color{red} コミットは失敗する}
    \end{itemize}
  \item 対処法
    \begin{itemize}
      \item まずは{\tt svn update}
      \item ファイル内の別の場所が更新されている場合: マージが成功するので、その後{\tt svn ci}
      \item ファイル内の同じ場所が更新されている場合: 衝突(コンフリクト)!

        ディレクトリ内に {\tt main.c.rXXX}, {\tt main.c.rYYY}, {\tt main.c.mine} ができる

        3つのファイルを参照しながら{\tt main.c}を編集し, {\tt svn resolved main.c}を実行した後、コミット
    \end{itemize}
  \end{itemize}
\end{frame}

\begin{frame}[t,fragile]{実習: Subversion (5)}
  \begin{itemize}
    \setlength{\itemsep}{1em}
  \item その他のコマンド
    \begin{itemize}
      \item svn mkdir dir
      \item svn move file1 file2
      \item svn copy file1 file2
      \item svn delete file
    \end{itemize}
  \item 変更履歴などを見る
    \begin{itemize}
      \item svn diff -r XXX {\ \# リビジョンXXXと現在の作業ファイル}
      \item svn diff -r XXX:YYY {\ \# 二つのリビジョン間}
      \item svn diff -c XXX {\ \# リビジョンXXX-1とXXX}
      \item svn log file
      \item svn annotate file {\ \# それぞれの行を誰がいつ変更したか}
    \end{itemize}
  \end{itemize}
\end{frame}


\section{}
\begin{frame}[t,fragile]{実習・講義予定}
  \begin{itemize}
    \setlength{\itemsep}{1em}
  \item 実習 EX1
    \begin{itemize}
    \item リモートログイン
    \item C言語, グラフ作成, \LaTeX
    \item 数値微分、Newton法
    \end{itemize}
  \item 実習 EX2
    \begin{itemize}
    \item バージョン管理システム
    \item 摩擦のあるバネの問題、解の精度
    \end{itemize}
  \item 講義 L3
    \begin{itemize}
    \item 連立一次方程式の解法、直接法と反復法
    \item ライブラリ(LAPACK)の利用
    \end{itemize}
  \end{itemize}
\end{frame}

\end{document}
